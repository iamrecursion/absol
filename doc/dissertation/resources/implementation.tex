% Present an overview of the software system, and a high-level discussion of the implementation proces.
% Reflection on the choice of languages, tooling and techniques (project management).
% Make sure to focus on the MAIN issues with implementation.
% Follow the same architectural ordering as the design section.

\chapter{Implementation} % (fold)
\label{cha:implementation}
This section will:
\begin{itemize}
    \item Follow a similar structure to algorithms, using code listings to illustrate how the abstract algorithms were turned into a concrete system. 
    \item Examine the compromises or changes to the algorithm that had to take place.
\end{itemize}

\section{Building the Application Framework} % (fold)
\label{sec:building_the_application_framework}
% Designing state into the parser (two stage impl)
% How did the NT tracker come about - why was it insufficient to track all nt parses the same way? -> they are all parsed the same, but they have different semantic meaning for the precondition verification. 

% section building_the_application_framework (end)

\section{Building the Lexer and Parser} % (fold)
\label{sec:building_the_lexer_and_parser}
% The precondition verifier is implemented in separate functions as much as possible. This ensures that it doesn't add unnecessary clutter to the parser itself. 

% section building_the_lexer_and_parser (end)

\subsection{Building the Verification Framework} % (fold)
\label{sub:building_the_verification_framework}
% Need to discuss the practicalities of solving the mutually recursive rule problem (the production trace and additional RuleTag types). 
% The theoretical underpinning of this. 


% subsection building_the_verification_framework (end)

\section{Tooling and Language Choices} % (fold)
\label{sec:tooling_and_language_choices}

\subsection{Reflecting on the Language Choice} % (fold)
\label{sub:reflecting_on_the_language_choice}

% subsection reflecting_on_the_language_choice (end)

% section tooling_and_language_choices (end)

% chapter implementation (end)
