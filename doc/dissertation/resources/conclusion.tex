\chapter{Conclusion} % (fold)
\label{cha:conclusion}
As a project, \gls{absol} has been a great success.
Despite the numerous failures of the project, the end result is a novel contribution to the state of the art in language verification. 
Despite the flaws in both the metalanguage and metacompiler, the achievements of this project cannot be ignored. 
The main successes of this project are twofold:
\begin{itemize}
    \item \textbf{Advancing the State of the Art in Language Verification:} With the combination of the integrated syntactic and semantic specification in the form of \gls{metaspec}, and the novel verification algorithms implemented in \gls{absol} itself, this project provides two major novel contributions to the language verification ecosystem.
    \item \textbf{An Extensible Metacompiler:} The final version of the metacompiler is a functional and extensible solution to the language verification problem for \glspl{dsl} with limited semantics.
    It allows the language designers to create in relative safety, providing clear and concise diagnostics to help diagnose errors.
\end{itemize}

As a tool, \gls{absol} came very close to meeting its initial project goals, with only the omission of the code-generation step really marring that record. 
Nevertheless, this toolchain goes a long way to making provably correct \glspl{dsl} a reality in the context of the kind of systems that inspired this project.
It allows for the definition of capable \glspl{dsl}, and also for their verification, ensuring that the language semantics are total and thus eliminating entire classes of bug in the host systems.
With the implementation of the remaining system components as outlined in \autoref{sub:improving_the_absol_toolchain} it would be a fully-fledged system for \gls{dsl} creation.\\

During the development of this system, the metalanguage and algorithmic design processes went incredibly smoothly, resultant from the weeks of careful thought that had been put into both. 
This meant that the corresponding development proceeded well, working directly from these comprehensive designs.
Conversely, the lack of rigid processes imposed on the development and testing of the system, resulted in a not-insignificant number of bugs being found far later than was ideal.
Despite this, the system works as intended (even though it's not \textit{quite} complete), and its operational nature is a big success. \\

The final result, then, is this: \gls{absol}, while a flawed system, is a complex and powerful tool for the creation of provably correct \glspl{dsl}. 
It has produced novel contributions to the state of the art in language verification through the creation of the \gls{metaspec} metalanguage and the dual-part semantic verification engine that allows automated semantic verification while retaining useful language semantics.
It hence fills a niche in the language verification ecosystem, bridging the gap between complex semantic systems and systems so inflexible as to be unusable.
While it still needs some work, the future is bright, with \gls{absol} both having a clear path for further development, and advancing the state of the art. 

% chapter conclusion (end)
