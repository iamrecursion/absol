\chapter{The Metaspec Grammar} % (fold)
\label{cha:the_metaspec_grammar}
Metaspec, as a language, has a fairly complex syntactic structure. 
The syntax for the language is represented using \gls{ebnf}, and is included in its entirety below. \\

For an overview of standard \gls{ebnf} syntax, please see the official standard for the metasyntactic notation published in \citet{standard1996ebnf}.

\begin{minted}[fontsize=\blockfont]{text}

(*
    This file defines the grammar for the Syntax of the metalanguage 'metaspec'.

    It uses the EBNF syntax defined in Sections 4 and 5 of the ISO-14977 EBNF
    standard.

    The grammar does not care about whitespace except in the case of single-line
    comments which are ended by an EOL character (\n, \r, \r\n), in a
    platform-specific manner.

    Comments, defined by the metaspec-comment grammar element, are stripped 
    before parsing.
*)

(*
    This section defines literals useful in the definition of the language.
    The UTF-8 literal is defined as all graphemes that can be represented by the
    UTF-8 transformation format as defined in RFC 3629.
    For reference, the special symbols have the following meaning:

        *       repetition
        -       except
        ,       concatenate
        |       disjunction / definition separator
        =       defining
        ;       rule terminator
        []      optional
        {}      repetition
        ()      group
        ? ?     special sequence
*)

utf-8-char = ? all-utf-8-glyphs ?;
text = { utf-8-char }-;
digit = "0" | "1" | "2" | "3" | "4" | "5" | "6" | "7" | "8" | "9";
natural-number = { digit }-,;
integer = [ "+" | "-" ], natural-number;
floating-point-number = natural-number, [ ".", natural-number ];
number = natural-number | integer | floating-point-number;
textual-glyph = utf-8-char - digit;
eol-symbol = ? EOL ?;
literal-quote = ? ASCII-double-quote-symbol ?;
newline = ? \n and \r ?

(*
    This section defines the terminal symbols of the language itself, including:
    - Symbols for grammar definition
    - Symbols for comments in the language
*)

(* Terminals used for defining the grammar *)

repeat-count-symbol = "*";
except-symbol = "-";
disjunction-symbol = "|";
defining-symbol = "::=";
rule-termination-symbol = ";";

optional-start-symbol = "[";
optional-end-symbol = "]";
group-start-symbol = "(";
group-end-symbol = ")";
repeat-start-symbol = "{";
repeat-end-symbol = "}";

special-sequence-start-symbol = "<?";
special-sequence-end-symbol = "?>";

start-symbol-start = "<<";
start-symbol-end = ">>";
non-terminal-start = "<";
non-terminal-end = ">";

(* Comment Symbols *)

line-comment-start-symbol = "//";
block-comment-start-symbol = "(*";
block-comment-end-symbol = "*)";

(* Semantic Definition Symbols *)

semantic-behaves-as = "-->";
evaluates-to = "<=";
where-symbol = ":";
semantic-and = ",";
semantic-assign = "=";

semantic-environment-symbol = "e";
semantic-environment-input-symbol = "<--"
environment-access-symbol = ".";
environment-defines-symbol = ":";
semantic-list-delimiter = ",";
semantic-disjunction = "|";

semantic-block-start = "{";
semantic-block-end = "}";
restriction-block-start = "(";
restriction-block-end = ")";

syntax-access-start-symbol = "[";
syntax-access-end-symbol = "]";

special-syntax-start = "(";
special-syntax-end = ")";

(*
    This section defines the allowed types of identifiers in metaspec.
*)

non-terminal-identifier = 
    textual-glyph, { textual-glyph | natural-number | "-" | "_"};
terminal-string =  { utf-8-char - newline };
semantic-identifier = 
    (textual-glyph, { textual-glyph | natural-number | "-" | "_"})
    - (semantic-type | metaspec-feature | semantic-special-syntax);
string-literal = literal-quote, text, literal-quote;
semantic-type
    = "any"
    | "none"
    | "bool" 
    | "natural"
    | "integer"
    | "int32"
    | "int64"
    | "uint32"
    | "uint64"
    | "float"
    | "double"
    | "integral"
    | "floating"
    | "number"
    | "string"
    | "list"
    | "matrix" ;

(*
    This section defines the grammar of metaspec itself.
    The start symbol is 'metaspec'.
*)

metaspec = metaspec-defblock; (* file cannot be empty *)

(* 
    Comments in metaspec are defined as follows, and are removed in a 
    preprocessing step. They are hence not represented in the rest of the 
    language grammar in an explicit fashion. 
*)
metaspec-comment =
    line-comment-start-symbol, { utf-8-char }, eol-symbol |
    block-comment-start-symbol, { utf-8-char }, block-comment-end-symbol;

(* 
    All of these blocks must be defined once in order.
*)
metaspec-defblock = 
    name-defblock, rule-termination-symbol, 
    version-defblock, rule-termination-symbol, 
    using-defblock, rule-termination-symbol, 
    truths-defblock, rule-termination-symbol, 
    language-defblock, rule-termination-symbol;

(* 
    Names can be arbitrary, parses to from first to last non-whitespace character
    before the `;`.
*)
name-defblock = "name", where-symbol, { utf-8-char }-,;

(* 
    Version strings can be arbitrary, parses to from first to last 
    non-whitespace character before the `;`.
*)
version-defblock = "version", where-symbol, { utf-8-char }-,;

using-defblock =
    "using",
    where-symbol,
    semantic-block-start,
    [ metaspec-feature, { semantic-list-delimiter, metaspec-feature }]
    semantic-block-end;

truths-defblock =
    "truths",
    where-symbol, 
    semantic-block-start,
    semantic-evaluation,
    { ", ", semantic-evaluation },
    semantic-block-end;

(* For defining the language itself *)
language-defblock =
    "language",
    where-symbol,
    semantic-block-start,
    language-definition,
    semantic-block-end;

(* 
    These features import language features into scope.

    The syntax and usage of these features is tbc, and they appear in semantic
    portions of the defined language.
*)
metaspec-feature
    = "base"
    | "number"
    | "string"
    | "list"
    | "matrix"
    | "traverse"
    | "funcall" ;

(*
    The language described by the metalanguage is defined in terms of rules 
    that combine syntax definitions and semantics.

    The start symbol must be defined first, followed by the productions of the
    language.
*)
language-definition = start-rule, { language-rule };

non-terminal = non-terminal-start, non-terminal-identifier, non-terminal-end;
terminal = literal-quote, terminal-string, literal-quote;
start-symbol = start-symbol-start, non-terminal-identifier, start-symbol-end;

start-rule = start-symbol, defining-symbol, language-rule-body;

language-rule = non-terminal, defining-symbol, language-rule-body;

language-rule-body = 
    syntax-expression,
    rule-termination-symbol;

(* These NTs use definitions adapted directly from ISO 14977 - EBNF *)
syntax-expression = 
    syntax-alternative, { disjunction-symbol, syntax-alternative };

syntax-alternative = syntax-term, { syntax-term }, [ language-rule-semantics ];

syntax-term = syntax-factor, [ except-symbol, syntax-exception ];

syntax-exception = 
    ? a syntax-factor that can be replaced by one containing no NT symbols ?;

syntax-factor = [ integer, repeat-count-symbol ], syntax-primary;

syntax-primary = 
    syntax-optional | 
    syntax-repeated |
    syntax-grouped |
    syntax-special |
    non-terminal |
    terminal;

syntax-optional = optional-start-symbol, syntax-expression, optional-end-symbol;

syntax-repeated = repeat-start-symbol, syntax-expression, repeat-end-symbol;

syntax-grouped = group-start-symbol, syntax-expression, group-end-symbol;

syntax-special =
    special-sequence-start-symbol,
    text,
    special-sequence-end-symbol;

(* These productions define the syntax of the semantic definition blocks *)
language-rule-semantics = 
    semantic-behaves-as,
    semantic-block-start,
    semantic-rule,
    semantic-block-end;

semantic-rule = 
    environment-input-rule |
    environment-access-rule |
    special-syntax-rule |
    semantic-evaluation-rule-list;

environment-input-rule =
    semantic-type, 
    semantic-environment-symbol,
    semantic-environment-input-symbol,
    syntax-access-block, (* key *)
    environment-defines-symbol,
    syntax-access-list;

syntax-access-block = non-terminal, syntax-accessor;

syntax-accessor = 
    syntax-access-start-symbol,
    natural-number,
    syntax-access-end-symbol;

syntax-access-list = 
    syntax-access-block, { semantic-list-delimiter, syntax-access-block };

environment-access-rule = 
    semantic-environment-symbol,
    environment-access-symbol,
    syntax-access-block,
    { environment-access-symbol, syntax-access-block };

special-syntax-rule = 
    semantic-special-syntax,
    special-syntax-start,
    [ syntax-access-block | environment-access-rule ],
    { semantic-list-delimiter, (syntax-access-block|environment-access-rule) },
    special-syntax-end;

semantic-evaluation-rule-list = 
    semantic-evaluation-rule,
    { semantic-disjunction, semantic-evaluation-rule };

semantic-evaluation-rule = 
    semantic-type,
    semantic-identifier,
    where-symbol,
    semantic-operation-list,
    semantic-restiction-list,
    where-symbol, 
    semantic-eveluation-list;

semantic-eveluation-list = 
    semantic-evaluation,
    { semantic-list-delimiter, semantic-evaluation };

semantic-evaluation = 
    semantic-block-start
    semantic-type, 
    semantic-identifier,
    evaluates-to,
    [ syntax-access-block | special-syntax-rule ],
    semantic-block-end;

semantic-operation-list = 
    semantic-block-start,
    semantic-operation-assignment
    { semantic-list-delimiter, semantic-operation-assignment },
    semantic-block-end;

semantic-operation-assignment =
    semantic-identifier,
    semantic-assign,
    semantic-operation;

// Revise this section in response to live code changes
semantic-operation 
    = semantic-identifier
    | semantic-value
    | semantic-identifier-access
    | "(", semantic-operation, ")"
    | prefix-semantic-unary-operator, semantic-operation
    | semantic-operation, postfix-semantic-unary-operator
    | semantic-operation, semantic-binary-operator, semantic-operation;

semantic-identifier-access = 
    semantic-identifier, "|", natural-number, "|" | "[", natural-number, "]";

semantic-restriction-list = 
    restriction-block-start
    semantic-restriction,
    { semantic-list-delimiter, semantic-restriction },
    restriction-block-end;

semantic-restriction = semantic-identifier, semantic-restriction-check-operator,
    semantic-value | identifier;

(* The symbols here are dependent on the language using imports *)
semantic-restriction-check-operator = "==" | "!=" | "<" | ">" | "<=" | ">=";

semantic-value = string-literal | number | semantic-boolean;

semantic-boolean = "true" | "false";

semantic-binary-operator = 
    "+" |
    "-" |
    "*" |
    "/" |
    "%" |
    ":" |
    "^" |
    "|" |
    "||" |
    "&&" |
    "&" |
    "==" |
    "!=" |
    "<" |
    ">" |
    "<=" |
    ">=";

prefix-semantic-unary-operator = "!" | "-" | "++" | "--";

postfix-semantic-unary-operator = "--" | "++";

semantic-special-syntax 
    = "map"
    | "fold"
    | "filter"
    | "defproc"
    | "deffun"
    | "callproc"
    | "callfun"

\end{minted}

% chapter the_metaspec_grammar (end)
