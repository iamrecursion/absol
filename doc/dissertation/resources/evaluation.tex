\chapter{Evaluation} % (fold)
\label{cha:evaluation}
The design and development of both \gls{metaspec} and the \gls{absol} metacompiler toolchain has involved an immense amount of careful thought, time and effort.
The final results of the project are impressive, but flawed.
This section aims to examine and discuss both the major successes of the \gls{absol} project, including how it succeeded at its high-level goals and advanced the state of the art in language verification.
It will also provide a discussion of the major failings of the project, particularly around scoping, design and the development process itself.
Finally, it will suggest future work in the area of \gls{dsl} verification, and improvements that could be made to the \gls{absol} toolchain.

\section{Successes of the Project} % (fold)
\label{sec:successes_of_the_project}
The key successes of the \gls{absol} project are twofold. 
The first major success is how well both \gls{metaspec} and the metacompiler toolchain itself were able to meet its high-level requirements: it has been able to achieve all of the requirements detailed for it.
The second major success is the contribution that this project has been able to make to the state of the art in language verification: through its constrained form of language semantics it has been able to provide a flexible and useful toolchain for the creation of provably correct languages. 

\subsection{Meeting the High-Level Goals} % (fold)
\label{sub:meeting_the_high_level_goals}
One of the main concerns during the design and development of both \gls{metaspec} and \gls{absol} has been meeting the high-level system requirements set out in \autoref{sec:high_level_requirements_specification} on \autopageref{sec:high_level_requirements_specification}. 
Both the metalanguage and the metacompiler program have been able to meet all of these requirements due to the comprehensive design process that both underwent. \\

\gls{metaspec} has been a particular success of the project as the detailed and careful design process for the language, as explored in \autoref{cha:designing_the_metalanguage}, has ensured that the metalanguage really appears as one \textit{integrated} entity, rather than being composed out of parts.
One of the most successful portions of the metaspec language is the method by which it provides an integrated syntax and semantic specification (Requirement~\reqref{req:IntegratedSyntaxandSemanticSpecification}).
While the syntax definition format required little adaptation from the \gls{ebnf} it was based upon, the linking of syntax and semantics through the `syntax-access' blocks (\autoref{sub:accessing_syntax_from_the_semantics}) is far more intuitive than would initially have been expected. 
The success of the combined specification is complemented by the amount of design work that went into defining the syntax (and hence features) of the user-defined semantics (\autoref{sub:user_defined_semantics}).
The comprehensive nature of the translation from big-step operational semantics to this language has retained the full set of semantic features, and is a big boon in the expression of \gls{dsl} semantics.\\

The second major success of the metalanguage arises from the ease with which it allows users to define languages. 
While part of this ease comes from the support of the metacompiler tools, the language itself provides obvious ways to accomplish the goals of the \gls{dsl} designer. 
While in part this comes from the neat integration of syntactic and semantic definitions, the designer also benefits significantly from the approach it takes to defining languages. 
Through the provision of the \mintinline{text}{truths} defblock, it becomes simple for language designers to declare when they know things terminate (such as the literals shown in \autoref{sec:an_example_language}).
Beyond this, the provision of semantic special-forms, especially the special-syntax rules, means that language designers are \textit{not} overly constrained when it comes to defining the capabilities of their language.\\

As much as the metalanguage has been one of the largest successes of the project, the \gls{absol} toolchain itself has also been a major success. 
Much like \gls{metaspec}, the metacompiler has also been able to meet all of its high-level goals placed upon it.
Within these goals, however, it has achieved two highly-significant successes.\\

The first of these successes comes through the verification capabilities of the metacompiler. 
Though the design process for the verification algorithms (\autoref{sec:the_core_algorithms}) was complex and involved, the final implementation of the verification theory has turned out to be elegant and fairly simple.
This means that the success is twofold: the verification portion of the metacompiler remains extensible, but also in the capabilities it affords the user.
In being able to verify the language as it is being developed, it provides the user with significant assurance through this development process.
Through the practice of developing example languages, such as `spreadsheet', discussed in \autoref{sec:an_example_language}, it became apparent that the support from the metacompiler allowed for more experimentation and less mental load.
As the verification of the language was offloaded to \gls{absol}, the language designer could better concern themselves with the \textit{form} of the language and the \textit{local} semantics, rather than constantly ensuring global semantic correctness.
While this success comes from a very limited sample-size, the benefits felt to a highly-experienced user could imply that the metacompiler would be an even more significant boon to the less-experienced user. \\

The other major success of the metacompiler itself is the language diagnostic facility that it provides.
Though derived from a difficult design task, the use of the \mintinline{haskell}{RuleTag} type and associated monoidal properties for tracking error information have allowed the metacompiler to provide the user with comprehensive diagnostics.
The resultant diagnostics are very obvious, providing both a message describing the problem and a diagnostic trace of where the problem arises from.
Hence, they allow the users of the metacompiler to quickly identify and fix errors in their language definition, enhancing the utility of the metacompiler.

% subsection meeting_the_high_level_goals (end)

\subsection{Contributions to the State of the Art} % (fold)
\label{sub:contributions_to_the_state_of_the_art}
% - Funcall and the other special forms are important for the quality of the project.
% - The ability to verify at the language level while still retaining useful functionality for the creation of capable languages. 
% - Integrated syntax and semantic specifications that operate together intuitively. 

As discussed above, the project has seen success in terms of meeting its high-level goals.
However, alone that does not make the project worthy of note.
Much of the major success of this project arises from the way in which it contributes to the state-of-the-art in the area of formal language verification.\\

While it would be possible to define a language purely based on big-step operational semantics, the restrictions required to prove termination about this language would make it \textit{useless}.
The main contribution of \gls{absol} to the state of the art is, resultantly, the ability to define useful (though still limited) languages while retaining the ability to prove them correct.
This contribution to the state of the art arises from two main factors:
\begin{enumerate}
    \item \textbf{The User-Defined Semantics:} The users of \gls{metaspec} are given the ability to define the semantics of their own \gls{dsl} within the restrictions imposed on these semantics.
    While sufficient for defining many common operations, they are not enough to make a useful language on their own.
    \item \textbf{The Special Semantic Forms:} The special semantic forms provide known-to-terminate functionality to the \gls{dsl} designer while allowing the implementation of more complex, useful language features.
\end{enumerate}

It is the \textit{combination} of both of these features in a context where the termination properties can be proved that is the real contribution of the project.
Previous frameworks have existed for specifying complex language semantics, such as Mosses' Action Semantics \citep{mosses1992action}. 
However, such semantic frameworks become too complex to automatically verify languages based upon them.
As a counterpoint simpler semantic frameworks exist, such as restricted forms of operational semantics, that \textit{can} be automatically proven to terminate (under certain restrictions) \citep{Zhang:2004:SSD:981009.981013}.
Much like the more complex frameworks these also have an issue, except here it is that they are incapable of representing more complex (and hence useful) language semantics while still being provable.\\

To this end, the combination of \gls{metaspec} and \gls{absol} lies in the middle of these two roads.
While it provides restricted methods for users to define semantics, allowing semantics forms such that they can automatically be shown to terminate,
it \textit{also} provides additional semantic forms.
These additional forms, while they have to be proved total externally, can then interact with the proof mechanism simply. 
It is this combination that represents the major contribution of the project, allowing the definition of capable languages while still retaining the ability to prove the language total.
This means that all programs written in these languages will also be total.\\

The second major contribution of this project to the state of the art in language verification is the creation of \gls{metaspec}. 
Prior to the creation of this language there appeared to be no metalanguages that allowed for the combined specification of both \textit{syntax} and \textit{semantics} in an integrated form. 
While \autoref{sub:specifying_language_semantics} explores multiple methods (of varying complexities) for the specification of language semantics, research found that none of them properly integrated with the syntactic descriptions of languages.
This separation meant that it was often difficult to reason about the semantics of a given program fragment, given that syntax and semantics are so intrinsically linked in a verification context.\\

This highlights that the creation of the \gls{metaspec} metalanguage offers a significant contribution to the state of the art in that it provides an \textit{integrated} method for specifying both syntax and semantics in one file.
This contrasts dramatically with existing approaches that keep syntax and semantics separate, or annotate existing metasyntactic notations rather than using a custom-designed notation.
As a result, it is clear that this integrated approach brings significant benefits to the usability of defining languages.

% subsection contributions_to_the_state_of_the_art (end)

% section successes_of_the_project (end)

\section{Failings of the Project} % (fold)
\label{sec:failings_of_the_project}

As much as the design and development of the project has been able to meet its high-level goals, this does not mean to imply that it has been without its problems. 
During the course of the project there have been multiple instances where portions of the project could have been better-designed, or processes being used could have been refined. 
This section aims to examine these major project failings and discuss how they could have been mitigated. 

\subsection{Scoping the Project} % (fold)
\label{sub:scoping_the_project}
One of the major failings of the project was an initial inability to define the scope of the system appropriately.
Initial requirements and the corresponding system designs intended to incorporate both the full semantic guard verification algorithm and to perform full code-generation for the \gls{dsl} compiler, as discussed in \autoref{sub:out_of_scope_requirements}. \\

While this would not have been a major issue had the project scope been decreased quickly, the project progressed under the impression that both of these features would be feasible to develop within the provided time frame.
This meant that some not insignificant effort was wasted performing design work for them. 
In particular, significant time was spent attempting to define an algorithm for the checking of guard completeness before this was determined as difficult enough to rule out of scope (see \nameref{ssub:checking_guard_completeness} on \autopageref{ssub:checking_guard_completeness}).
While this could have been ruled out of scope sooner through more targeted research (see \nameref{ssub:guard_completeness_and_project_scope} on \autopageref{ssub:guard_completeness_and_project_scope}), it was thought of as an easy problem and not worthy of detailed research. 
This was very much incorrect.\\

The other major scoping issue on the project was the insistence on the creation of a full `product' pipeline for \gls{absol}.
This meant that the generation of code for the \gls{dsl} compiler was considered as in scope for a significant portion of the project.
While it was always recognised as a large body of work (and a non-novel portion of the project), it did not have as much design work dedicated to it as the guard completeness problem had.
Nevertheless, the project under-appreciated how much work would be contained within the development of the parser and verification engine.
When this work became apparent, the code-generation module was swiftly ruled out of scope. \\

While the now useless design and development work dedicated to these portions of the project has not impacted on the project's completion in any undue fashion, it is clear that having this time available for other tasks would have achieved a much more efficient time utilisation.
As a result, the failure to scope the project appropriately is one of the major failings of the project.

% subsection scoping_the_project (end)

\subsection{Metalanguage Design Issues} % (fold) 
\label{sub:metalanguage_design_issues}
While \gls{metaspec} represents a big step in the design of metalanguages for syntactic and semantic specification, it is not without some significant issues.
The majority of these issues arise from portions of the language that were not given enough consideration during the design process, though some also arise from issues with the language grammar. 

\subsubsection{Issues with the Metaspec Grammar} % (fold)
\label{ssub:issues_with_the_metaspec_grammar}
One of the main issues with the current form of the \gls{metaspec} language grammar is that it is under-constrained in certain circumstances. 
This means that, from a grammatical standpoint, it will admit syntax that is not actually valid at the metalanguage semantics level.
There are a few main examples of this:
\begin{itemize}
    \item \textbf{Special-Syntax Arguments:} The grammar in its current form admits any number of arguments to each special syntax form. 
    This means that the verification of these arguments is not part of parsing, but later at the level of \gls{dsl} compilation time (when the types of the arguments are also checked).
    While this does not impinge on the correctness of the verification result, it would be a significant improvement to enforce the correct number of arguments for each piece of special syntax at the level of the language grammar. 
    \item \textbf{Semantic Associations:} The semantic association was joined to the existing syntactic grammar at the level of the alternation: each term in \mintinline{text}{<a> | <b> | ...} could each provide its own semantics.
    However, this addition did not recognise the fact that alternations can contain sub-alternations (e.g. \mintinline{text}{(<a> | <b>) | <c>}).
    At the level of the grammar, it is possible to write \gls{metaspec} statements which assign semantics to both \mintinline{text}{<a>} and \mintinline{text}{<b>}.
    While these don't have any semantic meaning and are ignored by all stages of the metacompiler pipeline, it would ideally not be possible to write such expressions in the grammar. 
\end{itemize}

Fixing both of these flaws in the current metalanguage design would result in minor, but still useful improvements to the metalanguage as a whole. 

% subsubsection issues_with_the_metaspec_grammar (end)

\subsubsection{The Design of the Environment} % (fold)
\label{ssub:the_design_of_the_environment}
%   - No direct use of environment accesses in basic semantics, requiring indirection.
%   - Fairly awkward use of the environment (no ability to delete properties, for example). 
% Environment not particularly well thought-out.

The notion behind the semantic environment was that it would be able to act as a global key-value store. 
However, this idea was not thought through to its full extent, leaving the current version of the environment with the following issues:
\begin{itemize}
    \item \textbf{Defaulting:} When the semantic rules attempt to retrieve a value for which a key does not exist, the environment will return a default for the appropriate type. 
    While this ensures that the semantics are always defined, it is relatively inelegant. 
    Ideally, even in the basic environment rules, the user would be able to specify the default value that they want to return, and even check for the presence of a given key.
    Neither of these are insurmountable obstacles, but would require careful semantic analysis and integration in the current design of metaspec.
    \item \textbf{Usability in Semantic Rules:} The current metaspec grammar precludes environment accesses from occurring in a number of locations where they would be useful (especially in semantic evaluation rules).
    This means that special-syntax provides functionality for performing this in these locations, but it would be much more elegant to just allow the use of standard environment syntax (which would bring back the benefit of all environment-based accesses being instantly visible --- see \autopageref{ssub:environment_input_rules}).
    \item \textbf{No Property Deletion:} Currently, any value stored under a given key in the environment is unable to be deleted.
    While it can be overwritten with some value that is semantically equivalent (in the \gls{dsl} domain) to \mintinline{text}{null}, this is not the same as being able to delete a property from the environment entirely.
    Future work on the environmental design would be able to introduce this feature and ensure that it works in an always-terminating manner. 
    \item \textbf{No Scoping:} The semantic environment is a purely un-scoped store of information at the moment.
    While this does work from a basic perspective, it restricts the ability of \gls{dsl} designers to define scope-based constructs in their language.
    While this is not a \textit{particularly onerous} restriction in the context of \glspl{dsl}, it would not impact the semantic correctness of the environment to allow it to work in a scoped fashion.
    Ideally, future revisions of \gls{metaspec} would allow this.
    \item \textbf{No Error States:} Currently, if a key already exists in the semantics, it will just have its value overwritten silently for any new input with that key.
    Ideally, this would instead provide information to the \gls{dsl} designer via some kind of feedback mechanism, allowing them to reliably ensure that the correct value is associated with the correct key in the environment.
\end{itemize}

Future work on \gls{metaspec} would definitely benefit from revising the nature of the environment to make it more useful to \gls{dsl} designers.
As it is, it's not a particularly useful feature due to the lack of thought that went into its design.

% subsubsection the_design_of_the_environment (end)

\subsubsection{Special-Syntax Flexibility} % (fold)
\label{ssub:special_syntax_flexibility}
While able to be rectified at the grammar level, this is a more in-depth issue than those discussed above. 
In their current form, arguments to a special-syntax rule can only take the form of two things: environment accesses, and syntax access blocks. 
However, as discussed in \autoref{sub:reflecting_on_the_language_design_process}, it could be quite useful to be able to pass semantic constants as arguments to the special-syntax rules as well.\\

The main reason for this is that the current inability to use constants breaks one of the core tenets of a dsl (see \autoref{sec:domain_specific_languages}) --- they should match the notation of their domain as closely as possible.
Much like the example in \autoref{sub:reflecting_on_the_language_design_process}, the ability to use values of the domain-syntax and translate these into the syntax of the underlying semantic functions provides a much more succinct interface at the level of the \gls{dsl}.
As key as the domain-notation-based nature of \glspl{dsl} is to their usage and utility, this is a \textit{major} oversight in the design of the metalanguage.\\

As allowing constants does not alter the termination properties of any of the special syntax rules, this is purely an issue at the metalanguage design level.
Future work on this project could significantly benefit from removing this restriction.
This would allow \gls{metaspec}-defined \glspl{dsl} to far better match their domain notation.

% subsubsection special_syntax_flexibility (end)

% subsection metalanguage_design_issues (end)

\subsection{Metacompiler Design Issues} % (fold)
\label{sub:metacompiler_design_issues}
Much like the metalanguage, the \gls{absol} toolchain itself has some not-insignificant flaws in its design. 
While it, too, provides some novelty and advances the state of the art through the form of its verification algorithms, the rectification of these design issues would improve it as a tool.

\subsubsection{Error Reporting Issues} % (fold)
\label{ssub:error_reporting_issues}
With the diagnostic capabilities of \gls{absol} proving so integral to the \gls{dsl} development process, it is clear that the clarity and explanatory power of these diagnostics is key.
However, while a diagnostic message is available for every error the metacompiler can diagnose, sometimes these messages are not specific enough.\\

This is particularly true of the criteria defined for the semantic operation evaluations.
As the algorithm in \nameref{ssub:verifying_the_evaluation_criterion} on \autopageref{ssub:verifying_the_evaluation_criterion} states, there are six criteria that must be satisfied.
However, the current error reporting groups all of these issues under a single error message, making it difficult for the user of the toolchain to find and correct the issue.\\

This error-reporting deficiency is also true in general.
While many of the errors that exist today \textit{are} helpful, most of these could still provide additional contextual information (e.g. exactly \textit{which} evaluation is wrong).
Future development of the toolchain should definitely dedicate some time to the improvement of these error messages.

% subsubsection error_reporting_issues (end)

\subsubsection{Precondition Verification} % (fold)
\label{ssub:precondition_verification}
It is the responsibility of the metacompiler to verify the preconditions for the verification algorithms are correct, and it does this.
However, from a user-utility perspective, the method by which this is performed is suboptimal.\\

As the algorithm in use performs many of its checks \textit{during} the parse of the input file, any error encountered and output will stop the parse entirely.
This means that even if there are other issues present in the file, the user has to go through this iterative cycle of `check' / `fix'.
It would be much more user-friendly to collate all of these errors during the parsing process and output them in a comprehensive fashion at the end.
This would provide a better experience for the user, and hence should be improved in future work on the metacompiler. 

% subsubsection precondition_verification (end)

% subsection metacompiler_design_issues (end)

\subsection{Development Issues} % (fold)
\label{sub:development_issues}
As discussed in \autoref{sub:the_development_process} and \autoref{sec:the_general_testing_approach}, the approaches to both development and testing on this project were somewhat ad-hoc. 
While the project has been completed successfully, it was not without its headaches, and many of these could have been mitigated or avoided entirely through use of more formal processes.
The lack of formalisation seen throughout this project is that it was mainly treated as a theoretical project, rather than a software-development project.
While it \textit{has been} highly theoretical, in hindsight the application of more rigorous processes and quality standards would have been beneficial.\\

The lack of formalisation of the design process, in particular, is what has led to many of the issues identified in \autoref{sub:metalanguage_design_issues} above.
Ideally, a more formal design process would have been employed by the project, bringing the benefits of additional design rigour.
Doing this would have likely improved the software architecture, algorithmic design and better identified the features required for the metalanguage. 
Applying such a process would likely have led to the environment being removed entirely as a feature until it could have been better developed, and given the list of issues with illustrated in \nameref{ssub:the_design_of_the_environment} above, this would not have been entirely negative. \\

Similarly, \autoref{sub:examining_the_testing_approach_in_hindsight} identifies some major problems with the lack of a rigorous (and automated) testing approach.
The regressions in the parser and verification engine, in particular, that were not caught until much later on cost the project some not-insignificant time to fix.
Employing a more rigorous testing procedure, much like a more rigorous design process, would most likely have identified these regressions immediately, or even prevented them from happening in the first place.\\

The initial intent behind the lack of formal processes was to allow a flexible and unrestricted approach to the design and development of \gls{absol}.
With the benefit of hindsight, however, this approach created more problems than it solved.
Doing the project again would involve far more rigorous design and development processes, as well as the employment of a fully automated test suite.

% subsection development_issues (end)

% section failings_of_the_project (end)

\section{Future Work} % (fold)
\label{sec:future_work}

While \gls{absol} is a substantial achievement, it is far from the end of the evolution of language verification. 
With the contributions to the state-of-the art that this project has made, the theoretical and practical advancements that have resulted from it should be able to be integrated into future work in this space. 
Similarly, the toolchain itself is still not complete, and future evolution of this project should focus on developing the final code-generation stage to take it from theory to useful tool.

\subsection{Evolving the Art of DSL Verification} % (fold)
\label{sub:evolving_the_art_of_dsl_verification}
As discussed above, the project has made two major contributions to the art of \gls{dsl} verification.
Taking these contributions forward could involve either of the following:
\begin{itemize}
    \item \textbf{Developing the Metalanguage:} Providing a novel approach to combining syntax and semantics for language development through its syntax access methodology, \gls{metaspec} provides a robust foundation for the further development of metalanguages.
    \item \textbf{Dual-Level Semantics:} The other innovation of this project was the combination of multiple levels of semantics.
    In doing so it was possible to both provide useful features for \gls{dsl} designers, but also enable provable correctness for these languages.
\end{itemize}

Future work in this area could involve expansion in both of these areas, as outlined below.
However, by the same token, it could involve totally unintended directions for this work.

\subsubsection{The Future of Metaspec} % (fold)
\label{ssub:the_future_of_metaspec}
While \gls{metaspec} itself is limited to the scope of the \gls{absol} project, this does not mean that the principles behind the metalanguage cannot be employed in other avenues. 
The key innovation in metaspec that is more broadly applicable is the syntax-access blocks, by which the syntax and semantics are tied together directly. 
Future work could certainly evolve the metalanguage, using this integration principle to combine the syntactic specification with more complex semantic specifications (e.g. Action Semantics).
These more complex semantic specifications would bring significantly increased expressive power.\\

While these expanded specifications may not always be able to be automatically proved, though \citet{Mosses:2009:CS:1596486.1596489} has pursued significant work in that area, this integrated approach brings significant benefits over existing approaches.
Combining syntax and semantics in future would make it much easier to reason about the semantic properties of languages, much as it has done for \gls{absol}.

% subsubsection the_future_of_metaspec (end)

\subsubsection{Multi-Level Semantic Proofs} % (fold)
\label{ssub:multi_level_semantic_proofs}
The multi-level semantic proofs could also be expanded.
Theoretically, this could be taken in two directions:
\begin{enumerate}
    \item \textbf{Additional Proof Levels:} Further levels of semantics could be implemented, allowing the proof of more complex language properties.
    \item \textbf{Creation of Additional Capabilities:} Additional capabilities could be added within the existing proof framework.
\end{enumerate}

Both of these are likely to be fruitful endeavours, though it will be important to ensure that the scope within which these operate is still constrained.
Without appropriate constraints on the language being constructed, the proof of semantics becomes impossible.\\

An alternate direction for the semantic proofs is to move the semantic checking to \textit{program} compile time.
While that is far out of scope for this project, it enables the proof of many more properties while increasing the expressive potential of the language.
This would rely more on the work of \citet{hinze2010reasoning} and \citet{nordstrom1988terminating} on the termination of general recursion than any insight brought by this project, but the development and use of \gls{absol} has demonstrated just how useful such proof-based approaches can be.

% subsubsection multi_level_semantic_proofs (end)

% subsection evolving_the_art_of_dsl_verification (end)

\subsection{Improving the ABSOL Toolchain} % (fold)
\label{sub:improving_the_absol_toolchain}
Aside from the developments of the state of the art based upon the work of this project, \gls{absol} itself is still unfinished.
Future work could certainly involve the development of the incomplete portions of the toolchain, and evolutions of existing features.
Some proposals for further development are as follows:
\begin{itemize}
    \item \textbf{Code Generation:} In order for \gls{absol} to become a truly useful tool, it needs to be capable of generating a compiler for the defined DSL.
    As the \gls{metaspec} files contain both syntax and semantics all of the necessary information is available to the metacompiler.
    Generation of syntax parsers based on the grammar is easy, especially with the use of combinator-based parsing.
    While the attribution of semantics to these new productions is harder, work by \citet{diehl1996semantics} has shown that it is doable.
    \item \textbf{Further Special Features:} As discussed in \autoref{sub:discounted_language_features}, there are still multiple special language features that could be proven to terminate and then implemented.
    These would bring additional expressive power and expand the capabilities of \gls{absol} \glspl{dsl}.
    \item \textbf{Proper Guard Verification:} As has been discussed throughout the document, particularly in \autoref{sub:guard_checking}, the checking of guards for completeness is a difficult problem.
    However, \autoref{sub:guard_completeness_checking} discusses a possible method for accomplishing it, and hence it is likely to be a fruitful avenue to pursue in the future. 
\end{itemize}

In addition to these undeveloped features of the toolchain, there is a wealth of improvements to be made, as discussed in the above sections.
Ideal future development of \gls{absol} would fix these issues, and then continue on to finalise toolchain development as discussed above.

% subsection improving_the_absol_toolchain (end)

% section future_work (end)

% chapter evaluation (end)
