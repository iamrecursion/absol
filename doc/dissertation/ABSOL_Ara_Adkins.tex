%DocumentClass
\documentclass[a4paper,11pt]{report}

%Additional Packages
\usepackage{floatrow}
\usepackage{scrextend} %for variable indentation
\usepackage[chapter]{algorithm} %For algorithms
\usepackage{algpseudocode} %For algorithms
\usepackage[]{array}
\usepackage{amsmath} %Extra formula-writing functionality
\usepackage{amssymb} %More formula-writing functionality
\usepackage[toc]{appendix}
\usepackage{bm}
\usepackage{bold-extra} %Small caps
\usepackage{caption}
\usepackage{color} %Colour stuff (mostly for the highlight custom command)
\usepackage{enumerate} %For lists
\usepackage{fancyhdr}
% \usepackage{float} %better float control
\usepackage[T1]{fontenc}
\usepackage[utf8]{inputenc} %for font encoding
\usepackage{framed} %For frames around blocks
\usepackage[top=3cm, bottom=3cm, left=3.5cm, right=3cm]{geometry} %Fix page margins
\usepackage[hidelinks]{hyperref} %For URL formleatting (makes clickable links)
\usepackage[toc, xindy, acronym, nonumberlist, nopostdot]{glossaries}
\usepackage{graphicx} %For including images, etc.
\usepackage{listings} %For including code
\usepackage{longtable} %for multi-page tables
\usepackage{mathrsfs} %For maths script fonts
\usepackage{newclude}
\usepackage{titling}
\usepackage{titlesec}
\usepackage[nottoc]{tocbibind}
\usepackage[]{natbib} %For the bibliography
%\usepackage[superscript]{cite}
\usepackage{pdfpages} %For including PDFs
\usepackage[]{pdflscape}
\usepackage{stmaryrd}
\usepackage{subcaption}
\usepackage{tabu} %more pretty tables
\usepackage{tabulary} %for nice tables
\usepackage{tabularx} %also for nice tables
\usepackage{ulem}
\usepackage{xcolor}
\usepackage{xparse}
\usepackage{xstring}
\usepackage{textcomp}
\usepackage[prefix=sol-]{xcolor-solarized}
\usepackage[]{marvosym}
\usepackage{microtype}
\usepackage{lmodern}
\usepackage[chapter]{minted}
\usepackage{epigraph}

\usepackage{relsize}

%CustomCommands
\newcommand{\highlight}[1]{\colorbox{yellow}{#1}} %Highlights text
\newcommand{\limplies}{\to} %Creates the logical implication sign
\newcommand{\liff}{\leftrightarrow} %Creates the double logical implication sign
\newcommand{\leftabs}{\left\lvert} %Left absolute value bracket
\newcommand{\rightabs}{\right\rvert} %Right absolute value bracket
\newcommand{\textbsc}[1]{\textsc{\textbf{#1}}}
%\renewcommand\thesubsection{(\alph{subsection})} %Make subsections alphabetical
\newcommand{\id}{\hspace*{12pt}}
\newcommand{\newpar}{\vspace{12pt}}
\newcommand{\lam}{$\lambda$}
\newcommand{\alp}{$\alpha$}
\newcommand{\bet}{$\beta$}
\newcommand{\aequiv}{=_\alpha}
\newcommand{\bequiv}{=_\beta}
\newcommand{\bconv}{\limplies_\beta}
\newcommand{\context}{$\Gamma$}
\newcommand{\rspace}{\;\;\;\;\;\;\;\;}
\newcommand{\eval}{\Downarrow}
\newcommand{\goesto}[0]{\MVRightarrow}
\newcommand{\mathgoesto}[0]{\mathord{\text{ \goesto\;}}}
\newcommand{\la}{\langle}
\newcommand{\ra}{\rangle}

% Some algorithm shorthand
\newcommand{\Map}[2]{\textbf{map} #1 #2}
\newcommand{\Reduce}[3]{\textbf{reduce} #1 #2 #3}
\algnewcommand{\LineComment}[1]{\State \(\triangleright\) #1}

\makeatletter
\newenvironment{breakablealgorithm}
  {% \begin{breakablealgorithm}
   \begin{center}
     \refstepcounter{algorithm}% New algorithm
     \hrule height.8pt depth0pt \kern2pt% \@fs@pre for \@fs@ruled
     \renewcommand{\caption}[2][\relax]{% Make a new \caption
       {\raggedright\textbf{\ALG@name~\thealgorithm} ##2\par}%
       \ifx\relax##1\relax % #1 is \relax
         \addcontentsline{loa}{algorithm}{\protect\numberline{\thealgorithm}##2}%
       \else % #1 is not \relax
         \addcontentsline{loa}{algorithm}{\protect\numberline{\thealgorithm}##1}%
       \fi
       \kern2pt\hrule\kern2pt
     }
  }{% \end{breakablealgorithm}
     \kern2pt\hrule\relax% \@fs@post for \@fs@ruled
   \end{center}
  }
\makeatother

\newcounter{requirementcounter}
\newcommand*{\reqlabel}[1]{R\refstepcounter{requirementcounter}\therequirementcounter\label{#1}}
\newcommand*{\reqref}[1]{R\ref{#1}}

\newcommand*{\mklabelcase}[1]{\lowercase{\StrSubstitute[0]{#1}{ }{\_}}}

\newcommand{\requirement}[3]{
    \StrSubstitute[0]{#1}{ }{}[\labelname]
    \textbf{Requirement \reqlabel{req:\labelname}}
    \begin{addmargin}[1em]{0em}
        \textbf{Name:} #1\\ 
        \textbf{Type:} #2\\
        \textbf{Description:}\\
        #3\\
    \end{addmargin}
}

% New Table Column Types
\newcolumntype{L}[1]{>{\raggedright\let\newline\\\arraybackslash\hspace{0pt}}m{#1}}
\newcolumntype{C}[1]{>{\centering\let\newline\\\arraybackslash\hspace{0pt}}m{#1}}
\newcolumntype{R}[1]{>{\raggedleft\let\newline\\\arraybackslash\hspace{0pt}}m{#1}}

% Syntax: \newdualentry[glossary options][acronym options]{label}{abbrv}{long}{description}
\DeclareDocumentCommand{\newdualentry}{ O{} O{} m m m m } {
    \newglossaryentry{gls-#3}{name={#5},text={#5\glsadd{#3}},
        description={#6},#1
    }
    % \makeglossaries
    \newacronym[first=#5, firstplural=#5s, see={[Glossary:]{gls-#3}},#2]{#3}{#4}{#5 \glsseeformat[Glossary:]{gls-#3}{#5}\glsadd{gls-#3}}
    % \newacronym[see={[Glossary:]{gls-#3}},#2]{#3}{#4}{#5\glsadd{gls-#3}}
}

% For quoting \quoteit{quote}{attribution}
\newcommand{\quoteit}[2]{
    \begin{longtable}{p{14cm}}
        \textit{``#1''} \\
        % \hspace{5mm} --- #2 
        \begin{tabular}{R{14cm}}
            --- #2
        \end{tabular}
    \end{longtable}
}

% For definitions \defblock{colsize}{name}{description}
\newcommand{\defblock}[3]{
    \begin{longtable}{l p{#1}} 
        \textbf{#2} & #3
    \end{longtable}
}

\input{resources/bath_dissertation_definitions.tex}

%Various Definitions
\setcounter{tocdepth}{2}
\definecolor{light-gray}{gray}{0.5}

\titleformat{\chapter}
    {\normalfont\huge}  % format
    {\thechapter.}      % label
    {10pt}              % separation
    {\huge\it}          % before-code

% \pagestyle{fancy}
% \lhead{\color{light-gray}Ara Adkins}
% \rhead{\color{light-gray}ABSOL}
% \cfoot{\thepage}
% \renewcommand{\headrulewidth}{0pt}
% \renewcommand{\footrulewidth}{0pt}

\setlength{\parindent}{0pt}
% \setlength{\headheight}{14pt}

\hypersetup{
    colorlinks,
    linkcolor={black},
    citecolor={black},
    urlcolor={black}
}

\makeatletter
\renewcommand\@dotsep{200}
\makeatother

\renewcommand{\labelenumii}{\theenumii}
\renewcommand{\theenumii}{\theenumi.\arabic{enumii}.}

% Listings Styles
\lstset{
    % How/what to match
    sensitive=true,
    % Border (above and below)
    frame=lines,
    % Extra margin on line (align with paragraph)
    xleftmargin=0.5cm,
    % Put extra space under caption
    belowcaptionskip=1\baselineskip,
    % Colors
    backgroundcolor=\color{sol-base3},
    basicstyle=\color{sol-base0}\footnotesize\ttfamily,
    keywordstyle=\color{sol-cyan},
    commentstyle=\color{sol-base01},
    stringstyle=\color{sol-blue},
    numberstyle=\color{sol-violet},
    identifierstyle=\color{sol-base00},
    % Break long lines into multiple lines?
    breaklines=true,
    % Show a character for spaces?
    showstringspaces=false,
    tabsize=4,
    numbers=left,
    numbersep=5pt,
    numberstyle=\tiny\color{sol-base1},
    rulecolor=\color{sol-base01},
    aboveskip=2em,
    belowskip=2em,
    upquote=true,
    basewidth={0.5em,0.5em}
}

\floatsetup[listing]{style=plain}
\newenvironment{longlisting}{\captionsetup{type=listing}}{}
\setminted{
    autogobble=true, % remove common leading whitespace
    breakanywhere=true,
    breakautoindent=true,
    numbers=left,
    mathescape=true, % allow maths symbols
    stripnl=false,
    tabsize=4,
    texcomments=true,
    resetmargins=true,
    % escapeinside=\#\#,
    % xleftmargin=\parindent
}

% \let\origthelstnumber\thelstnumber
% \makeatletter
% \newcommand*\Suppressnumber{%
%   \lst@AddToHook{OnNewLine}{%
%     \let\thelstnumber\relax%
%      \advance\c@lstnumber-\@ne\relax%
%     }%
% }

% \newcommand*\Reactivatenumber[1]{%
%   \lst@AddToHook{OnNewLine}{%
%    \let\thelstnumber\origthelstnumber%
%    \setcounter{lstnumber}{\numexpr#1-1\relax}%
%    %\advance\c@lstnumber\@ne\relax%
%   }%
% }

% \makeatother

\newcommand{\blockfont}{\footnotesize}

\renewcommand{\chapterautorefname}{Chapter}
\renewcommand{\sectionautorefname}{Section}
\renewcommand{\subsectionautorefname}{Subsection}
\renewcommand{\subsubsectionautorefname}{Subsection}
\newcommand{\listingautorefname}{Listing}
% \newcommand{\algorithmautorefname}{Algorithm}

% Title
% \pretitle{
% 	\begin{center}
% }
% \posttitle{
%     \end{center}
% }

\newcommand{\titletext}{ABSOL: Specification and Formal Verification of Domain-Specific Languages through Automatic Compiler Generation in Haskell}

\title{\titletext}
\author{Ara Adkins}
\date{Bachelor of Science, Computer Science with Honours\\The University of Bath\\2017}

% Glossaries
\makeglossaries
\newdualentry{dsl}{DSL}{Domain-Specific Language}{
    An abstraction of the problem domain in the form of a `small' programming language that provides appropriate notations and expressive power focused on a particular problem domain. 
}

\newglossaryentry{semantics}{
    name=Semantics,
    text=semantics,
    description={
        The definition of meaning of syntactically valid (see \gls{syntax}) strings in a language.
        This meaning defines how these programs \textit{behave} when executed, or as the \textit{effect} of executing these programs.
    }
}

\newglossaryentry{syntax}{
    name=Syntax,
    text=syntax,
    description={
        The ways in which the terminal symbols of a programming language may be combined to create programs that are well-formed in the language.
    }
}

\newdualentry{absol}{ABSOL}{Automatic Builder for Semantically Oriented Languages}{
    The \gls{metacompiler} program and toolchain that performs language analysis and verification.
}

\newglossaryentry{metacompiler}{
    name=Metacompiler,
    text=metacompiler,
    description={
        A metacompiler is a program that takes in a specification and generates code for a language compiler as its output. 
        In the case of this project, the metacompiler is responsible for the verification of the input specification, rather than code generation from it. 
    }
}

\newglossaryentry{halting_problem}{
    name=Halting Problem,
    description={
        The Halting Problem describes the problem of determining, from a description of an arbitrary computer program ($\delta$), for a Turing-Machine or equivalent, and an arbitrary input $x$ whether $\delta$ will complete execution or continue to run forever (\gls{diverge}) \citep{boyer1984mechanical}. 
    }
}

\newglossaryentry{declarative}{
    name=Declarative,
    text=declarative,
    description={
        A style of programming language that allows the user to express the logic of a computation without expressing the exact set of steps to be executed. 
        Examples include some portions of Haskell, and Prolog. 
    }
}

\newacronym{avopt}{AVOPT}{Analysis, Verification, Optimisation, Parallelisation and Transformation}

\newglossaryentry{transpilation}{
    name=Transpilation,
    text=transpilation,
    description={
        The transformation of one high level language $L_1$ into another $L_2$ via the \gls{ast} created from parsing $L_1$. 
        The \gls{ast} is used to generate code in the \textit{target} language ($L_2$), aiming to preserve the semantics of $L_1$ as accurately as possible. 
    }
}

\newdualentry{ast}{AST}{Abstract Syntax Tree}{
    An Abstract Syntax Tree (AST) is a tree-based representation of the syntactic structure of the source code of a program or other unit in some language, without any of the extraneous information required for parsing.
    It is the ideal form for programmatic manipulation of the language unit.
}

\newglossaryentry{source_map}{
    name=Source Map,
    text=source map,
    description={
        A predefined mapping of debug symbols in the target language back to the debug symbols in the source language \citep{kulkarnitranspiler}.
    }
}

\newacronym{bnf}{BNF}{Backus-Naur Form}

\newacronym{ebnf}{EBNF}{Extended Backus-Naur Form}

\newacronym{ffi}{FFI}{Foreign-Function Interface}

\newglossaryentry{metaspec}{
    name=Metaspec,
    description={
        The \gls{metalanguage} designed as part of the \acrshort{absol} project. 
        It is capable of describing both \gls{syntax} and \gls{semantics} of the target DSL with one unified syntax. 
    }
}

\newglossaryentry{metalanguage}{
    name=Metalanguage,
    text=metalanguage,
    description={
        A metalanguage is a language that is, itself, used to describe aspects of another language \citep{jakobson1980metalanguage}. 
    }
}

\newacronym{vcs}{VCS}{Version Control System}

\newglossaryentry{defblock}{
    name=Defblock,
    description={
        A defblock is a top-level definition block in \gls{metaspec}. 
    }
}

\newacronym{ssr}{SSR}{Special-Syntax Rule}

\newacronym{cli}{CLI}{Command-Line Interface}

\newglossaryentry{lexer}{
    name=Lexer,
    text=lexer,
    description={
        A lexer is a program that takes a string of characters and performs a process known as \textit{lexical analysis} to generate a stream of tokens (each of which may consist of one or more characters).
        Each of these tokens has semantic meaning in a given language.
    }
}

\newglossaryentry{parser}{
    name=Parser,
    text=parser,
    description={
        A parser is a program that takes a string of tokens (usually from a \gls{lexer}) and builds an \acrshort{ast} for out of these tokens based on the language grammar. 
    }
}

\newglossaryentry{diverge}{
    name=Diverge,
    text=diverge,
    description={
        A computer program is said to \textit{diverge} if it never terminates or terminates in some unintended or exceptional state.
    }
}

\newglossaryentry{group}{
    name=Group,
    text=group,
    description={
        A group is a structure $(G, \cdot)$ where $G$ is a set and $\cdot$ is an associative binary operation on elements of $G$.
        A group satisfies the following axioms:
        \begin{itemize}
            \item \textbf{Closure:} $\forall a, b \in G, a \cdot b \in G$
            \item \textbf{Associativity:} $\forall a, b, c\in G, (a \cdot b) \cdot c = a \cdot (b \cdot c)$.
            \item \textbf{Identity:} $\exists! e \in G. \forall a \in G, e \cdot a = a \cdot e = a$
            \item \textbf{Inverse:} $\forall a \in G. \exists b \in G, a \cdot b = b \cdot a = e$, where $e$ is the identity as above.
        \end{itemize}
    }
}

\newglossaryentry{semigroup}{
    name=Semigroup,
    text=semigroup,
    description={
        A semigroup is a structure $(G, \cdot)$ where $G$ is a set and $\cdot$ is an associative binary operation on elements of $G$ such that the following axioms hold:
        \begin{itemize}
            \item \textbf{Closure:} $\forall a, b \in G, a \cdot b \in G$
            \item \textbf{Associativity:} $\forall a, b, c\in G, (a \cdot b) \cdot c = a \cdot (b \cdot c)$.
        \end{itemize}
    }
}

\newglossaryentry{monoid}{
    name=Monoid,
    text=monoid,
    description={
        A monoid is a structure $(G,\cdot)$ where $G$ is a set and $\cdot$ is an associative binary operation on elements of $G$ such that the following axioms hold:
        \begin{itemize}
            \item \textbf{Closure:} $\forall a, b \in G, a \cdot b \in G$
            \item \textbf{Associativity:} $\forall a, b, c\in G, (a \cdot b) \cdot c = a \cdot (b \cdot c)$.
            \item \textbf{Identity:} $\exists! e \in G. \forall a \in G, e \cdot a = a \cdot e = a$
        \end{itemize}
    }
}

\newglossaryentry{noop}{
    name=Noop,
    text=noop,
    description={
        A noop or no-operation is a computer instruction that does nothing, but advances the program counter.
    }
}

\newacronym{ide}{IDE}{Integrated Development Environment}

\newacronym{ghci}{GHCi}{GHC Interpreter}

\newacronym{api}{API}{Application Programming Interface}

\newglossaryentry{dependent_typing}{
    name=Dependent Typing,
    description={
        The ability to encode expressions in the type system that are predicated on values.
        This allows the type system to encode pre- and post-conditions for functions in a dependently-typed language. 
    }
}

\newglossaryentry{dependently_typed}{
    name=Dependently Typed,
    text=dependently typed,
    description={
        See \gls{dependent_typing}.
    }
}

\newacronym{gpl}{GPL}{General-Purpose Programming Language}

\newglossaryentry{domain_specific_logic}{
    name=Domain-Specific Logic,
    text=domain-specific logic,
    description={
        Logic pertaining to how a system operates within its environment. 
    }
}

\newglossaryentry{coupling}{
    name=Coupling,
    text=coupling,
    description={
        The degree of interdependence between components of a software system. 
        High coupling is undesirable as it increases the difficulty of making changes to the system. 
    }
}

\newglossaryentry{undecidable}{
    name=Undecidable,
    text=undecidable,
    description={
        An undecidable computer problem is one for which a yes/no answer is required, but for which there is no possible program that will always compute the correct answer. 
    }
}

\glsaddall % add all items to the glossaries

%Document
\begin{document}

% Title
\maketitle
\newpage

% Dissertation Consultation Prohibition
\consultation{3}
\newpage

% Declaration
\declaration{\titletext}{Ara Adkins}

% Abtract Text
\begin{addmargin}[1em]{2em}
\begin{abstract}
    With modern software seeing significant increases in its complexity, domain-specific logic is becoming more and more integrated throughout these systems. 
    As a counterpoint to this integration is the re-emergence of Domain-Specific Languages (DSLs): application-specific languages representing domain logic using domain concepts and language.
    The increasing prevalence of these languages creates a problem however --- what happens if they go wrong?
    This project documents the design and development of a toolchain for the creation of provably correct DSLs, with the hope that by proving the language correct the scope for bugs in these critical pieces of infrastructure can be reduced. \\

    This document details a state-of-the-art system consisting of a metalanguage for the specification of the DSL, and a metacompiler toolchain capable of verifying that specification.
    By limiting the types of programs that it can represent, it can be shown that this system allows provable correctness --- here meaning that the language is always guaranteed to terminate --- and provides a novel approach to the development of provably correct languages.
\end{abstract}
\end{addmargin}


% Contents
\tableofcontents

\listoffigures

\listofalgorithms
\addcontentsline{toc}{chapter}{List of Algorithms}

\lstlistoflistings
\addcontentsline{toc}{chapter}{Listings}

\printglossaries

% Acknowledgements
\chapter*{Acknowledgements} % (fold)
\label{cha:acknowledgements}
\addcontentsline{toc}{chapter}{Acknowledgements}
Many thanks to the following people without whom this project would not have been completed:
\begin{itemize}
    \item \textbf{Guy McCusker:} Without whose insight, guidance, support and enthusiasm, the project would not have been possible or anywhere near as enjoyable.
    \item \textbf{Wyeth Wolnick:} Whose interest, questions and loving support has led to the solutions to many challenges.
    \item \textbf{Willem Heijltjes:} Whose kind willingness to discuss abstract Haskell topics has been crucial.
    \item \textbf{Damask Talary-Brown:} Whose faith and confidence in the project's success has been instrumental. 
    \item \textbf{Hillel Sims (Bloomberg):} Whose team's product's runtime configuration using OCaml inspired the entire project.
    \item \textbf{My Friends:} Whose support has buoyed the project along through its struggles. 
\end{itemize}

% chapter acknowledgements (end)


% Chapters
% Introduce the problem to be tackled
% Use this as an opportunity to talk about the inspiration of the project
% Contextualise the problem being tackled by the project
% Explain why the problem is worthy of study
% Outline the structure of the document and provide a framework for the reader of the dissertation. 

\chapter{Project Introduction} % (fold)
\label{cha:project_introduction}
This section will:
\begin{itemize}
    \item \textbf{Introduce the Project:} A brief overview of the inspiration for the project, and how that materialised as the project itself.
    \item \textbf{Overview of Initial Concept}: The initial concept for the system.
    \item \textbf{The Final System:} How that initial concept came to materialise as the final system.
    \item \textbf{Document Overview:} An overview of what the dissertation aims to achieve, and it structure. 
\end{itemize}

This project aims to create a framework and toolchain for the formal verification of embedded, executable Domain-Specific Languages.
The hope is to allow the generation of provably correct, tailored programming languages that can encapsulate domain logic inside another program.
These languages will be compiled through the use of a Haskell metacompiler (a compiler that generates a compiler) that takes both syntax and semantics for the Domain-Specific Language, generating executable libraries for embedding the DSL.

\section{An Introduction to Domain-Specific Languages} % (fold)
\label{sec:an_introduction_to_domain_specific_languages}
Modern software systems are increasing in complexity, with everything from flight computers to business software growing \citep{dvorak2009nasa}.
As these systems grow, the domain logic is becoming integrated throughout the systems, leading to increased levels of coupling throughout the code.
This means that there is an increased risk that changes to the software create modes of incorrect operation \citep{khawar2001developing}.\\

In reaction to the dispersion of domain logic throughout modern software systems, Domain-Specific Languages (DSLs) have seen ``a significant uptick in interest'' \citep{fowler2010domain}. 
A DSL is a limited, application-specific language that is integrated with another software system, providing a custom encoding of domain rules and logic that can often go beyond the syntactic constraints of the host language \citep{Mernik:2005:DDL:1118890.1118892}.
In the use of a DSL, software systems are able to centralise their domain logic using a common syntax, making it simple to change rules for the operation of the program.\\

The term `Domain-Specific Language' is a broad one, encompassing a variety of embedded syntaxes for defining domain logic.
Fowler defines `internal' and `external' DSLs.
An internal DSL is written in the host language and exposed via an API, also known as a fluent interface.
\cite{fowler2010domain} defines an external DSL, in contrast, as a language parsed independently of the host language.\\

In addition to their structure, Domain-Specific Languages can also be defined in terms of their `executability' \citep{Mernik:2005:DDL:1118890.1118892}.
Many DSLs are programming languages with `well-defined execution semantics'.
Examples of this are \LaTeX, a macro language for \TeX, and Template Haskell, which allows manipulation of Haskell code itself using Haskell \citep{Sheard:2002:TMH:581690.581691}.
As a counterpoint, some DSLs only define configuration information for a software system, having no executable semantics of their own.

% section an_introduction_to_domain_specific_languages (end)

\section{The Need for Correct DSLs} % (fold)
\label{sec:the_need_for_correct_dsls}
Initial inspiration for this project was engendered by the observation of a financial technology company using the general-purpose programming language OCaml for specifying logic at runtime.
As a Turing-complete programming language, this offered vastly greater expressive power than necessary for the domain in which it was applied.
This meant an increase in the potential for bugs in the domain logic, as suggested by \cite{subramanyam2003empirical}.
If a capable DSL could be designed that could have formally verified properties, then it would be a much better fit to the domain while reducing the capacity for bugs.\\

While the scope of common DSLs is necessarily very broad, encompassing everything from executable program logic to statically defined configuration, this project aims to focus on a small subset of DSLs.
As formal correctness only applies to executable programs, this project will focus on external, executable DSLs. 
While analysis could benefit internal DSLs as well, they are already covered by traditional code review mechanisms, and would be hard to integrate into this project.\\

Even with the ``detailed analysis and structuring of the application domain'' \citep{van2000domain} performed when initially defining the DSL, it can be difficult to verify the correctness of the resulting language. 
While current methods can utilise code-review to attempt to analyse the correctness of the DSL implementation, a lack of understanding of the domain by the application developers may cause logic errors to be missed. 
This is because the domain logic is often embodied by expert knowledge, as found by \cite{studer1998knowledge}. 
It can hence be difficult to verify the language implementation with those domain experts, through code-review, due to their lack of understanding of the implementing language.
As core domain logic, often integral to the program, is encapsulated by the DSL, it is instrumental that it be formally correct.\\

This project aims to provide a solution for the sixth step of DSL design as proposed by \cite{van2000domain}: ``design and implement a compiler that translates DSL programs''. 
While the logic that the DSL aims to encode can be verified with domain experts, it is often difficult to verify the resulting implementation.
As the language semantics are integral to the operation of the DSL, providing the semantic analysis and verification as part of the toolchain will allow the automatic generation of DSLs.
The generation from specification greatly reduces the potential for errors introduced during the implementation.
An automated system offers benefits over manual review, as the domain experts can work with the DSL designers to encode the correct semantics before it becomes an executable program.

% section the_need_for_correct_dsls (end)

\section{Verification of DSL Semantics} % (fold)
\label{sec:verification_of_dsl_semantics}
While it would be desirable to design a system for verification of arbitrary DSL semantics, there are certain classes of problem in semantic verification which are not decidable \citep{abdulla1994undecidable}. \\

This is most elegantly expressed in the duality of data and codata, or recursion and corecursion. 
While recursively defined structures are decomposed from a state into certain `base-cases', providing a well-defined termination property, corecursion aims to start from a `base-case' and build data from there \citep{hinze2010reasoning}.
As corecursion is dual to recursion, codata is the potentially infinite dual to data, which is necessarily finite. \\

If the program semantics are limited to those that can be defined recursively rather than corecursively; that is, via structural induction rather than coinduction, then termination can be proven for such programs \citep{nordstrom1988terminating}.\\

In the limited domain of recursively defined program semantics, this project aims to build on the work of \cite{doh2001composing}, drawing on the \textit{PlanCompS} project for specifying modular programming language semantics.
It will also draw upon the work of Godel's `System-T', a theory of arithmetic for finite types that allows reasoning about arbitrarily-sized, finite data structures via structural induction \cite{girard1989proofs,alves2010godel}.\\

The aim is to define a set of what Mosses terms `funcons' (\textit{fundamental constructs}); a set of semantic building blocks that can all be proven to terminate \cite{Churchill:2014:RCS:2577080.2577099}.
Alongside a metalanguage that is able to specify both syntax and semantics for the DSL, the project aims to create a toolchain for verifying the DSL semantics and producing an executable library for the DSL.\\

Using the metalanguage definition of the syntax and semantics for the DSL, the project toolchain will act as a metacompiler \cite{Mandell:1966:MDA:800267.810785}.
This means it will produce an executable for the DSL via metacompilation and compilation steps.
The provided semantics will allow decomposition, through structural induction, into the funcons, allowing the termination properties to be proved.
The funcons and associated program code can then be composed to generate the program that matches the semantics defined for the DSL.\\

% section verification_of_dsl_semantics (end)

\section{The Novel Contribution of this Project} % (fold)
\label{sec:the_novel_contribution_of_this_project}
This project aims to make a novel research contribution through the creation of two main components:
\begin{itemize}
    \item \textbf{A DSL Metalanguage:} A metalanguage for the specification of the syntax of a DSL, along with its associated semantics. 
    The syntactic notation will likely be based upon the Extended Backus-Naur Form, a metasyntax notation for expressing languages whose syntax matches a context-free grammar. 
    The corresponding semantic notations will likely be based on the conventional big-step operational semantics \citep{Schmidt:2003:PLS:1074100.1074733} devised for automatic compiler generation \citep{diehl1996semantics}.
    \item \textbf{A Metacompiler:} This metacompiler program will take a DSL description in the above metalanguage, and a program in the described DSL, and produce a working executable from this. 
    This may operate in a single-step or multipass process, depending on what makes for the most simple metacompiler architecture.
\end{itemize}

Through limiting scope to just recursively defined semantic constructs, and omitting consideration of corecursive constructs, this project will be able to produce a useful metacompiler for a set of language semantics within which arbitrary DSLs can be constructed. 

% section the_novel_contribution_of_this_project (end)

\section{Project Objectives} % (fold)
\label{sec:project_objectives}
The main objectives of this project are as follows:
\begin{itemize}
    \item Conduct research on methods for the syntactic description of fundamental program syntax and semantics.
    \item Use the knowledge gained from this research to develop the \textit{Absol}\footnote{Automatic Builder for Semantically Organised Languages: The name of the metacompiler toolchain} metalanguage to describe the syntax and semantics of an arbitrary domain-specific language.
    \item Define a set of basic \textit{funcons} as fundamental building blocks for programs with purely recursively defined semantics. 
    These building blocks will have trivially true termination properties.
    \item Describe methods for semantic decomposition, allowing the program semantics specified using \textit{Absol} to be decomposed to the \textit{funcons}, allowing termination proof for the semantics via structural induction (they are all total functions, and compositions of total functions are also total).
    Structurally recursive semantic definitions allow us to decompose the semantics to the \textit{funcons}, as long as we can express all DSL semantics in terms of these building blocks. 
    \item Development of a basic DSL using the metalanguage.
    \item Development of a minimum working example for the metacompiler using hand-implemented semantics based on the \textit{Absol} metalanguage.
    \item Development of a working metacompiler that can ingest a description of the DSL's syntax and semantics using \textit{Absol}, as well as a program in the DSL itself, and produce a working executable program.
    \item Creation of a more-capable and useful DSL with additional capabilities.
    \item Evaluate the metacompiler in terms of real-world utility and theoretical contribution based on the requirements and the theoretical and practical impact in relation to the literature surveyed.
\end{itemize}


% section project_objectives (end)

% chapter project_introduction (end)

\chapter{Literature and Technology Survey} % (fold)
\label{cha:literature_and_technology_survey}
As the project has proposed the investigation and development of a toolchain for the creation of provably-correct \glspl{dsl}.
Even with the scope constrained to the creation of \glspl{dsl} alone, such a project draws on a significant breadth of disciplines within Computer Science, necessitating a broad knowledge-base. \\

This project exists at the intersection of the study of \glspl{dsl}, the development of programming languages, and formalisations of program semantics. 
As such, it is important to understand the relevant work in these domains. 
While each of these fields is large in its own right, this review aims to distil the relevant bodies of knowledge. \\

This document provides a broad understanding of domain-specific languages, including their types, uses and limitations.
Additionally, it explores the state-of-the-art methods for specifying language syntax and semantics, with an accompanying critical evaluation of these techniques.\\

The second portion of this review examines methods for formal program verification while identifying the limitations of these techniques as they exist today. 
Finally, it provides an examination of techniques for automated compiler generation, and a discussion of relevant technologies to support such tasks. 

\section{Domain Specific Languages} % (fold)
\label{sec:domain_specific_languages}
\defblock{12cm}{DSL}{``A Domain-Specific Language (DSL) is a programming language or executable specification language that offers, through appropriate notations and abstractions, expressive power focused on ... a particular problem domain'' \citep{van2000domain}}

While a Turing-Complete General-Purpose Programming Language (GPL) is capable of expressing any algorithm that can be executed on a standard computer, \citet{fowler2010domain} finds that it is often the case that the use of a GPL provides the wrong level of abstraction in a problem domain.
When attempting to express domain knowledge and domain rules using a GPL, there is often a mismatch between the GPL and the knowledge to be expressed. 
DSLs allow expression of solutions ``at the level of abstraction of the problem domain'' \citep{van2000domain}.\\

Use of a DSL allows an encoding of ``bits of important logic that [don't] fit well within [GPLs]'', allowing an expression of domain-expert knowledge at a higher level of abstraction \citep{fowler2010domain,van2000domain}. 
This embodiment of domain knowledge was found by \citet{fowler2010domain} to ``enable a much richer communication channel'' between programmers and domain experts, allowing domain-experts to interact with the configuration and behaviour of complex software systems. \\

The term `DSL' encompasses a family of languages, as discussed in Section~\ref{sub:types_of_dsls}, so the definition provided by \citet{van2000domain} at the start of this section is perhaps too restrictive. 
DSLs encompass a wide range of programming styles, from declarative to functional, and a similarly varied set of execution models. 
However, in all cases the ``DSLs trade generality for expressiveness in a given domain'', and so the DSL approach should be chosen to maximise that expressiveness \citep{Mernik:2005:DDL:1118890.1118892}. \\

While DSLs offer significant benefits in their encapsulation of domain knowledge, there is significant challenge in creating a DSL to accurately reflect the domain. 
\citet{fowler2010domain} states that DSLs take ``narrow parts of programming'' and make them ``easier to understand and therefore quicker to write, quicker to modify and less likely to breed bugs''. 
Such a statement, however, is only true if the DSL has been implemented correctly, with ``a solid understanding of the domain'' \citep[pg. 1]{bosch1997domain}.\\

Even with such an understanding of the domain, it is possible to make mistakes in the DSL compiler, known as an \textit{application generator}.
Through automatic generation of such tools, this project aims to support the correct implementation of formally correct DSLs, hence avoiding the potential for mistakes in tooling. 
Such constraints on the DSL make it much more difficult to make damaging mistakes in systems where the DSL is in use. 

% While DSLs have multiple disadvantages (including the cost, limited availability, scoping and balance with the host language), this project deals with the support of correct use of DSLs, and through the generation of embeddable DSLs will help to constrain DSL scope. 

\subsection{Types of DSLs} % (fold)
\label{sub:types_of_dsls}
Knowing what a DSL \textit{is} explains little about what forms they might take. 
In reality, the term DSL can be used to describe any limited (non-GPL) programming language, and so one might expect a huge variety in the kinds of DSLs seen in use. \\

In order to better understand the broad variety of DSLs, \citet{van2000domain} proposes a taxonomic classification system, analysing domain-specific languages on five different axes:
\begin{itemize}
    \item \textbf{Execution Strategy:} Domain-Specific Lanaguages can be designed to be \textit{interpreted} (translated from program statements to executable code as they are run), or \textit{compiled} (the same translation performed ahead of time, providing additional opportunities for optimisation and domain-logic checking).
    \item \textbf{Design Strategies:} \citet{van2000domain} claims that a DSL can emerge from a \textit{restricted subset of a GPL}. 
    While this is evidenced by DSLs such as Promela++, a DSL for the construction and validation of protocols, this is limited in its ability to encapsulate domain knowledge \citep{basu1997language}.
    A DSL \textit{designed from scratch} is able to match both syntax and semantics to the domain without restriction. 
    \item \textbf{Implementation Strategy:} \citet{van2000domain} proposes a set of implementation strategies for a DSL:
    \begin{itemize}
        \item \textit{Embedded DSLs:} Such languages use mechanisms that exist in a GPL to define the DSL. 
        The literature surrounding embedded DSLs recognises the limitations imposed by the syntactic and semantic structure of the host language, risking compromises to ``the optimal domain-specific notation'' to work in the host language \citep[pg. 3]{van2000domain}.
        \item \textit{Preprocessed DSLs:} DSLs of this kind translate statements in the syntax of the DSL into statements in a GPL. 
        This, however, also suffers from semantic and syntactic constraints imposed by the macro language or preprocessor. 
        Furthermore, there is no understanding here of the DSL domain, preventing the incorporation of domain-level semantics checking \citep{van2000domain}.
        \item \textit{Compiler Extension:} The preprocessing phase is integrated into a compiler or interpreter, allowing better checking of syntax and types.
        This still provides no ability for checking program semantics at the domain level. 
        \item \textit{Compile from Scratch:} Compilation of DSL program code into executables that can be used from within a GPL. 
        This provides the potential for full domain-level semantic checking, as well as static type-checking and optimisation \citep{van2000domain}.
    \end{itemize}
\end{itemize}

While \citet{van2000domain} provides a useful classification system for Domain-Specific Languages, the work of \citet{Mernik:2005:DDL:1118890.1118892} proposes a connected taxonomy that expands upon the work of \citet{van2000domain}, while proposing additional axes for classification.
The taxonomy of \citet{Mernik:2005:DDL:1118890.1118892} conflates the execution and implementation strategy axes proposed by \citet{van2000domain}:
\begin{itemize}
    \item \textbf{Interpreter:} Recognition, decoding and execution of DSL constructs using a standard interpreter paradigm.
    \citet{Mernik:2005:DDL:1118890.1118892} suggests that this allows high levels of control over the DSL execution environment, and provides for easier language extension.
    \item \textbf{Application Generators:} The translation of DSL constructs to base language constructs and library calls (where the base language may be assembly or a GPL). 
    It is claimed that this allows full static analysis of the DSL program or specification, as the application generator can operate at the semantic level of the DSL.
    \item \textbf{Preprocessing:} Translation of DSL constructs to existing languages, with static analysis limited to that performed by the target language processor.
    \item \textbf{Embedding:} The creation of constructs (new abstract data-types and operators) in a GPL to model a given domain.
    This is most commonly found in the form of an application library. 
    \item \textbf{Compiler Extension:} Extension of an existing compiler or interpreter with domain-specific optimisation rules (e.g. Template Haskell Optimisation Rules, as discussed in Section~\ref{sec:technological_support}).
    \item \textbf{COTS:} Application of existing tools and notations to a new problem domain. 
    \item \textbf{Hybrid:} Any combination of the above approaches. 
\end{itemize}

While the classification system by \citet{Mernik:2005:DDL:1118890.1118892} agrees with both \citet{van2000domain} and \citet{fowler2010domain}, it is much broader. 
It proposes further axes for classification, the most important of which are highlighted below:
\begin{itemize}
    \item \textbf{Execution Style:} \citet{Mernik:2005:DDL:1118890.1118892} recognises that Domain-Specific Languages fall onto a spectrum of `executability', as it is termed. 
    This refers to the nature of the execution which the DSL undergoes in use, and has modes as follows:
    \begin{itemize}
        \item Well-defined execution semantics (e.g. Excel Macro Language).
        \begin{lstlisting}
=SUM(COUNTIF(A3:A24, 0))
        \end{lstlisting}
        \item Inputs to applications with a declarative character and less well-defined execution semantics (e.g. ATMOL, a DSL for the specification of atmospheric models, example from \citet{a2001atmol}).
        \begin{lstlisting}
p :: float(0..107000) dim ``Pa''
    field (x(grid), y(grid), z(grid))
        monotonic k(+) on i=1..n by j=1..m by k=1..l        
        \end{lstlisting}
        \item DSLs for application generation intended as non-executable input (e.g. Extended Backus-Naur Form, the example is for the specification of floating-point numbers in Python 3 \cite{Python3Lexical}).
        \begin{lstlisting}
floatnumber   ::=  pointfloat | exponentfloat
pointfloat    ::=  [intpart] fraction | intpart "."
exponentfloat ::=  (intpart | pointfloat) exponent
intpart       ::=  digit+
fraction      ::=  "." digit+
exponent      ::=  ("e" | "E") ["+" | "-"] digit+
        \end{lstlisting}
        \item Non-executable DSLs, such as those for declaration of static application configuration or definition of data structures (e.g. representation of data structures, such as a satellite's coverage \cite{s2001supporting}).
        \begin{lstlisting}
[(sat, gs) | 
    gs <- groundstations,
    sat <- satellites,
    (coverage sat) (location gs)]
        \end{lstlisting}
    \end{itemize}
    Some of these classifications are imprecise, however, and could benefit from further clarification.
    Nevertheless, the most important observation made on this classification axis is that not all DSLs must be executable.
    Domain-specific rules can be encoded as static configuration as well as executable specification, a facet that was left un-addressed by the taxonomy proposed by \citet{van2000domain}.
    \item \textbf{Resuability:} The nature of a DSL is to encapsulate domain-specific logic, configuration and behaviour. 
    \begin{itemize}
        \item \citet{Mernik:2005:DDL:1118890.1118892} recognises that a compiled DSL can act as a portable store for this information that can be reused across multiple systems. 
        \item In contrast, an embedded DSL is non-portable as its implementation is wedded to the language in which it is embedded. 
        This means that similar DSL concepts would have to be implemented again in other languages, increasing the potential for implementation errors. 
    \end{itemize}
\end{itemize}

The much broader classification scheme proposed by \citet{Mernik:2005:DDL:1118890.1118892} demonstrates the limited scope of the analysis provided by \citet{van2000domain}. \\

The literature also notes an important point about compiled DSLs: while embedded DSLs are Turing-Complete, compiled DSLs offer ``possibilities for ... verification ... that would be much harder or unfeasible (sic) if a GPL has been used'', due to their limited scope \citep[pg. 3]{Mernik:2005:DDL:1118890.1118892}.
This is very important to the progress of this project, as it proposes to produce compilers for DSLs in which the programs are provably correct. 

\subsubsection{Implementation Errors in DSLs} % (fold)
\label{ssub:implementation_errors_in_dsls}
With many DSLs acting as portable stores of domain-specific logic and configuration, the impact of errors in DSL implementations is potentially exacerbated.\\

In situations of reuse, an incorrectly implemented DSL will carry the same implementation flaws to everywhere that it is used.
Resultant from this, \citet{Mernik:2005:DDL:1118890.1118892} stresses the importance of performing the domain analysis phase correctly, as the results of such an analysis may persist for a long time. \\

If a DSL is going to persist for some time, it is all the more important that the domain analysis be correctly translated into the DSL implementation. 

% subsubsection implementation_errors_in_dsls (end)

\subsubsection{Detailed Implementation Strategies} % (fold)
\label{ssub:detailed_implementation_strategies}
\citet{Mernik:2005:DDL:1118890.1118892} claims that the application generator or compiled DSL approach and the embedded approach are the two most common DSL implementation strategies. 
As a result, the paper provides a detailed analysis of these strategies. \\

Compiled DSLs, in a general sense, is a category that encompasses any DSL that is translated directly to executable code, whether by an interpreter or compiler \citep{Mernik:2005:DDL:1118890.1118892}. 
Building such tools from scratch allows these DSLs to match the notation of domain experts as closely as possible, significantly reducing the cognitive load required to translate domain rules into program code \citep{fowler2010domain}.
The compiled approach allows opportunities for domain-semantics-related error reporting, rather than reporting based on the semantics of a host language.
Furthermore, it provides opportunities for domain-specific Analysis, Verification, Optimisation, Parallelisation and Transformation (AVOPT) that are unparalleled by other approaches \citep{Mernik:2005:DDL:1118890.1118892}.\\

Such DSLs are not without their downsides, however, as they require significant development effort due to the requirement for a complex language processor. 
Such processors are rarely designed with extension in mind, resulting in alterations to the DSL being difficult to achieve. 
However, in recognising these deficiencies, \citet{Mernik:2005:DDL:1118890.1118892} fails to note the existence of tools such as compiler generators that significantly decrease this implementation effort \citep{Mandell:1966:MDA:800267.810785}. \\

\citet{Mernik:2005:DDL:1118890.1118892} contrasts the compiled approach with the embedded approach, which refers to any DSL that uses extension mechanisms provided by the host language. 
Such DSLs are an almost perfect counterpoint to compiled DSLs, requiring a far smaller implementation effort due to the ability to reuse existing language features (and produce a more powerful DSL without additional effort). 
Furthermore, such DSLs often benefit from the tooling support around the host language, which has the potential to increase the DSL's ease of use. \\

Embedded DSLs, however, suffer significantly from suboptimal syntax and semantics due to host-language restrictions, with domain-specific constructs and abstractions unable to map to a GPL or GPL-based library \citep{Mernik:2005:DDL:1118890.1118892,van2000domain}. 
While some languages such as Lisp/Scheme allow arbitrary syntax extensions, in most cases this can produce a confusing mismatch between the domain knowledge and its representation \citep{jennings1999verischemelog}. 
These host language restrictions also manifest in the form of error reporting. 
This will take place in terms of the host language constructs, causing a conceptual mismatch between the domain and the language. 
Furthermore, \citet{Mernik:2005:DDL:1118890.1118892} suggests that the embedded approach restricts AVOPT, though there are some languages that can assist with this \citep{seefried2004optimising}.\\

While these categories were established by \citet{van2000domain}, \citet{Mernik:2005:DDL:1118890.1118892} recognises that these approaches are less restricted than they were initially credited as being. 
These two approaches are again corroborated by \citet{fowler2010domain}, who proposes the concepts of Internal DSLs (written in the host language and exposed via an API) and External DSLs (languages with their own semantics parsed independently of the host language). \\

It seems possible to utilise a hybrid approach that combines the syntactic and semantic freedom of compilation with the language power of embedding. 
Such an approach is discussed in \nameref{ssub:transpilation} below. 

% subsubsection detailed_implementation_strategies (end)

\subsubsection{Transpilation} % (fold)
\label{ssub:transpilation}
All of these classification systems, while useful, fail to explicitly recognise an additional implementation strategy: transpilation. \\

While transpiling a language can technically be classified under `hybrid' the transpilation process, as defined by \citet{Mernik:2005:DDL:1118890.1118892}, parses the source language into its Abstract Syntax Tree (AST).
The semantics associated with the nodes in the AST are then used to generate code in the target language \citep[pg. 4]{Bouraqadi:2016:MPT:2991041.2991051}.
This aims to totally preserve the semantics of the source language, while encapsulated in the different target language. \\

The transpilation process combines some benefits of both the compiled and embedded DSL approaches \citep{kulkarnitranspiler}.
It allows:
\begin{itemize}
    \item \textbf{Flexible Syntax and Semantics:} The DSL's syntax and semantics can be defined independently of the eventual target language. 
    This means that the DSL can match the domain as closely as possible. 
    \item \textbf{Domain-Specific AVOPT:} The source language's parser and compiler stages operate directly on the DSL and can perform analysis and verification based on the Domain-Level concepts. 
    \item \textbf{Domain-Level Error Reporting:} As errors can be detected and reported by the source-language semantic analysis, these can be reported in terms of domain-level concepts. 
    \item \textbf{Powerful Language Features:} As the actual semantic functionality of the DSL is being provided via the target language, complex language features are already available as a translation target.
    This removes the need for them to be recreated as for a normal compiler. 
\end{itemize}

While a transpilation approach brings significant benefits, it does not address issues around tooling for the DSL, nor does it address the issues with lack of extensibility in source language parsers, which would require implementation by hand. \\

Furthermore, while the transpiler is able to catch static errors (such as syntax errors, domain-logic errors and other errors expressed in program syntax), any runtime errors will still be expressed in terms of the target language's syntax and semantics. 
This issue can be addressed by a source-mapping technique, in which the debug symbols for the target language are mapped onto the debug symbols for the source language, but this requires additional work in the transpiler to maintain this information \citep{kulkarnitranspiler}.

% subsubsection transpilation (end)

% subsection types_of_dsls (end)

\subsection{Uses of DSLs} % (fold)
\label{sub:uses_of_dsls}
DSLs provide value in two main ways: ``improving productivity for developers and improving communication with domain experts'' \citep{fowler2010domain}. 
While DSLs are generally \textit{small} languages, as discussed by \citet{van2000domain}, they take a variety of forms.
\citet{fowler2010domain} shows that these forms range from small, declarative programming languages, to statically defined specifications and even include traditional functional programming-based rulesets for domain functionality.\\

Due to the wide variety of forms that DSLs can take, they are found in a comparatively wide range of uses.
Some of the key uses are listed below:
\begin{itemize}
    \item A ``common text that acts both as executable software and a description that domain experts can read to understand how their ideas are represented in a system'' \citep{fowler2010domain}.
    Much of the literature focuses on DSLs as enablers of communication through defining important program concepts in terms of the domain at hand.
    \item A form of ``end-user programming'', where domain experts are able to use programming to perform tasks or otherwise configure their tools \citep{van2000domain}.
    \item A portable form of domain-specific rules, configuration and behaviour. 
    Such ``executable specifications'', as they are termed in \citet{fabry2015taxonomy} provide a centralised repository of domain knowledge in a form that is easy to understand and reason about. 
    \item Improving testability and allowing validation of domain logic through expression of the logic in domain concepts \citep{van2000domain}.
    The opportunities for AVOPT are extended significantly in compiled (external) DSLs \citep{Mernik:2005:DDL:1118890.1118892}.
    \item Provides ``substantial gains in expressiveness and ease of use'' \citep[pg. 2]{Mernik:2005:DDL:1118890.1118892} over GPLs due to the tailored encoding of domain constructs and semantics. 
\end{itemize}

In all of the above situations it is important that the DSL both represents the correct domain constructs, and that the behaviour of the DSL is as intended by the language designers.
To this end, \citet{van2000domain} suggests a rigorous design methodology for a DSL, performing domain analysis and first implementing it as a library in a host language. \\

This reflects a seemingly predominant view of DSLs as evolutions of object-oriented design frameworks or application libraries designed for modelling a problem domain \citep{van2000domain,Mernik:2005:DDL:1118890.1118892}. 
While this is true, it reflects a limited understanding of the capabilities of DSLs, with small external languages featuring a much greater range of syntactic and semantic expressive power through their ability to encapsulate domain concepts \citep{fowler2010domain}.
Other language interface paradigms, such as use of a Foreign-Function Interface can permit far more flexible integration with a DSL.\\

In all cases, the key role of a DSL is to encapsulate the knowledge and rules embodied by a domain in a form that allows the validation of that knowledge. 
To that end, it is important to ensure that the resultant DSL accurately captures the semantics of the domain \citep{fowler2010domain}.

% subsection uses_of_dsls (end)

\subsection{Designing Domain-Specific Languages} % (fold)
\label{sub:designing_domain_specific_languages}
The design of DSLs can be a task that is ``sometimes error-prone and usually time-consuming'' \citep{karsai2014design}. 
This is due to the complex domain-analysis process that has to take place, elucidating the relevant domain constructs and semantics into a form ready for implementation. \\

Despite this difficulty, tool support for the design of DSLs usually encompasses the implementation phase, with little support for the analysis phase. 
Such tools are often encapsulated in a ``language development system'', which is capable of generating tools from a description of the DSL. 
Such tools may include consistency checkers, language interpreters or compilers and even Integrated Development Environments (IDEs) with integrated editors, analysis tools and debuggers \citep[pg. 19-20]{Mernik:2005:DDL:1118890.1118892}. \\

Such systems often take an opinionated stance on the design of the resultant DSL. 
Sprint, for example, assumes an interpreter for the DSL that uses a partial evaluation technique to lower the overhead of running the DSL program \citep{Consel98architecturingsoftware}.
Environments such as ASF+SDF \citep{van2001asf+}, DMS \citep{baxter2004dms} and Stratego \citep{visser2003stratego}, in contrast, allow definition of the DSL in a variety of forms:
\begin{itemize}
    \item \textbf{Interpretive:} A definition that provides the language semantics for direct execution by an interpreter.
    \item \textbf{Translational:} A definition that provides rules to support a source-to-source transpilation of the DSL into another programming language.
    The target language is usually a GPL \citep{Mernik:2005:DDL:1118890.1118892}.
    \item \textbf{Transformational:} A definition style that specifies the language semantics for direct translation to assembly for direct execution or embedding into another language via a Foreign-Function Interface (FFI) \citep{van2001asf+}.
\end{itemize}

Choice of the implementation style is one of the major decisions to be made as part of the design process.
Each implementation style result in a DSL with different capabilities, including how it can be used from other languages and opportunities for domain-level AVOPT. \\

In order to help make such design decisions, multiple papers offer recommendations.
\citet{Mernik:2005:DDL:1118890.1118892}, for example, suggests that embedded DSLs should be the initial strategy, unless AVOPT is required of the DSL.
Such a suggestion, however, is somewhat fallacious given that the paper previously argued for the importance of DSL syntax matching the domain, and the culpability of embedded DSLs for suboptimal syntax, as discussed in Section~\ref{sub:types_of_dsls}. \\

The choice of implementation strategy and design for a DSL should hence be made carefully, as it can have significant impacts on the usability of the final language. 

% subsection designing_domain_specific_languages (end)

% section domain_specific_languages (end)

\section{Metalanguages} % (fold)
\label{sec:metalanguages}
% TODO Can revisions to this section and the next make things more engaging? Or does that risk clarity?
\defblock{10.5cm}{\Glsname{metalanguage}}{\glsdesc*{metalanguage}}

As implementing verifiable DSLs requires formal descriptions of both the syntax and semantics of a DSL, it necessarily involves the use of one or more metalanguages. \\

Various aspects of DSLs are usually ``developed in terms of specialised metalanguages'', and it is often true that these languages are DSLs themselves \citep{Mernik:2005:DDL:1118890.1118892}.
These metalanguages are concerned with the specification of a property or some set of properties of the DSL such as its syntax or semantics. \\

Metalanguages are used to make user-defined abstractions into first-class citizens through accurate description of the abstraction \citep{Siek:2010:GPL:1706356.1706358}.
To this end, the project needs to create a combined notation, in itself a DSL, for the specification of both syntax and semantics of a DSL. \\

With the field having been in development for ``more than 40 years'', there are a wide variety of metalanguages in use today for the specification of language syntax and semantics \citep{Zhang:2004:SSD:981009.981013}. 
This section aims to explore and evaluate these techniques. 

\subsection{Specifying Language Syntax} % (fold)
\label{sub:specifying_language_syntax}
The syntax of a language defines the formal relationships between the language components (known as non-terminal symbols), and thereby provides a structured description of the valid strings in a language. 
Such formal definitions have three main uses in that they name the syntactic elements, define the valid sentences and provide the syntactic structure of sentences in the language \citep{Scowen:1982:SSM:947912.947917,standard1996ebnf}.

\defblock{11.8cm}{Syntax}{
    Programming language syntax refers to the ways in which language symbols may be combined to create well-formed sentences (otherwise known as programs) in a given languages \citep[pg. 1]{slonneger1995formal}.
}

Languages are defined by a grammar $<\Sigma,N,P,S>$, which consists of four parts \citep{slonneger1995formal}:
\begin{enumerate}
    \item A finite set of terminal symbols $\Sigma$ --- the alphabet of the language that is assembled to make sentences in the language.
    \item A finite set $N$ of nonterminal symbols, which represents the subphrases of sentences in the language.
    \item A finite set $P$ of productions that describe the definition of nonterminals in terms of the terminals and nonterminals in the language. 
    \item A special nonterminal $S$, the start symbol, that identifies the principle category being defined.
\end{enumerate}

In practice, programming language syntax is specified through use of a variant of Backus-Naur Form (BNF), a metalanguage \citep[pg. 21]{Mernik:2005:DDL:1118890.1118892}.
Rules in BNF are specified as follows, with nonterminals represented \lstinline{<category-name>} and productions as follows:
\begin{lstlisting}
    <declaration> ::= var <variable-list> : <type> ;
\end{lstlisting}
where:
\begin{itemize}
    \item \lstinline{var}, \lstinline{:} and \lstinline{;} are terminal symbols in the language. 
    \item \lstinline{::=} is a syntactic construct read to mean ``is defined to be'' or ``is composed of''.
\end{itemize}

\subsubsection{The Chomsky Hierarchy} % (fold)
\label{ssub:the_chomsky_hierarchy}
Languages fall into a set of categories defined by the Chomsky hierarchy \citep{slonneger1995formal}:
\begin{itemize}
    \item \textbf{Level 0 Grammars:} These are unrestricted grammars, which consist of all languages that can be recognised by a Turing Machine.
    Such languages are known as \textit{recursively enumerable}, and require that at least one nonterminal occur on the left side of the rule: $\alpha ::= \beta$
    \item \textbf{Level 1 Grammars:} These are Context-Sensitive grammars, and can be recognised by a linear bounded automaton. 
    These have the additional restriction that the right side contains no fewer symbols than the left side: $\alpha<B>\gamma ::= \alpha\beta\gamma$ where $<B>$ is a nonterminal.
    \item \textbf{Level 2 Grammars:} Known as Context-Free grammars, these are grammars that can be recognised by a nondeterministic pushdown automaton.
    These restrict the left side to being a single nonterminal: $<A> :: \alpha$.
    These correspond to the BNF grammar, and play a major role in the definition of programming languages. 
    \item \textbf{Level 3 Grammars:} These are regular grammars and recognise regular languages, and can be recognised by Finite Automata. 
    These are restricted to allowing only a terminal or terminal followed by a non-terminal on the right side: $<A> ::= \alpha$ or $<A> ::= \alpha<A>$.
\end{itemize}

While there are many categories, most Domain-Specific Languages fall into the category of Context-Free Grammars \citep{Siek:2010:GPL:1706356.1706358}.
However, there is additional contextual information that cannot be defined by standard BNF.\\

Examples of such context-sensitive language features are ``declaration of identifiers before use'' and ``well-typedness of expressions'' \citep{mosses1992action}.
Such syntax can either be defined by attribute grammars (which can be specified in EBNF, discussed in Section~\ref{ssub:extended_backus_naur_form}), or as static program semantics (semantics defined by program structure) \citep{mosses1992action}. 
Such features include:
\begin{itemize}
    \item Well-typedness information
    \item Constraints on programs
    \item Grouping of operations and ordering
\end{itemize}

% subsubsection the_chomsky_hierarchy (end)

\subsubsection{Extended Backus-Naur Form} % (fold)
\label{ssub:extended_backus_naur_form}
Due to limitations of the original BNF, such as an inability to express language elements that were similar to BNF syntax, it has been extended to produce many ``slightly different notations'' that are in use today \citep{standard1996ebnf}.\\

The EBNF is standardised in ISO Standard 14977, and includes the most widely adopted extensions to the original BNF \citep{standard1996ebnf}.
More importantly, the standard has been designed such that ``various extensions can be made in a natural way''.
While this was originally intended to support multi-level grammars (and hence allow BNF to support context-free grammars), ``the format between the special sequence characters [is] almost completely arbitrary'' \citep[pg. vii]{standard1996ebnf}.
This means that it could also be used to support semantic specifications about the grammar, or any other extensions required.\\

EBNF, as a result, provides a flexible, and more importantly extensible, mechanism for specifying language syntax.
These extensions can be used to specify meta-rules about the productions of the language, with \citet{slonneger1995formal} stating the three main forms as:
\begin{itemize}
    \item \textbf{Attributed Syntax Rules:} Rules that define the attributes of productions within a grammar, often used for the definition of context-sensitive syntactical structures. 
    Such rules are evaluated as the language is parsed. 
    \item \textbf{Conditional Rewrite Rules:} Syntactic rules that apply based on the evaluation of some condition to provide an alternative form for a given piece of syntax (e.g. an optimisation). 
    \item \textbf{Transition Rules:} Rules defining allowable transitions between states of the parser. 
\end{itemize}

% subsubsection extended_backus_naur_form (end)

\subsubsection{Abstract Syntax} % (fold)
\label{ssub:abstract_syntax}
\defblock{12cm}{\acrshort{ast}}{\glsdesc*{gls-ast}}

While the EBNF definition of a programming language is sometimes referred to as the \textit{concrete syntax} of the language, the language's structure can be expressed in a simplified form as an \textit{\gls{ast}}.
This is produced from a derivation tree (a tree showing the parsed result of a program expression) by removing any information only required by the parser.\\

This simplification enables the abstract syntax to communicate the structure of phrases in terms of their semantics in the programming language, and the rules for producing an AST are similar to the EBNF grammar for the language \citep{slonneger1995formal}.
However, the AST rules factor out any extraneous detail to help define the language semantics. \\

Nodes in an AST denote constructs in the source code of the program, and it is abstract in the sense that the tree will not explicitly represent all elements of the original source (e.g. grouping by parentheses may be implicit). 
This relationship between concrete and abstract syntax is defined by a relation $\textit{unparse} : \textit{AST} \mathgoesto \textit{concrete syntax}$.
This relation is, ideally, a function, and so restrictions are usually placed on the canonical representations for concrete phrases to disambiguate the reversal process. 
Correct application of unparse to the AST is able to demonstrate the correctness of the parsing algorithm \citep[pg. 29]{slonneger1995formal}.

% subsubsection abstract_syntax (end)

% subsection specifying_language_syntax (end)

\subsection{Specifying Language Semantics} % (fold)
\label{sub:specifying_language_semantics}
\defblock{11cm}{Semantics}{
    The semantics of a language defines the meaning of syntactically valid strings in the language.
    This can be viewed as either the \textit{behaviour} followed when executing a program in this language \citep{slonneger1995formal}.
    Alternatively, it is the \textit{effect} of executing these syntactically valid strings \citep{hennessy1990semantics}.
    These definitions differ due to the multitude of semantic options available. 
}

Much of the literature agrees that formal semantic description is key to the design of, and reasoning about, programming languages \citep{Zhang:2004:SSD:981009.981013,mosses1986use,Binsbergen:2016:TSC:2892664.2893464}.
However, most common semantic description methods are inadequate for describing the complex semantics of real-world GPLs, meaning that most language standards use natural-language descriptions of semantics, leaving much scope for ambiguity \citep{mosses1986use}.
This inadequacy is compounded by an almost complete lack of tooling to support the formal semantic definition of programming languages \citep{Binsbergen:2016:TSC:2892664.2893464}.\\

As the parser is a program that maps syntactically valid strings to an AST, program semantics can be modelled by semantic functions: expressions that map the abstract syntax of a program to the semantic entity that represents the program behaviour \citep{mosses1992action}:
\begin{equation}
    f_s(\text{abstract syntax}) \mathgoesto \text{semantics}
\end{equation}

\citet{mosses1992action} defines two semantic functions to be equivalent at a given level of decomposition if two phrases of program syntax are interchangeable without altering the meaning of the program. 
Consider, for example, the following Haskell code:
\begin{lstlisting}[language=haskell]
let x = sort(xs)
let y = heapsort(xs)
-- x == y -> true
\end{lstlisting}
In this case, both program phrases will have the same semantic meaning, as both ingest the list \lstinline[language=haskell]{xs} and return a sorted version of that list.\\

Any compositions of such phrases with the same semantics are called \textit{fully abstract}. 
When the semantics of a phrase composed of sub-phrases depends only on the semantics of its sub-phrases this is known as a \textit{compositional} semantics. 
While such semantics are simpler to reason about, not all program expressions can be represented as compositional semantics \citep{mosses1992action}.\\

Such notions of program semantics take into account only the result of the expression, and do not account for any complexity of the algorithm.
While such concerns are rarely considered as part of the program semantics, they can be useful when considering program synthesis from semantics \citep{kanovich1991efficient}.\\

In practice, there is a wide variety of semantic description frameworks, with the diversity suggested to stem from the fact that program behaviour ``exhibits far greater complexity than program structure'' \citep[pg. 14]{Zhang:2004:SSD:981009.981013}.
The paper suggests that the main semantic description frameworks are:
\begin{itemize}
    \item Operational Semantics
    \item Denotational Semantics
    \item Axiomatic Semantics
    \item Hybrid Semantics
\end{itemize}

These different frameworks are concerned with different elements of program execution. 
Some frameworks concern themselves with only the results of execution, while others are concerned with the details of execution and program continuation structure during statement execution \citep{mosses2001varieties}.

\subsubsection{Operational Semantics} % (fold)
\label{ssub:operational_semantics}
There are multiple kinds of operational semantics, but the two most common semantic frameworks are \textit{Structural Operational Semantics (SOS)} and \textit{Natural Semantics}.\\

Structural operational semantics (also known as small-step operational semantics) specify transition relations (semantic functions) that are characterised by phrase transitions depending only on the transitions of one or more of its sub-phrases \citep{Zhang:2004:SSD:981009.981013}:
\begin{itemize}
    \item Transition relations in this semantic framework are specified as sets of axioms and inference rules.
    \item Each transition modifies the syntax part of the state as a reflection of a portion of the execution of some sub-phrase. 
    \item When execution of a semantic sub-phrase is completed, the sub-phrase is replaced by the resultant value. 
\end{itemize}

Such semantics have seen wide use in program analysis due to their fine-grained nature, but this granularity has the potential for detail overload in the definition of programming languages. 
An example of such semantics can be seen in the following equation:
\begin{equation}
    [\text{asgn}_\text{sos}]: \langle x := \alpha, s \rangle \rightarrow s[x \mapsto \mathbb{A} \llbracket \alpha \rrbracket s]
\end{equation}

This equation defines a rule for assignment, stating that assigning the value of $\alpha$ to the variable $x$ in state $s$ results in a new state $\mathbb{A}\llbracket \alpha \rrbracket s$ where the value of $x$ is $\alpha$.\\

Natural semantics (also known as big-step operational semantics) exists as an alternative, hiding more execution details than SOS:
\begin{itemize}
    \item This semantics is concerned with the description of how the overall computed result of an expression was obtained, rather than the description of the individual steps.
    \item Evaluations in Natural semantics are also specified as sets of axioms and accompanying inference rules. 
\end{itemize}

While natural semantics has seen ``extensive'' use in the definition of programming languages, it is not suitable for describing concurrent or interleaved execution due to the lack of intermediate states \citep{Zhang:2004:SSD:981009.981013}.
An example of such semantics can be seen in the following equation:
\begin{equation}
    [\text{if}^\text{ tt}_\text{ ns}]: 
    \frac
    {\langle S_1, s \rangle \rightarrow s'}
    {\langle \text{if } b \text{ then } S_1 \text{ else } S_2, s \rangle \rightarrow s'} 
    \text{ if } \mathbb{B}\llbracket b \rrbracket s = {tt}
\end{equation}

This semantic description defines the rule for an `if' statement, evaluating statement $S_1$ if the value of $b \in s$ evaluates to $tt$, else evaluating $S2$.\\ 

While the above semantic frameworks have poor modularity, further operational semantics exist, such as the Modular Operational Semantics proposed by \cite{mosses2004modular}.
This is a variant on SOS that restricts states to syntax and computed values.
It incorporates all auxiliary entities (e.g. memory stores, loads and environments/continuations --- data structures that model the computational process at a given point in the program's execution) as labels on transitions.
The hope is that this brings additional modularity to the semantics to allow for the semantic phrases to be reused when embedded in more complex programming languages, but this has seen little uptake in practice.

% subsubsection operational_semantics (end)

\subsubsection{Denotational Semantics} % (fold)
\label{ssub:denotational_semantics}
Denotational semantics model the behaviour of program phrases through use of \textit{denotations} --- mathematical objects that are usually continuous functions \citep{Zhang:2004:SSD:981009.981013,mosses2001varieties}.
\begin{itemize}
    \item These functions reflect the contribution of a given program phrase to the overall program behaviour, focusing on the result of a given computation rather than how the result was obtained. 
    \item Program semantics can be defined through the composition of these denotational functions.
    \item They also provide the ability to denote sequencing via the use of continuation passing as arguments to the semantic functions. 
\end{itemize}

While such semantics provide a high-level overview of the computational semantics, they may not provide enough computational detail for use in a metacompiler system.
An example of denotational semantics can be seen below \citep{Zhang:2004:SSD:981009.981013}.
\begin{equation}
    S_{ds} \llbracket x := a \rrbracket = \lambda s.s [x \mapsto \mathbb{A}\llbracket a \rrbracket s]
\end{equation}

This semantic definition also defines a rule for assigning the value $\alpha$ to the variable $x$ in the state $s$. \\

Denotational and Operational semantics can be written to be fully equivalent; such semantics are known as \textit{fully abstract} \citep{mosses2001varieties}.

% subsubsection denotational_semantics (end)

\subsubsection{Axiomatic Semantics} % (fold)
\label{ssub:axiomatic_semantics}
Axiomatic Semantics describe the properties of a program as a set of constraints on behaviour, with programs modelled as transformations of these constraints \citep{Zhang:2004:SSD:981009.981013}.
These constraints are known as \textit{assertions}, and were initially developed to formalise the verification of abstract algorithms.
\begin{itemize}
    \item Axiomatic semantics are not particularly well suited to producing descriptions of programming languages due to their highly abstract nature.
    \item They are, however, suitable for proving properties about the programs that they describe, as the assertions act as rules to constrain program behaviour \citep{mosses2001varieties}.
\end{itemize}

An example of axiomatic semantics can be seen below \citep{Zhang:2004:SSD:981009.981013}.
\begin{equation}
    [if_{as}]: 
    \frac
    {\{t \lor P\} S_1 \{Q\}, \{\lnot t \lor P\} S_2 \{Q\}}
    {\{P\} \text{ if } b \text{ then } S_1 \text{ else } S_2 \{Q\}}
    \text{ where }
    t = \mathbb{B}\llbracket b \rrbracket 
\end{equation}

Much akin to the rule shown for the natural semantics, this axiomatic rule defines the properties of an `if' expression. 
It states that if the expression $t$, which has the value of evaluating the boolean expression $b$ is true, then expression $S_1$ is evaluated producing a state $Q$, otherwise the expression $S_2$ is evaluated, also producing a new state $Q$. 

% subsubsection axiomatic_semantics (end)

\subsubsection{Hybrid Semantics} % (fold)
\label{ssub:hybrid_semantics}
Hybrid Semantics is a term used to refer to any combination or augmentation of the previously listed frameworks to produce a new system for semantic description \citep{Zhang:2004:SSD:981009.981013}.\\

There have been multiple hybrid frameworks, but one of the most studied is the Action Semantics framework designed by \citet{mosses1992action}, and used in multiple projects such as those by \citet{brown1992actress} and \citet{diehl1996semantics}.
Action semantics recognised the inability of the traditional semantic description methods to scale to more complex programming languages such as the GPLs used in day-to-day systems programming.
\begin{itemize}
    \item It improves the modularity of denotational semantics by treating denotations as actions to be defined using a fixed \textit{action notation} consisting of primitives and combinators.
    \item It provides direct support for the specification of control flow, data flow, scoping and concurrent communication \citep{mosses1992action,doh2001composing}.
    \item As it is based upon operational and denotational semantics, it can be used to verify properties of the programs it is used to specify.
    \item Action semantics was extended to form Modular Action Semantics, which encapsulated the semantic description of each language construct in a separate module to allow for reuse \citep{Zhang:2004:SSD:981009.981013}. 
\end{itemize}

An example of action semantics can be seen below \citep{Zhang:2004:SSD:981009.981013}.
\begin{equation}
    \textbf{execute} \llbracket x := a \rrbracket = (\text{evaluate } a \textbf{ then } \text{store the primitive value in the cell bound } x)
\end{equation}

Action Semantics, however, is limited in that ``not all programming language concepts can be directly represented within action semantics'' \citep{wansbrough1997modular}. 
This restriction is imposed by the set of computational constructs originally envisioned by \citet{mosses1992action}, with the system proving hard to extend.\\

In the search for a more modular system, \citet{wansbrough1997modular} proposed a system called Modular Monadic Action Semantics (MMAS) based upon both Action Semantics and the earlier Modular Monadic Semantics.
This system aimed to combine the benefits of both systems into a single semantics framework.
\begin{itemize}
    \item MMAS provides a truly extensible version of Action Semantics, allowing the representation of additional programming concepts such as first-class continuations that cannot be expressed in Action Semantics \citep{wansbrough1997modular}.
    \item It replaces the original Structural Operation Semantics that underlie Action Semantics, maintaining the readability of Action Semantics alongside the flexibility of the Modular Monadic Semantics, allowing the semantic model to be refined or extended based upon the application to model new forms of computation.
\end{itemize}

% subsubsection hybrid_semantics (end)

\subsubsection{Modularity of Language Semantics} % (fold)
\label{ssub:modularity_of_language_semantics}
Despite the wide variety of ways in which language semantics can be specified, it is still an open problem to develop a truly flexible semantic framework. 
\cite{Churchill:2014:RCS:2577080.2577099} states that ``various semantic frameworks do not have good modularity'', and this is evidenced by the issues seen when extending existing semantic frameworks \citep{wansbrough1997modular}.\\

Simple semantic specifications such as Operational and Denotational semantics suffer from issues with extensibility and modularity, even though ``various programming constructs are common to many languages'' \citep{Churchill:2014:RCS:2577080.2577099,Zhang:2004:SSD:981009.981013,mosses2001varieties,mosses2004modular}. 
While it is possible to specify a limited semantic framework that encompasses a set of computational actions, it is very possible to create a new kind of computation that does not fit easily into the existing framework \citep{wansbrough1997modular}. \\

While there has been much research into the subject, with proposals such as Modular Monadic Semantics, Modular Action Semantics and even Modular Monadic Action Semantics, none of these frameworks have seen particular use \citep{wansbrough1997modular,mosses1992action,Zhang:2004:SSD:981009.981013,Mosses:2009:CS:1596486.1596489,mosses2001varieties}.
Peter Mosses has written extensively about modular semantic frameworks as embodied by his PLanCompS project, but such semantic models that he terms `Funcons' have seen little uptake in practice \citep{Mosses:2009:CS:1596486.1596489,Churchill:2014:RCS:2577080.2577099,Binsbergen:2016:TSC:2892664.2893464}.

% subsubsection modularity_of_language_semantics (end)

% subsection specifying_language_semantics (end)

% section metalanguages (end)

\section{Formal Program Verification} % (fold)
\label{sec:formal_program_verification}
Having designed the syntax and semantics of a domain-specific language, there still needs to be some mechanism by which the properties of a language can be proved. \\

The notion of program correctness and program verification refers to the ability to prove, via formal methods, that a computer program is ``totally correct''.
A program is called ``totally correct'' when it can be shown to both \textit{terminate} and \textit{perform the operations as defined by its specification} \citep{manna1974axiomatic}.
In general this is an undecidable problem \citep{walther1994proving}.\\

Proving program correctness is a complex task, and in many cases requires an examination of the environment in which the program executes.
With a total understanding of this \textit{environment} and a set of \textit{well-defined program semantics}, it is theoretically possible to apply deductive reasoning to sets of axiomatic program properties to reason about the program's correctness in that environment \citep{Hoare:1969:ABC:363235.363259}.\\

While \citet{Hoare:1969:ABC:363235.363259} and \citet{manna1974axiomatic} propose a generic axiomatic framework for reasoning about computer programs, the most appropriate framework to prove properties using is the well-specified semantics of the programming language in which the program is written.
Under such circumstances, if the language semantics can be proven correct regardless of the execution environment (for a subset of program operations as discussed in Section~\ref{sub:data_and_codata}), it is possible to prove properties of programs in that language.

\subsection{Data and Codata} % (fold)
\label{sub:data_and_codata}
While, in general, it is an undecidable problem to determine if an arbitrary program in a Turing-Complete language is correct, it is possible to restrict the set of allowable operations to a set that can be shown to terminate \citep{walther1994proving}.
Basing the semantics of a DSL upon these allowable operations will, theoretically, allow for a DSL that can be shown to be correct.\\

This restriction of allowable computational operations is based on the duality of \textit{data} and \textit{codata}.\\

Data is captured by the notion of \textit{inductive} data types, whose elements can be constructed in a finite number of steps.
This means that properties of data can be shown by well-founded induction on recursive programs \citep{hinze2010reasoning}.
\begin{itemize}
    \item As a result, it is possible in general to prove that well-founded recursive programs will terminate, and can hence be proven correct.
    \item While it is undecidable to prove termination for general recursion (unlike primitive recursion, for which it is always provable), it is shown by \citet{nordstrom1988terminating} that it is possible to prove termination for well-founded general recursion (recursion which operates on data, not unbounded constructs). 
    This is represented by the following recursion rule, which states that well-founded general recursion can be equated to primitive recursion over the natural numbers as long as $A$ is well-founded by $\prec_A$:
    \begin{equation}
        \frac{
            \text{Wellfounded}(A, \prec_A)\;\;\;\;
            p \in A\;\;\;\;
            e(x, y) \in C(x) [x \in A, y(z) \in C(z) [z \prec_A x]]
        }{\text{rec}(e, p) \in C(p)}
    \end{equation}
\end{itemize}

Dual to data is the concept of codata.
While data is defined inductively, with elements constructed in a finite number of steps, codata is constructed additively from a base-case, allowing it to represent infinite structures such as streams and infinite trees \citep{hinze2010reasoning}.
While it is possible to reason about data using total functions (those shown to terminate, have no side effects and not return error states), the same cannot be said for codata.\\

Reasoning about codata instead requires a technique known as \textit{coinduction} which describes how to destructure (break-down) codata to permit well-founded reasoning about it. 
Coinduction allows reasoning about the termination properties of codata, but is unable to prove termination of algorithms operating on codata in the general case. 

% subsection data_and_codata (end)

\subsection{Proving Termination} % (fold)
\label{sub:proving_termination}
Through a separation of data and codata, it is possible to define a language whose semantics can be reasoned about purely by well-founded induction, as defined by \citet{nordstrom1988terminating}. 
This encompasses proofs on arbitrary length, finite data structures (those that are, hence, well-formed); inductive reasoning applied to such structures is guaranteed to reach a base-case, as examined in Godel's System-T and System-F \citep{alves2010godel,girard1989proofs}.\\

This would, hopefully, allow it to be shown that all well-formed programs in the language terminate and behave according to their specification (i.e. that they are \textit{correct}).\\

Such a proof requires the definition of the following relation for a program $M$ with unique configurations $s$:
\begin{equation}
    s, M \rightarrow s', M'
\end{equation}
where:
\begin{itemize}
    \item A configuration $s$ refers to any additional computational state (which may include heap state, the continuation, etc).
    \item The relation $\rightarrow$ is termed ``converges to'', and is inductively defined.
\end{itemize}

However, the inductive definition of the convergence relation does not guarantee that the relation is total.\\

If it can be shown, by induction on the structure of $M$, that the rules by which $\rightarrow$ is inductively defined are defined in such a way that the convergence hypothesis for $M$ is given in terms of the sub-programs of $M$, the result follows as long as each base-case terminates. \\

If this holds, then it can be shown for every program $M$ and every unique configuration $s$, there exists a program and configuration $s', M'$ such that $s, M \rightarrow s', M'$.\\

If such a property can be shown for the all base-cases and each semantic construct in the language, it is possible to state that all programs written in this language terminate. 

% subsection proving_termination (end)

\subsection{Total Functional Programming} % (fold)
\label{sub:total_functional_programming}
The distinction is accurately represented by \citet{turner2004total}, where the notion of disciplining the use of a functional programming language to exclude the possibility of non-termination is proposed. 
The paper draws the same distinction between data and codata, and restricts the language to the use of only total functions. \\

In order to better illustrate this, \citet{turner2004total} proposes an augmentation to the language typing discipline, adding an explicit $\bot$ (bottom, a type which has no values) to denote functions that may error or not terminate (those functions that operate on codata). 
Languages such as Haskell already incorporate an expression for this idea via the notion of Monads and Monad Transformers.
Turner aims to omit any incidence of $\bot$ in a total functional program, hence restricting the program to the use of total functions, and omitting the use of partial functions. \\

While this is not feasible in practice for a general purpose programming language, it is interesting for a DSL, as all functions could be constrained to being total.
This can be done through imposition of the following restrictions:
\begin{itemize}
    \item Recursion is used to traverse data.
    \item Corecursion is used to traverse codata, where all the infinite structures (and corecursive functions) are total. 
\end{itemize}

Such a language is not Turing-Complete, and hence cannot express all programs (even if such programs terminate). 
Such a restricted language is unable, for example, to express its own parser \citep{turner2004total}.\\

Nevertheless, the notion of Total Functional Programming has interesting implications for the design of provably correct DSLs.

% subsection total_functional_programming (end)

% section formal_program_verification (end)

\section{Automating the Generation of the Compiler} % (fold)
\label{sec:automating_the_generation_of_the_compiler}
\defblock{10cm}{Compiler Generator}{
    A compiler generator, or metacompiler, is ``a tool that constructs a compiler automatically, given a syntactic and semantic description of the source language'' \citep{brown1992actress}.
}

The literature having established that it is possible to specify a language such that it terminates (and behaves as specified) for all programs in that language, some mechanism needs to exist for translating such programs into an executable form.
While it is possible to manually build a compiler for each DSL that is designed, this is a time-consuming undertaking requiring significant effort that may produce implementation defects, resulting in a non-correct compiler \citep{Mernik:2005:DDL:1118890.1118892}.\\

In order to verify that the resultant DSL is correct, it is better to utilise a compiler generator, a specific form of \textit{application generator}.
\citet{cleaveland1988building} suggests that application generators ``let you customise and reuse a general software design easily'', providing a significant benefit in the reduction of programming errors.\\

This decoupling of the language specification and implementation, as discussed by \citet{cleaveland1988building} is key to the implementation of correct languages.
As long as the metacompiler is correct, the resultant compilers should maintain the semantics of the input programs \citep{Gray:1992:ECF:129630.129637}.
Centralisation of the semantic properties of DSLs into the metacompiler allows the centralisation of the proof infrastructure to allow automation.

\subsection{Metacompiler Systems} % (fold)
\label{sub:metacompiler_systems}
Metacompilers can, in general, be categorised as syntax- and semantics-directed compilers \citep{Mandell:1966:MDA:800267.810785,diehl1996semantics}.
They ingest definitions of the language syntax and semantics and generate a compiler for that language. \\

Traditional compiler design divides the work of the compiler into multiple phases: syntactical and lexical analysis, semantic analysis, optimisation and code generation. 
Generator programs exist for multiple of these phases:
\begin{itemize}
    \item \textbf{LEX:} A generator for lexical analysers --- programs that convert a sequence of characters into a stream of syntax tokens.
    \item \textbf{YACC:} A generator for LALR (Look-Ahead, Left to Right) parsers --- programs that make sense of the source code based upon a provided grammar.
\end{itemize}

Such systems, however, suffer from a somewhat arbitrary mapping between source and target language constructs, with the translation schemes being predetermined, rather than generated alongside the parser. 
Such translation schemes must be implemented by hand, as they are not automatically generated. \\

\citet{diehl1996semantics} suggests that the generation of the compiler can be directed by \textit{semantics} in addition to the \textit{syntax}.
Doing so provides multiple advantages over handwriting compilers in accordance with a language specification:
\begin{itemize}
    \item \textbf{Correctness:} If the generator can be verified, this implies that the generated compilers are proved correct.
    \item \textbf{Readability:} The specification of a programming language is generally more intelligible than a compiler.
    \item \textbf{Maintainability:} Altering the language specification results in a compiler that automatically supports the new features after regeneration.
    \item \textbf{Portability:} Changing the definition of the target language allows generation of compilers for different architectures without changes to the source languages.
    \item \textbf{Insight:} Generation of compilers via semantics aims to use semantics preserving transformations which relate source code to target code. 
    By tracing these transformations, the target code can be explained. 
\end{itemize}

While these benefits are certainly true, there are some downsides to such generated compilers:
\begin{itemize}
    \item Verification of the metacompiler is a non-trivial task for a GPL.
    \item Programming language specifications, particularly those with formally specified semantics, are not necessarily legible to someone without appropriate experience. 
    This means that writing and altering such specifications could be a significant source of maintenance burden for such compilers.
    \item The design of the metacompiler may not easily permit alteration of the target language.
    It may be easier in practice to target a single language with a simple FFI such as the C FFI (e.g. Haskell), allowing easy interoperability with other programming languages. 
\end{itemize}

The metacompiler system proposed by \citet{diehl1996semantics} utilises an Action Semantics (see Section~\ref{ssub:hybrid_semantics}) based specification of the underlying language, generating a compiler based upon the semantics. 
The compiler targets an abstract machine, adding portability for native code generation, and uses a `Term Rewriting System' to generate both the compiler and program behaviour, deferring modifications of program and state to different portions of the compilation-execution pipeline \citep[pg. 59]{diehl1996semantics}.\\

These Term Rewrite Rules are defined in such a fashion that the semantics of the original language are preserved, ensuring that any properties of the original program still hold in the compiled version.
This means that the generated compiler will produce a program that is semantically correct insofar as the original language specification is semantically correct. \\

Such a capability is key to the correct operation of this project, and so a rigorous Term-Rewriting System could be pursued. 

% subsection metacompiler_systems (end)

\subsection{Program Transformations} % (fold)
\label{sub:program_transformations}
\defblock{10cm}{Transformation}{
    A transformation is a rule that operates on the parse-tree of the source language, providing well-specified constructions in the target language that maintain the semantics of the source language \citep{diehl1996semantics}.
    Such transformations may operate on either an AST (see Section~\ref{ssub:abstract_syntax}) or individual portions of the language grammar \citep{brabrand2003metafront}.
}

The intention of this project is to transpile (see Section~\ref{ssub:transpilation}) the DSL source program to an equivalent Haskell program.
The process of doing so will utilise program transformations, similar to those defined by \citet{brabrand2003metafront} for the Metafront Project.\\

Transformations are defined by \citet{brabrand2003metafront} to have three properties that are important for formal verification of the semantic correctness of the translation:
\begin{itemize}
    \item They are designed to allow only well-founded induction, so termination of the transformations is ensured.
    \item Transformations can be decided statically if they will map legal input to legal output. 
    \item The transformation rules can perform expressive transformations that rearrange ASTs in a non-local manner, while preserving the semantics of the program.
\end{itemize}

The fact that such transformation ensure preservation of the language semantics is significant for their use in a metacompiler system. \\

These semantics, however, are not without their limitations. 
They only allow inductively defined semantics, and hence disallow any action-semantics or monadic semantics representations. 
Additionally, they are only able to operate between context-free grammars. 
As the target language of a metacompiler may not be context-free, or able to be restricted to a context-free subset, the ability to use such transformations in a metacompiler may be limited.\\

Such limited translation rules can be contrasted with the Turing-Complete rewrite system used in language development systems such as ASF+SDF, which operate on the AST of the source language \citep{van2001asf+}.
Such transformations, however, do not provide the same guarantees as the Metafront transformation system given by \citep{brabrand2003metafront}.

% subsection program_transformations (end)

\subsection{Intermediate Representations} % (fold)
\label{sub:intermediate_representations}
Many compiler systems use an Intermediate Representation (IR).
This is as it ``hides details about the target execution platform'', allowing semantic-level optimisation and analysis of the program code \citep{Zhao:2012:FLI:2103621.2103709}.
It is, however, often the case that these IRs do not have well-defined formal semantics, precluding the proving of properties of programs expressed in these IRs. \\

In an effort to formalise the LLVM IR, their Low-Level Virtual Machine, \citet{Zhao:2012:FLI:2103621.2103709} found that the aggressive optimisations performed by industrial strength compilers often had non-deterministic semantics.
Such semantics preclude the proving of certain termination properties of the transformed programs, even if they use linguistic features that are terminating.
Such optimisation issues can be seen with the Glasgow Haskell Compiler (GHC), where certain heavy optimisations alter the semantic meaning of programs (see Section~\ref{sub:issues_with_ghc}.\\

While the use of IRs bring benefits in terms of semantic analysis and AVOPT, the complexity of such representations means that for a DSL-based project that it is unlikely to be worth employing one.

% subsection intermediate_representations (end)

% section automating_the_generation_of_the_compiler (end)

\section{Technological Support} % (fold)
\label{sec:technological_support}
A project such as this requires an implementation language with excellent support for symbolic manipulation, domain-specific optimisation, language parsing and control over program execution.
While languages such as Lisp allow first-class syntax definitions, it has less library support for parsing, and less robust execution control due to the default eager evaluation semantics. \\

Haskell, on the other hand, has robust support for metaprogramming, parsing and execution control, providing the \lstinline{seq} function to force evaluation of the default lazy semantics. 
This makes Haskell an excellent choice for the building of modular language implementations \citep{hudak1996building}.\\

Languages like Haskell and Lisp have excellent support for symbolic manipulation, and hence would be viable implementation languages for such a project. 
This is to be contrasted with more traditional programming languages such as C++, which are more suited to numeric manipulation.
Such languages, additionally require management of implementation features such as memory, which detracts from the ability to implement the required program features.\\

The premier Haskell compiler in use today is the Glasgow Haskell Compiler (GHC), which provides a robust suite of well-tested language extensions.

\subsection{Template Haskell} % (fold)
\label{sub:template_haskell}
One of these language extensions is Template Haskell, a robust metaprogramming mechanism to allow ``compile-time preprocessing of Haskell Source programs'', allowing the programmer to define new language syntax without compiler modification \citep{Sheard:2002:TMH:581690.581691,Czarnecki2004}.
Template Haskell provides support for the ``algorithmic construction of programs at compile time'', including techniques for language rewriting and optimisation, techniques discussed in Sections~\ref{sub:program_transformations} and~\ref{sub:types_of_dsls} respectively \citep{Sheard:2002:TMH:581690.581691,jones2001playing}.
\begin{itemize}
    \item Through the use of Template Haskell's compile-time IO mechanism, DSL programs can be read as input, and transformed into an output Haskell program for compilation \citep[pg. 9]{Czarnecki2004}.
    \item Template Haskell is capable of introducing new names and syntax, providing a mechanism for compile-time syntax extension \citep{Czarnecki2004}.
    \item It allows the alteration of program semantics through rewriting at compile time, allowing significant flexibility with the implementation of DSLs. 
    \item Through the ability to inspect the program AST as data (using a quoted expression), it can modify and parse the semantics of the input program as required, resulting in an equivalent Haskell program without performance compromises.
\end{itemize}

Furthermore, Template Haskell allows the specification of compile-time rewrite rules for language elements \citep{jones2001playing}.
Such capabilities allow for the optimisation of domain-specific logic through definitions of domain-specific optimisations that may not be visible to the Haskell compiler. \\

The major downside of the capabilities provided by Template Haskell is the introduction of additional complexity. 
While using the metaprogramming mechanism does not restrict the use of any other language features, it can significantly contribute to program complexity.

% subsection template_haskell (end)

\subsection{Language Parsing} % (fold)
\label{sub:language_parsing}
If, on the other hand, a more traditional parsing approach is wanted, Haskell still provides significant and flexible support for doing so. 
Through its extensive library support, Haskell provides significant tooling for the creation of parsers, including Happy and Parsec, two libraries for generating language parsers. \\

Parsec is the more capable library of the two.
It is a monadic parser combinator library that is capable of parsing context-sensitive, infinite-lookahead grammars \citep{leijen2001parsec}.
While traditional parser generators such as YACC use event-based parsing, combinator parsing is unique in that it allows the programmer to write expressions which appear to be language grammars, and yet describe a parser for such grammars.
\begin{itemize}
    \item This avoids the need for significant amounts of preprocessing by harnessing the power of Haskell itself. 
    The strict type-system of Haskell is ideal for the description of embedded DSLs \citep{swierstra2009combinator}.
    \item Parsec features a novel combinator-based implementation technique that enhances the parser efficiency and allow the production of useful error messages \citep{leijen2002parsec}.
    \item Useful error messages decrease programmer burden for the users of the end-result, and hence are a significant benefit to using the library. 
\end{itemize}

The main downside of Parsec is that it provides no guarantees about left-recursion in grammars, which has the potential to cause the parser to hang at runtime. 
Furthermore it does not perform as well as some traditional parsers due to the additional state maintained for error reporting, as evidenced by the existence of Attoparsec \citep{gummelt2011hindsight}. \\

While Parsec is much faster than many parser combinator libraries, it is still too slow for real-time use or efficient parsing of very complex grammars. 
Attoparsec, however, sacrifices readability of error messages for performance, meaning that Parsec itself is still a more sensible choice for the implementation of user-facing compilers. \\

Haskell's library support also provides Alex, a tool for generating lexical analysers. 
While not strictly required for parsing using either Happy or Parsec, it can greatly simplify implementation of more complex parsers through elimination of certain bug-classes through the type system (e.g. parsing a keyword as an identifier).

% subsection language_parsing (end)

\subsection{The Haskell FFI} % (fold)
\label{sub:the_haskell_ffi}
\defblock{12cm}{FFI}{A Foreign Function Interface (FFI) is a mechanism by which a program written in one programming language can call functions written in another programming language.}

A DSL is not useful if it isn't portable.
As discussed in Section~\ref{sub:types_of_dsls}, DSLs with implementations rooted in a given language become specific stores of domain knowledge rather than portable encapsulations.
\citet{Marlow:2004:EHF:1017472.1017479} suggests that the C FFI is the lowest-common-denominator for interactions between languages, and so is a useful target for DSL portability.\\

While there are some issues surrounding the Haskell FFI and concurrency, it is possible to both call C-FFI functions from Haskell, and call Haskell functions using a C-FFI \citep{Marlow:2004:EHF:1017472.1017479,HaskellWikiFFIFromC}.
This allows DSLs compiled with Haskell to act as a lowest-common-denominator interface to the DSL through automatic generation of the C-FFI function-call stubs. 
As a result, the DSL can act as a portable repository of Domain-Specific knowledge and configuration. 

% subsection the_haskell_ffi (end)

\subsection{Desirable Language Properties} % (fold)
\label{sub:desirable_language_properties}
Haskell also contributes additional, desirable language properties to the project:
\begin{itemize}
    \item Haskell has desirable properties when it comes to automated proofs of termination \citep{Giesl:2011:ATP:1890028.1890030}.
    This means that it is possible to automatically prove that, for well-founded implementations, the termination properties of the DSL can be verified in the final transpiled DSL code. 
    This analysis examines some of the linguistic complexities of Haskell, and shows that termination can be shown even in the presence of lazy evaluation, equation definition order, polymorphic typing and potentially infinite codata structures. 
    \item Embedded DSLs are difficult to optimise as the host language's optimising compiler only operates at the level of the host language constructs. 
    Future work may want to increase the efficiency of generated code, and the compile-time metaprogramming provided by Template Haskell to express compile-time optimisation opportunities in a declarative fashion such that the GHC optimising compiler can take advantage of them \citep{seefried2004optimising}.
\end{itemize}

% subsection desirable_language_properties (end)

\subsection{Issues with GHC} % (fold)
\label{sub:issues_with_ghc}
Despite all of the benefits that the use of Haskell and GHC can bring to the project, it is not without its flaws.
GHC is an optimising compiler and, as discussed in Section~\ref{sub:intermediate_representations}, some optimisations can alter the semantic meaning of programs.\\

While optimisations, in general aim to preserve the semantic correctness of programs, there are certain documented optimisations that GHC can perform that may alter the program semantics. 
These optimisations are collectively known as Short-Cut Fusion, and they aim to eliminate successive data-structure allocations in chained functional calls, optimising the code to a tight loop over the final structure \citep{HaskellWikiShortCutFusionCorrectness}.\\

While it is strongly conjectured that such an optimisation cannot create a non-terminating program from a terminating program, it has not been conclusively proven \citep{voigtlander2008semantics}.
This means that projects dealing with the correctness of programs must be careful when utilising certain optimisations in GHC, avoiding any optimisations that may semantically alter the code.

% subsection issues_with_ghc (end)

% section technological_support (end)

\section{Combining the Ideas} % (fold)
\label{sec:combining_the_ideas}
This literature and technology review has examined a broad base of previous work surrounding the creation of provably correct domain-specific languages.\\

It has focused on the study of DSLs themselves, as well as mechanisms for the formalisation of program syntax and semantics. 
Furthermore, it has explored methods for program verification and proving program correctness, and techniques for automated compiler generation. \\

As a result of the survey, it is clear that the project wants to pursue a limited set of DSLs, namely external DSLs, allowing a flexibility of syntax, with transpilation (ensuring semantic preservation) into Haskell.
The DSLs would have their syntax specified using EBNF, and semantics likely specified using some form of operational semantics as a compromise between simplicity and expressiveness. 
Such semantic specifications are hoped to allow the compiler system to prove the termination properties of input programs in the DSL. \\

The project will likely see an implementation of a metacompiler system in Haskell, ingesting the language specifications to produce a compiler for that language. 
The resultant language compilers will use a transpilation approach via program transformation rules to transform input programs in the DSL into equivalent Haskell programs, allowing use from a multitude of host languages. \\

It is hoped that these techniques will produce a flexible compiler system that is capable of generating programs written in provably correct Domain-Specific Languages.

% section combining_the_ideas (end)

% chapter literature_and_technology_survey (end)

% Identification of the requirements capture process
% Explanation of why the chosen technique was utilised over other potential competitors.
% Identify and discuss key requirements, with a focus on areas of challenge, difficulty or conflict.
% Be sure to show appropriate scoping. 
% Identify the areas of requirements analysis and specification that were particularly successful
% Be critical of areas where compromises were required. 

\chapter{Elucidation} % (fold)
\label{cha:elucidation}
With a necessarily broad body of work examined in the literature and technology survey (see Chapter~\ref{cha:literature_and_technology_survey}), it is necessary to consolidate this information as it applies to the project. 
This section aims to provide a concrete understanding of which portions of the research will apply directly to the project, fill in any gaps in the research and provide a high-level specification for the project as a whole.

\section{A Kind of Domain-Specific Language} % (fold)
\label{sec:a_kind_of_domain_specific_language}
As has been previously mentioned, it is computationally infeasible (in fact being an instance of the \gls{halting_problem}) to decide whether an arbitrary programming language will terminate in all cases. 
In order to avoid this, \gls{absol} focuses on \glspl{dsl}, and beyond that a specific type of DSL. \\

This section aims to consolidate the breadth of information presented in the literature review to provide a concrete requirement for the type of \glspl{dsl} with which the project will be dealing. 

\subsection{DSL Execution Strategy} % (fold)
\label{sub:dsl_execution_strategy}
As previously examined, \glspl{dsl} can provide a wide scope of different execution types. 
As this project focuses on the verification of \gls{dsl} \textit{semantics}, it is important that the DSLs considered have some form of executable behaviour. \\

As examined in Section~\ref{sub:types_of_dsls}, \citet{Mernik:2005:DDL:1118890.1118892} proposes two main categories of DSL with executable semantics:
\begin{itemize}
    \item Well-defined execution semantics
    \item \Gls{declarative} inputs to applications
\end{itemize}

While the latter is an interesting case, allowing for users to provide a simple set of configuration denoting the structure of a problem in their domain, the less well-defined execution semantics pose a problem for verification.
As well-defined execution semantics easily permit semantic analysis, \gls{absol} focuses on the first type of DSL mentioned above.\\

This means that creators of these \glspl{dsl} need to be able to fully specify the semantics associated with each kind of program statement. 
These semantics must have a restricted form (see Section~\ref{sub:choosing_a_semantic_form}) to ensure that the verification problem is tractable.  
The restriction does, however, mean that the kinds of programs that the language can represent is limited.

% subsection dsl_execution_strategy (end)

\subsection{DSL Implementation Strategy} % (fold)
\label{sub:dsl_implementation_strategy}
As examined by both \citet{Mernik:2005:DDL:1118890.1118892} and \citet{van2000domain}, one of the key merits of a \gls{dsl} is the ability to enable re-use of program specification. 
As a result, \gls{absol} focuses on providing languages that can be interfaced with from multiple host languages. \\

While it seems that the embedded approach is common due to its ease of implementation, this comes with restrictions around error reporting, and the form of the syntax. 
\gls{absol} will instead provide a fusion of the compiled and embedded approaches:
\begin{itemize}
    \item The language will be described in a metalanguage, allowing the generation of the compiler from both syntax and semantics.
    This allows opportunities for language-level analysis that would be otherwise impossible if the \gls{dsl} was expressed directly in a host language.
    \item DSL programs themselves will be compiled into a target language, with the compilation process allowing for a domain-specific approach to \gls{avopt}.
    \item The target language will need to provide some flexible mechanism for interfacing with other languages, most likely via the C \gls{ffi}, as this is somewhat of a lowest-common-denominator for cross-language interfaces \citep{van2001asf+}.
    \item While such an approach requires conventionally requires significant development effort, the two-stage, metacompiler-based approach taken by \gls{absol} will reduce that burden significantly.
\end{itemize}

Such an approach neatly sidesteps the issues encountered by many embedded DSLs as the user-specified syntax and semantics are not restricted by any host language, thereby avoiding suboptimal syntax or non-domain-oriented semantics, and poor error reporting.
That is not, however, to imply that there are no restrictions on the form of the syntax and semantics here, but only that these restrictions are not resultant from the choice of target language. \\

This approach involves a \gls{transpilation} step, which \citet{kulkarnitranspiler} finds to provide many of the benefits of both the embedded and compiled approaches. 
One particular benefit of such an approach, that \gls{absol} aims to use, is the ability to provide detailed domain-specific diagnostics and error messages, while utilising the power of the target language as much as possible.\\

The selection of a transpilation-based approach is not without its flaws, however as, while static errors can be expressed in terms of the source language semantics, any runtime errors will still be expressed in terms of the target language's semantics. 
This semantic mismatch can be handled via \glspl{source_map}, but this requires significant additional work, and is thus considered as out of scope for this project. 

% subsection dsl_implementation_strategy (end)

% section a_kind_of_domain_specific_language (end)

\section{Languages and Programs} % (fold)
\label{sec:languages_and_programs}
As a project, \gls{absol} deals with the specification of \glspl{dsl}. 
This means that, in addition to dealing with \gls{dsl} \textit{programs}, it first has to deal with \textit{languages}. 
In order to both constrain the things these languages can represent, and have a standard for for interacting with them, \gls{absol} must define a metalanguage. \\

Designing a metalanguage requires methods for specifying both the syntax and semantics of the defined language.
Research showed that there had been little work on metalanguages to perform both tasks at once, and so the onus fell on the project to create one. 

\subsection{Choosing a Syntactic Form} % (fold)
\label{sub:choosing_a_syntactic_form}
Determining the basic syntactic metalanguage --- the form of the syntax rules for the defined language --- proved to be a mostly simple endeavour. 
While both research and industry have used multiple syntactic metalanguages in the past, research found that most modern syntactic specifications use some variant of \gls{bnf}. \\

The standardised variant of \gls{bnf}, known as \gls{ebnf} and defined by \citet{standard1996ebnf}, it provides a flexible syntax for defining context-free language syntax. 
While it also supports the definition of context-sensitive syntax via extension rules, most simple programming languages can be defined using context-free productions. \\

One of the main issues with the standard \gls{ebnf} syntax, however, is the use of the concatenation operator (\lstinline{,}) which results in all productions appearing as lists of terminals and non-terminals.
While this is not a nonsensical representation, it can hamper the intuitive readability of the productions, and so has been changed for this project.

% subsection choosing_a_syntactic_form (end)

\subsection{Choosing a Semantic Form} % (fold)
\label{sub:choosing_a_semantic_form}
The literature survey explored multiple methods for the expression of language (and program) semantics.
Clearly for formal semantic analysis, the natural-language descriptions used by many GPLs would not suffice, meaning that the project had to use some form of formal semantics as a starting point. \\

The literature survey examined operational, denotational, axiomatic and hybrid semantics.
While most hybrid semantic frameworks (see Section~\ref{ssub:hybrid_semantics}) provide significant expressive power, they often involve correspondingly significant levels of complexity.
As the semantic metalanguage needed to be both easily understood and easily written, this restricted the choice of semantic framework.\\

The final choice for the main form of semantics was the natural operational semantics (see Section~\ref{ssub:operational_semantics}), as they describe how the \textit{overall computed result} for the computation is obtained.
This frees the language designer from dealing with intermediate states, but imposes limitations on the project: they are not suitable for detailing concurrent or interleaved execution.
In the examination of \glspl{dsl}, however, this is unlikely to be any real issue.\\

However, in choosing operational semantics, the project has to deal with two main issues:
\begin{enumerate}
    \item \textbf{Semantic Representation:} Standard operational semantic rules use a multi-level format for the axioms.
    Such a format is difficult to directly represent in a non-rich-text environment (e.g. a code editor), and so requires transformation to ensure that semantics can be intuitively expressed.
    \item \textbf{Specific Semantic Representations:} While most operational semantics can be represented with the sub-evaluations depending on structural sub-terms of the main evaluation, this is not true of all kinds
    of program semantics.
    While this restriction is necessary for the verification engine to operate on the semantics, this prevents representation of many useful language features.
\end{enumerate}

To this end, the final semantic format for the metalanguage must support both generic semantic evaluations, and some method of providing more complex semantic features.
As these features cannot be directly proved from the form of the semantics, they must be subject to generic, external proof. 

% subsection choosing_a_semantic_form (end)

\subsection{Language Verification} % (fold)
\label{sub:language_verification}
While the problem of program verification (and hence language verification) is generally undecidable, this project aims to place restrictions on the languages that it can represent to make it tractable. 
These restrictions come from the \textit{semantic form}, as discussed in Section~\ref{sub:choosing_a_semantic_form}, and thereby restrict the types of programs that can be represented.

\subsubsection{Language vs. Program Verification} % (fold)
\label{ssub:language_vs_program_verification}
While proving properties of \textit{languages} is more general than proving properties of \textit{specific programs}, this imposes further restrictions on the kinds of properties that it can prove. \\

As part of the investigation of Data and its dual Codata in Section~\ref{sub:data_and_codata}, it was found that it is possible to prove termination for well-founded general recursion over data. 
However, this proof can only be performed at the program level, as done by Idris \citep{idris_lang}. 
This restriction is because types are predicated on values, where none of the
values exist at a program level. \\

This means that the proof mechanism used by \gls{absol} is less general, and unable to prove that recursion between arbitrary functions terminates.
While this \textit{does} restrict the kinds of programs that metaspec languages can represent

% subsubsection language_vs_program_verification (end)

\subsubsection{Traversal of Data} % (fold)
\label{ssub:traversal_of_data}
That is not to say that the result is not usable.
While at the program level it is not possible to allow recursive function calls, it is possible to provide special-case semantics that allow for the traversal of data. \\

To this end, the metalanguage should ensure that all structures that it can define can be reasoned about via well-founded induction, as discussed in Section~\ref{sub:proving_termination}. 
As long as the language (and proof engine) enforce the rules given in the literature survey it is then possible to show that all possible programs that can be represented in the language terminate.\\

It is this idea that the project has pursued in light of providing a language-level proof mechanism. 

% subsubsection traversal_of_data (end)

% subsection language_verification (end)

% section languages_and_programs (end)

\section{Filling in the Gaps} % (fold)
\label{sec:filling_in_the_gaps}
While the Literature and Technology Survey provided a comprehensive overview of much of the material, there were still some areas that were either not examined in sufficient detail, or ignored entirely. 
This section provides brief explorations of these.

\subsection{Guard Completeness Checking} % (fold)
\label{sub:guard_completeness_checking}
As, at the time of performing the literature survey, no decisions had been made as to the semantic representation, it was not apparent that it would be required to verify guards in the language.
However, since the selection of operational semantics as the basic form of semantic representation, it is clear that some `guard'-style functionality is needed. \\

Guards, in this sense, refer to restrictions on the values of the sub-evaluations of a rule, as can be seen in the following pair of operational semantic rules for a basic \texttt{if-then-else} expression:
\begin{align}
    [\text{if}] &: \frac{\langle S_1, s \rangle \to s'}{\langle \text{if } b \text{ then } S_1 \text{ else } S_2, s\rangle \to s'} \text{ if } \mathbb{B}\llbracket b \rrbracket s = \textit{ true} \\
    [\text{if}] &: \frac{\langle S_2, s \rangle \to s'}{\langle \text{if } b \text{ then } S_1 \text{ else } S_2, s\rangle \to s'} \text{ if } \mathbb{B}\llbracket b \rrbracket s = \textit{ false}
\end{align}

In this case the, `$\text{if } \mathbb{B}\llbracket b \rrbracket s = \textit{ false}$' and `$\text{if }\mathbb{B}\llbracket b \rrbracket s = \textit{ true}$' are what the project terms the \textit{semantic restrictions}.\\

In order to guarantee that a language where semantics can contain these guards is complete, the verification process must be able to guarantee that there is a semantic rule for all possible values of the guarded variables.
In the case above, where the value is binary, this is trivial, but over more complex domains (e.g. $n \in \mathbb{Z}$, for some variable $n$), it becomes significantly more complex.\\

A set of guards for a given piece of program semantics are a set of constraints.
There is a well-known method for solving equations with sets of constraints: linear programming.

\subsubsection{Linear Programming} % (fold)
\label{ssub:linear_programming}
\defblock{10cm}{Linear Program}{
    A Linear Program is an optimisation problem in which the objective function is linear in the unknowns and the constraints consist of linear equalities and linear inequalities \citep{luenberger2016simplex}.
}

The set of constraints in the guard completeness problem can be viewed as constraints on the optimisation.
Hence, the question of ``do these guards cover all possibilities' can be transformed to ``do the negations of these guards have a solution''. 
The hope, is that the negated system does not have a solution.\\

Solving this is known as the \textit{Feasibility Problem}, the aim of which is to determine if the region defined by the set of constraints is a bounded search space \citep{luenberger2016simplex}. 
If the negations of the guards form a system of constraints that are not feasible, then the guards are complete over the domain. \\

The problem, however, is slightly more complex than directly checking for feasibility, as guard conditions may be of the form $X \land Y$, which when negated is of the form $\lnot X \lor \lnot Y$ (by De-Morgan's Laws).
This disjunction means that the problem is slightly more difficult, involving checking sets of guard conditions for being \textit{infeasible}.

% subsubsection linear_programming (end)

\subsubsection{Guard Completeness and Project Scope} % (fold)
\label{ssub:guard_completeness_and_project_scope}
Unfortunately the implementation of such a completeness checker would add significant complexity to an already complex project, and as such is considered as out of scope.
Nevertheless, it is an important and interesting problem to be solved as part of the future work.\\

This is not to imply that the guards will remain unchecked in the language, as that would admit possible non-termination due to no semantics existing for some evaluations.
To this end, a simpler method of ensuring guard completeness must be included, even if it is one that is inelegant. 

% subsubsection guard_completeness_and_project_scope (end)

% subsection guard_completeness_checking (end)

\subsection{Megaparsec --- Improved Parsing} % (fold)
\label{sub:megaparsec_improved_parsing}
As part of the technology survey, the availability of parsing libraries in Haskell was examined, finding that both Parsec and Happy were available (see Section~\ref{sub:language_parsing}). 
However, research failed to identify Megaparsec, a fork of the Parsec library that retains all of its strengths while fixing a multitude of issues with the long extant library \citep{megaparsec}.\\

The fixes that Megaparsec applied on top of the Parsec codebase included significant refactoring efforts, but most importantly for the project:
\begin{itemize}
    \item A greatly simplified Lexer interface, allowing for the creation of simple lexing code with ease.
    This removes the need for an external lexer such as Alex, and means that the parser code can be greatly simplified.
    \item Refactoring to the parsing of left-recursive expressions such as arithmetic and other binary operators. 
    Megaparsec provides an expression parser that handles all of the lookahead requirements automatically, providing the potential for further simplification to the parser code. 
    \item Significantly improved error messages, especially in the case of backtracking parsers. 
    \item General improvements to the interface provided by the libraries, including more intuitive naming for certain oft-used functions.
\end{itemize}

In light of these benefits over the standard Parsec library and the fact that it retains the other benefits of parsec, Megaparsec was chosen instead. 
Allowing simpler parser code and integrated lexing means that the parser portion of the project can be greatly simplified. 

% subsection megaparsec_improved_parsing (end)

% section filling_in_the_gaps (end) 

\section{High-Level Requirements Specification} % (fold)
\label{sec:high_level_requirements_specification}
This section provides a high-level outline of the requirements for the system as a whole. \\

The project consists of two main components:
\begin{itemize}
    \item The Metalanguage: Metaspec
    \item The Metacompiler: \gls{absol}
\end{itemize}

\subsection{The Requirements Generation Process} % (fold)
\label{sub:the_requirements_generation_process}
As this is a research project with a heavy software component, the requirements elicitation process is far less formal than for a purely software project.\\

As there are no stakeholders beyond those working on the project, the requirements in the following section were mostly generated from a vision of what an ideal DSL toolchain would look like. 
The requirements engineering process can be summarised in brief as follows:
\begin{enumerate}
    \item \textbf{Scenario Examination:} The informal but informed discussion of scenarios where such a toolchain might be used, inspired by the project's genesis at Bloomberg and in the wider industry.
    \item \textbf{Requirements Analysis:} Each scenario was, again informally, analysed to find what project requirements that it could generate.
\end{enumerate}

This method was selected due to the lack of any \textit{real} stakeholder, thus precluding more formal requirements analysis, and the research nature of the project, as it recognised the potentially in-flux nature as the research proceeded. 

% subsection the_requirements_generation_process (end)

\subsection{Requirements for the Metalanguage} % (fold)
\label{sub:requirements_for_the_metalanguage}
The first of the main project components is the metalanguage, Metaspec. 
It has the following requirements imposed upon it.\\

\requirement{Specify Language Syntax}{Functional}{
    The metalanguage must provide a flexible and intuitive way for the user to specify the syntax of the DSL.
}

\requirement{Specify Language Semantics}{Functional}{
    The metalanguage must provide an intuitive method for the user to specify thes language semantics in the general case, and also provide special-case semantics for things that cannot be proved by the proof mechanism.
}

\requirement{Semantic Typing}{Functional}{
    Language types should be enforced at the semantic level.
    The reasoning for this is twofold:
    \begin{enumerate}
        \item DSLs often have concise and clear syntax, and cluttering this with syntax-level typing compromises this goal somewhat.
        \item It is difficult to provide an extensible mechanism for language designers to add syntax-level typing.
    \end{enumerate}

    Combined, it is clear that the semantic level is the most appropriate place to enforce language types. 
}

\requirement{Integrated Syntax and Semantic Specification}{Non-Functional}{
    The forms of specification for both syntax (see Requirement~\reqref{req:SpecifyLanguageSyntax}) and semantics (see Requirement~\reqref{req:SpecifyLanguageSemantics}) should be integrated together in an intuitive fashion.
}

\requirement{Data Types}{Functional}{
    The metalanguage must provide a useful set of data-types to allow for useful computation to be performed. 
    These must include integer and floating point types, as well as list and matrix container types.
}

\requirement{Function Calls}{Functional}{
    The metalanguage must provide the ability to define procedures callable from host languages, and subroutines callable internally.
    The way in which these routines are defined and called should be congruent with being able to prove that the language semantics terminate.
}

\requirement{Data Traversal}{Functional}{
    The metalanguage should provide mechanisms for traversing the data types it defines, so as to avoid the need to allow general recursion.
}

\requirement{Ground-Truth Semantics}{Functional}{
    The metalanguage should provide a way to specify ground-truth semantics for non-terminals. 
    These are things that can be trivially assumed to terminate by the proof mechanism.
}

\requirement{Extension Mechanisms}{Functional}{
    The metalanguage should provide a set of `language features' that can be imported into scope for use by the DSL designer. 
    These should provide useful language constructs (e.g. non-terminals, special semantics) to aid in the construction of both syntax and semantics for the DSL.
}

\requirement{Environment Accesses}{Functional}{
    The metalanguage should provide syntactic constructs for both storing and accessing values in the environment. 
    This is required to support DSL creators defining their own function definitions and other such constructs in the DSL. 
}

\requirement{Language Metadata}{Functional}{
    The metalanguage should provide syntactic constructs for managing metadata about the language itself.
    These must include the language name and the language version.
}

\requirement{Intuitive File Structure}{Non-Functional}{
    The structure of a file in the metalanguage should have an intuitive structure.
    This means that the file should establish all prerequisites to the definition of the language before the language itself is defined. 
    This means that all the necessary context to understand the language definition is provided.
}

\requirement{Comments}{Functional}{
    The metalanguage must provide both line and block comments to allow for annotating language specifications.
    These comments may have no semantic meaning in the language and may be stripped at parse time. 
}

\requirement{Text Editor Ready}{Functional, Non-Functional}{
    The metasyntax specified by the metalanguage should be representable as plaintext with no markup required. 
    This ensures that it can be written in a standard text editor.
    It should, however, still have support for unicode glyphs as these can assist in matching the domain environment. 
}

% subsection requirements_for_the_metalanguage (end)

\subsection{Requirements for the Metacompiler} % (fold)
\label{sub:requirements_for_the_metacompiler}
The second major component of the project is the metacompiler, \gls{absol}.
It must conform with the following set of requirements.\\

\requirement{Parse Metaspec}{Functional}{
    The metacompiler must be capable of both lexing and parsing a metaspec file into an appropriate AST data structure. 
}

\requirement{Verify Language Construction}{Functional}{
    The metacompiler must be capable of verifying that all used non-terminal symbols are defined, and that no non-terminal is defined more than once.
    These are pre-requisites for the verification engine.
}

\requirement{Verify Semantic Form}{Functional}{
    The metacompiler must be capable of verifying that all semantic rules match their appropriate forms.
    This includes all types of special semantic rules, as well as the user-defined operational-style semantics.
}

\requirement{Verify Semantic Guards}{Functional}{
    The metacompiler must be capable of verifying that all semantic rules match their appropriate forms.
    This includes all types of special semantic rules, as well as the user-defined operational-style semantics.
}

\requirement{Generation of Verification Reports}{Functional}{
    In the case where the metacompiler is unable to verify an input language, it should generate detailed diagnostics that indicate the reason(s) why the language failed to verify.
    These diagnostics should provide a trace (from the start symbol of the language) to help the language developer determine the location of the error. 
}

\requirement{Defer Typechecking}{Functional}{
    The metacompiler does not deal with semantic type checking, as type holes cannot be inferred at the language level. 
}

\requirement{Prevention of Arbitrary Recursion}{Functional}{
    The metacompiler should check that languages do not enable arbitrary recursion over data, as this would make termination impossible to show at the language level. 
}

\requirement{Extensibility}{Non-Functional}{
    The metacompiler should be designed in such a fashion that it can easily be extended in the future to accommodate further developments on the way to productisation. 
    This means that it must be modular. 
}

% subsection requirements_for_the_metacompiler (end)

\subsection{Out of Scope Requirements} % (fold)
\label{sub:out_of_scope_requirements}
Over the course of the project, certain portions of the system that were initially in-scope had to be moved out of scope due to concerns over completing the project on time.
The main casualty of these time restrictions was the creation of a complete metacompiler pipeline.\\

While it is a shame that the project will not result in a complete `product', this is less of a problem than it may initially seem.
The \textit{novel} work of the project is concentrated in the metacompiler front end, which ingests, parses and verifies the language. 
Any further code-generation from the defined language has already seen significant exploration, particularly by \citet{diehl1996semantics}.
To this end, the following requirements have been moved out of scope.\\

\requirement{DSL Compiler Generation}{Functional}{
    Generation of a compiler for the DSL specified in the input file from both the syntax and semantics contained therein.
    This DSL compiler must be capable of taking programs in the specified DSL and transpiling them to Haskell code.
    This resultant Haskell code should be ready for use via the C \gls{ffi}.
}

While an important part of `productising' this toolchain, the generation of the DSL compiler from the metaspec AST is out-of-scope.
This is because it is not novel work, and yet constitutes a significant amount of implementation effort. 
As it doesn't really contribute much to the state of the art, it is not considered as part of this project.\\

\requirement{Full Semantic Guard Checking}{Functional}{
    The metacompiler verification stage must ensure that all sets of guards for user-specified semantics are complete. 
    \textit{Complete} means that there is no set of values that the program can create which will not satisfy \textit{at least one} of the extant guards.
    This helps to ensure that the program always has defined semantics.
}

As discussed in Section~\ref{sub:guard_completeness_checking}, it is possible to develop a sophisticated mechanism for checking the completeness of the guards.
However, the significant development effort this would require is unlikely to be achievable within the project time-frame, and is hence ruled as out of scope. 
As mentioned in Section~\ref{sub:requirements_for_the_metacompiler} this doesn't, however, absolve the metacompiler of needing to check guard completeness. 

% subsection out_of_scope_requirements (end)

\subsubsection{Evaluating the Requirements Specification} % (fold)
\label{ssub:evaluating_the_requirements_specification}
The requirements contained in Sections~\ref{sub:requirements_for_the_metalanguage} and~\ref{sub:requirements_for_the_metacompiler} provide a high-level overview of the goals that the metacompiler toolchain is expected to meet. 
While it does not provide a high-level of detail, this is appropriate for the nature of the project.\\

Due to the research-based focus of the project, more specific requirements would be in a state of constant flux, while these higher-level specifications are broad enough that they can remain in place for the duration of the project.

% subsubsection evaluating_the_requirements_specification (end)

% section high_level_requirements_specification (end)

% chapter elucidation (end)

\chapter{Designing the Metalanguage} % (fold)
\label{cha:designing_the_metalanguage}
As a project, \gls{absol} has had a very heavy research bent. 
The experimental nature of the toolchain resulted in a heavy up-front design load and, combined with the highly theoretical nature of the language verification algorithms, this meant that language design and algorithmic development dominated the time spent on the project.
This section aims to illustrate the significant design work that was put into the first of the two main project components: the metalanguage --- \gls{metaspec}. \\

\gls{metaspec} is the metalanguage for the \gls{absol} project, allowing the language designers to specify both the syntax and semantics of their DSL, as well as associated metadata, in a unified form. 
The final syntax for Metaspec is the result of significant design work, and consequentially the syntax discussed below has been through some evolution. \\

Metaspec is, in itself, a \gls{dsl}, and hence its design process was an interesting insight into how people might use the language to design their own \glspl{dsl}. 
The complete grammar for Metaspec can be found in Appendix~\ref{cha:the_metaspec_grammar}, and is written in standard \gls{ebnf} notation. 
The same notation will be used throughout this section of the document. 

\section{The Top-Level Definitions} % (fold)
\label{sec:the_top_level_definitions}
The top-level structure of a Metaspec file consists of a series of ordered top-level definitions.
The presence of these definitions emerged from the seemingly disparate nature of a number of the requirements placed upon the language.
They are as follows:
\begin{enumerate}
    \item The language name (Requirement~\reqref{req:LanguageMetadata})
    \item The language version (Requirement~\reqref{req:LanguageMetadata})
    \item Language feature imports (Requirement~\reqref{req:ExtensionMechanisms})
    \item Ground truths for the proof engine (Requirement~\reqref{req:Ground-TruthSemantics})
    \item The language itself (Requirement~\reqref{req:IntegratedSyntaxandSemanticSpecification})
\end{enumerate}

Tying these somewhat disparate areas together is the requirement for language definition files to read in an ``intuitive'' fashion (Requirement~\reqref{req:IntuitiveFileStructure}). 
This provided an initial sense for the ordering of the language definitions, as each block assisted in providing the contextual foundation for the language definition itself. \\

To this end, the decision was made to enforce the ordering of these in the language grammar itself (as seen in Listing~\ref{lst:top_level_metaspec_definition_blocks}, with the ordering as above.

\begin{listing}[!htb]
\begin{minted}[firstnumber=156]{text}
metaspec-defblock = 
    name-defblock, rule-termination-symbol, 
    version-defblock, rule-termination-symbol, 
    using-defblock, rule-termination-symbol, 
    truths-defblock, rule-termination-symbol, 
    language-defblock, rule-termination-symbol;
\end{minted}
\caption{Top-Level Metaspec Definition Blocks}
\label{lst:top_level_metaspec_definition_blocks}
\end{listing}

Placing the metadata fields first was a natural way to provide some initial context as to the language and version, and also assists with at-a-glance determining the language version in a \gls{vcs}.
As the extensions import list, the \mintinline{text}{using-defblock}, can contain dependencies of the other two blocks, it made a significant amount of sense to put this next, as it aids in establishing context for the next two blocks.\\

The ordering of the termination truths (\mintinline{text}{truths-defblock}) and the language definition itself (\mintinline{text}{language-defblock}) was similarly natural. 
While the truths are mainly depended upon by the termination proof mechanism (see \autoref{sub:metaverify_the_verification_engine}), they can also act as a guide to the language designer to indicate where they will be required to provide semantics for a language production. 
This quite clearly indicates that the termination truths should come before the language definition itself, as they provide additional context for the language designer.\\

While one might argue for the separation of these top-level elements, particularly the metadata, into multiple files, it seems far more natural to combine them all under one umbrella.
This means that each language definition is a self-contained unit with all the context required to understand it. 
While this single-file requirement does mean that definitions for large DSL could become unwieldy, taking a single-file approach significantly simplifies implementation.\\

Each of these top-level blocks have fairly distinct forms, and each was designed very carefully to aid the language designer's understanding of the file.

\subsection{The Metadata Blocks} % (fold)
\label{sub:the_metadata_blocks}
These two blocks, defining the language name and language version are almost trivial to define. 
However, care was taken not to enforce any particular naming or version scheme on the users of Metaspec.
\citet{raemaekers2014semantic} found that the semantics of a version string vary dramatically among software projects, and so it made little sense to impose a particular scheme in this case. \\

To this end, both the name and version strings for the language consist of a string of utf-8 characters.
They are parsed from the first non-whitespace character to the last that occurs before the rule termination symbol (\mintinline{text}{;}), as seen in Listing~\ref{lst:metadata_block_definitions}.
This ensures the maximum flexibility for the users of Metaspec. 

\begin{listing}[!htb]
\begin{minted}[firstnumber=156]{text}
name-defblock = "name", where-symbol, { utf-8-char }-,;
version-defblock = "version", where-symbol, { utf-8-char }-,;
\end{minted}
\caption{Metadata Block Definitions}
\label{lst:metadata_block_definitions}
\end{listing}

% subsection the_metadata_blocks (end)

\subsection{The Imports Definition Block} % (fold)
\label{sub:the_imports_definition_block}
In defining the style of imports for the language extension features (Requirement~\reqref{req:ExtensionMechanisms}) there were two main forms considered for use in Metaspec.\\

The first style of import declaration that was considered was what the project terms the `one-line-per-import' style. 
This is seen in myriad programming languages, and has each import as a single statement. 
Such a style can be seen in \autoref{lst:haskell_import_style} below.
This does have the benefit of possibly making complicated import statements easier to understand for the user, but in the context of Metaspec it doesn't belong from a stylistic standpoint.

\begin{listing}[!htb]
\begin{minted}[numbers=none]{haskell}
import qualified Data.Map                     as M
import           Data.Maybe                   (fromJust, isJust)
import qualified Data.Set                     as S
\end{minted}
\caption{Haskell Import Style}
\label{lst:haskell_import_style}
\end{listing}

The alternative style is to have imports declared as a comma-separated list. 
This is conceptually simpler, but has the downside of making it more complex to have variants of import statements (as can be seen in \autoref{lst:haskell_import_style}).
In the case of Metaspec, however, imports are simply keywords. 
As a result, the list format was chosen as it both simplifies the form of the statement, and is a better stylistic fit with the other elements of the metaspec syntax. \\

To that end, the \mintinline{text}{using} definition block is defined as a simple comma-separated list of feature keywords.
This definition can be seen in \autoref{lst:the_using_definition_block} below.

\begin{listing}[!htb]
\begin{minted}[firstnumber=175]{text}
using-defblock =
    "using",
    where-symbol,
    semantic-block-start,
    [ metaspec-feature, { semantic-list-delimiter, metaspec-feature }]
    semantic-block-end;
\end{minted}
\caption{The Using Definition Block}
\label{lst:the_using_definition_block}
\end{listing}

% subsection the_imports_definition_block (end)

\subsection{The Language Truths Block} % (fold)
\label{sub:the_language_truths_block}
Another key part of the setup to understanding a language defined in Metaspec is to understand the termination truths.
These are the trivial `base-cases' that can be assumed to terminate by the termination proof engine.\\

Initially, there was some consideration given to having these be automatic, with only those given by the language features available.
This was dismissed quickly, however, as it would remove significant flexibility from the language. 
All the usefulness of metaspec comes from allowing users to specify the semantics exactly as they want to, and having these base-cases automated went against that.
Thus, the truths are specified directly by the user, even for non-terminal symbols that they did not provide. 
This ensures the maximum amount of flexibility in the termination base-cases for the language.\\

The format of the truths themselves is taken almost directly from the language semantics (see Section~\ref{sec:specifying_the_language_semantics} for a discussion on the design of this form).
The main reason for this is to allow a kind of `visual pattern matching' on behalf of the language designer. 
The hope is that it easily allows them to find the termination truths for language elements visually, and hence aid in writing the language semantics.\\

As a result, truths are basically given as a list of evaluation blocks for non-terminals in the language, the semantics of each meaning ``this evaluation always terminates''. 
The exact syntactic specification for the truths definition block can be seen in \autoref{lst:the_truths_definition_block}. 

\begin{listing}[!htb]
\begin{minted}[firstnumber=182]{text}
truths-defblock =
    "truths",
    where-symbol, 
    semantic-block-start,
    semantic-evaluation,
    { ", ", semantic-evaluation },
    semantic-block-end;
\end{minted}
\caption{The Truths Definition Block}
\label{lst:the_truths_definition_block}
\end{listing}

% subsection the_language_truths_block (end)

\subsection{The Language Definition Block} % (fold)
\label{sub:the_language_definition_block}
The language definition block is simple, and required very little thought. 
It does nothing but act as a container for the productions of the language.
The form of these productions arises from the design work explored in Sections~\ref{sec:specifying_the_language_syntax} and~\ref{sec:specifying_the_language_semantics}, and is the result of combining the two as discussed in Section~\ref{sec:combining_syntax_and_semantics}.\\

The definition block is effectively a container for the language definitions, and has the syntactic form shown in \autoref{sub:the_language_definition_block} below.

\begin{listing}[!htb]
\begin{minted}[firstnumber=191]{text}
language-defblock =
    "language",
    where-symbol,
    semantic-block-start,
    language-definition,
    semantic-block-end;
\end{minted}
\caption{The Language Definition Block}
\label{lst:the_language_definition_block}
\end{listing}

% subsection the_language_definition_block (end)

\subsection{Comments in Metaspec} % (fold)
\label{sub:comments_in_metaspec}
During the consideration of the high-level structure of the metalanguage, it was important to determine the appearance of comments in the language.
Comments are crucial in any programming or specification language, and hence had to be included in Metaspec. \\

Requirement~\reqref{req:Comments} states that these comments need not have any semantic meaning in the language (contrasted with Python Docstrings, which are compiled with their functions \citep{python_docstrings}).
This meant that the commenting syntax could be very simple, allowing the parser to strip them directly, rather than be parsed.
To this end, it is stated in the language specification that all comments are removed in a preprocessing step. \\

The choice of \mintinline{text}{line-comment-start-symbol}, \mintinline{text}{block-comment-start-symbol} and \mintinline{text}{block-comment-end-symbol} was effectively arbitrary.
The block comment style was taken directly from \gls{ebnf}, as specified in \cite{standard1996ebnf}, but there was no line comment style to go on as EBNF does not provide one. 
As a result, comments themselves are defined as follows, with both a line comment and block comment style provided:
\begin{listing}[!htb]
\begin{minted}[firstnumber=156]{text}
metaspec-comment =
    line-comment-start-symbol, { utf-8-char }, eol-symbol |
    block-comment-start-symbol, { utf-8-char }, block-comment-end-symbol;
\end{minted}
\caption{Comments in Metaspec}
\label{lst:comments_in_metaspec}
\end{listing}

While block comments can be used on a single line they contribute additional visual noise with their end-delimiter, and hence it seemed appropriate to include a line-comment style as well.
With no prior reference to go on, the C-style line comment operator was chosen as it has been adopted my many programming languages.
As a result, the comment symbols are defined in Metaspec as follows.
\begin{listing}[!htb]
\begin{minted}[firstnumber=156]{text}
line-comment-start-symbol = "//";
block-comment-start-symbol = "(*";
block-comment-end-symbol = "*)";
\end{minted}
\caption{Metaspec Comment Symbols}
\label{lst:metaspec_comment_symbols}
\end{listing}

% subsection comments_in_metaspec (end)

% section the_top_level_definitions (end)

\section{Specifying the Language Syntax} % (fold)
\label{sec:specifying_the_language_syntax}
% How does the syntax interact with the semantics?
Metaspec, as a hybrid metalanguage, needed to have the ability to specify the syntax of the language it describes. 
As discussed in \autoref{sub:choosing_a_syntactic_form}, the predominant notation for specifying language syntax that is used today is \gls{ebnf}.
Metaspec adapts the \gls{ebnf} syntactic specification language, taking the set of definitions for specifying productions almost directly.
This decision was made for a few main reasons:
\begin{itemize}
    \item \textbf{Flexibility of Syntactic Definition:} \gls{ebnf} is already capable of representing both context-free and context-sensitive language grammars.
    Furthermore, \gls{ebnf} places \textit{no restriction} on the format of the terminals of the language, which is important for allowing \glspl{dsl} to get as close to the domain notation as possible. 
    Any form of syntactic specification that would be devised as part of this project would likely be flawed in some way, shape or form, or not be as flexible as the extant notation in the form of \gls{ebnf}. 
    \item \textbf{User Familiarity:} Due to the prevalent use of \gls{ebnf} in the programming language community and its standardised nature, it was felt that providing a syntactic notation that the users would be familiar with would ease use of Metaspec. 
    \item \textbf{Correctness:} Adaptation of a pre-defined and well-studied metasyntax notation is far less likely to result in errors than implementing a new notation from scratch.
\end{itemize}

That is not to suggest, however, that the \gls{ebnf} productions were taken entirely as-is. 
As part of adapting the long-standing metasyntactic notation, there were a few changes made to its syntax.
These changes were in aid of providing better operation with other portions of the toolchain, as well as fixing some of the less-intuitive elements of the \gls{ebnf} grammar. 

\subsection{Adapting the Form of Non-Terminals} % (fold)
\label{sub:adapting_the_form_of_non_terminals}
In standard \gls{ebnf}, a non-terminal is represented (approximately) by any string of textual characters, including spaces. 
This means that they can be difficult to visually identify at first glance for a reader of the grammar, and also that concatenation of terminals and non-terminals has to be defined using an operator (\mintinline{text}{,}).
This was found to be quite inelegant, and hence the metasyntactic notation used by \gls{metaspec} adapts this. \\

Instead of using an explicit term concatenation operator, Metaspec chooses to delimit the non-terminal symbols of the language, using the symbols shown in \autoref{lst:non_terminals_in_metaspec}.
The use of the distinctive angle brackets both helps to visually distinguish the non-terminals of the language, and syntactically delimit the non-terminals.
As a result, there is no need for an explicit concatenation operator, reducing visual noise in the Metaspec definitions.

\begin{listing}[!htb]
\begin{minted}[numbers=none, fontsize=\blockfont]{text}
non-terminal-start = "<";
non-terminal-end = ">";
non-terminal-identifier = 
    textual-glyph, { textual-glyph | natural-number | "-" | "_"};
non-terminal = non-terminal-start, non-terminal-identifier, non-terminal-end;
\end{minted}
\caption{Non-Terminals in Metaspec}
\label{lst:non_terminals_in_metaspec}
\end{listing}

Further visual distinction was provided through a restriction of the kinds of symbols used to name a non-terminal.
As \autoref{lst:non_terminals_in_metaspec} shows, the names of non-terminals are textual characters, numbers, hyphens and underscores. \\

This altered syntax for non-terminals also brings benefits when it comes to parsing the semantics. 
The semantics of a language defined in metaspec may contain variable identifiers (as discussed in REFERENCE), but also require access to elements of the syntax (see \autoref{sub:accessing_syntax_from_the_semantics}).
These identifiers look much like standard \gls{ebnf} non-terminal names, and so the alterations made for \gls{metaspec} provide some additional ability to disambiguate at the parser level. 

% subsection adapting_the_form_of_non_terminals (end)

\subsection{Specification of the Language Start Rule} % (fold)
\label{sub:specification_of_the_language_start_rule}
One of the things that \gls{ebnf} lacks is an explicit method of representing the `start rule' of a grammar. 
The start rule (or start symbol) in any grammar is the place from which a parse either \textit{starts}, or where a parse will \textit{end} \cite{slonneger1995formal}.
For automated tools working with syntax, this is clearly important to define. \\

Beyond the requirement to use the grammar portion of the specification for parser generation, the notion of a start symbol is also useful when it comes to verifying the language semantics.
The start symbol can act as the point from which the verification starts, and thus brings the ability to check for other criteria on the language (unused productions, syntax-only productions, etc). \\

While it could be simple as to provide a metadata field specifying the name of the start symbol for the \gls{dsl}, that seemed inelegant. 
Instead, a special type of non-terminal declaration was added; where non-terminals are enclosed in angle brackets --- \mintinline{text}{<nt>} --- the start symbol is enclosed in a set of double angle brackets: \mintinline{text}{<<start>>}. 
This means that it can be trivially identified during the parsing stage, and used as needed. 
The start symbol and start rule (the production associated with the start symbol) are hence defined as in \autoref{lst:the_start_symbol_and_start_rule_for_metaspec}.

\begin{listing}[!htb]
\begin{minted}[numbers=none, fontsize=\blockfont]{text}
start-symbol-start = "<<";
start-symbol-end = ">>";
start-symbol = start-symbol-start, non-terminal-identifier, start-symbol-end;
start-rule = start-symbol, defining-symbol, language-rule-body;
\end{minted}
\caption{The Start Symbol and Start Rule for Metaspec}
\label{lst:the_start_symbol_and_start_rule_for_metaspec}
\end{listing}

% subsection specification_of_the_language_start_rule (end)

\subsection{Removal of the Empty Syntax} % (fold)
\label{sub:removal_of_the_empty_syntax}
As part of its syntactic specification, \gls{ebnf} provides the notion of an `empty syntax'.
This is a piece of syntax that can exist when there is no actual syntax in a place where some would otherwise be expected. 
As one might imagine, this poses some significant difficulties to parsers, and hence has been removed in Metaspec for practical purposes. \\

There were some concerns that this would compromise the expressive power of the metasyntactic notation used in Metaspec, but careful examination of the \gls{ebnf} standard only showed one case where the empty syntax was used. 
\gls{ebnf} provides facilities for a syntactic exception, and this can be combined with the repetition notation and empty syntax to ensure that at least one repetition exists: \mintinline{text}{prod = {some-term}-;}.
While not allowing this notation does compromise the \textit{conciseness} of certain productions in Metaspec, it does not prevent representation of such terms --- the above can be represented as \mintinline{text}{prod = some-term, {some-term};}.

% subsection removal_of_the_empty_syntax (end)

\subsection{Altering the Assignment Operator} % (fold)
\label{sub:altering_the_assignment_operator}
One of the other changes that \gls{metaspec} makes to the standard \gls{ebnf} notation for language productions is to alter the assignment symbol.
Where \gls{ebnf} uses \mintinline{text}{=}, \gls{metaspec} uses \mintinline{text}{::=}.\\

The reasoning behind this is similar to some of the reasoning for redefining the form of non-terminals: it provides a more significant visual cue to the language designer. 
As the \mintinline{text}{=} symbol often appears in programming languages as some form of assignment, it is not unlikely that a \gls{dsl} designer would want to use it as such.
This means that it would have to occur in a production as \mintinline{text}{"="}. 
The \mintinline{text}{::=} as chosen by \gls{metaspec} is far less common, and this should help to improve the readability of the syntactic notation at a glance. 

% subsection altering_the_assignment_operator (end)

\subsection{The Final Language Syntax} % (fold)
\label{sub:the_final_language_syntax}
As a result, the final syntactic metalanguage for \gls{metaspec} is very similar to that of \gls{ebnf}.
Including the changes mentioned in the preceding sections, it is capable of the specification of flexible syntactic rules containing the following syntactic features:
\begin{itemize}
    \item Optionality of syntax.
    \item Grouping of syntax.
    \item Alternation of syntax terms.
    \item Repetition, in both the multiplicative (\mintinline{text}{3 * <nt>}) and repeated-group (\mintinline{text}) forms, for syntax terms.
    \item Syntactic exceptions (\mintinline{text}{<nt1> - <nt2>}).
    \item Arbitrary non-terminals and terminals.
\end{itemize}

Overall, this provides an excellent metasyntactic foundation for the metalanguage, and lets users describe a wide-set of possible languages.
Hence, it allows \gls{dsl} designers to get as close to their intended domain-specific notation as possible. 

% subsection the_final_language_syntax (end)

% section specifying_the_language_syntax (end)

\section{Specifying the Language Semantics} % (fold)
\label{sec:specifying_the_language_semantics}
% Semantics and the unspecified cases. 
% Talking about typing here, rather than at a syntactic level
% Access into the syntax.
% Issues with the final form of the semantics
% Ability to omit semantics in certain cases -> terminals?
% Need to talk about the kinds of semantics here: env access, env store, normal and special.
% How does typing work at the semantic level?
% Determining which productions have semantics? 
% Worked example of the transformation

While \gls{metaspec} now had a way of specifying language syntax, its hybrid nature meant that it also needed a way to specify the language semantics. 
As discussed in \autoref{sub:choosing_a_semantic_form}, the basic semantic descriptions in metaspec are based on big-step operational semantics (see Subsection~\ref{ssub:operational_semantics}). 
However, requirements dictated that the language support other forms of semantics, including environment accesses (Requirement~\reqref{req:EnvironmentAccesses}) and special-case semantic rules (Requirement~\reqref{req:SpecifyLanguageSemantics}).\\

As a result, the metalanguage has four different kinds of semantic rules, each of which is explored below:
\begin{itemize}
    \item \textbf{Standard Semantics:} These are an adaptation of the big-step operational semantics form.
    They provide restrictions and evaluation control, and are the main method for specifying language semantics.
    \item \textbf{Special-Case Semantics:} These special rules have a distinct syntactic form, and provide semantics that cannot be proved by the standard mechanism.
    \item \textbf{Environment Accesses:} Accesses to items stored in the global environment, akin to a key-value store.
    \item \textbf{Environment Stores:} The ability to store values in the global environment under a key. 
\end{itemize}

The following sections aim to explain the design process that the syntax for the semantic forms underwent, broadly grouping them into the categories of ``User-Defined Semantics'' (\autoref{sub:user_defined_semantics}), which took the most design effort, and ``Special-Case Semantics'' (\autoref{sub:special_case_semantics}).

\subsection{Accessing Syntax from the Semantics} % (fold)
\label{sub:accessing_syntax_from_the_semantics}
One of the first things to notice about the form of a big-step operational semantics rule (see \autoref{eq:example_of_big_step_operational_semantics_if} for an example) is that the sub-evaluations (those above the evaluation line) usually contain sub-terms of the main expression (below the line).
It should be noted that, in general, these sub-evaluations do not \textit{have} to be purely sub-terms of the main expression. 

\begin{equation}
    [\text{if}] : \frac{\langle S_1, s \rangle \to s'}{\langle \text{if } b \text{ then } S_1 \text{ else } S_2, s\rangle \to s'} \text{ if } \mathbb{B}\llbracket b \rrbracket s = \text{true}
    \label{eq:example_of_big_step_operational_semantics_if}
\end{equation}

In a basic sense, every piece of syntax in Metaspec has some kind of semantics associated with it.
As a result, the syntax often defines the `main expression' of the semantic rule.
Given this fact, \gls{metaspec} needs a way for the semantic rules to `access' the syntax, that is: refer directly to portions of the syntax of a given production. \\

As has been established in \autoref{sec:specifying_the_language_syntax}, \gls{metaspec} supports a flexible notation for defining productions of the DSL. 
However, the important thing to recognise is that the only terms that one would need to refer to in the semantics are the non-terminals of the language. 
Any syntactic production decomposes to a set of terminals and non-terminals, and as terminals are fixed they can be explicitly written into the semantics where required. \\

\begin{listing}[!htb]
\begin{minted}[numbers=none]{text}
<example> ::= <foo> "(" <bar> "," <bar> ")"
\end{minted}
\caption{An Example Production for Syntax Access}
\label{lst:an_example_production_for_syntax_access}
\end{listing}

Consider an example production of the form seen in \autoref{lst:an_example_production_for_syntax_access}. 
The most intuitive way to access the syntax is to refer directly to the non-terminals directly.
This means that \mintinline{text}{<foo>} in the semantics would refer to the first instance of the non-terminal in the syntax, and similarly for \mintinline{text}{<bar>}.
However, this positional argument-based nature is a non-intuitive way to access the syntax, as it forces the \gls{dsl} designer to formulate their semantics in the order of the syntax. 
This would impose significant restrictions in the case of semantic special syntax (which have fixed argument positions), and was hence seen as undesirable.\\

As the above situation was deemed unsatisfactory, the final access form for \gls{metaspec} takes inspiration from array accesses in other programming languages, using the square-bracket notation (\mintinline{text}{[]}) for familiarity.
Each access into the syntax must hence specify the (zero-indexed) \textit{position} of the specific non-terminal to which it refers: \mintinline{text}{<bar>[1]} refers to the second instance of the non-terminal, for example.
These positions must hence be checked at language verification time (see \autoref{sub:user_defined_semantic_form_verification} for details).\\

This provides an intuitive way to access syntactic elements within the semantics, without restricting the form in which the semantics can be written. 

% subsection accessing_syntax_from_the_semantics (end)

\subsection{Semantic Typing} % (fold)
\label{sub:semantic_typing}
The semantics are the location where the notion of data types was designed into \gls{metaspec} with each semantic rule carrying some type information. 
This was imposed upon the language by Requirement~\reqref{req:SemanticTyping} for two main reasons:
\begin{itemize}
    \item \textbf{Conciseness:} A common feature of \glspl{dsl} is that they have a clear and concise syntax that matches the domain representation of knowledge as closely as possible (\autoref{sec:domain_specific_languages}). 
    As a result, having typing imposed at the syntactic level would add unnecessary visual clutter to the \gls{dsl} syntax. 
    \item \textbf{Clarity of Language Implementation:} It would likely introduce significant complexity to the form of the language semantics to introduce a method for providing syntactic typing to the language designer.
    Furthermore, the provision of such a feature would likely require alterations to the verification algorithm, and hence potentially impact correctness. 
\end{itemize}

To this end, it was decided to impose types at the semantic level.
In this case, each semantic expression (where the type could not be inferred) is required to have a result type, where the types may be one of the types allowed by the current language context. 
These types are expressions as part of the language grammar, and are contextually checked to be in scope at parse time (see \autoref{sub:metaparse_ast_generation}).\\

For information on the checking of types in the language specification, please see \autoref{sub:type_checking} on \autopageref{sub:type_checking}. 

% subsection semantic_typing (end)

\subsection{User-Defined Semantics} % (fold)
\label{sub:user_defined_semantics}
% Have subsubs in here to break it up, there's a lot to talk about
% Analysing the form of the semantics, have examples of non-representable rules? Or is that better suited for the verification step?
% Talk about the omission of the 'bottom' row of the rule as it is effecitvely the syntax production itself. 

While \textit{all} semantics for a language designed in \gls{metaspec} are technically user-defined, this section focuses on the examination of the main form of the semantics --- those based upon big-step operational semantics. 
These semantic rules form core of languages written in \gls{metaspec}, as they provide a flexible way to specify the behaviour of the language. 

\subsubsection{Examining the form of Big-Step Operational Semantics} % (fold)
\label{ssub:examining_the_form_of_big_step_operational_semantics}
\begin{equation}
    [\text{if}] : \frac{\langle S_1, s \rangle \to s'}{\langle \text{if } b \text{ then } S_1 \text{ else } S_2, s\rangle \to s'} \text{ if } \mathbb{B}\llbracket b \rrbracket s = \textit{ true} 
    \label{eq:a_basic_big_step_operational_semantics_rule}
\end{equation}

A big-step operational rule such as the one seen in \autoref{eq:a_basic_big_step_operational_semantics_rule} can be broken down into a number of components:
\begin{itemize}
    \item \textbf{The Primary Expression:} This is the portion of the syntax to be evaluated below the line, to the left of the evaluation arrow.
    Here, this is $\langle \text{if } b \text{ then } S_1 \text{ else } S_2$.
    \item \textbf{The Evaluation Result:} The result of evaluating the primary expression, this is found to the right of the evaluation arrow below the line.
    Here it is $s'$. 
    \item \textbf{The Restriction:} A restriction placed on the evaluation of this rule, such that the rule is only evaluated if the restriction holds. 
    Here, this is $\text{ if } \mathbb{B}\llbracket b \rrbracket s = \textit{ true}$.
    \item \textbf{The Sub-Evaluation:} A specification of how each component of the primary expression is evaluated, located above the line.
    Here, this is $\langle S_1, s \rangle \to s'$. 
\end{itemize}

\begin{equation}
    [+] : \frac{\langle A_1, s_0\rangle \to \langle n_1, s_1\rangle \;\;\;\; \langle A_2, s_1 \rangle \to \langle n_2, s_2\rangle}{\langle A_1 + A_2, s_0 \rangle \to \langle n, s_2 \rangle}\;\; n = n_1 + n_2
    \label{eq:an_alternate_big_step_operational_semantics_rule}
\end{equation}

There are further components that are not indicated in such a rule. 
Consider the rule shown in \autoref{eq:an_alternate_big_step_operational_semantics_rule}.
This exhibits an additional set of components as follows:
\begin{itemize}
    \item \textbf{The Semantic Evaluation:} This is a portion of the rule that specifies how the result of the semantics is calculated, located to the right of the evaluation rule. 
    Here, it is $n = n_1 + n_2$.
    \item \textbf{Multiple Sub-Evaluations:} This rule also illustrates that it is possible to have multiple sub-evaluations taking place as part of a given semantic rule. 
    The importance of this is the \textit{order} in which they are executed, as this may impact the results of computations. 
\end{itemize}

Each of these components must be accounted for in the semantic form that represents these semantic rules, and have been incorporated into the metasyntax for the semantics as discussed in \autoref{ssub:the_final_form_of_the_user_defined_semantics}.

% subsubsection examining_the_form_of_big_step_operational_semantics (end)

\subsubsection{Transforming Big-Step Operational Semantics} % (fold)
\label{ssub:transforming_big_step_operational_semantics}
% Adding the types

Having identified the key portions of the big-step semantic rules, the next task was to represent these components effectively in a textual format. 
The working assumption for designing this syntactic representation is that these \gls{metaspec} language specifications would be written with nothing but a text editor, as specified by Requirement~\reqref{req:TextEditorReady}.
To this end, the notation should be simplistic and not require complex formatting (thus also simplifying the parsing process). \\

In one sense, semantics represent a flow of information or a \textit{pipeline} of evaluation. 
Semantically, each portion of the evaluation is predicated on the next portion of the evaluation, so they can be separated by what is effectively a \textit{where} clause. 
For the purposes of these semantic rules, the \mintinline{text}{:} operator was chosen to represent this.
To this end, the portions of the semantic rules discussed in \autoref{ssub:examining_the_form_of_big_step_operational_semantics} can be represented in order as follows:
\begin{enumerate}
    \item \textbf{The Output Variable:} A variable identifier and associated type (as discussed in \autoref{sub:semantic_typing}) as the leftmost portion of the rule: \mintinline{text}{<type> <identifier>}.
    This variable defines the final result of the semantic evaluations.
    \item \textbf{The Semantic Operations:} A list of operations that define how the results of the sub-evaluations are combined: \mintinline{text}{(<eval> {"," <eval>})}.
    These operations are defined in terms of an allowed set of semantic operations defined by metaspec. % do I want to talk about these?
    \item \textbf{The Semantic Restrictions:} A list of restrictions that constrain the circumstances under which this semantic rule can operate.
    These are restrictions on the results of the sub-evaluations that act as boolean conditions, and hence look like a list of standard conditions: \mintinline{text}{([<key> <op> <key>] {"," <key> <op> <key>})}.
    Restrictions do not have to exist, however, and hence it is perfectly valid to have an empty restriction block.
    \item \textbf{The Semantic Evaluations:} These define evaluations on sub-terms of the syntax, with each sub-term addressing non-terminals as discussed in \autoref{sub:accessing_syntax_from_the_semantics}. 
    Each of these terms must define a result type and variable for use in the semantic operations section: \mintinline{text}{<type> <identifier> "<=" <syntax-access>}.
    As there can be multiple of these, they are given as a list.
\end{enumerate}

It is apparent at this point that the \textit{primary expression} identified in appears to have been ignored.
While it was initially included as a component in the semantic rule it was quickly noticed that the syntax of the language production itself suffices as the primary expression.
It was hence omitted from the semantics directly, with the syntax accesses sufficing to combine the two. 
The final basic form of the syntax for these semantic rules can be seen in \autoref{lst:the_semantic_evaluation_rule_grammar}.

\begin{listing}[!htb]
\begin{minted}[firstnumber=314]{text}
semantic-evaluation-rule = 
    semantic-type,
    semantic-identifier,
    where-symbol,
    semantic-operation-list,
    semantic-restiction-list,
    where-symbol, 
    semantic-eveluation-list;
\end{minted}
\caption{The Semantic Evaluation Rule Grammar}
\label{lst:the_semantic_evaluation_rule_grammar}
\end{listing}

Combining these elements provides the form for a single evaluation rule.
A single rule is, however, not sufficient to represent a defined set of semantics for a given production. 
As a result, the grammar was expanded to accommodate a `semantic alternation', with multiple semantic rules separated by \mintinline{text}{|}, appearing as a logical disjunction.
This notion perfectly represents the concept, as only one semantic rule will evaluated based upon the evaluation of the guards. 
This is represented in the concept of the evaluation list, as defined in \autoref{lst:the_semantic_evaluation_rule_list}.

\begin{listing}[!htb]
\begin{minted}[firstnumber=310]
semantic-evaluation-rule-list = 
    semantic-evaluation-rule,
    { semantic-disjunction, semantic-evaluation-rule };
\end{minted}
\caption{The Semantic Evaluation Rule List}
\label{lst:the_semantic_evaluation_rule_list}
\end{listing}

Multiple portions of this syntax are required to match a certain form for verification to be able to occur. 
These restrictions are discussed in detail in \autoref{sub:user_defined_semantic_form_verification}, and are crucial to the ability of \gls{absol} to verify the language semantics.

% subsubsection transforming_big_step_operational_semantics (end)

\subsection{An Example of User-Defined Semantics} % (fold)
\label{sub:an_example_of_user_defined_semantics}
Having established the form of the semantics as above, it is now possible to formulate an example of the transformation from standard big-step operational semantics to the \gls{metaspec} syntax for the same.
Consider the rules for exponentiation as described in Equations~\ref{eq:big_step_exponentiation_special_case} and~\ref{eq:big_step_exponentiation_base_case} below.
These rules have a special case for illustrative purposes. 

\begin{align}
    [\string^] &: \frac{\la A_2, s_0 \ra \to \la n_2, s_1 \ra}
    {\la A_1 \string^ A_2, s_0 \ra \to \la n, s_1 \ra} n = 1 \text{ where } n_2 == 0 \label{eq:big_step_exponentiation_special_case} \\
    [\string^] &: \frac{\la A_2, s_0 \ra \to \la n_2, s_1 \ra \la A_1, s_1 \ra \to \la n_1, s_2 \ra}{\la A_1 \string^ A_2, s_0 \ra \to \la n, s_2 \ra} n = n_1^{n_2} \label{eq:big_step_exponentiation_base_case}
\end{align}

Working through the special-case rule in \autoref{eq:big_step_exponentiation_special_case} it is possible to compose the equivalent \gls{metaspec} expression as follows (assuming types and that $A_1, A_2$ are the same non-terminal): 
\begin{enumerate}
    \item \textbf{Determine the Output Variable:} In this case it is clear that it is $n$. 
    The expression is hence initially:
    \begin{minted}[xleftmargin=2cm, numbers=none]{text}
        number n : 
    \end{minted}
    \item \textbf{Determine the Evaluation Rule:} This is also clear as $n = 1$, in the presence of a restriction.
    The Metaspec rule is hence:
    \begin{minted}[xleftmargin=2cm, numbers=none]{text}
        number n : {n = 1}
    \end{minted}
    \item \textbf{Determine any Restrictions:} This rule has a restriction, and so this can also be found and added. 
    It is $n_2 == 0$, making the Metaspec rule into:
    \begin{minted}[xleftmargin=2cm, numbers=none]{text}
        number n : {n = 1}(n2 == 0) : 
    \end{minted}
    \item \textbf{Write the Sub-Evaluations:} This rule only has a single sub-evaluation as follows: $\la A_2, s_0 \ra \to \la n_2, s_1 \ra$.
    This is transformed into Metaspec, giving the resultant rule (as it is the \textit{second} instance of that `non-terminal' in the rule):
    \begin{minted}[xleftmargin=2cm, numbers=none]{text}
        number n : {n = 1}(n2 == 0) : {number n2 <= <a-nt>[1]}
    \end{minted}
\end{enumerate}

Performing the same process for the generic exponentiation rule in \autoref{eq:big_step_exponentiation_base_case} and combining using an alternation (as discussed in \autoref{ssub:transforming_big_step_operational_semantics}), the final semantic expression can be found.
This is seen in \autoref{lst:metaspec_semantic_rules_for_exponentiation}, and can be assumed to be linked to a syntactic rule of the form \mintinline{text}{<exp> ::= <a-nt> "^" <a-nt>;} (or similar).

\begin{listing}[!htb]
\begin{minted}[numbers=none]{text}
number n : {n = 1}(n2 == 0) : {number n2 <= <a-nt>[1]} |
number n : {n = n1 ^ n2}() : 
    {number n2 <= <a-nt[1]}, {number n1 <= <a-nt>[0]}
\end{minted}
\caption{Metaspec Semantic Rules for Exponentiation}
\label{lst:metaspec_semantic_rules_for_exponentiation}
\end{listing}

% the user is able to decide which evaluations to make, and in which order

% subsection an_example_of_user_defined_semantics (end)

\subsubsection{The Final Form of the User-Defined Semantics} % (fold)
\label{ssub:the_final_form_of_the_user_defined_semantics}
At this point, the syntax for defining language semantics is concise and clear.
However, it has not yet been established exactly how one of these operations is evaluated.
Much like the big-step operational semantics from which they are derived, the evaluation semantics are crucial to understanding these expressions. \\

As a result, it is important to formalise the evaluation semantics of the metaspec expressions. 
A metaspec semantic rule is evaluated according to the algorithm given in \autoref{alg:metaspec_semantic_evaluation_algorithm}, taking an alternation of metaspec semantic rules as input.
Assume that \textbf{evaluate} is a function that evaluates the provided evaluation or computation, and that \textbf{checkRestriction} is a function that returns true if the restrictions are all true.
With formalised execution order, this ensures that the \gls{dsl} creators are provided with full control over the execution semantics of their language.

\begin{algorithm}
\begin{algorithmic}
\State rules $\gets$ the alternation of semantic rules
\State
\ForAll{rule $\in$ rules}
    \State evals $\gets$ sub-evaluations in rule
    \State computations $\gets$ semantic evaluations in rule
    \State restrictions $\gets$ restrictions in rule
    \State 
    \ForAll {eval $\in$ evals}
        \State \textbf{evaluate}(eval)
    \EndFor
    \State
    \If{\textbf{checkRestrictions}(restrictions)}
        \ForAll {computation $\in$ computations}
            \State \textbf{evaluate}(computation)
        \EndFor
        \State
        \State output the result
        \State \textbf{terminate}
    \Else
        \State \textbf{continue}
    \EndIf
\EndFor
\end{algorithmic}
\caption{Metaspec Semantic Evaluation Algorithm}
\label{alg:metaspec_semantic_evaluation_algorithm}
\end{algorithm}

% subsubsection the_final_form_of_the_user_defined_semantics (end)

% subsection user_defined_semantics (end)

\subsection{Special-Case Semantics} % (fold)
\label{sub:special_case_semantics}
In addition to the user-defined semantics based upon the big-step operational semantics, Requirements~\reqref{req:EnvironmentAccesses} and~\reqref{req:ExtensionMechanisms} mean that other kinds of semantics must be defined.
These include the special-case semantics for providing language features that cannot be proved by the termination mechanism, and semantic rules for environment accesses and stores.
The below sections explore the design of these additional semantic features.

\subsubsection{Semantic Special Syntax} % (fold)
\label{ssub:semantic_special_syntax}
% Look like library functions as the best analogy for them.
% Why these can occur in the user-defined semantics (to allow complex operations). 

The semantic \glspl{ssr} are a method for \gls{metaspec} to provide additional, useful language features that cannot be proven by the usual termination proof mechanism. 
While they might technically be able to be expressed in big-step operational semantics, the restrictions placed on the standard semantic form (\autoref{sub:user_defined_semantics}) mean that they cannot be expressed in metaspec.
This includes features like data traversal and function calls. \\

The syntactic design for these pieces of special syntax was inspired by the notion of `library functions' in other languages.
Much like Haskell has \mintinline{haskell}{fmap fn s}, these pieces of special syntax provide advanced functionality to the user for little effort on their part.
As a result, the syntax was designed to look like a function call: \mintinline{text}{<ssr> "("[<arg>] {"," <arg>} ")"}.\\

\begin{listing}[!htb]
\begin{minted}[firstnumber=303, fontsize=\blockfont]{text}
special-syntax-rule = 
    semantic-special-syntax,
    special-syntax-start,
    [ syntax-access-block | environment-access-rule ],
    { semantic-list-delimiter, (syntax-access-block|environment-access-rule) },
    special-syntax-end;
\end{minted}
\caption{Semantic Special Syntax}
\label{lst:semantic_special_syntax}
\end{listing}

Syntactically, they consist of a language keyword and an argument list, and the exact semantics of each is defined in \autoref{sec:special_language_features}.
The argument list may consist of syntax accesses (see \autoref{sub:accessing_syntax_from_the_semantics} or environment accesses (see \autoref{ssub:environment_access_rules}), the latter of which allow them to retrieve values from the environment.
They are described by the \gls{ebnf} expression in \autoref{lst:semantic_special_syntax} and are subject to verification as described in \autoref{ssub:verification_of_special_syntax_rules}.\\

These \glspl{ssr} can occur inside the user-defined semantics (\autoref{sub:user_defined_semantics}) as they can provide useful functionality for defining the semantics of more complex expressions.
In allowing them to be nested in such expressions, it removes the need to indirect through another non-terminal where these might occur. 
This brings no additional complexity to the proof mechanism, but makes removes the potential for complexity in the language specification.
This is likely to reduce the potential for bugs somewhat. 

% subsubsection semantic_special_syntax (end)

\subsubsection{Environment Input Rules} % (fold)
\label{ssub:environment_input_rules}
Environment input rules are intended to provide mechanisms for \gls{dsl} designers to store and retrieve values in the environment. 
The environment was conceptualised as an unscoped key-value store to allow for variable definitions, function definitions and any other use that the user can come up with. \\

Syntactically, the environment was represented by a special reserved symbol in \gls{metaspec}, \mintinline{text}{e}. 
The notion behind this was to ensure that all environment stores and accesses could be found at a glance by the user, aiding their understanding of any stateful behaviour in the language semantics. 
Unfortunately this has turned out to not be entirely true, as some special syntax calls define elements in the environment without direct use of the symbol.\\

Beyond that, the store into the environment uses a newly introduced operator \mintinline{text}{<--}. 
This operator was designed to visually mimic the \mintinline{haskell}{<-} monadic extraction operator, and aims to imply getting something out of the syntax and storing it somewhere else. 
Stores in the environment operate in a key-value function, with the key and value separated by the \mintinline{text}{:} (defining) symbol.
In short: \mintinline{text}{e <-- <key>[n] : <values...>}, for some arbitrary non-terminals \mintinline{text}{<key>} and \mintinline{text}{<values>}.
Environment input rules are hence defined by the \gls{ebnf} expression given in \autoref{lst:environment_input_rules}.

\begin{listing}[!htb]
\begin{minted}[firstnumber=279]{text}
environment-input-rule =
    semantic-type, 
    semantic-environment-symbol,
    semantic-environment-input-symbol,
    syntax-access-block, (* key *)
    environment-defines-symbol,
    syntax-access-list;
\end{minted}
\caption{Environment Input Rules}
\label{lst:environment_input_rules}
\end{listing}

% subsubsection environment_input_rules (end)

\subsubsection{Environment Access Rules} % (fold)
\label{ssub:environment_access_rules}
Much like for environment inputs, environment accesses also took syntactic inspiration from features of common programming languages. 
Initial syntax for the environment accesses treated it much like a dictionary, where a key was given to retrieve a value: \mintinline{text}{e[<key>]}.
However, it is possible that certain structures in the environment have properties of their own, and it was immediately apparent that it could be syntactically awkward to access these.\\

To remove this awkwardness, the final syntax in \gls{metaspec} defines \mintinline{text}{.} as the environment access symbol, taking inspiration from object property access in most C++ derived languages.
The idea behind this is that it can be chained in a visually pleasing way, accessing properties in the environment and the environment's children: \mintinline{text}{e.<nt-1>.<nt-2> ...}
These accesses are based on the syntax of the triggering rule, and hence the \gls{ebnf} expression for such productions is as in \autoref{lst:environment_access_rules}.

\begin{listing}[!htb]
\begin{minted}[firstnumber=297]{text}
environment-access-rule = 
    semantic-environment-symbol,
    environment-access-symbol,
    syntax-access-block,
    { environment-access-symbol, syntax-access-block }; 
\end{minted}
\caption{Environment Access Rules}
\label{lst:environment_access_rules}
\end{listing}

% subsubsection environment_access_rules (end)

% subsection special_case_semantics (end)

% section specifying_the_language_semantics (end)

\section{Combining Syntax and Semantics} % (fold)
\label{sec:combining_syntax_and_semantics}
With a robust metasyntactic format for both syntax and semantics in \gls{metaspec}, the remaining design challenge of the metalanguage was to combine the two. 
This combination is crucial to the design of the language, as stated in Requirement~\reqref{req:IntegratedSyntaxandSemanticSpecification}.
The combination of the two had to admit all of the four categories of semantic rule (see \autoref{sec:specifying_the_language_semantics}), and associate them with a syntactic production, and hence the choice of both \textit{syntax} and the \textit{combination point} was important. 

\subsection{Choosing the Combination Syntax} % (fold)
\label{sub:choosing_the_combination_syntax}
As the marriage of syntax and semantic specification is at the crux of \gls{metaspec} itself, the syntax for combining the two metasyntactic notations had some significance. \\

The initial inspiration for the syntax came from the notion of logical implication: that the syntax implied a given behaviour. 
There were concerns, however, about this being too literal an idea, and hence rather than \mintinline{text}{->}, the extended arrow operator (\mintinline{text}{-->}) was used. 
This was coupled with a block syntax, wrapping all the semantics for a given production up in braces (\mintinline{text}{{}}) to act as a visual delimiter.
The resultant \gls{ebnf} production can be seen in \autoref{lst:language_rule_semantics}. 

\begin{listing}[!htb]
\begin{minted}[firstnumber=267]{text}
language-rule-semantics = 
    semantic-behaves-as,
    semantic-block-start,
    semantic-rule,
    semantic-block-end;
\end{minted}
\caption{Language Rule Semantics}
\label{lst:language_rule_semantics}
\end{listing}

% subsection choosing_the_combination_syntax (end)

\subsection{Determining the Combination Point} % (fold)
\label{sub:determining_the_combination_point}
Just as important as the combination syntax was the choice of where to combine the syntactic and semantic descriptions.\\

The initial choice of the connection point went somewhat awry, with the grammar specifying that the semantics came directly after the full syntactic specification: \mintinline{text}{production, "-->", "{", semantics, "}"}.
For some fairly obvious reasons, this did not work.
All of these boil down to one essential fact: syntactic productions contain alternations:
\begin{itemize}
    \item It is trivial to imagine a production where a non-terminal \mintinline{text}{<a>} occurs in the first alternation and is referred to in the semantics, but does not occur in the second alternation. 
    \item It is just as simple to imagine each alternations containing different terminal symbols (e.g. \mintinline{text}{<a> "+" <a> | <a> "-" <a>}) which may want to result in different behaviour. 
\end{itemize}

Clearly, the semantics should instead be specified at the level of the alternation instead: \mintinline{text}{<a> --> {} | <b> --> {} | ...} and so on.
This provides the maximum amount of flexibility to the language designers, allowing them to specify more complex productions that are logically grouped. An example of this is the `arithmetic operation' production shown in \autoref{lst:arithmetic_operations_in_metaspec}.

\begin{listing}[!htb]
\begin{minted}[numbers=none]{text} 
<arith-op> ::=
    <arith-expr> "+" <arith-expr> --> {
        num n : {n = n1 + n2}() :
            {num n1 <= <arith-expr>[0]}, {num n2 <= <arith-expr>[1]}
    } |
    <arith-expr> "-" <arith-expr> --> {
        num n : {n = n1 - n2}() :
            {num n1 <= <arith-expr>[0]}, {num n2 <= <arith-expr>[1]}
    } |
    <arith-expr> "*" <arith-expr> --> {
        num n : {n = n1 * n2}() :
            {num n1 <= <arith-expr>[0]}, {num n2 <= <arith-expr>[1]}
    } |
    <arith-expr> "/" <arith-expr> --> {
        num n : {n = n1 / n2}() :
            {num n1 <= <arith-expr>[0]}, {num n2 <= <arith-expr>[1]}
    } |
    <arith-expr> "^" <arith-expr> --> {
        num n : {n = n1 ^ n2}() :
            {num n1 <= <arith-expr>[0]}, {num n2 <= <arith-expr>[1]}
    };
\end{minted}
\caption{Arithmetic Operations in Metaspec}
\label{lst:arithmetic_operations_in_metaspec}
\end{listing}

It is hence clear that the decision to associate semantics with each block in a top-level alternation is the most flexible choice for \gls{metaspec} and affords the \gls{dsl} designers the most flexibility without creating an overly convoluted syntax. 

% subsection determining_the_combination_point (end)

% section combining_syntax_and_semantics (end)

% chapter designing_the_metalanguage (end)

% How the problem is analysed to create the solution.
% Overall architecture of the design.
% Examine the design approaches taken.
% Identification of areas of the design that account for the requirements and resolve potential conflicts. 
% 

\chapter{Architecture and Algorithms} % (fold)
\label{cha:architecture_and_algorithms}
% This section will contain:
% \begin{itemize}
%     \item \textbf{Architectural Overview:} An overview of the system architecture, linked to the high-level requirements.
%     This will touch on the portions of the system that were ruled out of scope later on, and why.
%     \item \textbf{System Component Examination:} An examination of the design of each of the system components in detail.
%     \item \textbf{The Development of Metaspec:} The process of developing the metaspec language itself, as well as some syntax examples and full syntax description. 
%     \item \textbf{Algorithmic Descriptions:} The verification engine in particular is very heavy on algorithmic work. 
%     These algorithms will be explained here, accompanied by any additional proof work that they require. 
%     This will include proofs for the special form semantics.
%     It will also look at initial or discarded algorithmic designs as part of the process. 
% \end{itemize}

As a project, \gls{absol} has had a very heavy research bent. 
The experimental nature of the toolchain resulted in a heavy up-front design load and, combined with the highly theoretical nature of the language verification algorithms, this meant that design and algorithmic development dominated the time spent on the project.\\

This section aims to illustrate the significant design work that was put into all facets of the project.
It first explores the design process for the metalanguage, \gls{metaspec}, demonstrating the careful thought that went into the final result. 
It then provides a high-level overview of the architecture of the toolchain, showing the main system components and tying them to the overarching system requirements.
Finally, it concludes with the design of both the core algorithms and special language features that are integral to the operation of \gls{absol}. 

\section{Designing the Metalanguage --- Metaspec} % (fold)
\label{sec:designing_the_metalanguage_metaspec}
% Careful consideration of why each bit of syntax looks like it does, what it allows the user to do. 
% Design for intuition and flexibility.
% Look like the source kind of semantics -> provide examples. 
% Consideration of where the semantics needed to be placed in the syntax for it to make real sense. 

\gls{metaspec} is the metalanguage for the \gls{absol} project, allowing the language designers to specify both the syntax and semantics of their DSL, as well as associated metadata, in a unified form. 
The final syntax for Metaspec is the result of significant design work, and consequentially the syntax discussed below has been through some evolution. \\

Metaspec is, in itself, a \gls{dsl}, and hence its design process was an interesting insight into how people might use the language to design their own \glspl{dsl}. \\

The complete grammar for Metaspec can be found in Appendix~\ref{cha:the_metaspec_grammar}, and is written in standard \gls{ebnf} notation. 
The same notation will be used throughout this section of the document. 

\subsection{The Top-Level Definitions} % (fold)
\label{sub:the_top_level_definitions}
The top-level structure of a Metaspec file consists of a series of ordered top-level definitions.
The presence of these definitions emerged from the seemingly disparate nature of a number of the requirements placed upon the language.
They are as follows:
\begin{enumerate}
    \item The language name (Requirement~\reqref{req:LanguageMetadata})
    \item The language version (Requirement~\reqref{req:LanguageMetadata})
    \item Language feature imports (Requirement~\reqref{req:ExtensionMechanisms})
    \item Ground truths for the proof engine (Requirement~\reqref{req:Ground-TruthSemantics})
    \item The language itself (Requirement~\reqref{req:IntegratedSyntaxandSemanticSpecification})
\end{enumerate}

Tying these somewhat disparate areas together is the requirement for language definition files to read in an ``intuitive'' fashion (Requirement~\reqref{req:IntuitiveFileStructure}). 
This provided an initial sense for the ordering of the language definitions, as each block assisted in providing the contextual foundation for the language definition itself. \\

To this end, the decision was made to enforce the ordering of these in the language grammar itself (as seen in Listing~\ref{lst:top_level_metaspec_definition_blocks}, with the ordering as above.
\begin{lstlisting}[caption={Top-Level Metaspec Definition Blocks}, label={lst:top_level_metaspec_definition_blocks}, firstnumber=156]
metaspec-defblock = 
    name-defblock, rule-termination-symbol, 
    version-defblock, rule-termination-symbol, 
    using-defblock, rule-termination-symbol, 
    truths-defblock, rule-termination-symbol, 
    language-defblock, rule-termination-symbol;
\end{lstlisting}

Placing the metadata fields first was a natural way to provide some initial context as to the language and version, and also assists with at-a-glance determining the language version in a \gls{vcs}.
As the extensions import list, the \lstinline{using-defblock}, can contain dependencies of the other two blocks, it made a significant amount of sense to put this next, as it aids in establishing context for the next two blocks.\\

The ordering of the termination truths (\lstinline{truths-defblock}) and the language definition itself (\lstinline{language-defblock}) was similarly natural. 
While the truths are mainly depended upon by the termination proof mechanism (see Section~\ref{sub:the_verification_engine}), they can also act as a guide to the language designer to indicate where they will be required to provide semantics for a language production. 
This quite clearly indicates that the termination truths should come before the language definition itself, as they provide additional context for the language designer.\\

While one might argue for the separation of these top-level elements, particularly the metadata, into multiple files, it seems far more natural to combine them all under one umbrella.
This means that each language definition is a self-contained unit with all the context required to understand it. 
While this single-file requirement does mean that definitions for large DSl

Each of these top-level blocks have fairly distinct forms, and each was designed very carefully to aid the language designer's understanding of the file.

\subsubsection{The Metadata Blocks} % (fold)
\label{ssub:the_metadata_blocks}
These two blocks, defining the language name and language version are almost trivial to define. 
However, care was taken not to enforce any particular naming or version scheme on the users of Metaspec.
\citet{raemaekers2014semantic} found that the semantics of a version string vary dramatically among software projects, and so it made little sense to impose a particular scheme in this case. \\

To this end, both the name and version strings for the language consist of a string of utf-8 characters.
They are parsed from the first non-whitespace character to the last that occurs before the rule termination symbol (\lstinline{;}), as seen in Listing~\ref{lst:metadata_block_definitions}.
This ensures the maximum flexibility for the users of Metaspec. 

\begin{lstlisting}[numbers=none, caption={Metadata Block Definitions}, label={lst:metadata_block_definitions}]
name-defblock = "name", where-symbol, { utf-8-char }-,;
version-defblock = "version", where-symbol, { utf-8-char }-,;
\end{lstlisting}

% subsubsection the_metadata_blocks (end)

\subsubsection{The Imports Definition Block} % (fold)
\label{ssub:the_imports_definition_block}
In defining the style of imports for the language extension features (Requirement~\reqref{req:ExtensionMechanisms}) there were two main forms considered for use in Metaspec.\\

The first style of import declaration that was considered was what the project terms the `one-line-per-import' style. 
This is seen in myriad programming languages, and has each import as a single statement. 
This does feature benefits for user parsing, as well as complicated import statements possibly being easier to read. 
Such a style can be seen in \autoref{lst:haskell_import_style} below.

\begin{listing}
\begin{minted}[numbers=none]{haskell}
import qualified Data.Map                     as M
import           Data.Maybe                   (fromJust, isJust)
import qualified Data.Set                     as S
\end{minted}
\caption{Haskell Import Style}
\label{lst:haskell_import_style}
\end{listing}

% subsubsection the_imports_definition_block (end)

\subsubsection{Comments in Metaspec} % (fold)
\label{ssub:comments_in_metaspec}
During the consideration of the high-level structure of the metalanguage, it was important to determine the appearance of comments in the language.
Comments are crucial in any programming or specification language, and hence had to be included in Metaspec. \\

Requirement~\reqref{req:Comments} states that these comments need not have any semantic meaning in the language (contrasted with Python Docstrings, which are compiled with their functions \citep{python_docstrings}).
This meant that the commenting syntax could be very simple, allowing the parser to strip them directly, rather than be parsed.
To this end, it is stated in the language specification that all comments are removed in a preprocessing step. \\

Comments themselves are defined as follows, and both a line comment and block comment style are provided:
\begin{lstlisting}[caption={Comments in Metaspec}, label={lst:comments_in_metaspec}, firstnumber=149]
metaspec-comment =
    line-comment-start-symbol, { utf-8-char }, eol-symbol |
    block-comment-start-symbol, { utf-8-char }, block-comment-end-symbol;
\end{lstlisting}

The choice of \lstinline{line-comment-start-symbol}, \lstinline{block-comment-start-symbol} and \lstinline{block-comment-end-symbol} was effectively arbitrary.
The block comment style was taken directly from \gls{ebnf}, as specified in \cite{standard1996ebnf}, but there was no line comment style to go on as EBNF does not provide one. \\

While block comments can be used on a single line they contribute additional visual noise with their end-delimiter, and hence it seemed appropriate to include a line-comment style as well.
With no prior reference to go on, the C-style line comment operator was chosen as it has been adopted my many programming languages. \\

As a result, the comment symbols are defined in Metaspec as follows.
\begin{lstlisting}[caption={Metaspec Comment Symbols}, label={lst:metaspec_comment_symbols}, numbers=none]
line-comment-start-symbol = "//";
block-comment-start-symbol = "(*";
block-comment-end-symbol = "*)";
\end{lstlisting}

% subsubsection comments_in_metaspec (end)

% subsection the_top_level_definitions (end)

\subsection{Specifying the Language Syntax} % (fold)
\label{sub:specifying_the_language_syntax}
% Why were the changes from EBNF made? 
% Removal of empty syntax

% subsection specifying_the_language_syntax (end)

\subsection{Specifying the Language Semantics} % (fold)
\label{sub:specifying_the_language_semantics}
% Semantics and the unspecified cases. 

% subsection specifying_the_language_semantics (end)

\subsection{Combining Syntax and Semantics} % (fold)
\label{sub:combining_syntax_and_semantics}
% Challenges choosing the combination point
% Initially at the production level, but this was nonsensical
% What does having it at the alternation level allow users to do?
% Why the particular form was chosen: --> {};

% subsection combining_syntax_and_semantics (end)

% section designing_the_metalanguage_metaspec (end)

\section{Designing the Metacompiler --- ABSOL} % (fold)
\label{sec:designing_the_metacompiler_absol}
% TODO Parser: Lexer, Parser, Etc.
% TODO Verifier: Different Verification Modules, talk about the recursive nature. Preprocessor.
% Use this same style of architectural overview in the development section. 
% Architectural overview focuses on system components, not code-level components.
% Suitability of a pipeline-style architecture for the system as a whole. 

As for any large system, it is important to be able to visualise the way in which the individual system components interact and are integrated. \\

\gls{absol}, the metacompiler system 

\subsection{Lexing and Parsing} % (fold)
\label{sub:lexing_and_parsing}
% Talk about the design for the lexer and parser, informed by the use of Megaparsec.
% Talk about the process and use of datatypes to represent a very strongly typed AST

% subsection lexing_and_parsing (end)

\subsection{The Verification Engine} % (fold)
\label{sub:the_verification_engine}
% Talk about the different modules, the verification preprocessor, the recursive nature of the algorithm for traversing the metaspec AST.

% subsection the_verification_engine (end)

% section designing_the_metacompiler_absol (end)

\section{The Core Algorithms} % (fold)
\label{sec:the_core_algorithms}
% Talk about the design of each of the core algorithms in detail.

\subsection{Verifier Traversal} % (fold)
\label{sub:verifier_traversal}

% subsection verifier_traversal (end)

\subsection{Semantic Form Verification} % (fold)
\label{sub:semantic_form_verification}
% All three criteria (evals, subterms, etc)

% subsection semantic_form_verification (end)

\subsection{Guard Checking} % (fold)
\label{sub:guard_checking}

% subsection guard_checking (end)

% section the_core_algorithms (end)

\section{Special Language Features} % (fold)
\label{sec:special_language_features}
% Talk about the design of each of the language features, and prove the required termination properties here.

\subsection{Feature --- \texttt{base}} % (fold)
\label{sub:feature_base}

% subsection feature_base (end)

\subsection{Feature --- \texttt{number}} % (fold)
\label{sub:feature_number}

% subsection feature_number (end)

\subsection{Feature --- \texttt{string}} % (fold)
\label{sub:feature_string}

% subsection feature_string (end)

\subsection{Feature --- \texttt{list}} % (fold)
\label{sub:feature_list}

% subsection feature_list (end)

\subsection{Feature --- \texttt{matrix}} % (fold)
\label{sub:feature_matrix}

% subsection feature_matrix (end)

\subsection{Feature --- \texttt{traverse}} % (fold)
\label{sub:feature_traverse}

% subsection feature_traverse (end)

\subsection{Feature --- \texttt{funcall}} % (fold)
\label{sub:feature_funcall}

% subsection feature_funcall (end)

% section special_language_features (end)

% chapter architecture_and_algorithms (end)

% Present an overview of the software system, and a high-level discussion of the implementation proces.
% Reflection on the choice of languages, tooling and techniques (project management).
% Make sure to focus on the MAIN issues with implementation.
% Follow the same architectural ordering as the design section.

\chapter{Implementation} % (fold)
\label{cha:implementation}
This section will:
\begin{itemize}
    \item Follow a similar structure to algorithms, using code listings to illustrate how the abstract algorithms were turned into a concrete system. 
    \item Examine the compromises or changes to the algorithm that had to take place.
\end{itemize}

\section{Building the Application Framework} % (fold)
\label{sec:building_the_application_framework}
% Designing state into the parser (two stage impl)
% How did the NT tracker come about - why was it insufficient to track all nt parses the same way? -> they are all parsed the same, but they have different semantic meaning for the precondition verification. 

% section building_the_application_framework (end)

\section{Building the Lexer and Parser} % (fold)
\label{sec:building_the_lexer_and_parser}
% The precondition verifier is implemented in separate functions as much as possible. This ensures that it doesn't add unnecessary clutter to the parser itself. 

% section building_the_lexer_and_parser (end)

\subsection{Building the Verification Framework} % (fold)
\label{sub:building_the_verification_framework}
% Need to discuss the practicalities of solving the mutually recursive rule problem (the production trace and additional RuleTag types). 
% The theoretical underpinning of this. 


% subsection building_the_verification_framework (end)

\section{Tooling and Language Choices} % (fold)
\label{sec:tooling_and_language_choices}

\subsection{Reflecting on the Language Choice} % (fold)
\label{sub:reflecting_on_the_language_choice}

% subsection reflecting_on_the_language_choice (end)

% section tooling_and_language_choices (end)

% chapter implementation (end)

\chapter{Testing} % (fold)
\label{cha:testing}
Like any well-designed piece of software, \gls{absol} underwent a significant amount of testing during its development.
As it was mainly developed as an embodiment of the theoretical algorithms developed as part of the project, rather than as a software product itself, the ad-hoc testing approach seemed to prove sufficient for the context of the project.\\

This chapter of the document aims to both outline the testing approach taken, with an examination of how it integrated with the development process.
It also provides examples of the tests carried out with an examination of their result. 
Finally, it provides an example of a language developed in \gls{metaspec} and verified by \gls{absol}, with a description of that language's development process, in order to showcase the utility of the metacompiler toolchain.

\section{The General Testing Approach} % (fold)
\label{sec:the_general_testing_approach}
\gls{absol} is a large software system, and like any large software system it is likely to have bugs.
This meant that having some testing approach was paramount in order to ensure that the software operated correctly.
With the need for testing clear, the choice to be made was as to what kind of testing approach should be employed.\\

Initial examination of the testing ecosystem in Haskell highlighted two main complementary testing approaches that could be unified by tooling.
Both of these testing approaches could be unified under a common testing toolchain integrated with Stack in the form of Tasty \citep{tasty_haskell}. 
Tasty is a testing framework for Haskell that provides the ability to combine diverse testing approaches into a single test suite that can be run from within the Stack build tool. 
The approaches were as follows:
\begin{itemize}
    \item \textbf{HUnit:} Akin to the JUint framework for Java, HUnit provides a unit testing framework for Haskell that allows software engineers to write tests in terms of assertions on results of functions \citep{hunit}.
    \item \textbf{QuickCheck:} An automated property-based testing framework for Haskell, QuickCheck provides specifications of function behaviour that are then automatically tested over a large random search space \citep{quick_check}.
    With integrated QuickCheck support in Megaparsec, this would have integrated nicely with an automated testing approach.
\end{itemize}

However, despite the obviously robust support for software testing provided by the Haskell ecosystem, the project did not take an automated approach.
While it would have brought significant benefits to the project, initial evaluation of such a testing strategy in the context of the non-product nature of the \gls{absol} toolchain indicated that the effort to maintain the tests would outweigh the benefits they would bring.
This seemed to be especially true of the QuickCheck specifications, which are written in a \gls{dsl} themselves. \\

As an alternative, the project decided to rely on a dual-pronged approach that combined manual software testing with the significant strengths of Haskell as a language: its strong type-safety:
\begin{itemize}
    \item \textbf{Haskell's Type-Safety:} It is often heard in folklore around Haskell that ``if it compiles, it probably works''. 
    This quote stems from the fact that in specifying the type of a function, and keeping functions small and composable, it is possible to encode a significant amount of information about the function itself.
    This meant that the first part of this testing approach involved the strict specification of types for all functions, not relying on Haskell's robust type-inference mechanism.
    If the resultant code compiled then there was some guarantee of safety.
    It should be noted that this is not on par with the kinds of tests that can be encoded in the type system using a \gls{dependently_typed} system.
    \item \textbf{Manual Testing:} The type-safety alone, however, was far from sufficient. 
    While it can guarantee that the functions making up a program operate in the correct domain, it cannot necessarily show that the behaviour of these functions is \textit{correct}.
    In order to ensure correctness, the decision was made to additionally perform manual testing using a test file.
    This test file would be modified to test the piece of functionality that was currently a concern.
    The test file was also intended to act as some kind of regression test, as no working piece of syntax or semantics would have been removed from it. 
\end{itemize}

The main benefit of such a testing approach, and the main reason why it was chosen over more formal automated testing was that it required little maintenance effort.
In a project so constrained for time, this was seen as a large boon. 
To this end, this was the chosen approach for the project, with the hope that it would be sufficient. 
The chosen testing strategy was interleaved with the development process, testing each feature as they were developed. 

\subsection{Examining the Testing Approach in Hindsight} % (fold)
\label{sub:examining_the_testing_approach_in_hindsight}
While the testing approach chosen did \textit{help} the development of \gls{absol} it turned out to be woefully insufficient in hindsight.
The main failure of the chosen testing approach was that it provided very little in the way of \textit{regression testing}: tests to ensure that further development did not break or alter the function of existing features. \\

The \gls{metaspec} test file, which can be seen in \autoref{cha:the_absol_testing_file} on \autopageref{cha:the_absol_testing_file}, was able to act as some kind of regression testing.
This file was used to test the new features as they were added, testing both accepting and rejecting scenarios.
However, as a feature became `finished' or development moved on from that point, the corresponding part of the file had to be left in an `accepting' state, meaning that it only tested the success criteria.
This meant that while the test file acted somewhat like a regression test, it would only detect breakages where something correct or valid was no longer accepted by \gls{absol}.
As a result, it entirely ignored functionality in the cases where things should be broken.\\

In order to attempt to compensate for this difficulty, manual breakages were introduced to the file at regular intervals in order to examine the behaviour in these cases. 
However, it was all too easy to forget to re-test a certain case, and this led to multiple occasions where alterations to the verification engine or parser caused breakages that were not detected until much later.
This late discovery of bugs often made fixing them more difficult due to the mental context switch involved. \\

In hindsight, it would have been far better to have applied the extra work required to maintain both HUnit and QuickCheck tests for \gls{absol}.
A fully automated test suite would have acted as an efficient regression testing mechanism, and also provided other testing guarantees that would potentially have allowed more rapid feature development. 
The HUnit-based tests would have been able to check the behaviour of the support modules and Metaverify, while sets of QuickCheck properties could have helped automate the testing of both the parser and the verification engine.
Overall, the chosen testing strategy turned out to be a significant misstep during the development of \gls{absol}.

% subsection examining_the_testing_approach_in_hindsight (end)

% section the_general_testing_approach (end)

\section{Testing During Development} % (fold)
\label{sec:testing_during_development}
Testing during the development process took place using manual test cases created in \gls{metaspec} (the language) itself.
These test cases were kept throughout development, and can be seen in \autoref{cha:the_absol_testing_file}.
While most of the tests that took place operated successfully, often due to the correct Haskell type-signatures constraining function behaviour, there were some occurrences where this manual testing approach was able to expose significant problems.
These usually occurred either with the \textit{algorithms} underlying the metacompiler or the \textit{implementation} of these algorithms.

\subsection{Infinite Recursion in Mutually Recursive Productions} % (fold)
\label{sub:infinite_recursion_in_mutually_recursive_productions}
As mentioned in \nameref{ssub:the_implementation_influencing_design} on \autopageref{ssub:the_implementation_influencing_design}, it was not uncommon for the development process to influence the design of the underlying algorithms.
In the particular case highlighted here, the verification engine was failing to terminate for a set of mutually recursive productions specifically designed to test behaviour in this case. 
They are as seen in \autoref{lst:the_test_case_for_mutually_recursive_semantic_verification}, and are still visible in the metaspec test file:
\begin{listing}[!htb]
\begin{minted}[numbers=none, fontsize=\blockfont]{text}
<arith-expr> ::= <my-number> | <arith-op> ;
...
<arith-op> ::= 
    <arith-expr> "+" <arith-expr> --> {
        number n : {n = n1 + n2}() :
            {number n1 <= <arith-expr>[0]}, {number n2 <= <arith-expr>[1]}
    } | ... ;
\end{minted}
\caption{The Test Case for Mutually Recursive Semantic Verification}
\label{lst:the_test_case_for_mutually_recursive_semantic_verification}
\end{listing}

In this case it is obvious that the verification for \mintinline{text}{<arith-expr>} depends on the verification for \mintinline{text}{<arith-op>}, and the converse is also true. 
This test case was initially written into the file at the time of testing the parser, but once the development of the verification for the user-specified semantics occurred in Metaverify, it was able to expose the issue in verifying such productions.
The metacompiler was observed to hang, and using the test case contained within the testing file it was possible to diagnose the issue and implement the solution as described in the final version of the non-terminal verification algorithm (see \autoref{sub:verifier_traversal}).

% subsection infinite_recursion_in_mutually_recursive_productions (end)

\subsection{False Successes for Semantic Evaluations} % (fold)
\label{sub:false_successes_for_semantic_evaluations}
Another case where the testing approach was able to identify an issue was in one of the rules for verification of the user-defined semantics. 
While \autoref{alg:verification_of_the_semantic_evaluation_criterion} had provided the correct set of criteria to validate these evaluations since it was designed (see \autoref{ssub:verifying_the_evaluation_criterion} on \autopageref{ssub:verifying_the_evaluation_criterion}), the testing approach highlighted a bug in the implementation of one of the criteria checks.\\

As stated in the above section, all temporary variables declared in the evaluation list must only be used \textit{to the left} of where they were declared. 
The test case created to check this verification was working as intended was as seen in \autoref{lst:testing_part_of_the_semantic_evaluation_criteria}.

\begin{listing}[!htb]
\begin{minted}[numbers=none, fontsize=\blockfont]{text}
<arith-op> ::= 
    ... |
    <arith-expr> "^" <arith-expr> --> {
        ... |
        number n : {n = n1 ^ n3, n3 = n3 * 0 + n2}() :
            {number n1 <= <arith-expr>[0]}, {number n2 <= <arith-expr>[1]}
    };
\end{minted}
\caption{Testing Part of the Semantic Evaluation Criteria}
\label{lst:testing_part_of_the_semantic_evaluation_criteria}
\end{listing}

This test case was designed to fail immediately, but it turned out that it validated without issue. 
Careful tracing of this result through the verification algorithm allowed the discovery that the collation algorithm for obtaining the variables used after each definition had not been implemented properly. 
Fixing this implementation allowed the metacompiler to act as expected, failing the language with this semantic rule:

\begin{minted}[numbers=none]{text}
Incorrect Semantic Form.
    REASON: Malformed semantic operation(s).
    IN: <statement> -> <arith-expr> -> <arith-op>

Incorrect Semantic Form.
    REASON: Malformed semantic operation(s).
    IN: <statement> -> <assignment> -> <arith-expr> -> <arith-op>
\end{minted}

% subsection false_successes_for_semantic_evaluations (end)

% section testing_during_development (end)

\section{Testing Error States} % (fold)
\label{sec:testing_error_states}
In general, the testing approach focused on establishing that each feature of the metacompiler worked as expected as it was completed. 
The following section aims to provide some evidence of the correct functionality of \gls{absol} through an application of the standard testing approach discussed in \autoref{sec:the_general_testing_approach}. \\

As Metaparse was the first system component requiring testing to be developed, initial testing focused on the correct operation of the parser.
As discussed in \autoref{sub:metaparse_ast_generation}, the parser in \gls{absol} has two main functions:
\begin{itemize}
    \item Parsing the \gls{metaspec} language specification.
    \item Ensuring that the preconditions are met for the verifier.
\end{itemize}

Each of these pieces of functionality was manually tested as it was developed. 

\subsection{Syntax Errors} % (fold)
\label{sub:syntax_errors}
All of the syntax error detection and reporting functionality built into \gls{absol} comes as a result of the use of Megaparsec, which provides these abilities by default.
As a result, testing this was as simple as introducing syntax errors into the input file and ensuring that the resultant parse error made sense. 
Consider the introduction of the syntax error in \autoref{lst:introducing_a_syntax_error}, which omits the closing \mintinline{text}{>} of a non-terminal. 
The resultant error, shown in \autoref{lst:the_syntax_error_diagnostic}, correctly diagnoses what's wrong as part of the built in functionality of Megaparsec. 

\begin{listing}[!htb]
\begin{minted}[numbers=none]{text}
<my-number> ::= <integer> | <floating ;
\end{minted}
\caption{Introducing a Syntax Error}
\label{lst:introducing_a_syntax_error}
\end{listing}

\begin{listing}[!htb]
\begin{minted}[numbers=none]{text}
metaspec/simple_test.meta:26:38:
unexpected space
expecting '>' or alphanumeric character
\end{minted}
\caption{The Syntax Error Diagnostic}
\label{lst:the_syntax_error_diagnostic}
\end{listing}

While the syntax error reporting behaviour does degrade in the presence of backtracking parsing, testing shows that it is still able to retain some fairly sensible diagnostics in the case of errors in the backtracking portion of the parser. 
If you omit the type before a special syntax rule, for example, it has enough context to suggest a list of types: \mintinline{text}{expecting "any", "bool", ...}, an appropriate suggestion based on the parser structure and language grammar.

% subsection syntax_errors (end)

\subsection{Precondition Verification Errors} % (fold)
\label{sub:precondition_verification_errors}
More interesting is the testing of the precondition verification algorithm.
The errors here are fully part of the \gls{absol} implementation and thus not provided by a library. 
This means that it was very important to test both that the errors manifested as expected, \textit{and} that they provided the appropriate diagnostic information.
The precondition verification algorithm aims to check four different criteria (see \autoref{sub:verifier_precondition_validation}), each of which was tested independently

\subsubsection{Testing Used Non-Terminals Defined In Scope} % (fold)
\label{ssub:testing_used_non_terminals_defined_in_scope}
To test this it is as simple as introducing the usage of a non-terminal into the language definition that does not exist in the language scope. 
Conversely, it is possible to remove the import for a given non-terminal from the \mintinline{text}{using} definition block for the language.
To test this a non-terminal \mintinline{text}{<tmp>} is added to the start-rule production of the test file. 
This means that the non-terminal is used but never defined, and should hence cause an error.

\begin{listing}[!htb]
\begin{minted}[numbers=none]{text}
metaspec/simple_test.meta:82:1:
The following Non-Terminals are used but not defined: <bar>... 
\end{minted}
\caption{Error for a Non-Terminal Used While Not In-Scope}
\label{lst:error_for_a_non_terminal_used_while_not_in_scope}
\end{listing}

Doing this results in the error seen in \autoref{lst:error_for_a_non_terminal_used_while_not_in_scope}, which correctly diagnoses the issue.
In the context where the missing non-terminal is defined by a language feature, it would also be capable of suggesting the corresponding import to the \gls{dsl} designer.

% subsubsection testing_used_non_terminals_defined_in_scope (end)

\subsubsection{Testing the Single Definition Principle} % (fold)
\label{ssub:testing_the_single_definition_principle}
Much like the above, it is very simple to test.
To do so it is sufficient to introduce a secondary definition for a non-terminal that has already been defined. 
In this case, a duplicate definition of \mintinline{text}{<integer>} is added.
This non-terminal is defined by the \mintinline{text}{number} language feature.

\begin{listing}[!htb]
\begin{minted}[numbers=none, fontsize=\blockfont]{text}
metaspec/simple_test.meta:28:11:
Non-Terminal with name "integer" already defined. Defined by language feature(s): number.
\end{minted}
\caption{Error for Duplicate Non-Terminal Definitions}
\label{lst:error_for_duplicate_non_terminal_definitions}
\end{listing}

As seen in \autoref{lst:error_for_duplicate_non_terminal_definitions}, this produces a helpful error. 
In this case, the error is able to recognise that, rather than being defined in the document body itself, the non-terminal was originally defined by a language feature, and that information is provided. 

% subsubsection testing_the_single_definition_principle (end)

\subsection{Testing Types and Special-Syntax in Scope} % (fold)
\label{sub:testing_types_and_special_syntax_in_scope}
Much like the other portions of the precondition verifier, the testing of both of these was simple. 
The basic test file makes use of the \mintinline{text}{map} special syntax, as well as heavy use of the \mintinline{text}{<integer>} type, and so removing the relevant imports produced errors as seen in \autoref{lst:error_for_types_and_special_syntax_missing_from_scope}.
In both cases, the missing elements are defined by language features, and so the metacompiler is able to suggest the relevant imports to help the \gls{dsl} designer. 

\begin{listing}[!htb]
\begin{minted}[numbers=none, fontsize=\blockfont]{text}
metaspec/simple_test.meta:74:12:
Special Syntax "map" not in scope. Please import one of the following: traverse.

metaspec/simple_test.meta:14:14:
Type "integer" not in scope. Defined in language feature(s): number.
\end{minted}
\caption{Error for Types and Special-Syntax Missing from Scope}
\label{lst:error_for_types_and_special_syntax_missing_from_scope}
\end{listing}

% subsection testing_types_and_special_syntax_in_scope (end)

% subsection precondition_verification_errors (end)

% section testing_error_states (end)

\section{Testing Metaverify Errors} % (fold)
\label{sec:testing_metaverify_errors}
Having established that the parser worked properly, it was time to develop the verification engine. 
Much like Metaparse, the verification engine also underwent manual testing to ensure that it detected all the possible conditions that it was meant to detect. 
This, too, used the same testing methodology to help establish whether Metaverify was able to operate correctly, and thus that its conclusions were also correct.\\

Metaverify was the second system component that was developed for \gls{absol} that required testing.
However, it was developed piece-by-piece so the testing approach focused on each portion of the verification algorithm in turn. 

\subsection{Testing the Semantic Inference} % (fold)
\label{sub:testing_the_semantic_inference}
One of the major features of the verification algorithm is its ability to infer the semantics for simple productions that just consist of alternations. 
Ensuring this operates correctly requires both checking that it does not infer semantics when it reasonably cannot, but also that it will infer semantics correctly where possible.
The first test was to ensure that productions of the form where it \textit{should} operate correctly had their semantics inferred. \\

The test file used for most of the testing of the toolchain contains multiple examples of productions who satisfy the required form for semantic inference. 
One of these can be seen in \autoref{lst:a_production_with_inferred_semantics} below. 
When executing the metacompiler on this file, the semantics of the language are verified correctly, stating that it terminates.
This implies that the semantic inference is working in this case. 

\begin{listing}[!htb]
\begin{minted}[numbers=none]{text}
<arith-expr> ::= <my-number> | <arith-op> ;
\end{minted}
\caption{A Production with Inferred Semantics}
\label{lst:a_production_with_inferred_semantics}
\end{listing}

It remains to be shown, however, that the inference fails in appropriate cases.
In any situation other than a single non-terminal or terminal in an alternation, the inference engine should fail with an appropriate message. 
Altering the above production to read \mintinline{text}{... | <arith-op> "+" ;} introduces a situation in which the inference algorithm should fail, and indeed it does (as seen in \autoref{lst:inference_failure}).

\begin{listing}[!htb]
\begin{minted}[numbers=none]{text}
Unable to infer semantics for rule.
    REASON: Cannot infer semantics for rule.
    IN: <statement> -> <arith-expr>
\end{minted}
\caption{Inference Failure}
\label{lst:inference_failure}
\end{listing}

% subsection testing_the_semantic_inference (end)

\subsection{Testing the Alternative Semantic Forms} % (fold)
\label{sub:testing_the_alternative_semantic_forms}
The second kind of semantic verification that was developed were the special types of semantic rule.
This was broadly because their verification is generally simple, revolving around special-case semantic proofs, and the termination of the non-terminals involved.
The termination of all special semantic forms (special syntax rules, environment input rules and environment access rules) is guaranteed as long as the involved non-terminals terminate and exist in the associated syntactic production. \\

As the verification procedure is the same for each of these, the following example pertains only to the verification of special-syntax rules. 
The test file can already be shown to terminate with the valid special-syntax rule on line 73, but this does not show that it fails appropriately.
Consider a production (defined in the test file) as shown in, and introduce some semantic failure into \mintinline{text}{<arith-op>}.
As shown in \autoref{lst:a_failing_special_syntax_rule}, this produces the appropriate errors in the metacompiler output. 

\subsubsection{Testing Special-Syntax Rules} % (fold)
\label{ssub:testing_special_syntax_rules}
The termination of special syntax rules depends both on the external termination proof (as seen in \autoref{sec:special_language_features}), and the termination of all the non-terminals used in the semantic rule. 
Testing this is hence simple, introducing an error into one of the non-terminals used in the rule.

\begin{listing}[!htb]
\begin{minted}[numbers=none, fontsize=\blockfont]{text}
PRODUCTION: <ssr-fail>
STATUS: Does not terminate.

Refers to non-existent subterms.
    REASON: Non-terminal <arith-op> with index 2 is not defined in this production.
    IN:

Incorrect Semantic Form.
    REASON: Malformed semantic operation(s).
    IN: <arith-op>

Incorrect Semantic Form.
    REASON: Malformed semantic operation(s).
    IN: <arith-op>
\end{minted}
\caption{A Failing Special-Syntax Rule}
\label{lst:a_failing_special_syntax_rule}
\end{listing}

The other kinds of alternative semantic form were similarly tested, and the verification algorithms for each kind of semantics all appeared to operate correctly.
This meant that they detected the appropriate error conditions and produced the correct corresponding error messages.

% subsubsection testing_special_syntax_rules (end)

\subsection{Testing the User-Defined Semantics Checks} % (fold)
\label{sub:testing_the_user_defined_semantics_checks}
Testing the verification algorithm for the user-defined semantics is a touch more involved due to the number of separate components that exist within it. 
Each of these individual component tests was developed individually, and so was tested individually. 

\subsubsection{Testing Sub-Term Criterion Verification} % (fold)
\label{ssub:testing_sub_term_criterion_verification}
The sub-term criterion states that all sub-evaluations in a rule must be strict subterms of the main body, and that the semantics of these sub-terms also terminate. 
The implementation is a direct adaptation of the algorithm in \nameref{ssub:verifying_the_sub_term_criterion} on \autopageref{ssub:verifying_the_sub_term_criterion}, and this provides the grounds to test the implementation.

\begin{listing}[!htb]
\begin{minted}[numbers=none, fontsize=\blockfont]{text}
PRODUCTION: <arith-op>
STATUS: Does not terminate.

Refers to non-existent subterms.
    REASON: Non-terminal <arith-expr> with index 2 is not defined in this production.
    IN:
\end{minted}
\caption{Failing Due to a Non-Existent Sub-Term}
\label{lst:failing_due_to_a_non_existent_sub_term}
\end{listing}

Introducing a sub-term that does not exist in the syntax should result in an error as it is not a sub-term of the evaluated semantics.
Introducing a bad index into the \mintinline{text}{<arith-op> production} does indeed result in an appropriate error, as seen in \autoref{lst:failing_due_to_a_non_existent_sub_term}.
By the same token, adapting the semantic rule to contain non-terminating productions will also cause the verification to fail.

% subsubsection testing_sub_term_criterion_verification (end)

\subsubsection{Testing the Evaluation Criterion} % (fold)
\label{ssub:testing_the_evaluation_criterion}
The second main component of verifying the user semantic form is ensuring that the evaluation rules match the required form (described in \nameref{ssub:verifying_the_evaluation_criterion} on \autopageref{ssub:verifying_the_evaluation_criterion}).
It is possible to break this criterion in multiple ways, and so it is not worth showcasing all of the tests that were run on it here.\\

Instead, this demonstration will focus on a case where the conditions are broken by using a non-existent evaluation variable as part of the computation.
Working from the basic, valid, test file, the evaluation on line 42 can be transformed to read as in \autoref{lst:testing_the_evaluation_criterion} to introduce a dependency on a non-existent variable \mintinline{text}{n3}.

\begin{listing}[!htb]
\begin{minted}[numbers=none]{text}
<arith-expr> "+" <arith-expr> --> {
    number n : {n = n1 + n3}() :
        {number n1 <= <arith-expr>[0]}, {number n2 <= <arith-expr>[1]}
} | ...
\end{minted}
\caption{Testing the Evaluation Criterion}
\label{lst:testing_the_evaluation_criterion}
\end{listing}

Running the metacompiler on this file with the alteration produces an error in the output, as expected. 
The error can be seen in \autoref{lst:failure_to_verify_the_semantic_operations}, and it clearly diagnoses the issue. 

\begin{listing}[!htb]
\begin{minted}[numbers=none]{text}
PRODUCTION: <arith-op>
STATUS: Does not terminate.

Incorrect Semantic Form.
    REASON: Malformed semantic operation(s).
    IN:
\end{minted}
\caption{Failure to Verify the Semantic Operations}
\label{lst:failure_to_verify_the_semantic_operations}
\end{listing}

% subsubsection testing_the_evaluation_criterion (end)

\subsubsection{Testing the Guard Checking} % (fold)
\label{ssub:testing_the_guard_checking}
\begin{listing}[!htb]
\begin{minted}[numbers=none]{text}
 <arith-expr> "^" <arith-expr> --> {
    number n : {n = 1}(n1 == 1) :
        {number n1 <= <arith-expr>[0]}, {number n2 <= <arith-expr>[1]} |
    number n : {n = n1 * n1}(n2 == 2) :
        {number n1 <= <arith-expr>[0]}, {number n2 <= <arith-expr>[1]}
\end{minted}
\caption{A Set of Invalid Guards}
\label{lst:a_set_of_invalid_guards}
\end{listing}

The final element of checking the user-defined semantics is to ensure that there is always a semantic rule to execute, no matter the values of the guards. 
This is fairly simple to perform in this mode, checking just that there is a catch-all guard, as discussed in \autoref{sub:guard_checking}. 
This makes checking that it works as intended simple - remove the catch-all guard from a set of semantic rules and check that it detects it appropriately.
It is simple enough to alter the semantic rules beginning on line 58 of the test file to not include the catch-all guard, as seen in \autoref{lst:a_set_of_invalid_guards}. \\

Executing the metacompiler on the file containing the error produces a diagnostic message as expected.
This message can be seen in \autoref{lst:an_invalid_guards_error}.

\begin{listing}[!htb]
\begin{minted}[numbers=none]{text}
PRODUCTION: <arith-op>
STATUS: Does not terminate.

Guards incomplete.
    REASON: Guards must contain a catch-all clause.
    IN:
\end{minted}
\caption{An Invalid Guards Error}
\label{lst:an_invalid_guards_error}
\end{listing}

% subsubsection testing_the_guard_checking (end)

% subsection testing_the_user_defined_semantics_checks (end)

% section testing_metaverify_errors (end)

\section{An Example Language} % (fold)
\label{sec:an_example_language}
The other main component to testing \gls{absol} was the creation of a (potentially) useful \gls{dsl} as a proof-of-concept for the metacompiler system. 
This section aims to outline the process through which a simple \gls{dsl} could be created, and demonstrate the final language here. 

\subsection{The Idea for the Language} % (fold)
\label{sub:the_idea_for_the_language}
The notion for this language, known as \textit{spreadsheet}, is that it provides a \gls{dsl} for the manipulation of homogeneous, spreadsheet-based data. 
This means that it needs to be able to:
\begin{itemize}
    \item Ingest and output `spreadsheets' to the host language via the FFI (and so package \mintinline{text}{funcall} is needed).
    \item Perform basic operations over spreadsheets, allowing users to define functions to apply to rows and columns (\mintinline{text}{traverse}) will be required. 
    \item The spreadsheets should be able to contain numeric data for processing (hence \mintinline{text}{number} is required). 
\end{itemize}

Considering all of these language design goals, another import required will be \mintinline{text}{matrix}, providing a table-like container for homogeneous data. 
Finally, \mintinline{text}{base} should be included as it provides useful special features for the implementation of any language. 

% subsection the_idea_for_the_language (end)

\subsection{Designing the Example Language} % (fold)
\label{sub:designing_the_example_language}
This section aims to guide the reader through the design process for this language, emphasising the use of the various features of \gls{metaspec} to produce a language that is successfully verified by \gls{absol}. 
The design process involves multiple stages. 

\subsubsection{Specifying the Language Metadata} % (fold)
\label{ssub:specifying_the_language_metadata}
The metadata fields in a \gls{metaspec} provide a foundation from which to develop the language in question. 
To that end, the language is contextualised with a name and initial version, and the established list of imports is added to the \mintinline{text}{using} defblock. 
In addition to this, it is worth inserting both the truths defblock (though empty), and the language defblock with a basic start-rule. 
Doing this ensures that you can check if the language parses and terminates in its current state.
From this, the current state of the language is seen in \autoref{lst:the_initial_version_of_spreadsheet}. 

\begin{listing}[!htb]
\begin{minted}[numbers=none]{text}
name : spreadsheet;

version : 0.0.1;

using : {
    base,
    number,
    matrix,
    traverse,
    funcall
};

truths : {

};

language : {

<<spreadsheet>> ::= "";

};
\end{minted}
\caption{The Initial Version of Spreadsheet}
\label{lst:the_initial_version_of_spreadsheet}
\end{listing}

% subsubsection specifying_the_language_metadata (end)

\subsubsection{Designing the Language Itself} % (fold)
\label{ssub:designing_the_language_itself}
Designing the language itself is a difficult process to describe as it relies on both experience and intuition. 
It was built using a top-down process, envisioning what the user would be required to do to write useful programs in `spreadsheet', and developing the productions from there.
Nevertheless, the development of the \gls{dsl} relied heavily on the metacompiler to provide feedback about the termination state of the language. 
At any given time it was possible to run the metacompiler on it, and let the analysis explain what is invalid and needs to be improved, changed or completed. 
The final definition of `spreadsheet' can be seen below:

\begin{minted}[fontsize=\blockfont]{text}

name : spreadsheet;

version : 0.0.1;

using : {
    base,
    number,
    matrix,
    traverse,
    funcall,
    string
};

truths : {
    {number n <= <number>},
    {matrix n <= <matrix>},
    {any n <= <nondigit>},
    {any n <= <digit>},

    // Termination for literals
    {any n <= <proc-name>},
    {any n <= <variable-name>}
};

language : {

<<spreadsheet>> ::= <procedure-def> | <function-def> ;

// The procedure definitions are available in the host language. 
<procedure-def> ::= 
    "proc " <proc-name> "(" <argument-list> ")" "{" <proc-body> "}" ";" --> {
        // Suppress any value returned.
        none defproc(<proc-name>[0], <argument-list>[0], <proc-body>[0])
    };

// Restricted form for procedure naming, is a literal so is in truths
<proc-name> ::= <nondigit> { <nondigit> | <digit> | "-" | "_" };

// The argument list is a list of literals or variables
<argument-list> ::= <arg-nt> { "," <arg-nt> } --> {
    any semanticsOf(<arg-nt>[0], <arg-nt>[1])
};

// Argument non-terminals can either be literals or variables
<arg-nt> ::= <literal> | <variable-name> | <statement> ;

// These have default terminating semantics by the truths
<literal> ::= <matrix> | <number> ;

// Variable names take a standard form
<variable-name> ::= { <utf-8-char> } ;

// The procedure body is a list of statements.
<proc-body> ::= { <statement> } --> {
    any semanticsOf(<statement>[0])
};

// Statements make up the program in the language
<statement> ::= <process-sheet> | <reduce-sheet> | <binary-op> ;

// Process-sheet allows the programmer to specify a function to operate on
<process-sheet> ::= 
    "process " "(" <function-name> "," <arg-nt> "," <dim> ")" --> {
    any map(<function-name>[0], <arg-nt>[0], <dim>[0])
};

// Reduce-sheet allows reduction of the rows or columns of the sheet with a 
// function and value
<reduce-sheet> ::= 
    "reduce " "(" <function-name> "," <arg-nt> "," <arg-nt> "," <dim> ")" --> {
        any fold(<function-name>[0], <arg-nt>[0], <arg-nt>[1], <dim>[0])
    };

// Dimension markers for traversal
<dim> ::= "0" | "1" ;

// Mathematical operations to provide to process-sheet and reduce-sheet
<binary-op> ::= <number> "+" <number> --> {
        number n : {n = n1 + n2}() :
            {number n1 <= <number>[0]}, {number n2 <= <number>[1]}
    } |
    <number> "-" <number> --> {
        number n : {n = n1 - n2}() :
            {number n1 <= <number>[0]}, {number n2 <= <number>[1]}
    } |
    <number> "*" <number> --> {
        number n : {n = n1 * n2}() :
            {number n1 <= <number>[0]}, {number n2 <= <number>[1]}
    } |
    <number> "/" <number> --> {
        number n : {n = n1 / n2}() :
            {number n1 <= <number>[0]}, {number n2 <= <number>[1]}
    };

// Functions may only have single expressions in their bodies. 
<function-def> ::= 
    "fun " <function-name> "(" <argument-list> ")" "{" <fun-body> "}" ";" --> {
        none deffun(<function-name>[0], <argument-list>[0], <fun-body>[0])
    };

// These both get inferred semantics.
<function-name> ::= <proc-name> ;
<fun-body> ::= <statement> ;

};

\end{minted}

% subsubsection designing_the_language_itself (end)

% subsection designing_the_example_language (end)

\subsection{An Example Program} % (fold)
\label{sub:an_example_program}
Having  defined the language `spreadsheet', it remains to define an example program in this \gls{dsl} to exemplify what it is capable of.
For the purpose of this demonstration, this program will sum all the columns in an input matrix, and then get the sum of those results. 
It is defined in \autoref{lst:an_example_spreadsheet_program}.

\begin{listing}[!htb]
\begin{minted}[]{text}
proc sum_matrix (matrix) {
    // Sum down columns then across the results (rows)
    reduce(sum, 0, reduce(sum, matrix, 1), 0)
};

fun sum (val1, val2) { val1 + val2 };
\end{minted}
\caption{An Example Spreadsheet Program}
\label{lst:an_example_spreadsheet_program}
\end{listing}

This program defines a top-level procedure, with the return value being the value of the last statement. 
It computes the sum of each column of an $m \times n$ matrix, which results in a row vector (a $1 \times n$ matrix), which is then reduced over the row dimension to compute the overall sum.
This procedure relies on the function \mintinline{text}{sum}, which computes the sum of its two arguments. 

% subsection an_example_program (end)

\subsection{Reflecting on the Language Design Process} % (fold)
\label{sub:reflecting_on_the_language_design_process}
Overall, the process of designing this language was fairly simple.
Using the design documentation specified in \autoref{sec:special_language_features} on \autopageref{sec:special_language_features} it was clearly possible to define the language as required:
\begin{itemize}
    \item \textbf{Truths:} It was quite clear that when a production was only intended to have a literal meaning that it should be put in the \mintinline{text}{truths} definition block.
    Choosing the type for this was sometimes non-obvious, however, but if in doubt the \mintinline{text}{any} type would suffice as it would be bound properly at \gls{dsl} compile-time.
    \item \textbf{Meta-Thinking:} The main issue in defining the language was to think at the \textit{language} level, rather than the \textit{program} level.
    There were multiple occurrences where behaviour became confused between the language and a program in that language, requiring alterations to the language definition. 
    \item \textbf{Metaspec Deficiencies:} It was an oversight to not allow for the definition of constants as arguments to the special-syntax forms. 
    This would allow the creation of more elegant interfaces to some of these forms, and hence a better experience for the language designer.
    An example of this is illustrated in \autoref{lst:an_alternative_definition_for_reduce_sheet}, which would allow a more fluent interface for users by using better domain terminology, rather than syntax enforced by the special syntax form. 
\begin{listing}[!htb]
\begin{minted}[numbers=none, fontsize=\blockfont, xleftmargin=1cm]{text}
<reduce-sheet> ::= 
    "reduce " "(" <function-name> "," <arg-nt> "," <arg-nt> "," <dim> ")" --> {
        any n : {n = n1}(n2 == "rows") : 
            {any n1 <= any fold(<function-name>[0], <arg-nt>[0], <arg-nt>[1], 0)},
            {any n2 <= <dim>[0]} |
        any n : {n = n1}(n2 == "cols") : 
            {any n1 <= any fold(<function-name>[0], <arg-nt>[0], <arg-nt>[1], 1)},
            {any n2 <= <dim>[0]} |
        any n : {n = n1}() : 
            {any n1 <= any fold(<function-name>[0], <arg-nt>[0], <arg-nt>[1], 0)},
            {any n2 <= <dim>[0]}
    };

<dim> ::= "rows" | "cols" ;
\end{minted}
\caption{An Alternative Definition for \texttt{<reduce-sheet>}}
\label{lst:an_alternative_definition_for_reduce_sheet}
\end{listing}

    \item \textbf{Metacompiler Support:} Much like the edit-compile-test interactive development cycle when writing programs, the \gls{absol} metacompiler enables a similar kind of feedback for the people defining the \gls{dsl}.
    During the course of developing `spreadsheet' it was quickly made apparent that this support, and the notion that the user would immediately be alerted to errors with their language definition, allowed the \gls{dsl} designer a greater ability to experiment with both syntax and semantics.
    This liberty to experiment led to the creation of more intuitive syntax for the \gls{dsl}, but also allowed the language to remain correct.
    It should be noted that, in part, this interactivity is enabled by the performance of \gls{absol}, which requires no noticeable wait on the behalf of the language designer. 
\end{itemize}

% subsection reflecting_on_the_language_design_process (end)

% section an_example_language (end)

% chapter testing (end)

\chapter{Evaluation} % (fold)
\label{cha:evaluation}
This section will include:
\begin{itemize}
    \item An examination of how the system meets it high-level goals.
    \item Identification and discussion of areas requiring improvement, alteration and further work.
    \item Concluding with discussion of the major successes and failures of the ABSOL system.
\end{itemize}

% Flaws of the metalanguage:
%   - No direct use of environment accesses in basic semantics, requiring indirection.
%   - Fairly awkward use of the environment (no ability to delete properties, for example). 
%   - Grammar admits some nonsensical terms such as alternations with semantics not at the top-level. These are ignored by the metacompiler, but are inelegant. Ideally this would not be allowed by the grammar at all.

% Contributions
% Failures

% chapter evaluation (end)

\chapter{Conclusion} % (fold)
\label{cha:conclusion}
As a project, \gls{absol} has been a great success.
Despite the numerous failures of the project, the end result is a novel contribution to the state of the art in language verification. 
Despite the flaws in both the metalanguage and metacompiler, the achievements of this project cannot be ignored. 
The main successes of this project are twofold:
\begin{itemize}
    \item \textbf{Advancing the State of the Art in Language Verification:} The combination of the integrated syntactic and semantic specification in the form of \gls{metaspec}, and the novel verification algorithms implemented in \gls{absol} itself, this project provides two major novel contributions to the language verification ecosystem.
    \item \textbf{An Extensible Metacompiler:} The final version of the metacompiler is a functional and extensible solution to the language verification problem for \glspl{dsl} with limited semantics.
    It allows the language designers to create in relative safety, providing clear and concise diagnostics to help diagnose errors.
\end{itemize}

As a tool, \gls{absol} came very close to meeting its initial project goals, with only the omission of the code-generation step really marring that record. 
Nevertheless, this toolchain goes a long way to making provably correct \glspl{dsl} a reality in the context of the kind of systems that inspired this project.
It allows for the definition of capable \glspl{dsl}, and also for their verification, ensuring that the language semantics are total and thus eliminating entire classes of bug in the host systems.
With the implementation of the remaining system components as outlined in \autoref{sub:improving_the_absol_toolchain} it would be a fully-fledged system for \gls{dsl} creation.\\

During the development of this system, the metalanguage and algorithmic design processes went incredibly smoothly, resultant from the weeks of careful thought that had been put into both. 
This meant that the corresponding development proceeded well, working directly from these comprehensive designs.
Conversely, the lack of rigid processes imposed on the development and testing of the system, resulted in a not-insignificant number of bugs being found far later than was ideal.
Despite this, the system works as intended (even though it's not \textit{quite} complete), and its operational nature is a big success. \\

The final result, then, is this: \gls{absol}, while a flawed system, is a complex and powerful tool for the creation of provably correct \glspl{dsl}. 
It has produced novel contributions to the state of the art in language verification through the creation of the \gls{metaspec} metalanguage and the dual-part semantic verification engine that allows automated semantic verification while retaining useful language semantics.
It hence fills a niche in the language verification ecosystem, bridging the gap between complex semantic systems and systems so inflexible as to be unusable.
While it still needs some work, the future is bright, with \gls{absol} both having a clear path for further development, and advancing the state of the art. 

% chapter conclusion (end)


% Bibliography
\bibliographystyle{abbrvnat}
\bibliography{resources/bibliography}

\begin{appendix}

\chapter{Software Readme} % (fold)
\label{cha:software_readme}
This section will contain a guide to the project as a whole, as well as the necessary details to write languages and run the metacompiler on them.

\section{The Project Structure} % (fold)
\label{sec:the_project_structure}

% section the_project_structure (end)

\subsection{Building the Metacompiler} % (fold)
\label{sub:building_the_metacompiler}

% subsection building_the_metacompiler (end)

\subsection{Executing the Metacompiler} % (fold)
\label{sub:executing_the_metacompiler}

% subsection executing_the_metacompiler (end)

% chapter software_readme (end)

\chapter{The Metaspec Grammar} % (fold)
\label{cha:the_metaspec_grammar}
Metaspec, as a language, has a fairly complex syntactic structure. 
The syntax for the language is represented using \gls{ebnf}, and is included in its entirety below. \\

For an overview of standard \gls{ebnf} syntax, please see the official standard for the metasyntactic notation published in \citet{standard1996ebnf}.

\begin{lstlisting}
(*
    This file defines the grammar for the Syntax of the metalanguage 'metaspec'.

    It uses the EBNF syntax defined in Sections 4 and 5 of the ISO-14977 EBNF
    standard.

    The grammar does not care about whitespace except in the case of single-line
    comments which are ended by an EOL character (\n, \r, \r\n), in a
    platform-specific manner.

    Comments, defined by the metaspec-comment grammar element, are stripped 
    before parsing.
*)

(*
    This section defines literals useful in the definition of the language.
    The UTF-8 literal is defined as all graphemes that can be represented by the
    UTF-8 transformation format as defined in RFC 3629.
    For reference, the special symbols have the following meaning:

        *       repetition
        -       except
        ,       concatenate
        |       disjunction / definition separator
        =       defining
        ;       rule terminator
        []      optional
        {}      repetition
        ()      group
        ? ?     special sequence
*)

utf-8-char = ? all-utf-8-glyphs ?;
text = { utf-8-char }-;
digit = "0" | "1" | "2" | "3" | "4" | "5" | "6" | "7" | "8" | "9";
natural-number = { digit }-,;
integer = [ "+" | "-" ], natural-number;
floating-point-number = natural-number, [ ".", natural-number ];
number = natural-number | integer | floating-point-number;
textual-glyph = utf-8-char - digit;
eol-symbol = ? EOL ?;
literal-quote = ? ASCII-double-quote-symbol ?;
newline = ? \n and \r ?

(*
    This section defines the terminal symbols of the language itself, including:
    - Symbols for grammar definition
    - Symbols for comments in the language
*)

(* Terminals used for defining the grammar *)

repeat-count-symbol = "*";
except-symbol = "-";
disjunction-symbol = "|";
defining-symbol = "::=";
rule-termination-symbol = ";";

optional-start-symbol = "[";
optional-end-symbol = "]";
group-start-symbol = "(";
group-end-symbol = ")";
repeat-start-symbol = "{";
repeat-end-symbol = "}";

special-sequence-start-symbol = "<?";
special-sequence-end-symbol = "?>";

start-symbol-start = "<<";
start-symbol-end = ">>";
non-terminal-start = "<";
non-terminal-end = ">";

(* Comment Symbols *)

line-comment-start-symbol = "//";
block-comment-start-symbol = "(*";
block-comment-end-symbol = "*)";

(* Semantic Definition Symbols *)

semantic-behaves-as = "-->";
evaluates-to = "<=";
where-symbol = ":";
semantic-and = ",";
semantic-assign = "=";

semantic-environment-symbol = "e";
semantic-environment-input-symbol = "<--"
environment-access-symbol = ".";
environment-defines-symbol = ":";
semantic-list-delimiter = ",";
semantic-disjunction = "|";

semantic-block-start = "{";
semantic-block-end = "}";
restriction-block-start = "(";
restriction-block-end = ")";

syntax-access-start-symbol = "[";
syntax-access-end-symbol = "]";

special-syntax-start = "(";
special-syntax-end = ")";

(*
    This section defines the allowed types of identifiers in metaspec.
*)

non-terminal-identifier = 
    textual-glyph, { textual-glyph | natural-number | "-" | "_"};
terminal-string =  { utf-8-char - newline };
semantic-identifier = 
    (textual-glyph, { textual-glyph | natural-number | "-" | "_"})
    - (semantic-type | metaspec-feature | semantic-special-syntax);
string-literal = literal-quote, text, literal-quote;
semantic-type
    = "any"
    | "none"
    | "bool" 
    | "natural"
    | "integer"
    | "int32"
    | "int64"
    | "uint32"
    | "uint64"
    | "float"
    | "double"
    | "integral"
    | "floating"
    | "number"
    | "string"
    | "list"
    | "matrix" ;

(*
    This section defines the grammar of metaspec itself.
    The start symbol is 'metaspec'.
*)

metaspec = { metaspec-def | metaspec-comment }-,; (* file cannot be empty *)

metaspec-comment =
    line-comment-start-symbol, { utf-8-char }, eol-symbol |
    block-comment-start-symbol, { utf-8-char }, block-comment-end-symbol;

comment-list = { metaspec-comment };

metaspec-def = metaspec-defblock, rule-termination-symbol;

(* 
    All of these blocks must be defined once in order.
*)
metaspec-defblock = 
    name-defblock, 
    version-defblock, 
    using-defblock, 
    truths-defblock,
    language-defblock ;

(* names can be arbitrary *)
name-defblock = "name", where-symbol, { utf-8-char }-,;

(* Version strings can be alphanumeric *)
version-defblock = "version", where-symbol, { utf-8-char }-,;

using-defblock =
    "using",
    where-symbol,
    semantic-block-start,
    [ metaspec-feature, { semantic-list-delimiter, metaspec-feature }]
    semantic-block-end;

truths-defblock =
    "truths",
    where-symbol, 
    semantic-block-start,
    semantic-evaluation,
    { ", ", semantic-evaluation },
    semantic-block-end;

(* For defining the language itself *)
language-defblock =
    "language",
    where-symbol,
    semantic-block-start,
    language-definition,
    semantic-block-end;

(* 
    These features import language features into scope.

    The syntax and usage of these features is tbc, and they appear in semantic
    portions of the defined language.
*)
metaspec-feature
    = "base"
    | "number"
    | "string"
    | "list"
    | "matrix"
    | "traverse"
    | "funcall" ;

(*
    The language described by the metalanguage is defined in terms of rules 
    that combine syntax definitions and semantics.

    The start symbol must be defined first, followed by the productions of the
    language.
*)
language-definition = start-rule, { language-rule };

non-terminal = non-terminal-start, non-terminal-identifier, non-terminal-end;
terminal = literal-quote, terminal-string, literal-quote;
start-symbol = start-symbol-start, non-terminal-identifier, start-symbol-end;

start-rule = start-symbol, defining-symbol, language-rule-body;

language-rule = non-terminal, defining-symbol, language-rule-body;

language-rule-body = 
    syntax-expression,
    rule-termination-symbol;

(* These NTs use definitions adapted directly from ISO 14977 - EBNF *)
syntax-expression = 
    syntax-alternative, { disjunction-symbol, syntax-alternative };

syntax-alternative = syntax-term, { syntax-term }, [ language-rule-semantics ];

syntax-term = syntax-factor, [ except-symbol, syntax-exception ];

syntax-exception = 
    ? a syntax-factor that can be replaced by one containing no NT symbols ?;

syntax-factor = [ integer, repeat-count-symbol ], syntax-primary;

syntax-primary = 
    syntax-optional | 
    syntax-repeated |
    syntax-grouped |
    syntax-special |
    non-terminal |
    terminal;

syntax-optional = optional-start-symbol, syntax-expression, optional-end-symbol;

syntax-repeated = repeat-start-symbol, syntax-expression, repeat-end-symbol;

syntax-grouped = group-start-symbol, syntax-expression, group-end-symbol;

syntax-special =
    special-sequence-start-symbol,
    text,
    special-sequence-end-symbol;

(* These productions define the syntax of the semantic definition blocks *)
language-rule-semantics = 
    semantic-behaves-as,
    semantic-block-start,
    semantic-rule,
    semantic-block-end;

semantic-rule = 
    environment-input-rule |
    environment-access-rule |
    special-syntax-rule |
    semantic-evaluation-rule-list;

environment-input-rule =
    semantic-type, 
    semantic-environment-symbol,
    semantic-environment-input-symbol,
    syntax-access-block, (* key *)
    environment-defines-symbol,
    syntax-access-list;

syntax-access-block = non-terminal, syntax-accessor;

syntax-accessor = 
    syntax-access-start-symbol,
    natural-number,
    syntax-access-end-symbol;

syntax-access-list = 
    syntax-access-block, { semantic-list-delimiter, syntax-access-block };

environment-access-rule = 
    semantic-environment-symbol,
    environment-access-symbol,
    syntax-access-block,
    { environment-access-symbol, syntax-access-block };

special-syntax-rule = 
    semantic-special-syntax,
    special-syntax-start,
    [ syntax-access-block | environment-access-rule ],
    { semantic-list-delimiter, (syntax-access-block|environment-access-rule) },
    special-syntax-end;

semantic-evaluation-rule-list = 
    semantic-evaluation-rule,
    { semantic-disjunction, semantic-evaluation-rule };

semantic-evaluation-rule = 
    semantic-type,
    semantic-identifier,
    where-symbol,
    semantic-operation-list,
    semantic-restiction-list,
    where-symbol, 
    semantic-eveluation-list;

semantic-eveluation-list = 
    semantic-evaluation,
    { semantic-list-delimiter, semantic-evaluation };

semantic-evaluation = 
    semantic-block-start
    semantic-type, 
    semantic-identifier,
    evaluates-to,
    [ syntax-access-block | special-syntax-rule ],
    semantic-block-end;

semantic-operation-list = 
    semantic-block-start,
    semantic-operation-assignment
    { semantic-list-delimiter, semantic-operation-assignment },
    semantic-block-end;

semantic-operation-assignment =
    semantic-identifier,
    semantic-assign,
    semantic-operation;

// Revise this section in response to live code changes
semantic-operation 
    = semantic-identifier
    | semantic-value
    | semantic-identifier-access
    | "(", semantic-operation, ")"
    | prefix-semantic-unary-operator, semantic-operation
    | semantic-operation, postfix-semantic-unary-operator
    | semantic-operation, semantic-binary-operator, semantic-operation;

semantic-identifier-access = 
    semantic-identifier, "|", natural-number, "|" | "[", natural-number, "]";

semantic-restriction-list = 
    restriction-block-start
    semantic-restriction,
    { semantic-list-delimiter, semantic-restriction },
    restriction-block-end;

semantic-restriction = semantic-identifier, semantic-restriction-check-operator,
    semantic-value | identifier;

(* The symbols here are dependent on the language using imports *)
semantic-restriction-check-operator = "==" | "!=" | "<" | ">" | "<=" | ">=";

semantic-value = string-literal | number | semantic-boolean;

semantic-boolean = "true" | "false";

semantic-binary-operator = 
    "+" |
    "-" |
    "*" |
    "/" |
    "%" |
    ":" |
    "^" |
    "|" |
    "||" |
    "&&" |
    "&" |
    "==" |
    "!=" |
    "<" |
    ">" |
    "<=" |
    ">=";

prefix-semantic-unary-operator = "!" | "-" | "++" | "--";

postfix-semantic-unary-operator = "--" | "++";

semantic-special-syntax 
    = "map"
    | "fold"
    | "filter"
    | "defproc"
    | "deffun"
    | "callproc"
    | "callfun"
\end{lstlisting}

% chapter the_metaspec_grammar (end)

\chapter{The ABSOL Testing File} % (fold)
\label{cha:the_absol_testing_file}
This appendix contains the test file used to check the functionality of the metacompiler both during and after its development. 
The file is contained in the following listing and defines a toy language that doesn't do anything useful.
Its main intention is to test the parsing functionality of \gls{absol}, as well as the semantic verification facilities. 

\begin{minted}[fontsize=\blockfont]{text}
// Language metadata
name : simple_test;

version : 0.0.1a;

using : {
    number,
    base,
    traverse
};

// Specification of the base-case semantics for this test language
truths : {
    {integer n <= <integer>},
    {floating n <= <floating>},
    {number n <= <number>}
};

language : {

// The start symbol
<<simple_test>> ::= <statement> | <a> ;

// Productions whose semantics should be inferred.
<statement> ::= <my-number> | <number> | <arith-expr> | <test> | <assignment> ;
<my-number> ::= <integer> | <floating> ;

// Should have inferred semantics, mutually recurses with <arith-op>.
<arith-expr> ::= <my-number> | <arith-op> ;

// Mutually recursive productions with infinite syntax.
// These should not impact on successful verification.
<a> ::= <b> ;
<b> ::= <a> ;

// A production that should not have semantics as it's never used semantically.
<empty> ::= ""; 

// Testing user-defined semantics with both kinds of alternations. 
// Testing guard verification.
<arith-op> ::= 
    <arith-expr> "+" <arith-expr> --> {
        number n : {n = n1 + n2}() :
            {number n1 <= <arith-expr>[0]}, {number n2 <= <arith-expr>[1]}
    } |
    <arith-expr> "-" <arith-expr> --> {
        number n : {n = n1 - n2}() :
            {number n1 <= <arith-expr>[0]}, {number n2 <= <arith-expr>[1]}
    } |
    <arith-expr> "*" <arith-expr> --> {
        number n : {n = n1 * n2}() :
            {number n1 <= <arith-expr>[0]}, {number n2 <= <arith-expr>[1]}
    } |
    <arith-expr> "/" <arith-expr> --> {
        number n : {n = n1 / n2}() :
            {number n1 <= <arith-expr>[0]}, {number n2 <= <arith-expr>[1]}
    } |
    <arith-expr> "^" <arith-expr> --> {
        number n : {n = 1}(n1 == 1) :
            {number n1 <= <arith-expr>[0]}, {number n2 <= <arith-expr>[1]} |
        number n : {n = n1 * n1}(n2 == 2) :
            {number n1 <= <arith-expr>[0]}, {number n2 <= <arith-expr>[1]} |
        number n : {n = n1 ^ n2}() :
            {number n1 <= <arith-expr>[0]}, {number n2 <= <arith-expr>[1]}
    };

// Test for parsing and checking of environment input rules. 
<assignment> ::= <statement> "=" <arith-expr> <empty> --> {
    none e <-- <statement>[0] : <arith-expr>[0]
};

// Test for parsing and checking of special syntax rules. 
<test> ::= <statement> "traverse" <statement> --> {
    any map(<statement>[0], <statement>[1])
};

<env-access> ::= <statement> --> {
    any e.<statement>[0]
};

};

\end{minted}

% chapter the_absol_testing_file (end)


\end{appendix}

\end{document}
