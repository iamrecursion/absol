%DocumentClass
\documentclass[a4paper,11pt]{report}

%Additional Packages
\usepackage{floatrow}
\usepackage{scrextend} %for variable indentation
\usepackage[chapter]{algorithm} %For algorithms
\usepackage{algpseudocode} %For algorithms
\usepackage[]{array}
\usepackage{amsmath} %Extra formula-writing functionality
\usepackage{amssymb} %More formula-writing functionality
\usepackage[toc]{appendix}
\usepackage{bm}
\usepackage{bold-extra} %Small caps
\usepackage{caption}
\usepackage{color} %Colour stuff (mostly for the highlight custom command)
\usepackage{enumerate} %For lists
\usepackage{fancyhdr}
% \usepackage{float} %better float control
\usepackage[T1]{fontenc}
\usepackage[utf8]{inputenc} %for font encoding
\usepackage{framed} %For frames around blocks
\usepackage[top=3cm, bottom=3cm, left=3.5cm, right=3cm]{geometry} %Fix page margins
\usepackage[hidelinks]{hyperref} %For URL formleatting (makes clickable links)
\usepackage[toc, xindy, acronym, nonumberlist, nopostdot]{glossaries}
\usepackage{graphicx} %For including images, etc.
\usepackage{listings} %For including code
\usepackage{longtable} %for multi-page tables
\usepackage{mathrsfs} %For maths script fonts
\usepackage{newclude}
\usepackage{titling}
\usepackage{titlesec}
\usepackage[nottoc]{tocbibind}
\usepackage[]{natbib} %For the bibliography
%\usepackage[superscript]{cite}
\usepackage{pdfpages} %For including PDFs
\usepackage[]{pdflscape}
\usepackage{stmaryrd}
\usepackage{subcaption}
\usepackage{tabu} %more pretty tables
\usepackage{tabulary} %for nice tables
\usepackage{tabularx} %also for nice tables
\usepackage{ulem}
\usepackage{xcolor}
\usepackage{xparse}
\usepackage{xstring}
\usepackage{textcomp}
\usepackage[prefix=sol-]{xcolor-solarized}
\usepackage[]{marvosym}
\usepackage{microtype}
\usepackage{lmodern}
\usepackage[chapter]{minted}
\usepackage{epigraph}

\usepackage{relsize}

%CustomCommands
\newcommand{\highlight}[1]{\colorbox{yellow}{#1}} %Highlights text
\newcommand{\limplies}{\to} %Creates the logical implication sign
\newcommand{\liff}{\leftrightarrow} %Creates the double logical implication sign
\newcommand{\leftabs}{\left\lvert} %Left absolute value bracket
\newcommand{\rightabs}{\right\rvert} %Right absolute value bracket
\newcommand{\textbsc}[1]{\textsc{\textbf{#1}}}
%\renewcommand\thesubsection{(\alph{subsection})} %Make subsections alphabetical
\newcommand{\id}{\hspace*{12pt}}
\newcommand{\newpar}{\vspace{12pt}}
\newcommand{\lam}{$\lambda$}
\newcommand{\alp}{$\alpha$}
\newcommand{\bet}{$\beta$}
\newcommand{\aequiv}{=_\alpha}
\newcommand{\bequiv}{=_\beta}
\newcommand{\bconv}{\limplies_\beta}
\newcommand{\context}{$\Gamma$}
\newcommand{\rspace}{\;\;\;\;\;\;\;\;}
\newcommand{\eval}{\Downarrow}
\newcommand{\goesto}[0]{\MVRightarrow}
\newcommand{\mathgoesto}[0]{\mathord{\text{ \goesto\;}}}
\newcommand{\la}{\langle}
\newcommand{\ra}{\rangle}

% Some algorithm shorthand
\newcommand{\Map}[2]{\textbf{map} #1 #2}
\newcommand{\Reduce}[3]{\textbf{reduce} #1 #2 #3}
\algnewcommand{\LineComment}[1]{\State \(\triangleright\) #1}

\makeatletter
\newenvironment{breakablealgorithm}
  {% \begin{breakablealgorithm}
   \begin{center}
     \refstepcounter{algorithm}% New algorithm
     \hrule height.8pt depth0pt \kern2pt% \@fs@pre for \@fs@ruled
     \renewcommand{\caption}[2][\relax]{% Make a new \caption
       {\raggedright\textbf{\ALG@name~\thealgorithm} ##2\par}%
       \ifx\relax##1\relax % #1 is \relax
         \addcontentsline{loa}{algorithm}{\protect\numberline{\thealgorithm}##2}%
       \else % #1 is not \relax
         \addcontentsline{loa}{algorithm}{\protect\numberline{\thealgorithm}##1}%
       \fi
       \kern2pt\hrule\kern2pt
     }
  }{% \end{breakablealgorithm}
     \kern2pt\hrule\relax% \@fs@post for \@fs@ruled
   \end{center}
  }
\makeatother

\newcounter{requirementcounter}
\newcommand*{\reqlabel}[1]{R\refstepcounter{requirementcounter}\therequirementcounter\label{#1}}
\newcommand*{\reqref}[1]{R\ref{#1}}

\newcommand*{\mklabelcase}[1]{\lowercase{\StrSubstitute[0]{#1}{ }{\_}}}

\newcommand{\requirement}[3]{
    \StrSubstitute[0]{#1}{ }{}[\labelname]
    \textbf{Requirement \reqlabel{req:\labelname}}
    \begin{addmargin}[1em]{0em}
        \textbf{Name:} #1\\ 
        \textbf{Type:} #2\\
        \textbf{Description:}\\
        #3\\
    \end{addmargin}
}

% New Table Column Types
\newcolumntype{L}[1]{>{\raggedright\let\newline\\\arraybackslash\hspace{0pt}}m{#1}}
\newcolumntype{C}[1]{>{\centering\let\newline\\\arraybackslash\hspace{0pt}}m{#1}}
\newcolumntype{R}[1]{>{\raggedleft\let\newline\\\arraybackslash\hspace{0pt}}m{#1}}

% Syntax: \newdualentry[glossary options][acronym options]{label}{abbrv}{long}{description}
\DeclareDocumentCommand{\newdualentry}{ O{} O{} m m m m } {
    \newglossaryentry{gls-#3}{name={#5},text={#5\glsadd{#3}},
        description={#6},#1
    }
    % \makeglossaries
    \newacronym[first=#5, firstplural=#5s, see={[Glossary:]{gls-#3}},#2]{#3}{#4}{#5 \glsseeformat[Glossary:]{gls-#3}{#5}\glsadd{gls-#3}}
    % \newacronym[see={[Glossary:]{gls-#3}},#2]{#3}{#4}{#5\glsadd{gls-#3}}
}

% For quoting \quoteit{quote}{attribution}
\newcommand{\quoteit}[2]{
    \begin{longtable}{p{14cm}}
        \textit{``#1''} \\
        % \hspace{5mm} --- #2 
        \begin{tabular}{R{14cm}}
            --- #2
        \end{tabular}
    \end{longtable}
}

% For definitions \defblock{colsize}{name}{description}
\newcommand{\defblock}[3]{
    \begin{longtable}{l p{#1}} 
        \textbf{#2} & #3
    \end{longtable}
}

%%
%% Package includes to provide the basic style
%%
\usepackage{harvard}    % Uses harvard style referencing
\usepackage{graphicx}   % Permits import of various graphics formats
\usepackage{hyperref}   % Provides hyperlinks to sections automatically
\usepackage{pdflscape}  % Provides landscape mode for end code listings
\usepackage{multicol}   % Provides ability to split output into columns
\usepackage{listings}   % Provides styled code listings


%%
%% Set some page size changes from the standard article class
%%
\usepackage{calc}
\setlength{\parskip}{6pt}
\setlength{\parindent}{0pt}
\addtolength{\hoffset}{-0.5cm}
\addtolength{\textwidth}{2.5cm}


%%
%% Format definitions for the style
%%
\bibliographystyle{agsm}  %{alpha}
\citationstyle{dcu}
\pagestyle{headings}
\fussy


%%
%% Definitions to provide layout in the dissertation title pages
%%
\newenvironment{spaced}[1]
  {\begin{minipage}[c]{\textwidth}\vspace{#1}}
  {\end{minipage}}


\newenvironment{centrespaced}[2]
  {\begin{center}\begin{minipage}[c]{#1}\vspace{#2}}
  {\end{minipage}\end{center}}


\newcommand{\declaration}[2]{
  \thispagestyle{empty}
  \begin{spaced}{4em}
    \begin{center}
      \LARGE\textbf{#1}
    \end{center}
  \end{spaced}
  \begin{spaced}{3em}
    \begin{center}
      Submitted by: #2
    \end{center}
  \end{spaced}
  \begin{spaced}{5em}
    \section*{COPYRIGHT}

    Attention is drawn to the fact that copyright of this dissertation rests
    with its author. The Intellectual Property Rights of the products
    produced as part of the project belong to the author unless otherwise specified
    below, in accordance with the University of Bath's policy on intellectual property 
   (see http://www.bath.ac.uk/ordinances/22.pdf).

    This copy of the dissertation has been supplied on condition that anyone
    who consults it is understood to recognise that its copyright rests with its
    author and that no quotation from the dissertation and no information
    derived from it may be published without the prior written consent of
    the author.

    \section*{Declaration}
    This dissertation is submitted to the University of Bath in accordance
    with the requirements of the degree of Bachelor of Science in the
    Department of Computer Science. No portion of the work in this dissertation
    has been submitted in support of an application for any other degree
    or qualification of this or any other university or institution of learning.
    Except where specifically acknowledged, it is the work of the author.
  \end{spaced}

  \begin{spaced}{5em}
    Signed:
  \end{spaced}
  }


\newcommand{\consultation}[1]{%
\thispagestyle{empty}
\begin{centrespaced}{0.8\textwidth}{0.4\textheight}
\ifnum #1 = 0
This dissertation may be made available for consultation within the
University Library and may be photocopied or lent to other libraries
for the purposes of consultation.
\else
This dissertation may not be consulted, photocopied or lent to other
libraries without the permission of the author for #1 
\ifnum #1 = 1
year
\else
years
\fi
from the date of submission of the dissertation.
\fi
\vspace{4em}

Signed:
\end{centrespaced}
}

%%
%% END OF DEFINITIONS
%%



%Various Definitions
\setcounter{tocdepth}{2}
\definecolor{light-gray}{gray}{0.5}

\titleformat{\chapter}
    {\normalfont\huge}  % format
    {\thechapter.}      % label
    {10pt}              % separation
    {\huge\it}          % before-code

% \pagestyle{fancy}
% \lhead{\color{light-gray}Ara Adkins}
% \rhead{\color{light-gray}ABSOL}
% \cfoot{\thepage}
% \renewcommand{\headrulewidth}{0pt}
% \renewcommand{\footrulewidth}{0pt}

\setlength{\parindent}{0pt}
% \setlength{\headheight}{14pt}

\hypersetup{
    colorlinks,
    linkcolor={black},
    citecolor={black},
    urlcolor={black}
}

\makeatletter
\renewcommand\@dotsep{200}
\makeatother

\renewcommand{\labelenumii}{\theenumii}
\renewcommand{\theenumii}{\theenumi.\arabic{enumii}.}

% Listings Styles
\lstset{
    % How/what to match
    sensitive=true,
    % Border (above and below)
    frame=lines,
    % Extra margin on line (align with paragraph)
    xleftmargin=0.5cm,
    % Put extra space under caption
    belowcaptionskip=1\baselineskip,
    % Colors
    backgroundcolor=\color{sol-base3},
    basicstyle=\color{sol-base0}\footnotesize\ttfamily,
    keywordstyle=\color{sol-cyan},
    commentstyle=\color{sol-base01},
    stringstyle=\color{sol-blue},
    numberstyle=\color{sol-violet},
    identifierstyle=\color{sol-base00},
    % Break long lines into multiple lines?
    breaklines=true,
    % Show a character for spaces?
    showstringspaces=false,
    tabsize=4,
    numbers=left,
    numbersep=5pt,
    numberstyle=\tiny\color{sol-base1},
    rulecolor=\color{sol-base01},
    aboveskip=2em,
    belowskip=2em,
    upquote=true,
    basewidth={0.5em,0.5em}
}

\floatsetup[listing]{style=plain}
\newenvironment{longlisting}{\captionsetup{type=listing}}{}
\setminted{
    autogobble=true, % remove common leading whitespace
    breakanywhere=true,
    breakautoindent=true,
    numbers=left,
    mathescape=true, % allow maths symbols
    stripnl=false,
    tabsize=4,
    texcomments=true,
    resetmargins=true,
    % escapeinside=\#\#,
    % xleftmargin=\parindent
}

% \let\origthelstnumber\thelstnumber
% \makeatletter
% \newcommand*\Suppressnumber{%
%   \lst@AddToHook{OnNewLine}{%
%     \let\thelstnumber\relax%
%      \advance\c@lstnumber-\@ne\relax%
%     }%
% }

% \newcommand*\Reactivatenumber[1]{%
%   \lst@AddToHook{OnNewLine}{%
%    \let\thelstnumber\origthelstnumber%
%    \setcounter{lstnumber}{\numexpr#1-1\relax}%
%    %\advance\c@lstnumber\@ne\relax%
%   }%
% }

% \makeatother

\newcommand{\blockfont}{\footnotesize}

\renewcommand{\chapterautorefname}{Chapter}
\renewcommand{\sectionautorefname}{Section}
\renewcommand{\subsectionautorefname}{Subsection}
\renewcommand{\subsubsectionautorefname}{Subsection}
\newcommand{\listingautorefname}{Listing}
% \newcommand{\algorithmautorefname}{Algorithm}

% Title
% \pretitle{
% 	\begin{center}
% }
% \posttitle{
%     \end{center}
% }

\newcommand{\titletext}{ABSOL: Specification and Formal Verification of Domain-Specific Languages through Automatic Compiler Generation in Haskell}

\title{\titletext}
\author{Ara Adkins}
\date{Bachelor of Science, Computer Science with Honours\\The University of Bath\\\today}

% Glossaries
\makeglossaries
\newdualentry{dsl}{DSL}{Domain-Specific Language}{
    An abstraction of the problem domain in the form of a `small' programming language that provides appropriate notations and expressive power focused on a particular problem domain. 
}

\newglossaryentry{semantics}{
    name=Semantics,
    text=semantics,
    description={
        The definition of meaning of syntactically valid (see \gls{syntax}) strings in a language.
        This meaning defines how these programs \textit{behave} when executed, or as the \textit{effect} of executing these programs.
    }
}

\newglossaryentry{syntax}{
    name=Syntax,
    text=syntax,
    description={
        The ways in which the terminal symbols of a programming language may be combined to create programs that are well-formed in the language.
    }
}

\newdualentry{absol}{ABSOL}{Automatic Builder for Semantically Oriented Languages}{
    The \gls{metacompiler} program and toolchain that performs language analysis and verification.
}

\newglossaryentry{metacompiler}{
    name=Metacompiler,
    text=metacompiler,
    description={
        A metacompiler is a program that takes in a specification and generates code for a language compiler as its output. 
        In the case of this project, the metacompiler is responsible for the verification of the input specification, rather than code generation from it. 
    }
}

\newglossaryentry{halting_problem}{
    name=Halting Problem,
    description={
        The Halting Problem describes the problem of determining, from a description of an arbitrary computer program ($\delta$), for a Turing-Machine or equivalent, and an arbitrary input $x$ whether $\delta$ will complete execution or continue to run forever (diverge) \citep{boyer1984mechanical}. 
    }
}

\newglossaryentry{declarative}{
    name=Declarative,
    text=declarative,
    description={
        A style of programming language that allows the user to express the logic of a computation without expressing the exact set of steps to be executed. 
        Examples include some portions of Haskell, and Prolog. 
    }
}

\newacronym{avopt}{AVOPT}{Analysis, Verification, Optimisation, Parallelisation and Transformation}

\newglossaryentry{transpilation}{
    name=Transpilation,
    text=transpilation,
    description={
        The transformation of one high level language $L_1$ into another $L_2$ via the \gls{ast} created from parsing $L_1$. 
        The \gls{ast} is used to generate code in the \textit{target} language ($L_2$), aiming to preserve the semantics of $L_1$ as accurately as possible. 
    }
}

\newdualentry{ast}{AST}{Abstract Syntax Tree}{
    An Abstract Syntax Tree (AST) is a tree-based representation of the syntactic structure of the source code of a program or other unit in some language, without any of the extraneous information required for parsing.
    It is the ideal form for programmatic manipulation of the language unit.
}

\newglossaryentry{source_map}{
    name=Source Map,
    text=source map,
    description={
        A predefined mapping of debug symbols in the target language back to the debug symbols in the source language \citep{kulkarnitranspiler}.
    }
}

\newacronym{bnf}{BNF}{Backus-Naur Form}

\newacronym{ebnf}{EBNF}{Extended Backus-Naur Form}

\newacronym{ffi}{FFI}{Foreign-Function Interface}

\newglossaryentry{metaspec}{
    name=Metaspec,
    description={
        The \gls{metalanguage} designed as part of the \acrshort{absol} project. 
        It is capable of describing both \gls{syntax} and \gls{semantics} of the target DSL with one unified syntax. 
    }
}

\newglossaryentry{metalanguage}{
    name=Metalanguage,
    text=metalanguage,
    description={
        A metalanguage is a language that is, itself, used to describe aspects of another language \citep{jakobson1980metalanguage}. 
    }
}

\newacronym{vcs}{VCS}{Version Control System}

\newglossaryentry{defblock}{
    name=Defblock,
    description={
        A defblock is a top-level definition block in \gls{metaspec}. 
    }
}

\newacronym{ssr}{SSR}{Special-Syntax Rule}

\newacronym{cli}{CLI}{Command-Line Interface}

\newglossaryentry{lexer}{
    name=Lexer,
    text=lexer,
    description={
        A lexer is a program that takes a string of characters and performs a process known as \textit{lexical analysis} to generate a stream of tokens (each of which may consist of one or more characters).
        Each of these tokens has semantic meaning in a given language.
    }
}

\newglossaryentry{parser}{
    name=Parser,
    text=parser,
    description={
        A parser is a program that takes a string of tokens (usually from a \gls{lexer}) and builds an \acrshort{ast} for out of these tokens based on the language grammar. 
    }
}

\newglossaryentry{diverge}{
    name=Diverge,
    text=diverge,
    description={
        A computer program is said to \textit{diverge} if it never terminates or terminates in some unintended or exceptional state.
        (CITE)? 
    }
}

\newglossaryentry{group}{
    name=Group,
    text=group,
    description={
        A group is a structure $(G, \cdot)$ where $G$ is a set and $\cdot$ is an associative binary operation on elements of $G$.
        A group satisfies the following axioms:
        \begin{itemize}
            \item \textbf{Closure:} $\forall a, b \in G, a \cdot b \in G$
            \item \textbf{Associativity:} $\forall a, b, c\in G, (a \cdot b) \cdot c = a \cdot (b \cdot c)$.
            \item \textbf{Identity:} $\exists! e \in G. \forall a \in G, e \cdot a = a \cdot e = a$
            \item \textbf{Inverse:} $\forall a \in G. \exists b \in G, a \cdot b = b \cdot a = e$, where $e$ is the identity as above.
        \end{itemize}
    }
}

\newglossaryentry{semigroup}{
    name=Semigroup,
    text=semigroup,
    description={
        A semigroup is a structure $(G, \cdot)$ where $G$ is a set and $\cdot$ is an associative binary operation on elements of $G$ such that the following axioms hold:
        \begin{itemize}
            \item \textbf{Closure:} $\forall a, b \in G, a \cdot b \in G$
            \item \textbf{Associativity:} $\forall a, b, c\in G, (a \cdot b) \cdot c = a \cdot (b \cdot c)$.
        \end{itemize}
    }
}

\newglossaryentry{monoid}{
    name=Monoid,
    text=monoid,
    description={
        A monoid is a structure $(G,\cdot)$ where $G$ is a set and $\cdot$ is an associative binary operation on elements of $G$ such that the following axioms hold:
        \begin{itemize}
            \item \textbf{Closure:} $\forall a, b \in G, a \cdot b \in G$
            \item \textbf{Associativity:} $\forall a, b, c\in G, (a \cdot b) \cdot c = a \cdot (b \cdot c)$.
            \item \textbf{Identity:} $\exists! e \in G. \forall a \in G, e \cdot a = a \cdot e = a$
        \end{itemize}
    }
}

\newglossaryentry{noop}{
    name=Noop,
    text=noop,
    description={
        A noop or no-operation is a computer instruction that does nothing, but advances the program counter.
    }
}

\newacronym{ide}{IDE}{Integrated Development Environment}

\newacronym{ghci}{GHCi}{GHC Interpreter}

\newacronym{api}{API}{Application Programming Interface}

\glsaddall % add all items to the glossaries

%Document
\begin{document}

% Title
\maketitle
\newpage

% Dissertation Consultation Prohibition
\consultation{3}
\newpage

% Declaration
\declaration{\titletext}{Ara Adkins}

% Abtract Text
\begin{addmargin}[1em]{2em}
\begin{abstract}
    Lorem ipsum dolor sit amet, consectetur adipisicing elit, sed do eiusmod
    tempor incididunt ut labore et dolore magna aliqua. Ut enim ad minim veniam,
    quis nostrud exercitation ullamco laboris nisi ut aliquip ex ea commodo
    consequat. Duis aute irure dolor in reprehenderit in voluptate velit esse
    cillum dolore eu fugiat nulla pariatur. Excepteur sint occaecat cupidatat non
    proident, sunt in culpa qui officia deserunt mollit anim id est laborum.
\end{abstract}
\end{addmargin}
% TODO Write an appropriate abstract for the dissertation.


% Contents
\tableofcontents

\listoffigures

\listofalgorithms
\addcontentsline{toc}{chapter}{List of Algorithms}

\lstlistoflistings
\addcontentsline{toc}{chapter}{Listings}

\printglossaries

% TODO Proof Read
% TODO Formatting

% Acknowledgements
\chapter*{Acknowledgements} % (fold)
\label{cha:acknowledgements}
\addcontentsline{toc}{chapter}{Acknowledgements}
Many thanks to the following people without whom this project would not have been completed:
\begin{itemize}
    \item \textbf{Guy McCusker:} Without whose insight, guidance, support and enthusiasm, the project would not have been possible. 
    \item \textbf{Wyeth Wolnick:} Whose interest, questions and loving support has led to the solutions to many challenges.
    \item \textbf{Willem Heijltjes:} Whose kind willingness to discuss abstract Haskell topics has been crucial.
    \item \textbf{Damask Talary-Brown:} Whose faith and confidence in the project's success has been instrumental. 
    \item \textbf{Hillel Sims (Bloomberg):} Whose team's product's runtime configuration using OCaml inspired the entire project.
    \item \textbf{My Friends:} Whose support has buoyed the project along through its struggles. 
\end{itemize}

% chapter acknowledgements (end)


% Chapters
\chapter{Introduction} % (fold)
\label{cha:introduction}
\epigraph{When asked for a formal semantics of the formal semantics Milner's head explodes.}{James Iry \citep{incomplete_iry}}

Modern software systems are seeing significant increases in complexity, with everything from flight computers to business software becoming harder to develop and reason about \citep{dvorak2009nasa}.
As these systems grow, domain-specific logic becomes integrated throughout their code, leading to increased levels of coupling and more potential for bugs.
With it, such a high level of coupling brings an increased risk that changes to the software cause modes of incorrect operation \citep{khawar2001developing}.\\

In a world where domain-logic is crucial to many large software systems, there has been a ``significant uptick in interest'' in the use of \glspl{dsl} \citep{fowler2010domain}.
A \gls{dsl} is a limited, application-specific language that is integrated with another software system.
The intent is thus to encode the domain rules and logic in a fashion extending beyond the syntactic constraints of the host language, represented using domain constructs and terminology \citep{Mernik:2005:DDL:1118890.1118892}.
Through the use of a \gls{dsl}, a software system is able to avoid the aforementioned dispersion, centralising this domain logic using a common syntax, and making this domain logic easier to both change and verify.\\

However, with this increasing centralisation of such core logic comes a risk: if that logic is wrong, then the entire system is going to act incorrectly.
So what if it were possible to ensure that the \gls{dsl} was correct?
What if it were possible to guarantee that it would always terminate, and always compute the right result?
\textit{This} is the inspiration for this project, this is the reason behind \gls{absol}: providing a toolchain, including a \gls{metalanguage}, \gls{metacompiler} and verification framework for the creation and formal verification of embedded, executable Domain-Specific Languages.

\section{A Problem Worth Studying} % (fold)
\label{sec:a_problem_worth_studying}
\glspl{dsl} are used throughout many industries, and they can bring major benefits in industries where safety is most critical.
However, it is in these very industries that they cause significant problems should they go wrong.\\

During time spent at a financial technology company, one team was observed to be using the \gls{gpl} OCaml for the specification of application logic at runtime. 
With the vastly greater than necessary expressive power offered by this language, the potential for bugs was significantly increased \citep{subramanyam2003empirical}.
What if it were possible to design a capable \gls{dsl} that could be formally verified to replace the use of OCaml?
It would both be a \textit{much} better fit to the domain \textit{and} reduce the potential for bugs.\\

It is this kind of safety-critical system with which this project is concerned.
The ability to provide a method for designing and verifying \glspl{dsl} in these domains would be a boon for the safety of such systems and their runtime configuration behaviour.
While, in general, language verification is undecideable, this project establishes it as a manageable task through limiting the kinds of programs that these languages can represent.
While this is an onerous restriction for \gls{gpl}, but one that is acceptable for a \gls{dsl}, its imposition makes it possible to ensure that a given \gls{dsl} is correct.\\

Correctness, however, is a slightly nebulous term. 
In some areas it is used to mean compliance with the program's specification, while in others it means the production of the correct result.
For the purposes of this project, a language being \textit{correct} means that its semantics are \textit{always} defined, and that all programs in the language are \textit{guaranteed to terminate}. \\

The project hence aimed to produce two main novel contributions to the state of the art of language verification:
\begin{itemize}
    \item \textbf{A Metalanguage:} A language for the specification of both the \gls{syntax} and \gls{semantics} of the \gls{dsl}. 
    It would allow the representation of restricted forms of semantics to ensure that it is possible to prove the language correct.
    As a result, the language specification could then be verified before being used to generate a \gls{dsl} compiler. 
    \item \textbf{A Metacompiler:} A Haskell-based program that aims to generate a compiler, analysing the input language specification for correctness along the way.
    This would ensure that no bugs can be introduced in the \gls{dsl} compiler implementation as it is generated directly from specification.
\end{itemize}

Later during the project it became apparent that, while useful, the code-generation step was not novel.
As it consisted of a large amount of work, the feature was ruled out of scope.
Nevertheless, the final metacompiler system provides both the metalanguage and verification engine as robust contributions to the state of the art. 

% section a_problem_worth_studying (end)

\section{Conceptualising ABSOL} % (fold)
\label{sec:conceptualising_absol}
The entire toolchain developed by this project is known as \acrshort{absol}, the Automatic Builder for Semantically Oriented Languages. 
This term encompasses both the metalanguage and the metacompiler, though is often used to refer to the metacompiler directly.\\

The initial concept for the system was to have a metalanguage for the specification of these \glspl{dsl}, and then have a software program (the metacompiler) that would be capable of processing this specification.
At first, the notion of `processing' encompassed the parsing of the input file, its verification and then the generation of a \gls{dsl} compiler from that verified specification. 
This was a mammoth task, and it swiftly became apparent that while the above stages would be the ideal end result, the project time-frame would not allow for it.\\

Over time, \gls{absol} and its components underwent significant and rigorous design work, further solidifying the roles of each of the parts of the toolchain.
It was through this development that the metalanguage became \gls{metaspec}, and evolved a robust feature-set for the designing of \glspl{dsl}.
This same process resulted in the Metaparse and Metaverify modules of the metacompiler, performing language parsing and specification verification respectively.
At the same time, however, this also led to the generation of the \gls{dsl} compiler being pushed out of the project scope.

% section conceptualising_absol (end)

\section{Outlining the Dissertation} % (fold)
\label{sec:outlining_the_dissertation}
This document aims to provide a comprehensive overview of the design and development of \gls{absol}. 
\begin{itemize}
    \item It will start by providing an in-depth examination of the required background material in the form of the \nameref{cha:literature_and_technology_survey}, giving the reader a firm grounding in the theory required to understand the project as a whole.
    \item Following from the \nameref{cha:literature_and_technology_survey}, the document will examine the context in which the project exists, drawing conclusions and forming project requirements as part of the \nameref{cha:elucidation}.
    \item With the requirements for \gls{absol} firmly established, the document will go on to explore the process by which \gls{metaspec}, the metalanguage, came to exist in \nameref{cha:designing_the_metalanguage}.
    \item With the metalanguage, a precursor to designing the toolchain itself, now established, the following chapter will examine the design of the metacompiler and the algorithms underlying the verification in detail.
    This is \nameref{cha:architecture_and_algorithms}.
    \item Having provided the reader with a comprehensive overview of the design work of the project, the next chapter proceeds to detail the \nameref{cha:implementation} of the metacompiler toolchain.
    \item Implementation complete, the document discusses the \nameref{cha:testing} process applied to the software, highlighting key results and demonstrating an application of the toolchain via an example \gls{dsl}.
    \item Finally, the document concludes with an \nameref{cha:evaluation} of the project as a whole, highlighting the contributions to the state of the art as well as discussing the major successes and deficiencies of the project.
\end{itemize}

% TBC on formality here
This provides but a brief overview of what the reader has in store over the course of this document, so the best thing is to dive right in.
Please, don't forget to enjoy yourself!

% section outlining_the_dissertation (end)

% chapter introduction (end)

\chapter{Literature and Technology Survey} % (fold)
\label{cha:literature_and_technology_survey}
The project has proposed the investigation and development of a toolchain for the creation of provably-correct \glspl{dsl}.
Even with the scope constrained to the creation of \glspl{dsl} alone, such a project draws on a significant breadth of disciplines within Computer Science, necessitating a broad knowledge-base. \\

\gls{absol} exists at the intersection of the study of \glspl{dsl}, the development of programming languages, and formalisations of program semantics. 
As such, it is important to understand the relevant work in these domains. 
While each of these fields is large in its own right, this review aims to distil the relevant bodies of knowledge. \\

This document provides a broad understanding of domain-specific languages, including their types, uses and limitations.
Additionally, it explores the state-of-the-art methods for specifying language syntax and semantics, with an accompanying critical evaluation of these techniques.
The second portion of this review examines methods for formal program verification while identifying the limitations of these techniques as they exist today. 
Finally, it provides an examination of techniques for automated compiler generation, and a discussion of relevant technologies to support such tasks. 

\section{Domain Specific Languages} % (fold)
\label{sec:domain_specific_languages}
\defblock{12cm}{DSL}{``A Domain-Specific Language (DSL) is a programming language or executable specification language that offers, through appropriate notations and abstractions, expressive power focused on ... a particular problem domain'' \citep[pg. 26]{van2000domain}}

While a Turing-Complete, General-Purpose Programming Language (GPL) is capable of expressing any algorithm that can be executed on a standard computer, \citet{fowler2010domain} finds that it is often the case that the use of a \gls{gpl} provides the wrong level of abstraction in a problem domain.
When attempting to express domain knowledge and domain rules using a \gls{gpl}, there is often a mismatch between the \gls{gpl} and the knowledge to be expressed. 
\glspl{dsl}, instead, allow for the expression of solutions ``at the level of abstraction of the problem domain'' \citep[pg. 27]{van2000domain}.\\

Use of a \gls{dsl} allows an encoding of ``bits of important logic that [don't] fit well within [GPLs]'', allowing an expression of domain-expert knowledge at a higher level of abstraction \citep{fowler2010domain,van2000domain}. 
This embodiment of domain knowledge was found by \citet{fowler2010domain} to ``enable a much richer communication channel'' between programmers and domain experts, allowing domain-experts to interact with the configuration and behaviour of complex software systems. \\

The term `\gls{dsl}' encompasses a family of languages, as discussed in Section~\ref{sub:types_of_dsls}, so the definition provided by \citet{van2000domain} at the start of this section is perhaps too restrictive. 
\glspl{dsl} encompass a wide range of programming styles, from declarative to functional, and a similarly varied set of execution models. 
However, in all cases the ``DSLs trade generality for expressiveness in a given domain'', and so the \gls{dsl} approach should be chosen to maximise that expressiveness \citep[pg. 317]{Mernik:2005:DDL:1118890.1118892}. \\

While DSLs offer significant benefits in their encapsulation of domain knowledge, there is significant challenge in creating a DSL to accurately reflect the domain. 
\citet{fowler2010domain} states that DSLs take ``narrow parts of programming'' and make them ``easier to understand and therefore quicker to write, quicker to modify and less likely to breed bugs''. 
Such a statement, however, is only true if the DSL has been implemented correctly, with ``a solid understanding of the domain'' \citep[pg. 1]{bosch1997domain}.\\

Even with such an understanding of the domain, it is possible to make mistakes in the DSL compiler, known as an \textit{application generator}.
Through automatic generation of such tools, this project aims to support the correct implementation of formally correct DSLs through avoiding the potential for mistakes in tooling. 
Such constraints on the DSL make it much more difficult to make damaging mistakes in systems where the DSL is in use. 

\subsection{Types of DSLs} % (fold)
\label{sub:types_of_dsls}
Knowing what a DSL \textit{is} explains little about what forms they might take. 
In reality, the term DSL can be used to describe any limited (non-GPL) programming language, and so one might expect a huge variety in the kinds of DSLs seen in use. \\

In order to better understand the broad variety of DSLs, \citet{van2000domain} proposes a taxonomic classification system, analysing domain-specific languages on five different axes:
\begin{itemize}
    \item \textbf{Execution Strategy:} Domain-Specific Lanaguages can be designed to be \textit{interpreted} (translated from program statements to executable code as they are run), or \textit{compiled} (the same translation performed ahead of time, providing additional opportunities for optimisation and domain-logic checking).
    \item \textbf{Design Strategies:} \citet{van2000domain} claims that a DSL can emerge from a \textit{restricted subset of a GPL}. 
    While this is evidenced by DSLs such as Promela++, a DSL for the construction and validation of protocols, this is limited in its ability to encapsulate domain knowledge \citep{basu1997language}.
    A DSL \textit{designed from scratch} is able to match both syntax and semantics to the domain without restriction. 
    \item \textbf{Implementation Strategy:} \citet{van2000domain} proposes a set of implementation strategies for a DSL:
    \begin{itemize}
        \item \textit{Embedded DSLs:} Such languages use mechanisms that exist in a GPL to define the DSL. 
        The literature surrounding embedded DSLs recognises the limitations imposed by the syntactic and semantic structure of the host language, risking compromises to ``the optimal domain-specific notation'' to work in the host language \citep[pg. 3]{van2000domain}.
        \item \textit{Preprocessed DSLs:} DSLs of this kind translate statements in the syntax of the DSL into statements in a GPL. 
        This, however, also suffers from semantic and syntactic constraints imposed by the macro language or preprocessor. 
        Furthermore, there is no understanding here of the DSL domain, preventing the incorporation of domain-level semantics checking \citep{van2000domain}.
        \item \textit{Compiler Extension:} The preprocessing phase is integrated into a compiler or interpreter, allowing better checking of syntax and types.
        This still provides no ability for checking program semantics at the domain level. 
        \item \textit{Compile from Scratch:} Compilation of DSL program code into executables that can be used from within a GPL. 
        This provides the potential for full domain-level semantic checking, as well as static type-checking and optimisation \citep{van2000domain}.
    \end{itemize}
\end{itemize}

While \citet{van2000domain} provides a useful classification system for Domain-Specific Languages, the work of \citet{Mernik:2005:DDL:1118890.1118892} proposes a connected taxonomy that expands upon the work of \citet{van2000domain}, while proposing additional axes for classification.
The taxonomy of \citet{Mernik:2005:DDL:1118890.1118892} conflates the execution and implementation strategy axes proposed by \citet{van2000domain}:
\begin{itemize}
    \item \textbf{Interpreter:} Recognition, decoding and execution of DSL constructs using a standard interpreter paradigm.
    \citet{Mernik:2005:DDL:1118890.1118892} suggests that this allows high levels of control over the DSL execution environment, and provides for easier language extension.
    \item \textbf{Application Generators:} The translation of DSL constructs to base language constructs and library calls (where the base language may be assembly or a GPL). 
    It is claimed that this allows full static analysis of the DSL program or specification, as the application generator can operate at the semantic level of the DSL.
    \item \textbf{Preprocessing:} Translation of DSL constructs to existing languages, with static analysis limited to that performed by the target language processor.
    \item \textbf{Embedding:} The creation of constructs (new abstract data-types and operators) in a GPL to model a given domain.
    This is most commonly found in the form of an application library. 
    \item \textbf{Compiler Extension:} Extension of an existing compiler or interpreter with domain-specific optimisation rules (e.g. Template Haskell Optimisation Rules, as discussed in Section~\ref{sec:technological_support}).
    \item \textbf{Commercial Off-The-Shelf (COTS):} Application of existing tools and notations to a new problem domain. 
    \item \textbf{Hybrid:} Any combination of the above approaches. 
\end{itemize}

While the classification system by \citet{Mernik:2005:DDL:1118890.1118892} agrees with both \citet{van2000domain} and \citet{fowler2010domain}, it is much broader. 
It proposes further axes for classification, the most important of which are highlighted below:
\begin{itemize}
    \item \textbf{Execution Style:} \citet{Mernik:2005:DDL:1118890.1118892} recognises that Domain-Specific Languages fall onto a spectrum of `executability', as it is termed. 
    This refers to the nature of the execution which the DSL undergoes in use, and has modes as follows:
    \begin{itemize}
        \item Well-defined execution semantics (e.g. Excel Macro Language).
        \begin{lstlisting}
=SUM(COUNTIF(A3:A24, 0))
        \end{lstlisting}
        \item Inputs to applications with a declarative character and less well-defined execution semantics (e.g. ATMOL, a DSL for the specification of atmospheric models, example from \citet{a2001atmol}).
        \begin{lstlisting}
p :: float(0..107000) dim ``Pa''
    field (x(grid), y(grid), z(grid))
        monotonic k(+) on i=1..n by j=1..m by k=1..l        
        \end{lstlisting}
        \item DSLs for application generation intended as non-executable input (e.g. Extended Backus-Naur Form, the example is for the specification of floating-point numbers in Python 3 \cite{Python3Lexical}).
        \begin{lstlisting}
floatnumber   ::=  pointfloat | exponentfloat
pointfloat    ::=  [intpart] fraction | intpart "."
exponentfloat ::=  (intpart | pointfloat) exponent
intpart       ::=  digit+
fraction      ::=  "." digit+
exponent      ::=  ("e" | "E") ["+" | "-"] digit+
        \end{lstlisting}
        \item Non-executable DSLs, such as those for declaration of static application configuration or definition of data structures (e.g. representation of data structures, such as a satellite's coverage \cite{s2001supporting}).
        \begin{lstlisting}
[(sat, gs) | 
    gs <- groundstations,
    sat <- satellites,
    (coverage sat) (location gs)]
        \end{lstlisting}
    \end{itemize}
    Some of these classifications are imprecise, however, and could benefit from further clarification.
    Nevertheless, the most important observation made on this classification axis is that not all DSLs must be executable.
    Domain-specific rules can be encoded as static configuration as well as executable specification, a facet that was left un-addressed by the taxonomy proposed by \citet{van2000domain}.
    \item \textbf{Resuability:} The nature of a DSL is to encapsulate domain-specific logic, configuration and behaviour. 
    \begin{itemize}
        \item \citet{Mernik:2005:DDL:1118890.1118892} recognises that a compiled DSL can act as a portable store for this information that can be reused across multiple systems. 
        \item In contrast, an embedded DSL is non-portable as its implementation is wedded to the language in which it is embedded. 
        This means that similar DSL concepts would have to be implemented again in other languages, increasing the potential for implementation errors. 
    \end{itemize}
\end{itemize}

The much broader classification scheme proposed by \citet{Mernik:2005:DDL:1118890.1118892} demonstrates the limited scope of the analysis provided by \citet{van2000domain}. \\

The literature also notes an important point about compiled DSLs: while embedded DSLs are Turing-Complete, compiled DSLs offer ``possibilities for ... verification ... that would be much harder or unfeasible (sic) if a GPL has been used'', due to their limited scope \citep[pg. 3]{Mernik:2005:DDL:1118890.1118892}.
This is very important to the progress of this project, as it proposes to produce compilers for DSLs in which the programs are provably correct. 

\subsubsection{Implementation Errors in DSLs} % (fold)
\label{ssub:implementation_errors_in_dsls}
With many DSLs acting as portable stores of domain-specific logic and configuration, the impact of errors in DSL implementations is potentially exacerbated.\\

In situations of reuse, an incorrectly implemented DSL will carry the same implementation flaws to everywhere that it is used.
Resultant from this, \citet{Mernik:2005:DDL:1118890.1118892} stresses the importance of performing the domain analysis phase correctly, as the results of such an analysis may persist for a long time. 
If a DSL is going to persist for some time, it is all the more important that the domain analysis be correctly transformed into the DSL implementation. 

% subsubsection implementation_errors_in_dsls (end)

\subsubsection{Detailed Implementation Strategies} % (fold)
\label{ssub:detailed_implementation_strategies}
\citet{Mernik:2005:DDL:1118890.1118892} claims that the application generator or compiled DSL approach and the embedded approach are the two most common DSL implementation strategies. 
As a result, the paper provides a detailed analysis of these strategies. \\

Compiled DSLs, in a general sense, is a category that encompasses any DSL that is translated directly to executable code, whether by an interpreter or compiler \citep{Mernik:2005:DDL:1118890.1118892}. 
Building such tools from scratch allows these DSLs to match the notation of domain experts as closely as possible, significantly reducing the cognitive load required to translate domain rules into program code \citep{fowler2010domain}.
The compiled approach allows opportunities for domain-semantics-related error reporting, rather than reporting based on the semantics of a host language.
Furthermore, it provides opportunities for domain-specific \gls{avopt} that are unmatched by other approaches \citep{Mernik:2005:DDL:1118890.1118892}.\\

Such DSLs are not without their downsides, however, as they require significant development effort due to the requirement for a complex language processor. 
Such processors are rarely designed with extension in mind, resulting in alterations to the DSL being difficult to achieve. 
However, in recognising these deficiencies, \citet{Mernik:2005:DDL:1118890.1118892} fails to note the existence of tools such as compiler generators that significantly decrease this implementation effort \citep{Mandell:1966:MDA:800267.810785}. \\

\citet{Mernik:2005:DDL:1118890.1118892} contrasts the compiled approach with the embedded approach, which refers to any DSL that uses extension mechanisms provided by the host language. 
Such DSLs are an almost perfect counterpoint to compiled DSLs, requiring a far smaller implementation effort due to the ability to reuse existing language features (and produce a more powerful DSL without additional effort). 
Furthermore, such DSLs often benefit from the tooling support around the host language, which has the potential to increase the DSL's ease of use. \\

Embedded DSLs, however, suffer significantly from suboptimal syntax and semantics due to host-language restrictions, with domain-specific constructs and abstractions unable to map to a GPL or GPL-based library \citep{Mernik:2005:DDL:1118890.1118892,van2000domain}. 
While some languages such as Lisp/Scheme allow arbitrary syntax extensions, in most cases this can produce a confusing mismatch between the domain knowledge and its representation \citep{jennings1999verischemelog}. 
These host language restrictions also manifest in the form of error reporting. 
This will take place in terms of the host language constructs, causing a conceptual mismatch between the domain and the language. 
Furthermore, \citet{Mernik:2005:DDL:1118890.1118892} suggests that the embedded approach restricts \gls{avopt}, though there are some languages that can assist with this \citep{seefried2004optimising}.\\

While these categories were established by \citet{van2000domain}, \citet{Mernik:2005:DDL:1118890.1118892} recognises that these approaches are less restricted than they were initially credited as being. 
These two approaches are again corroborated by \citet{fowler2010domain}, who proposes the concepts of Internal DSLs (written in the host language and exposed via an API) and External DSLs (languages with their own semantics parsed independently of the host language). \\

It seems possible to utilise a hybrid approach that combines the syntactic and semantic freedom of compilation with the language power of embedding. 
Such an approach is discussed in \nameref{ssub:transpilation} below. 

% subsubsection detailed_implementation_strategies (end)

\subsubsection{Transpilation} % (fold)
\label{ssub:transpilation}
All of these classification systems, while useful, fail to explicitly recognise an additional implementation strategy: transpilation. \\

While transpiling a language can technically be classified under `hybrid' approaches, the transpilation process, as defined by \citet{Mernik:2005:DDL:1118890.1118892}, parses the source language into its Abstract Syntax Tree (AST).
The semantics associated with the nodes in the AST are then used to generate code in the target language \citep{Bouraqadi:2016:MPT:2991041.2991051}.
This aims to totally preserve the semantics of the source language, while encapsulated in the different target language. \\

The transpilation process combines some benefits of both the compiled and embedded DSL approaches \citep{kulkarnitranspiler}.
It allows:
\begin{itemize}
    \item \textbf{Flexible Syntax and Semantics:} The DSL's syntax and semantics can be defined independently of the eventual target language. 
    This means that the DSL can match the domain as closely as possible. 
    \item \textbf{Domain-Specific AVOPT:} The source language's parser and compiler stages operate directly on the DSL and can perform analysis and verification based on the Domain-Level concepts. 
    \item \textbf{Domain-Level Error Reporting:} As errors can be detected and reported by the source-language semantic analysis, these can be reported in terms of domain-level concepts. 
    \item \textbf{Powerful Language Features:} As the actual semantic functionality of the DSL is being provided via the target language, complex language features are already available as a translation target.
    This removes the need for them to be recreated as for a normal compiler. 
\end{itemize}

While a transpilation approach brings significant benefits, it does not address issues around tooling for the DSL, nor does it address the issues with lack of extensibility in source language parsers, which would require implementation by hand. \\

Furthermore, while the transpiler is able to catch static errors (such as syntax errors, domain-logic errors and other errors expressed in program syntax), any runtime errors will still be expressed in terms of the target language's syntax and semantics. 
This issue can be addressed by a source-mapping technique, in which the debug symbols for the target language are mapped onto the debug symbols for the source language, but this requires additional work in the transpiler to maintain this information \citep{kulkarnitranspiler}.

% subsubsection transpilation (end)

% subsection types_of_dsls (end)

\subsection{Uses of DSLs} % (fold)
\label{sub:uses_of_dsls}
DSLs provide value in two main ways: ``improving productivity for developers and improving communication with domain experts'' \citep{fowler2010domain}. 
While DSLs are generally \textit{small} languages, as discussed by \citet{van2000domain}, they take a variety of forms.
\citet{fowler2010domain} shows that these forms range from small, declarative programming languages, to statically defined specifications and even include traditional functional programming-based rulesets for domain functionality.\\

Due to the wide variety of forms that DSLs can take, they are found in a comparatively wide range of uses.
Some of the key uses are listed below:
\begin{itemize}
    \item A ``common text that acts both as executable software and a description that domain experts can read to understand how their ideas are represented in a system'' \citep{fowler2010domain}.
    Much of the literature focuses on DSLs as enablers of communication through defining important program concepts in terms of the domain at hand.
    \item A form of `end-user programming', where domain experts are able to use programming to perform tasks or otherwise configure their tools \citep{van2000domain}.
    \item A portable form of domain-specific rules, configuration and behaviour. 
    Such ``executable specifications'', as they are termed in \citet{fabry2015taxonomy} provide a centralised repository of domain knowledge in a form that is easy to understand and reason about. 
    \item Improving testability and allowing validation of domain logic through expression of the logic in domain concepts \citep{van2000domain}.
    The opportunities for AVOPT are extended significantly in compiled (external) DSLs \citep{Mernik:2005:DDL:1118890.1118892}.
    \item Provides ``substantial gains in expressiveness and ease of use'' \citep[pg. 2]{Mernik:2005:DDL:1118890.1118892} over GPLs due to the tailored encoding of domain constructs and semantics. 
\end{itemize}

In all of the above situations it is important that the DSL both represents the correct domain constructs, and that the behaviour of the DSL is as intended by the language designers.
To this end, \citet{van2000domain} suggests a rigorous design methodology for a DSL, performing domain analysis and first implementing it as a library in a host language. \\

This reflects a seemingly predominant view of DSLs as evolutions of object-oriented design frameworks or application libraries designed for modelling a problem domain \citep{van2000domain,Mernik:2005:DDL:1118890.1118892}. 
While this is true, it reflects a limited understanding of the capabilities of DSLs, with small external languages featuring a much greater range of syntactic and semantic expressive power through their ability to encapsulate domain concepts \citep{fowler2010domain}.
Other language interface paradigms, such as use of a Foreign-Function Interface can permit far more flexible integration with a DSL.\\

In all cases, the key role of a DSL is to encapsulate the knowledge and rules embodied by a domain in a form that allows the validation of that knowledge. 
To that end, it is important to ensure that the resultant DSL accurately captures the semantics of the domain \citep{fowler2010domain}.

% subsection uses_of_dsls (end)

\subsection{Designing Domain-Specific Languages} % (fold)
\label{sub:designing_domain_specific_languages}
The design of DSLs can be a task that is ``sometimes error-prone and usually time-consuming'' \citep[pg. 1]{karsai2014design}. 
This is due to the complex domain-analysis process that has to take place, elucidating the relevant domain constructs and semantics into a form ready for implementation. \\

Despite this difficulty, tool support for the design of DSLs usually encompasses the implementation phase, with little support for the analysis phase. 
Such tools are often encapsulated in a ``language development system'', which is capable of generating tools from a description of the DSL. 
Such tools may include consistency checkers, language interpreters or compilers and even Integrated Development Environments (IDEs) with integrated editors, analysis tools and debuggers \citep[pg. 19-20]{Mernik:2005:DDL:1118890.1118892}. \\

Such systems often take an opinionated stance on the design of the resultant DSL. 
Sprint, for example, assumes an interpreter for the DSL that uses a partial evaluation technique to lower the overhead of running the DSL program \citep{Consel98architecturingsoftware}.
Environments such as ASF+SDF \citep{van2001asf+}, DMS \citep{baxter2004dms} and Stratego \citep{visser2003stratego}, in contrast, allow definition of the DSL in a variety of forms:
\begin{itemize}
    \item \textbf{Interpretive:} A definition that provides the language semantics for direct execution by an interpreter.
    \item \textbf{Translational:} A definition that provides rules to support a source-to-source transpilation of the DSL into another programming language.
    The target language is usually a GPL \citep{Mernik:2005:DDL:1118890.1118892}.
    \item \textbf{Transformational:} A definition style that specifies the language semantics for direct translation to assembly for direct execution or embedding into another language via a Foreign-Function Interface (FFI) \citep{van2001asf+}.
\end{itemize}

Choice of the implementation style is one of the major decisions to be made as part of the design process.
Each implementation style result in a DSL with different capabilities, including how it can be used from other languages and opportunities for domain-level AVOPT. \\

In order to help make such design decisions, multiple papers offer recommendations.
\citet{Mernik:2005:DDL:1118890.1118892}, for example, suggests that embedded DSLs should be the initial strategy, unless AVOPT is required of the DSL.
Such a suggestion, however, is somewhat fallacious given that the paper previously argued for the importance of DSL syntax matching the domain, and the culpability of embedded DSLs for suboptimal syntax, as discussed in Section~\ref{sub:types_of_dsls}. \\

The choice of implementation strategy and design for a DSL should hence be made carefully, as it can have significant impacts on the usability of the final language. 

% subsection designing_domain_specific_languages (end)

% section domain_specific_languages (end)

\section{Metalanguages} % (fold)
\label{sec:metalanguages}
% TODO Can revisions to this section and the next make things more engaging? Or does that risk clarity?
\defblock{10.5cm}{\Glsname{metalanguage}}{\glsdesc*{metalanguage}}

As implementing verifiable DSLs requires formal descriptions of both the syntax and semantics of a DSL, it necessarily involves the use of one or more metalanguages. \\

Various aspects of DSLs are usually ``developed in terms of specialised metalanguages'', and it is often true that these languages are DSLs themselves \citep[pg. 334]{Mernik:2005:DDL:1118890.1118892}.
These metalanguages are concerned with the specification of a property or some set of properties of the DSL such as its syntax or semantics. \\

Metalanguages are used to make user-defined abstractions into first-class citizens through accurate description of the abstraction \citep{Siek:2010:GPL:1706356.1706358}.
To this end, the project needs to create a combined notation, in itself a DSL, for the specification of both syntax and semantics of a DSL. 
With the field having been in development for ``more than 40 years'', there are a wide variety of metalanguages in use today for the specification of language syntax and semantics \citep{Zhang:2004:SSD:981009.981013}. 
This section aims to explore and evaluate these techniques. 

\subsection{Specifying Language Syntax} % (fold)
\label{sub:specifying_language_syntax}
The syntax of a language defines the formal relationships between the language components (known as non-terminal symbols), and thereby provides a structured description of the valid strings in a language. 
Such formal definitions have three main uses in that they name the syntactic elements, define the valid sentences and provide the syntactic structure of sentences in the language \citep{Scowen:1982:SSM:947912.947917,standard1996ebnf}.

\defblock{11.8cm}{Syntax}{
    Programming language syntax refers to the ways in which language symbols may be combined to create well-formed sentences (otherwise known as programs) in a given languages \citep[pg. 1]{slonneger1995formal}.
}

Languages are defined by a grammar $<\Sigma,N,P,S>$, which consists of four parts \citep{slonneger1995formal}:
\begin{enumerate}
    \item A finite set of terminal symbols $\Sigma$ --- the alphabet of the language that is assembled to make sentences in the language.
    \item A finite set $N$ of nonterminal symbols, which represents the subphrases of sentences in the language.
    \item A finite set $P$ of productions that describe the definition of nonterminals in terms of the terminals and nonterminals in the language. 
    \item A special nonterminal $S$, the start symbol, that identifies the principle category being defined.
\end{enumerate}

In practice, programming language syntax is specified through use of a variant of Backus-Naur Form (BNF), a metalanguage \citep[pg. 21]{Mernik:2005:DDL:1118890.1118892}.
Rules in BNF are specified as follows, with nonterminals represented \lstinline{<category-name>} and productions as follows:
\begin{lstlisting}
    <declaration> ::= var <variable-list> : <type> ;
\end{lstlisting}
where:
\begin{itemize}
    \item \lstinline{var}, \lstinline{:} and \lstinline{;} are terminal symbols in the language. 
    \item \lstinline{::=} is a syntactic construct read to mean ``is defined to be'' or ``is composed of''.
\end{itemize}

\subsubsection{The Chomsky Hierarchy} % (fold)
\label{ssub:the_chomsky_hierarchy}
Languages fall into a set of categories defined by the Chomsky hierarchy \citep{slonneger1995formal}:
\begin{itemize}
    \item \textbf{Level 0 Grammars:} These are unrestricted grammars, which consist of all languages that can be recognised by a Turing Machine.
    Such languages are known as \textit{recursively enumerable}, and require that at least one nonterminal occur on the left side of the rule: $\alpha ::= \beta$
    \item \textbf{Level 1 Grammars:} These are Context-Sensitive grammars, and can be recognised by a linear bounded automaton. 
    These have the additional restriction that the right side contains no fewer symbols than the left side: $\alpha<B>\gamma ::= \alpha\beta\gamma$ where $<B>$ is a nonterminal.
    \item \textbf{Level 2 Grammars:} Known as Context-Free grammars, these are grammars that can be recognised by a nondeterministic pushdown automaton.
    These restrict the left side to being a single nonterminal: $<A> :: \alpha$.
    These correspond to the BNF grammar, and play a major role in the definition of programming languages. 
    \item \textbf{Level 3 Grammars:} These are regular grammars and recognise regular languages, and can be recognised by Finite Automata. 
    These are restricted to allowing only a terminal or terminal followed by a non-terminal on the right side: $<A> ::= \alpha$ or $<A> ::= \alpha<A>$.
\end{itemize}

While there are many categories, most Domain-Specific Languages fall into the category of Context-Free Grammars \citep{Siek:2010:GPL:1706356.1706358}.
However, there is additional contextual information that cannot be defined by standard BNF.\\

Examples of such context-sensitive language features are ``declaration of identifiers before use'' and ``well-typedness of expressions'' \citep{mosses1992action}.
Such syntax can either be defined by attribute grammars (which can be specified in EBNF, discussed in Section~\ref{ssub:extended_backus_naur_form}), or as static program semantics (semantics defined by program structure) \citep{mosses1992action}. 
Such features include:
\begin{itemize}
    \item Well-typedness information
    \item Constraints on programs
    \item Grouping of operations and ordering
\end{itemize}

% subsubsection the_chomsky_hierarchy (end)

\subsubsection{Extended Backus-Naur Form} % (fold)
\label{ssub:extended_backus_naur_form}
Due to limitations of the original BNF, such as an inability to express language elements that were similar to BNF syntax, it has been extended to produce many ``slightly different notations'' that are in use today \citep{standard1996ebnf}.\\

The EBNF is standardised in ISO Standard 14977, and includes the most widely adopted extensions to the original BNF \citep{standard1996ebnf}.
More importantly, the standard has been designed such that ``various extensions can be made in a natural way''.
While this was originally intended to support multi-level grammars (and hence allow BNF to support context-free grammars), ``the format between the special sequence characters [is] almost completely arbitrary'' \citep[pg. vii]{standard1996ebnf}.
This means that it could also be used to support semantic specifications about the grammar, or any other extensions required.\\

EBNF, as a result, provides a flexible, and more importantly extensible, mechanism for specifying language syntax.
These extensions can be used to specify meta-rules about the productions of the language, with \citet{slonneger1995formal} stating the three main forms as:
\begin{itemize}
    \item \textbf{Attributed Syntax Rules:} Rules that define the attributes of productions within a grammar, often used for the definition of context-sensitive syntactical structures. 
    Such rules are evaluated as the language is parsed. 
    \item \textbf{Conditional Rewrite Rules:} Syntactic rules that apply based on the evaluation of some condition to provide an alternative form for a given piece of syntax (e.g. an optimisation). 
    \item \textbf{Transition Rules:} Rules defining allowable transitions between states of the parser. 
\end{itemize}

% subsubsection extended_backus_naur_form (end)
\newpage
\subsubsection{Abstract Syntax} % (fold)
\label{ssub:abstract_syntax}
\defblock{12cm}{\acrshort{ast}}{\glsdesc*{gls-ast}}

While the EBNF definition of a programming language is sometimes referred to as the \textit{concrete syntax} of the language, the language's structure can be expressed in a simplified form as an \textit{\gls{ast}}.
This is produced from a derivation tree (a tree showing the parsed result of a program expression) by removing any information only required by the parser.\\

This simplification enables the abstract syntax to communicate the structure of phrases in terms of their semantics in the programming language, and the rules for producing an AST are similar to the EBNF grammar for the language \citep{slonneger1995formal}.
However, the AST rules factor out extraneous detail to define the language semantics. \\

Nodes in an AST denote constructs in the source code of the program, and it is abstract in the sense that the tree will not explicitly represent all elements of the original source (e.g. grouping by parentheses may be implicit). 
This relationship between concrete and abstract syntax is defined by a relation $\textit{unparse} : \textit{AST} \mathgoesto \textit{concrete syntax}$.
This relation is, ideally, a function, and so restrictions are usually placed on the canonical representations for concrete phrases to disambiguate the reversal process. 
Correct application of unparse to the AST is able to demonstrate the correctness of the parsing algorithm \citep[pg. 29]{slonneger1995formal}.

% subsubsection abstract_syntax (end)

% subsection specifying_language_syntax (end)

\subsection{Specifying Language Semantics} % (fold)
\label{sub:specifying_language_semantics}
\defblock{11cm}{Semantics}{
    The semantics of a language defines the meaning of syntactically valid strings in the language.
    This can be viewed as either the \textit{behaviour} followed when executing a program in this language \citep{slonneger1995formal}.
    Alternatively, it is the \textit{effect} of executing these syntactically valid strings \citep{hennessy1990semantics}.
    These definitions differ due to the multitude of semantic options available. 
}

Much of the literature agrees that formal semantic description is key to the design of, and reasoning about, programming languages \citep{Zhang:2004:SSD:981009.981013,mosses1986use,Binsbergen:2016:TSC:2892664.2893464}.
However, most common semantic description methods are inadequate for describing the complex semantics of real-world GPLs, meaning that most language standards use natural-language descriptions of semantics, leaving much scope for ambiguity \citep{mosses1986use}.
This inadequacy is compounded by an almost complete lack of tooling to support the formal semantic definition of programming languages \citep{Binsbergen:2016:TSC:2892664.2893464}.\\

As the parser is a program that maps syntactically valid strings to an AST, program semantics can be modelled by semantic functions: expressions that map the abstract syntax of a program to the semantic entity that represents the program behaviour \citep{mosses1992action}:
\begin{equation}
    f_s(\text{abstract syntax}) \mathgoesto \text{semantics}
\end{equation}

\citet{mosses1992action} defines two semantic functions to be equivalent, at a given level of decomposition, if two phrases of program syntax are interchangeable without altering the meaning of the program. 
Consider, for example, the following Haskell code:
\begin{lstlisting}[language=haskell]
let x = sort(xs)
let y = heapsort(xs)
-- x == y -> true
\end{lstlisting}
In this case, both program phrases will have the same semantic meaning, as both ingest the list \lstinline[language=haskell]{xs} and return a sorted version of that list.\\

Any compositions of such phrases with the same semantics are called \textit{fully abstract}. 
When the semantics of a phrase composed of sub-phrases depends only on the semantics of its sub-phrases this is known as a \textit{compositional} semantics. 
While such semantics are simpler to reason about, not all program expressions can be represented as compositional semantics \citep{mosses1992action}.\\

Such notions of program semantics take into account only the result of the expression, and do not account for any complexity of the algorithm.
While such concerns are rarely considered as part of the program semantics, they can be useful when considering program synthesis from semantics \citep{kanovich1991efficient}.\\

In practice, there is a wide variety of semantic description frameworks, with the diversity suggested to stem from the fact that program behaviour ``exhibits far greater complexity than program structure'' \citep[pg. 14]{Zhang:2004:SSD:981009.981013}.
The paper suggests that the main semantic description frameworks are:
\begin{itemize}
    \item Operational Semantics
    \item Denotational Semantics
    \item Axiomatic Semantics
    \item Hybrid Semantics
\end{itemize}

These different frameworks are concerned with different elements of program execution. 
Some frameworks concern themselves with only the results of execution, while others are concerned with the details of execution and program continuation structure during statement execution \citep{mosses2001varieties}.

\subsubsection{Operational Semantics} % (fold)
\label{ssub:operational_semantics}
There are multiple kinds of operational semantics, but the two most common semantic frameworks are \textit{Structural Operational Semantics (SOS)} and \textit{Natural Semantics}.\\

Structural operational semantics (also known as small-step operational semantics) specify transition relations (semantic functions) that are characterised by phrase transitions depending only on the transitions of one or more of its sub-phrases \citep{Zhang:2004:SSD:981009.981013}:
\begin{itemize}
    \item Transition relations in this semantic framework are specified as sets of axioms and inference rules.
    \item Each transition modifies the syntax part of the state as a reflection of a portion of the execution of some sub-phrase. 
    \item When execution of a semantic sub-phrase is completed, the sub-phrase is replaced by the resultant value. 
\end{itemize}

Such semantics have seen wide use in program analysis due to their fine-grained nature, but this granularity has the potential for detail overload in the definition of programming languages. 
An example of such semantics can be seen in the following equation:
\begin{equation}
    [\text{asgn}_\text{sos}]: \langle x := \alpha, s \rangle \rightarrow s[x \mapsto \mathbb{A} \llbracket \alpha \rrbracket s]
\end{equation}

This equation defines a rule for assignment, stating that assigning the value of $\alpha$ to the variable $x$ in state $s$ results in a new state $\mathbb{A}\llbracket \alpha \rrbracket s$ where the value of $x$ is $\alpha$.\\

Natural semantics (also known as big-step operational semantics) exists as an alternative, hiding more execution details than SOS:
\begin{itemize}
    \item This semantics is concerned with the description of how the overall computed result of an expression was obtained, rather than the description of the individual steps.
    \item Evaluations in Natural semantics are also specified as sets of axioms and accompanying inference rules. 
\end{itemize}

While natural semantics has seen ``extensive'' use in the definition of programming languages, it is not suitable for describing concurrent or interleaved execution due to the lack of intermediate states \citep{Zhang:2004:SSD:981009.981013}.
An example of such semantics can be seen in the following equation:
\begin{equation}
    [\text{if}^\text{ tt}_\text{ ns}]: 
    \frac
    {\langle S_1, s \rangle \rightarrow s'}
    {\langle \text{if } b \text{ then } S_1 \text{ else } S_2, s \rangle \rightarrow s'} 
    \text{ if } \mathbb{B}\llbracket b \rrbracket s = {tt}
\end{equation}

This semantic description defines the rule for an `if' statement, evaluating statement $S_1$ if the value of $b \in s$ evaluates to $tt$, else evaluating $S2$.\\ 

While the above semantic frameworks have poor modularity, further operational semantics exist, such as the Modular Operational Semantics proposed by \cite{mosses2004modular}.
This is a variant on SOS that restricts states to syntax and computed values.
It incorporates all auxiliary entities (e.g. memory stores, loads and environments/continuations --- data structures that model the computational process at a given point in the program's execution) as labels on transitions.
The hope is that this brings additional modularity to the semantics to allow for the semantic phrases to be reused when embedded in more complex programming languages, but this has seen little uptake in practice.

% subsubsection operational_semantics (end)

\subsubsection{Denotational Semantics} % (fold)
\label{ssub:denotational_semantics}
Denotational semantics model the behaviour of program phrases through use of \textit{denotations} --- mathematical objects that are usually continuous functions \citep{Zhang:2004:SSD:981009.981013,mosses2001varieties}.
\begin{itemize}
    \item These functions reflect the contribution of a given program phrase to the overall program behaviour, focusing on the result of a given computation rather than how the result was obtained. 
    \item Program semantics can be defined through the composition of these denotational functions.
    \item They also provide the ability to denote sequencing via the use of continuation passing as arguments to the semantic functions. 
\end{itemize}

While such semantics provide a high-level overview of the computational semantics, they may not provide enough computational detail for use in a metacompiler system.
An example of denotational semantics can be seen below \citep{Zhang:2004:SSD:981009.981013}.
\begin{equation}
    S_{ds} \llbracket x := a \rrbracket = \lambda s.s [x \mapsto \mathbb{A}\llbracket a \rrbracket s]
\end{equation}

This semantic definition also defines a rule for assigning the value $\alpha$ to the variable $x$ in the state $s$. \\

Denotational and Operational semantics can be written to be fully equivalent; such semantics are known as \textit{fully abstract} \citep{mosses2001varieties}.

% subsubsection denotational_semantics (end)

\subsubsection{Axiomatic Semantics} % (fold)
\label{ssub:axiomatic_semantics}
Axiomatic Semantics describe the properties of a program as a set of constraints on behaviour, with programs modelled as transformations of these constraints \citep{Zhang:2004:SSD:981009.981013}.
These constraints are known as \textit{assertions}, and were initially developed to formalise the verification of abstract algorithms.
\begin{itemize}
    \item Axiomatic semantics are not particularly well suited to producing descriptions of programming languages due to their highly abstract nature.
    \item They are, however, suitable for proving properties about the programs that they describe, as the assertions act as rules to constrain program behaviour \citep{mosses2001varieties}.
\end{itemize}

An example of axiomatic semantics can be seen below \citep{Zhang:2004:SSD:981009.981013}.
\begin{equation}
    [if_{as}]: 
    \frac
    {\{t \lor P\} S_1 \{Q\}, \{\lnot t \lor P\} S_2 \{Q\}}
    {\{P\} \text{ if } b \text{ then } S_1 \text{ else } S_2 \{Q\}}
    \text{ where }
    t = \mathbb{B}\llbracket b \rrbracket 
\end{equation}

Much akin to the rule shown for the natural semantics, this axiomatic rule defines the properties of an `if' expression. 
It states that if the expression $t$, which has the value of evaluating the boolean expression $b$ is true, then expression $S_1$ is evaluated producing a state $Q$, otherwise the expression $S_2$ is evaluated, also producing a new state $Q$. 

% subsubsection axiomatic_semantics (end)

\subsubsection{Hybrid Semantics} % (fold)
\label{ssub:hybrid_semantics}
Hybrid Semantics is a term used to refer to any combination or augmentation of the previously listed frameworks to produce a new system for semantic description \citep{Zhang:2004:SSD:981009.981013}.\\

There have been multiple hybrid frameworks, but one of the most studied is the Action Semantics framework designed by \citet{mosses1992action}, and used in multiple projects such as those by \citet{brown1992actress} and \citet{diehl1996semantics}.
Action semantics recognised the inability of the traditional semantic description methods to scale to more complex programming languages such as the GPLs used in day-to-day systems programming.
\begin{itemize}
    \item It improves the modularity of denotational semantics by treating denotations as actions to be defined using a fixed \textit{action notation} consisting of primitives and combinators.
    \item It provides direct support for the specification of control flow, data flow, scoping and concurrent communication \citep{mosses1992action,doh2001composing}.
    \item As it is based upon operational and denotational semantics, it can be used to verify properties of the programs it is used to specify.
    \item Action semantics was extended to form Modular Action Semantics, which encapsulated the semantic description of each language construct in a separate module to allow for reuse \citep{Zhang:2004:SSD:981009.981013}. 
\end{itemize}

An example of action semantics can be seen below \citep{Zhang:2004:SSD:981009.981013}.
\begin{equation}
    \textbf{execute} \llbracket x := a \rrbracket = (\text{evaluate } a \textbf{ then } \text{store the primitive value in the cell bound } x)
\end{equation}

Action Semantics, however, is limited in that ``not all programming language concepts can be directly represented within action semantics'' \citep[pg. 3]{wansbrough1997modular}. 
This restriction is imposed by the set of computational constructs originally envisioned by \citet{mosses1992action}, with the system proving hard to extend.\\

In the search for a more modular system, \citet{wansbrough1997modular} proposed a system called Modular Monadic Action Semantics (MMAS) based upon both Action Semantics and the earlier Modular Monadic Semantics.
This system aimed to combine the benefits of both systems into a single semantics framework.
\begin{itemize}
    \item MMAS provides a truly extensible version of Action Semantics, allowing the representation of additional programming concepts such as first-class continuations that cannot be expressed in Action Semantics \citep{wansbrough1997modular}.
    \item It replaces the original Structural Operation Semantics that underlie Action Semantics, maintaining the readability of Action Semantics alongside the flexibility of the Modular Monadic Semantics, allowing the semantic model to be refined or extended based upon the application to model new forms of computation.
\end{itemize}

% subsubsection hybrid_semantics (end)

\subsubsection{Modularity of Language Semantics} % (fold)
\label{ssub:modularity_of_language_semantics}
Despite the wide variety of ways in which language semantics can be specified, it is still an open problem to develop a truly flexible semantic framework. 
\cite{Churchill:2014:RCS:2577080.2577099} states that ``various semantic frameworks do not have good modularity'', and this is evidenced by the issues seen when extending existing semantic frameworks \citep{wansbrough1997modular}.\\

Simple semantic specifications such as Operational and Denotational semantics suffer from issues with extensibility and modularity, even though ``various programming constructs are common to many languages'' \citep{Churchill:2014:RCS:2577080.2577099,Zhang:2004:SSD:981009.981013,mosses2001varieties,mosses2004modular}. 
While it is possible to specify a limited semantic framework that encompasses a set of computational actions, it is very possible to create a new kind of computation that does not fit easily into the existing framework \citep{wansbrough1997modular}. \\

While there has been much research into the subject, with proposals such as Modular Monadic Semantics, Modular Action Semantics and even Modular Monadic Action Semantics, none of these frameworks have seen particular use \citep{wansbrough1997modular,mosses1992action,Zhang:2004:SSD:981009.981013,Mosses:2009:CS:1596486.1596489,mosses2001varieties}.
Peter Mosses has written extensively about modular semantic frameworks as embodied by his PLanCompS project, but such semantic models that he terms `Funcons' have seen little uptake in practice \citep{Mosses:2009:CS:1596486.1596489,Churchill:2014:RCS:2577080.2577099,Binsbergen:2016:TSC:2892664.2893464}.

% subsubsection modularity_of_language_semantics (end)

% subsection specifying_language_semantics (end)

% section metalanguages (end)

\section{Formal Program Verification} % (fold)
\label{sec:formal_program_verification}
Having designed the syntax and semantics of a domain-specific language, there still needs to be some mechanism by which the properties of a language can be proved. \\

The notion of program correctness and program verification refers to the ability to prove, via formal methods, that a computer program is ``totally correct''.
A program is called ``totally correct'' when it can be shown to both \textit{terminate} and \textit{perform the operations as defined by its specification} \citep{manna1974axiomatic}.
In general this is an undecidable problem \citep{walther1994proving}.\\

Proving program correctness is a complex task, and in many cases requires an examination of the environment in which the program executes.
With a total understanding of this \textit{environment} and a set of \textit{well-defined program semantics}, it is theoretically possible to apply deductive reasoning to sets of axiomatic program properties to reason about the program's correctness in that environment \citep{Hoare:1969:ABC:363235.363259}.\\

While \citet{Hoare:1969:ABC:363235.363259} and \citet{manna1974axiomatic} propose a generic axiomatic framework for reasoning about computer programs, the most appropriate framework to prove properties using is the well-specified semantics of the programming language in which the program is written.
Under such circumstances, if the language semantics can be proven correct regardless of the execution environment (for a subset of program operations as discussed in Section~\ref{sub:data_and_codata}), it is possible to prove properties of programs in that language.

\subsection{Data and Codata} % (fold)
\label{sub:data_and_codata}
While, in general, it is an undecidable problem to determine if an arbitrary program in a Turing-Complete language is correct, it is possible to restrict the set of allowable operations to a set that can be shown to terminate \citep{walther1994proving}.
Basing the semantics of a DSL upon these allowable operations will, theoretically, allow for a DSL that can be shown to be correct.\\

This restriction of allowable computational operations is based on the duality of \textit{data} and \textit{codata}.\\

Data is captured by the notion of \textit{inductive} data types, whose elements can be constructed in a finite number of steps.
This means that properties of data can be shown by well-founded induction on recursive programs \citep{hinze2010reasoning}.
\begin{itemize}
    \item As a result, it is possible in general to prove that well-founded recursive programs will terminate, and can hence be proven correct.
    \item While it is undecidable to prove termination for general recursion (unlike primitive recursion, for which it is always provable), it is shown by \citet{nordstrom1988terminating} that it is possible to prove termination for well-founded general recursion (recursion which operates on data, not unbounded constructs). 
    This is represented by the following recursion rule, which states that well-founded general recursion can be equated to primitive recursion over the natural numbers as long as $A$ is well-founded by $\prec_A$:
    \begin{equation}
        \frac{
            \text{Wellfounded}(A, \prec_A)\;\;\;\;
            p \in A\;\;\;\;
            e(x, y) \in C(x) [x \in A, y(z) \in C(z) [z \prec_A x]]
        }{\text{rec}(e, p) \in C(p)}
    \end{equation}
\end{itemize}

Dual to data is the concept of codata.
While data is defined inductively, with elements constructed in a finite number of steps, codata is constructed additively from a base-case, allowing it to represent infinite structures such as streams and infinite trees \citep{hinze2010reasoning}.
While it is possible to reason about data using total functions (those shown to terminate, have no side effects and not return error states), the same cannot be said for codata.\\

Reasoning about codata instead requires a technique known as \textit{coinduction} which describes how to destructure (break-down) codata to permit well-founded reasoning about it. 
Coinduction allows reasoning about the termination properties of codata, but is unable to prove termination of algorithms operating on codata in the general case. 

% subsection data_and_codata (end)

\subsection{Proving Termination} % (fold)
\label{sub:proving_termination}
Through a separation of data and codata, it is possible to define a language whose semantics can be reasoned about purely by well-founded induction, as defined by \citet{nordstrom1988terminating}. 
This encompasses proofs on arbitrary length, finite data structures (those that are, hence, well-formed); inductive reasoning applied to such structures is guaranteed to reach a base-case, as examined in Godel's System-T and System-F \citep{alves2010godel,girard1989proofs}.\\

This would, hopefully, allow it to be shown that all well-formed programs in the language terminate and behave according to their specification (i.e. that they are \textit{correct}).\\

Such a proof requires the definition of the following relation for a program $M$ with unique configurations $s$:
\begin{equation}
    s, M \rightarrow s', M'
\end{equation}
where:
\begin{itemize}
    \item A configuration $s$ refers to any additional computational state (which may include heap state, the continuation, etc).
    \item The relation $\rightarrow$ is termed ``converges to'', and is inductively defined.
\end{itemize}

However, the inductive definition of the convergence relation does not guarantee that the relation is total.\\

If it can be shown, by induction on the structure of $M$, that the rules by which $\rightarrow$ is inductively defined are defined in such a way that the convergence hypothesis for $M$ is given in terms of the sub-programs of $M$, the result follows as long as each base-case terminates. \\

If this holds, then it can be shown for every program $M$ and every unique configuration $s$, there exists a program and configuration $s', M'$ such that $s, M \rightarrow s', M'$.\\

If such a property can be shown for the all base-cases and each semantic construct in the language, it is possible to state that all programs written in this language terminate. 

% subsection proving_termination (end)

\subsection{Total Functional Programming} % (fold)
\label{sub:total_functional_programming}
The distinction is accurately represented by \citet{turner2004total}, where the notion of disciplining the use of a functional programming language to exclude the possibility of non-termination is proposed. 
The paper draws the same distinction between data and codata, and restricts the language to the use of only total functions. \\

In order to better illustrate this, \citet{turner2004total} proposes an augmentation to the language typing discipline, adding an explicit $\bot$ (bottom, a type which has no values) to denote functions that may error or not terminate (those functions that operate on codata). 
Languages such as Haskell already incorporate an expression for this idea via the notion of Monads and Monad Transformers.
Turner aims to omit any incidence of $\bot$ in a total functional program, hence restricting the program to the use of total functions, and omitting the use of partial functions. \\

While this is not feasible in practice for a general purpose programming language, it is interesting for a DSL, as all functions could be constrained to being total.
This can be done through imposition of the following restrictions:
\begin{itemize}
    \item Recursion is used to traverse data.
    \item Corecursion is used to traverse codata, where all the infinite structures (and corecursive functions) are total. 
\end{itemize}

Such a language is not Turing-Complete, and hence cannot express all programs (even if such programs terminate). 
Such a restricted language is unable, for example, to express its own parser \citep{turner2004total}.\\

Nevertheless, the notion of Total Functional Programming has interesting implications for the design of provably correct DSLs.

% subsection total_functional_programming (end)

% section formal_program_verification (end)

\section{Automating the Generation of the Compiler} % (fold)
\label{sec:automating_the_generation_of_the_compiler}
\defblock{10cm}{Compiler Generator}{
    A compiler generator, or metacompiler, is ``a tool that constructs a compiler automatically, given a syntactic and semantic description of the source language'' \citep[pg. 95]{brown1992actress}.
}

The literature having established that it is possible to specify a language such that it terminates (and behaves as specified) for all programs in that language, some mechanism needs to exist for translating such programs into an executable form.
While it is possible to manually build a compiler for each DSL that is designed, this is a time-consuming undertaking requiring significant effort that may produce implementation defects, resulting in a non-correct compiler \citep{Mernik:2005:DDL:1118890.1118892}.\\

In order to verify that the resultant DSL is correct, it is better to utilise a compiler generator, a specific form of \textit{application generator}.
\citet{cleaveland1988building} suggests that application generators ``let you customise and reuse a general software design easily'', providing a significant benefit in the reduction of programming errors.\\

This decoupling of the language specification and implementation, as discussed by \citet{cleaveland1988building} is key to the implementation of correct languages.
As long as the metacompiler is correct, the resultant compilers should maintain the semantics of the input programs \citep{Gray:1992:ECF:129630.129637}.
Centralisation of the semantic properties of DSLs into the metacompiler allows the centralisation of the proof infrastructure to allow automation.

\subsection{Metacompiler Systems} % (fold)
\label{sub:metacompiler_systems}
Metacompilers can, in general, be categorised as syntax- and semantics-directed compilers \citep{Mandell:1966:MDA:800267.810785,diehl1996semantics}.
They ingest definitions of the language syntax and semantics and generate a compiler for that language. \\

Traditional compiler design divides the work of the compiler into multiple phases: syntactical and lexical analysis, semantic analysis, optimisation and code generation. 
Generator programs exist for multiple of these phases:
\begin{itemize}
    \item \textbf{LEX:} A generator for lexical analysers --- programs that convert a sequence of characters into a stream of syntax tokens.
    \item \textbf{YACC:} A generator for LALR (Look-Ahead, Left to Right) parsers --- programs that make sense of the source code based upon a provided grammar.
\end{itemize}

Such systems, however, suffer from a somewhat arbitrary mapping between source and target language constructs, with the translation schemes being predetermined, rather than generated alongside the parser. 
Such translation schemes must be implemented by hand, as they are not automatically generated. \\

\citet{diehl1996semantics} suggests that the generation of the compiler can be directed by \textit{semantics} in addition to the \textit{syntax}.
Doing so provides multiple advantages over handwriting compilers in accordance with a language specification:
\begin{itemize}
    \item \textbf{Correctness:} If the generator can be verified, this implies that the generated compilers are proved correct.
    \item \textbf{Readability:} The specification of a programming language is generally more intelligible than a compiler.
    \item \textbf{Maintainability:} Altering the language specification results in a compiler that automatically supports the new features after regeneration.
    \item \textbf{Portability:} Changing the definition of the target language allows generation of compilers for different architectures without changes to the source languages.
    \item \textbf{Insight:} Generation of compilers via semantics aims to use semantics preserving transformations which relate source code to target code. 
    By tracing these transformations, the target code can be explained. 
\end{itemize}

While these benefits are certainly true, there are some downsides to such generated compilers:
\begin{itemize}
    \item Verification of the metacompiler is a non-trivial task for a GPL.
    \item Programming language specifications, particularly those with formally specified semantics, are not necessarily legible to someone without appropriate experience. 
    This means that writing and altering such specifications could be a significant source of maintenance burden for such compilers.
    \item The design of the metacompiler may not easily permit alteration of the target language.
    It may be easier in practice to target a single language with a simple FFI such as the C FFI (e.g. Haskell), allowing easy interoperability with other programming languages. 
\end{itemize}

The metacompiler system proposed by \citet{diehl1996semantics} utilises an Action Semantics (see Section~\ref{ssub:hybrid_semantics}) based specification of the underlying language, generating a compiler based upon the semantics. 
The compiler targets an abstract machine, adding portability for native code generation, and uses a `Term Rewriting System' to generate both the compiler and program behaviour, deferring modifications of program and state to different portions of the compilation-execution pipeline \citep[pg. 59]{diehl1996semantics}.\\

These Term Rewrite Rules are defined in such a fashion that the semantics of the original language are preserved, ensuring that any properties of the original program still hold in the compiled version.
This means that the generated compiler will produce a program that is semantically correct insofar as the original language specification is semantically correct. \\

Such a capability is key to the correct operation of this project, and so a rigorous Term-Rewriting System could be pursued. 

% subsection metacompiler_systems (end)

\subsection{Program Transformations} % (fold)
\label{sub:program_transformations}
\defblock{10cm}{Transformation}{
    A transformation is a rule that operates on the parse-tree of the source language, providing well-specified constructions in the target language that maintain the semantics of the source language \citep{diehl1996semantics}.
    Such transformations may operate on either an AST (see Section~\ref{ssub:abstract_syntax}) or individual portions of the language grammar \citep{brabrand2003metafront}.
}

The intention of this project is to transpile (see Section~\ref{ssub:transpilation}) the DSL source program to an equivalent Haskell program.
The process of doing so will utilise program transformations, similar to those defined by \citet{brabrand2003metafront} for the Metafront Project.\\

Transformations are defined by \citet{brabrand2003metafront} to have three properties that are important for formal verification of the semantic correctness of the translation:
\begin{itemize}
    \item They are designed to allow only well-founded induction, so termination of the transformations is ensured.
    \item Transformations can be decided statically if they will map legal input to legal output. 
    \item The transformation rules can perform expressive transformations that rearrange ASTs in a non-local manner, while preserving the semantics of the program.
\end{itemize}

The fact that such transformation ensure preservation of the language semantics is significant for their use in a metacompiler system. \\

These semantics, however, are not without their limitations. 
They only allow inductively defined semantics, and hence disallow any action-semantics or monadic semantics representations. 
Additionally, they are only able to operate between context-free grammars. 
As the target language of a metacompiler may not be context-free, or able to be restricted to a context-free subset, the ability to use such transformations in a metacompiler may be limited.\\

Such limited translation rules can be contrasted with the Turing-Complete rewrite system used in language development systems such as ASF+SDF, which operate on the AST of the source language \citep{van2001asf+}.
Such transformations, however, do not provide the same guarantees as the Metafront transformation system given by \citep{brabrand2003metafront}.

% subsection program_transformations (end)

\subsection{Intermediate Representations} % (fold)
\label{sub:intermediate_representations}
Many compiler systems use an Intermediate Representation (IR).
This is as it ``hides details about the target execution platform'', allowing semantic-level optimisation and analysis of the program code \citep[pg. 427]{Zhao:2012:FLI:2103621.2103709}.
It is, however, often the case that these IRs do not have well-defined formal semantics, precluding the proving of properties of programs expressed in these IRs. \\

In an effort to formalise the LLVM IR, their Low-Level Virtual Machine, \citet{Zhao:2012:FLI:2103621.2103709} found that the aggressive optimisations performed by industrial strength compilers often had non-deterministic semantics.
Such semantics preclude the proving of certain termination properties of the transformed programs, even if they use linguistic features that are terminating.
Such optimisation issues can be seen with the Glasgow Haskell Compiler (GHC), where certain heavy optimisations alter the semantic meaning of programs (see \autoref{sub:issues_with_ghc}).\\

While the use of IRs bring benefits in terms of semantic analysis and \gls{avopt}, the complexity of such representations means that for a DSL-based project that it is unlikely to be worth employing one.

% subsection intermediate_representations (end)

% section automating_the_generation_of_the_compiler (end)

\section{Technological Support} % (fold)
\label{sec:technological_support}
A project such as this requires an implementation language with excellent support for symbolic manipulation, domain-specific optimisation, language parsing and control over program execution.
While languages such as Lisp allow first-class syntax definitions, it has less library support for parsing, and less robust execution control due to the default eager evaluation semantics. \\

Haskell, on the other hand, has robust support for metaprogramming, parsing and execution control, providing the \mintinline{haskell}{seq} function to force evaluation of the default lazy semantics. 
This makes Haskell an excellent choice for the building of modular language implementations \citep{hudak1996building}.\\

Languages like Haskell and Lisp have excellent support for symbolic manipulation, and hence would be viable implementation languages for such a project. 
This is to be contrasted with more traditional programming languages such as C++, which are more suited to numeric manipulation.
Such languages, additionally require management of implementation features such as memory, which detracts from the ability to implement the required program features.\\

The premier Haskell compiler in use today is the Glasgow Haskell Compiler (GHC), which provides a robust suite of well-tested language extensions.

\subsection{Template Haskell} % (fold)
\label{sub:template_haskell}
One of these language extensions is Template Haskell, a robust metaprogramming mechanism to allow ``compile-time preprocessing of Haskell Source programs'', allowing the programmer to define new language syntax without compiler modification \citep{Sheard:2002:TMH:581690.581691,Czarnecki2004}.
Template Haskell provides support for the ``algorithmic construction of programs at compile time'', including techniques for language rewriting and optimisation, techniques discussed in Sections~\ref{sub:program_transformations} and~\ref{sub:types_of_dsls} respectively \citep{Sheard:2002:TMH:581690.581691,jones2001playing}.
\begin{itemize}
    \item Through the use of Template Haskell's compile-time IO mechanism, DSL programs can be read as input, and transformed into an output Haskell program for compilation \citep[pg. 9]{Czarnecki2004}.
    \item Template Haskell is capable of introducing new names and syntax, providing a mechanism for compile-time syntax extension \citep{Czarnecki2004}.
    \item It allows the alteration of program semantics through rewriting at compile time, allowing significant flexibility with the implementation of DSLs. 
    \item Through the ability to inspect the program AST as data (using a quoted expression), it can modify and parse the semantics of the input program as required, resulting in an equivalent Haskell program without performance compromises.
\end{itemize}

Furthermore, Template Haskell allows the specification of compile-time rewrite rules for language elements \citep{jones2001playing}.
Such capabilities allow for the optimisation of domain-specific logic through definitions of domain-specific optimisations that may not be visible to the Haskell compiler. \\

The major downside of the capabilities provided by Template Haskell is the introduction of additional complexity. 
While using the metaprogramming mechanism does not restrict the use of any other language features, it can significantly contribute to program complexity.

% subsection template_haskell (end)

\subsection{Language Parsing} % (fold)
\label{sub:language_parsing}
If, on the other hand, a more traditional parsing approach is wanted, Haskell still provides significant and flexible support for doing so. 
Through its extensive library support, Haskell provides significant tooling for the creation of parsers, including Happy and Parsec, two libraries for generating language parsers. \\

Parsec is the more capable library of the two.
It is a monadic parser combinator library that is capable of parsing context-sensitive, infinite-lookahead grammars \citep{leijen2001parsec}.
While traditional parser generators such as YACC use event-based parsing, combinator parsing is unique in that it allows the programmer to write expressions which appear to be language grammars, and yet describe a parser for such grammars.
\begin{itemize}
    \item This avoids the need for significant amounts of preprocessing by harnessing the power of Haskell itself. 
    The strict type-system of Haskell is ideal for the description of embedded DSLs \citep{swierstra2009combinator}.
    \item Parsec features a novel combinator-based implementation technique that enhances the parser efficiency and allow the production of useful error messages \citep{leijen2002parsec}.
    \item Useful error messages decrease programmer burden for the users of the end-result, and hence are a significant benefit to using the library. 
\end{itemize}

The main downside of Parsec is that it provides no guarantees about left-recursion in grammars, which has the potential to cause the parser to hang at runtime. 
Furthermore it does not perform as well as some traditional parsers due to the additional state maintained for error reporting, as evidenced by the existence of Attoparsec \citep{gummelt2011hindsight}. \\

While Parsec is much faster than many parser combinator libraries, it is still too slow for real-time use or efficient parsing of very complex grammars. 
Attoparsec, however, sacrifices readability of error messages for performance, meaning that Parsec itself is still a more sensible choice for the implementation of user-facing compilers. \\

Haskell's library support also provides Alex, a tool for generating lexical analysers. 
While not strictly required for parsing using either Happy or Parsec, it can greatly simplify implementation of more complex parsers through elimination of certain bug-classes through the type system (e.g. parsing a keyword as an identifier).

% subsection language_parsing (end)

\subsection{The Haskell FFI} % (fold)
\label{sub:the_haskell_ffi}
\defblock{12cm}{FFI}{A Foreign Function Interface (FFI) is a mechanism by which a program written in one programming language can call functions written in another programming language.}

A DSL is not useful if it isn't portable.
As discussed in Section~\ref{sub:types_of_dsls}, DSLs with implementations rooted in a given language become specific stores of domain knowledge rather than portable encapsulations.
\citet{Marlow:2004:EHF:1017472.1017479} suggests that the C FFI is the lowest-common-denominator for interactions between languages, and so is a useful target for DSL portability.\\

While there are some issues surrounding the Haskell FFI and concurrency, it is possible to both call C-FFI functions from Haskell, and call Haskell functions using a C-FFI \citep{Marlow:2004:EHF:1017472.1017479,HaskellWikiFFIFromC}.
This allows DSLs compiled with Haskell to act as a lowest-common-denominator interface to the DSL through automatic generation of the C-FFI function-call stubs. 
As a result, the DSL can act as a portable repository of Domain-Specific knowledge and configuration. 

% subsection the_haskell_ffi (end)

\subsection{Desirable Language Properties} % (fold)
\label{sub:desirable_language_properties}
Haskell also contributes additional, desirable language properties to the project:
\begin{itemize}
    \item Haskell has desirable properties when it comes to automated proofs of termination \citep{Giesl:2011:ATP:1890028.1890030}.
    This means that it is possible to automatically prove that, for well-founded implementations, the termination properties of the DSL can be verified in the final transpiled DSL code. 
    This analysis examines some of the linguistic complexities of Haskell, and shows that termination can be shown even in the presence of lazy evaluation, equation definition order, polymorphic typing and potentially infinite codata structures. 
    \item Embedded DSLs are difficult to optimise as the host language's optimising compiler only operates at the level of the host language constructs. 
    Future work may want to increase the efficiency of generated code, and the compile-time metaprogramming provided by Template Haskell to express compile-time optimisation opportunities in a declarative fashion such that the GHC optimising compiler can take advantage of them \citep{seefried2004optimising}.
\end{itemize}

% subsection desirable_language_properties (end)

\subsection{Issues with GHC} % (fold)
\label{sub:issues_with_ghc}
Despite all of the benefits that the use of Haskell and GHC can bring to the project, it is not without its flaws.
GHC is an optimising compiler and, as discussed in Section~\ref{sub:intermediate_representations}, some optimisations can alter the semantic meaning of programs.\\

While optimisations, in general aim to preserve the semantic correctness of programs, there are certain documented optimisations that GHC can perform that may alter the program semantics. 
These optimisations are collectively known as Short-Cut Fusion, and they aim to eliminate successive data-structure allocations in chained functional calls, optimising the code to a tight loop over the final structure \citep{HaskellWikiShortCutFusionCorrectness}.\\

While it is strongly conjectured that such an optimisation cannot create a non-terminating program from a terminating program, it has not been conclusively proven \citep{voigtlander2008semantics}.
This means that projects dealing with the correctness of programs must be careful when utilising certain optimisations in GHC, avoiding any optimisations that may semantically alter the code.

% subsection issues_with_ghc (end)

% section technological_support (end)

\section{Combining the Ideas} % (fold)
\label{sec:combining_the_ideas}
This literature and technology review has examined a broad base of previous work surrounding the creation of provably correct domain-specific languages.\\

It has focused on the study of DSLs themselves, as well as mechanisms for the formalisation of program syntax and semantics. 
Furthermore, it has explored methods for program verification and proving program correctness, and techniques for automated compiler generation. \\

As a result of the survey, it is clear that the project wants to pursue a limited set of DSLs, namely external DSLs, allowing a flexibility of syntax, with transpilation (ensuring semantic preservation) into Haskell.
The DSLs would have their syntax specified using EBNF, and semantics likely specified using some form of operational semantics as a compromise between simplicity and expressiveness. 
Such semantic specifications are hoped to allow the compiler system to prove the termination properties of input programs in the DSL. \\

The project will likely see an implementation of a metacompiler system in Haskell, ingesting the language specifications to produce a compiler for that language. 
The resultant language compilers will use a transpilation approach via program transformation rules to transform input programs in the DSL into equivalent Haskell programs, allowing use from a multitude of host languages. \\

It is hoped that these techniques will produce a flexible compiler system that is capable of generating programs written in provably correct Domain-Specific Languages.

% section combining_the_ideas (end)

% chapter literature_and_technology_survey (end)

% Identification of the requirements capture process
% Explanation of why the chosen technique was utilised over other potential competitors.
% Identify and discuss key requirements, with a focus on areas of challenge, difficulty or conflict.
% Be sure to show appropriate scoping. 
% Identify the areas of requirements analysis and specification that were particularly successful
% Be critical of areas where compromises were required. 

\chapter{Elucidation} % (fold)
\label{cha:elucidation}
With a necessarily broad body of work examined in the literature and technology survey (see Chapter~\ref{cha:literature_and_technology_survey}), it is necessary to consolidate this information as it applies to the project. 
This section aims to provide a concrete understanding of which portions of the research will apply directly to the project, fill in any gaps in the research and provide a high-level specification for the project as a whole.

\section{A Kind of Domain-Specific Language} % (fold)
\label{sec:a_kind_of_domain_specific_language}
As has been previously mentioned, it is computationally infeasible (in fact being an instance of the \gls{halting_problem}) to decide whether an arbitrary programming language will terminate in all cases. 
In order to avoid this, \gls{absol} focuses on \glspl{dsl}, and beyond that a specific type of DSL. \\

This section aims to consolidate the breadth of information presented in the literature review to provide a concrete requirement for the type of \glspl{dsl} with which the project will be dealing. 

\subsection{DSL Execution Strategy} % (fold)
\label{sub:dsl_execution_strategy}
As previously examined, \glspl{dsl} can provide a wide scope of different execution types. 
As this project focuses on the verification of \gls{dsl} \textit{semantics}, it is important that the DSLs considered have some form of executable behaviour. \\

As examined in Section~\ref{sub:types_of_dsls}, \citet{Mernik:2005:DDL:1118890.1118892} proposes two main categories of DSL with executable semantics:
\begin{itemize}
    \item Well-defined execution semantics
    \item \Gls{declarative} inputs to applications
\end{itemize}

While the latter is an interesting case, allowing for users to provide a simple set of configuration denoting the structure of a problem in their domain, the less well-defined execution semantics pose a problem for verification.
As well-defined execution semantics easily permit semantic analysis, \gls{absol} focuses on the first type of DSL mentioned above.\\

This means that creators of these \glspl{dsl} need to be able to fully specify the semantics associated with each kind of program statement. 
These semantics must have a restricted form (see Section~\ref{sub:choosing_a_semantic_form}) to ensure that the verification problem is tractable.  
The restriction does, however, mean that the kinds of programs that the language can represent is limited.

% subsection dsl_execution_strategy (end)

\subsection{DSL Implementation Strategy} % (fold)
\label{sub:dsl_implementation_strategy}
As examined by both \citet{Mernik:2005:DDL:1118890.1118892} and \citet{van2000domain}, one of the key merits of a \gls{dsl} is the ability to enable re-use of program specification. 
As a result, \gls{absol} focuses on providing languages that can be interfaced with from multiple host languages. \\

While it seems that the embedded approach is common due to its ease of implementation, this comes with restrictions around error reporting, and the form of the syntax. 
\gls{absol} will instead provide a fusion of the compiled and embedded approaches:
\begin{itemize}
    \item The language will be described in a metalanguage, allowing the generation of the compiler from both syntax and semantics.
    This allows opportunities for language-level analysis that would be otherwise impossible if the \gls{dsl} was expressed directly in a host language.
    \item DSL programs themselves will be compiled into a target language, with the compilation process allowing for a domain-specific approach to \gls{avopt}.
    \item The target language will need to provide some flexible mechanism for interfacing with other languages, most likely via the C \gls{ffi}, as this is somewhat of a lowest-common-denominator for cross-language interfaces \citep{van2001asf+}.
    \item While such an approach requires conventionally requires significant development effort, the two-stage, metacompiler-based approach taken by \gls{absol} will reduce that burden significantly.
\end{itemize}

Such an approach neatly sidesteps the issues encountered by many embedded DSLs as the user-specified syntax and semantics are not restricted by any host language, thereby avoiding suboptimal syntax or non-domain-oriented semantics, and poor error reporting.
That is not, however, to imply that there are no restrictions on the form of the syntax and semantics here, but only that these restrictions are not resultant from the choice of target language. \\

This approach involves a \gls{transpilation} step, which \citet{kulkarnitranspiler} finds to provide many of the benefits of both the embedded and compiled approaches. 
One particular benefit of such an approach, that \gls{absol} aims to use, is the ability to provide detailed domain-specific diagnostics and error messages, while utilising the power of the target language as much as possible.\\

The selection of a transpilation-based approach is not without its flaws, however as, while static errors can be expressed in terms of the source language semantics, any runtime errors will still be expressed in terms of the target language's semantics. 
This semantic mismatch can be handled via \glspl{source_map}, but this requires significant additional work, and is thus considered as out of scope for this project. 

% subsection dsl_implementation_strategy (end)

% section a_kind_of_domain_specific_language (end)

\section{Languages and Programs} % (fold)
\label{sec:languages_and_programs}
As a project, \gls{absol} deals with the specification of \glspl{dsl}. 
This means that, in addition to dealing with \gls{dsl} \textit{programs}, it first has to deal with \textit{languages}. 
In order to both constrain the things these languages can represent, and have a standard for for interacting with them, \gls{absol} must define a metalanguage. \\

Designing a metalanguage requires methods for specifying both the syntax and semantics of the defined language.
Research showed that there had been little work on metalanguages to perform both tasks at once, and so the onus fell on the project to create one. 

\subsection{Choosing a Syntactic Form} % (fold)
\label{sub:choosing_a_syntactic_form}
Determining the basic syntactic metalanguage --- the form of the syntax rules for the defined language --- proved to be a mostly simple endeavour. 
While both research and industry have used multiple syntactic metalanguages in the past, research found that most modern syntactic specifications use some variant of \gls{bnf}. \\

The standardised variant of \gls{bnf}, known as \gls{ebnf} and defined by \citet{standard1996ebnf}, it provides a flexible syntax for defining context-free language syntax. 
While it also supports the definition of context-sensitive syntax via extension rules, most simple programming languages can be defined using context-free productions. \\

One of the main issues with the standard \gls{ebnf} syntax, however, is the use of the concatenation operator (\lstinline{,}) which results in all productions appearing as lists of terminals and non-terminals.
While this is not a nonsensical representation, it can hamper the intuitive readability of the productions, and so has been changed for this project.

% subsection choosing_a_syntactic_form (end)

\subsection{Choosing a Semantic Form} % (fold)
\label{sub:choosing_a_semantic_form}
The literature survey explored multiple methods for the expression of language (and program) semantics.
Clearly for formal semantic analysis, the natural-language descriptions used by many GPLs would not suffice, meaning that the project had to use some form of formal semantics as a starting point. \\

The literature survey examined operational, denotational, axiomatic and hybrid semantics.
While most hybrid semantic frameworks (see Section~\ref{ssub:hybrid_semantics}) provide significant expressive power, they often involve correspondingly significant levels of complexity.
As the semantic metalanguage needed to be both easily understood and easily written, this restricted the choice of semantic framework.\\

The final choice for the main form of semantics was the natural operational semantics (see Section~\ref{ssub:operational_semantics}), as they describe how the \textit{overall computed result} for the computation is obtained.
This frees the language designer from dealing with intermediate states, but imposes limitations on the project: they are not suitable for detailing concurrent or interleaved execution.
In the examination of \glspl{dsl}, however, this is unlikely to be any real issue.\\

However, in choosing operational semantics, the project has to deal with two main issues:
\begin{enumerate}
    \item \textbf{Semantic Representation:} Standard operational semantic rules use a multi-level format for the axioms.
    Such a format is difficult to directly represent in a non-rich-text environment (e.g. a code editor), and so requires transformation to ensure that semantics can be intuitively expressed.
    \item \textbf{Specific Semantic Representations:} While most operational semantics can be represented with the sub-evaluations depending on structural sub-terms of the main evaluation, this is not true of all kinds
    of program semantics.
    While this restriction is necessary for the verification engine to operate on the semantics, this prevents representation of many useful language features.
\end{enumerate}

To this end, the final semantic format for the metalanguage must support both generic semantic evaluations, and some method of providing more complex semantic features.
As these features cannot be directly proved from the form of the semantics, they must be subject to generic, external proof. 

% subsection choosing_a_semantic_form (end)

\subsection{Language Verification} % (fold)
\label{sub:language_verification}
While the problem of program verification (and hence language verification) is generally undecidable, this project aims to place restrictions on the languages that it can represent to make it tractable. 
These restrictions come from the \textit{semantic form}, as discussed in Section~\ref{sub:choosing_a_semantic_form}, and thereby restrict the types of programs that can be represented.

\subsubsection{Language vs. Program Verification} % (fold)
\label{ssub:language_vs_program_verification}
While proving properties of \textit{languages} is more general than proving properties of \textit{specific programs}, this imposes further restrictions on the kinds of properties that it can prove. \\

As part of the investigation of Data and its dual Codata in Section~\ref{sub:data_and_codata}, it was found that it is possible to prove termination for well-founded general recursion over data. 
However, this proof can only be performed at the program level, as done by Idris \citep{idris_lang}. 
This restriction is because types are predicated on values, where none of the
values exist at a program level. \\

This means that the proof mechanism used by \gls{absol} is less general, and unable to prove that recursion between arbitrary functions terminates.
While this \textit{does} restrict the kinds of programs that metaspec languages can represent

% subsubsection language_vs_program_verification (end)

\subsubsection{Traversal of Data} % (fold)
\label{ssub:traversal_of_data}
That is not to say that the result is not usable.
While at the program level it is not possible to allow recursive function calls, it is possible to provide special-case semantics that allow for the traversal of data. \\

To this end, the metalanguage should ensure that all structures that it can define can be reasoned about via well-founded induction, as discussed in Section~\ref{sub:proving_termination}. 
As long as the language (and proof engine) enforce the rules given in the literature survey it is then possible to show that all possible programs that can be represented in the language terminate.\\

It is this idea that the project has pursued in light of providing a language-level proof mechanism. 

% subsubsection traversal_of_data (end)

% subsection language_verification (end)

% section languages_and_programs (end)

\section{Filling in the Gaps} % (fold)
\label{sec:filling_in_the_gaps}
While the Literature and Technology Survey provided a comprehensive overview of much of the material, there were still some areas that were either not examined in sufficient detail, or ignored entirely. 
This section provides brief explorations of these.

\subsection{Guard Completeness Checking} % (fold)
\label{sub:guard_completeness_checking}
As, at the time of performing the literature survey, no decisions had been made as to the semantic representation, it was not apparent that it would be required to verify guards in the language.
However, since the selection of operational semantics as the basic form of semantic representation, it is clear that some `guard'-style functionality is needed. \\

Guards, in this sense, refer to restrictions on the values of the sub-evaluations of a rule, as can be seen in the following pair of operational semantic rules for a basic \texttt{if-then-else} expression:
\begin{align}
    [\text{if}] &: \frac{\langle S_1, s \rangle \to s'}{\langle \text{if } b \text{ then } S_1 \text{ else } S_2, s\rangle \to s'} \text{ if } \mathbb{B}\llbracket b \rrbracket s = \textit{ true} \\
    [\text{if}] &: \frac{\langle S_2, s \rangle \to s'}{\langle \text{if } b \text{ then } S_1 \text{ else } S_2, s\rangle \to s'} \text{ if } \mathbb{B}\llbracket b \rrbracket s = \textit{ false}
\end{align}

In this case the, `$\text{if } \mathbb{B}\llbracket b \rrbracket s = \textit{ false}$' and `$\text{if }\mathbb{B}\llbracket b \rrbracket s = \textit{ true}$' are what the project terms the \textit{semantic restrictions}.\\

In order to guarantee that a language where semantics can contain these guards is complete, the verification process must be able to guarantee that there is a semantic rule for all possible values of the guarded variables.
In the case above, where the value is binary, this is trivial, but over more complex domains (e.g. $n \in \mathbb{Z}$, for some variable $n$), it becomes significantly more complex.\\

A set of guards for a given piece of program semantics are a set of constraints.
There is a well-known method for solving equations with sets of constraints: linear programming.

\subsubsection{Linear Programming} % (fold)
\label{ssub:linear_programming}
\defblock{10cm}{Linear Program}{
    A Linear Program is an optimisation problem in which the objective function is linear in the unknowns and the constraints consist of linear equalities and linear inequalities \citep{luenberger2016simplex}.
}

The set of constraints in the guard completeness problem can be viewed as constraints on the optimisation.
Hence, the question of ``do these guards cover all possibilities' can be transformed to ``do the negations of these guards have a solution''. 
The hope, is that the negated system does not have a solution.\\

Solving this is known as the \textit{Feasibility Problem}, the aim of which is to determine if the region defined by the set of constraints is a bounded search space \citep{luenberger2016simplex}. 
If the negations of the guards form a system of constraints that are not feasible, then the guards are complete over the domain. \\

The problem, however, is slightly more complex than directly checking for feasibility, as guard conditions may be of the form $X \land Y$, which when negated is of the form $\lnot X \lor \lnot Y$ (by De-Morgan's Laws).
This disjunction means that the problem is slightly more difficult, involving checking sets of guard conditions for being \textit{infeasible}.

% subsubsection linear_programming (end)

\subsubsection{Guard Completeness and Project Scope} % (fold)
\label{ssub:guard_completeness_and_project_scope}
Unfortunately the implementation of such a completeness checker would add significant complexity to an already complex project, and as such is considered as out of scope.
This complexity was not appreciated initially due to insufficient research having been performed into the problem, and so it remained within the project scope for a significant amount of time. 
Nevertheless, it is an important and interesting problem to be solved as part of the future work.\\

This is not to imply that the guards will remain unchecked in the language, as that would admit possible non-termination due to no semantics existing for some evaluations.
To this end, a simpler method of ensuring guard completeness must be included, even if it is one that is inelegant. 

% subsubsection guard_completeness_and_project_scope (end)

% subsection guard_completeness_checking (end)

\subsection{Megaparsec --- Improved Parsing} % (fold)
\label{sub:megaparsec_improved_parsing}
As part of the technology survey, the availability of parsing libraries in Haskell was examined, finding that both Parsec and Happy were available (see Section~\ref{sub:language_parsing}). 
However, research failed to identify Megaparsec, a fork of the Parsec library that retains all of its strengths while fixing a multitude of issues with the long extant library \citep{megaparsec}.\\

The fixes that Megaparsec applied on top of the Parsec codebase included significant refactoring efforts, but most importantly for the project:
\begin{itemize}
    \item A greatly simplified Lexer interface, allowing for the creation of simple lexing code with ease.
    This removes the need for an external lexer such as Alex, and means that the parser code can be greatly simplified.
    \item Refactoring to the parsing of left-recursive expressions such as arithmetic and other binary operators. 
    Megaparsec provides an expression parser that handles all of the lookahead requirements automatically, providing the potential for further simplification to the parser code. 
    \item Significantly improved error messages, especially in the case of backtracking parsers. 
    \item General improvements to the interface provided by the libraries, including more intuitive naming for certain oft-used functions.
\end{itemize}

In light of these benefits over the standard Parsec library and the fact that it retains the other benefits of parsec, Megaparsec was chosen instead. 
Allowing simpler parser code and integrated lexing means that the parser portion of the project can be greatly simplified. 

% subsection megaparsec_improved_parsing (end)

% section filling_in_the_gaps (end) 

\section{High-Level Requirements Specification} % (fold)
\label{sec:high_level_requirements_specification}
This section provides a high-level outline of the requirements for the system as a whole. \\

The project consists of two main components:
\begin{itemize}
    \item The Metalanguage: Metaspec
    \item The Metacompiler: \gls{absol}
\end{itemize}

\subsection{The Requirements Generation Process} % (fold)
\label{sub:the_requirements_generation_process}
As this is a research project with a heavy software component, the requirements elicitation process is far less formal than for a purely software project.\\

As there are no stakeholders beyond those working on the project, the requirements in the following section were mostly generated from a vision of what an ideal DSL toolchain would look like. 
The requirements engineering process can be summarised in brief as follows:
\begin{enumerate}
    \item \textbf{Scenario Examination:} The informal but informed discussion of scenarios where such a toolchain might be used, inspired by the project's genesis at Bloomberg and in the wider industry.
    \item \textbf{Requirements Analysis:} Each scenario was, again informally, analysed to find what project requirements that it could generate.
\end{enumerate}

This method was selected due to the lack of any \textit{real} stakeholder, thus precluding more formal requirements analysis, and the research nature of the project, as it recognised the potentially in-flux nature as the research proceeded. 

% subsection the_requirements_generation_process (end)

\subsection{Requirements for the Metalanguage} % (fold)
\label{sub:requirements_for_the_metalanguage}
The first of the main project components is the metalanguage, Metaspec. 
It has the following requirements imposed upon it.\\

\requirement{Specify Language Syntax}{Functional}{
    The metalanguage must provide a flexible and intuitive way for the user to specify the syntax of the DSL.
}

\requirement{Specify Language Semantics}{Functional}{
    The metalanguage must provide an intuitive method for the user to specify thes language semantics in the general case, and also provide special-case semantics for things that cannot be proved by the proof mechanism.
}

\requirement{Semantic Typing}{Functional}{
    Language types should be enforced at the semantic level.
    The reasoning for this is twofold:
    \begin{enumerate}
        \item DSLs often have concise and clear syntax, and cluttering this with syntax-level typing compromises this goal somewhat.
        \item It is difficult to provide an extensible mechanism for language designers to add syntax-level typing.
    \end{enumerate}

    Combined, it is clear that the semantic level is the most appropriate place to enforce language types. 
}

\requirement{Integrated Syntax and Semantic Specification}{Non-Functional}{
    The forms of specification for both syntax (see Requirement~\reqref{req:SpecifyLanguageSyntax}) and semantics (see Requirement~\reqref{req:SpecifyLanguageSemantics}) should be integrated together in an intuitive fashion.
}

\requirement{Data Types}{Functional}{
    The metalanguage must provide a useful set of data-types to allow for useful computation to be performed. 
    These must include integer and floating point types, as well as list and matrix container types.
}

\requirement{Function Calls}{Functional}{
    The metalanguage must provide the ability to define procedures callable from host languages, and subroutines callable internally.
    The way in which these routines are defined and called should be congruent with being able to prove that the language semantics terminate.
}

\requirement{Data Traversal}{Functional}{
    The metalanguage should provide mechanisms for traversing the data types it defines, so as to avoid the need to allow general recursion.
}

\requirement{Ground-Truth Semantics}{Functional}{
    The metalanguage should provide a way to specify ground-truth semantics for non-terminals. 
    These are things that can be trivially assumed to terminate by the proof mechanism.
}

\requirement{Extension Mechanisms}{Functional}{
    The metalanguage should provide a set of `language features' that can be imported into scope for use by the DSL designer. 
    These should provide useful language constructs (e.g. non-terminals, special semantics) to aid in the construction of both syntax and semantics for the DSL.
}

\requirement{Environment Accesses}{Functional}{
    The metalanguage should provide syntactic constructs for both storing and accessing values in the environment. 
    This is required to support DSL creators defining their own function definitions and other such constructs in the DSL. 
}

\requirement{Language Metadata}{Functional}{
    The metalanguage should provide syntactic constructs for managing metadata about the language itself.
    These must include the language name and the language version.
}

\requirement{Intuitive File Structure}{Non-Functional}{
    The structure of a file in the metalanguage should have an intuitive structure.
    This means that the file should establish all prerequisites to the definition of the language before the language itself is defined. 
    This means that all the necessary context to understand the language definition is provided.
}

\requirement{Comments}{Functional}{
    The metalanguage must provide both line and block comments to allow for annotating language specifications.
    These comments may have no semantic meaning in the language and may be stripped at parse time. 
}

\requirement{Text Editor Ready}{Functional, Non-Functional}{
    The metasyntax specified by the metalanguage should be representable as plaintext with no markup required. 
    This ensures that it can be written in a standard text editor.
    It should, however, still have support for unicode glyphs as these can assist in matching the domain environment. 
}

% subsection requirements_for_the_metalanguage (end)

\subsection{Requirements for the Metacompiler} % (fold)
\label{sub:requirements_for_the_metacompiler}
The second major component of the project is the metacompiler, \gls{absol}.
It must conform with the following set of requirements.\\

\requirement{Parse Metaspec}{Functional}{
    The metacompiler must be capable of both lexing and parsing a metaspec file into an appropriate AST data structure. 
}

\requirement{Verify Language Construction}{Functional}{
    The metacompiler must be capable of verifying that all used non-terminal symbols are defined, and that no non-terminal is defined more than once.
    These are pre-requisites for the verification engine.
}

\requirement{Verify Semantic Form}{Functional}{
    The metacompiler must be capable of verifying that all semantic rules match their appropriate forms.
    This includes all types of special semantic rules, as well as the user-defined operational-style semantics.
}

\requirement{Verify Semantic Guards}{Functional}{
    The metacompiler must be capable of verifying that all semantic rules match their appropriate forms.
    This includes all types of special semantic rules, as well as the user-defined operational-style semantics.
}

\requirement{Generation of Verification Reports}{Functional}{
    In the case where the metacompiler is unable to verify an input language, it should generate detailed diagnostics that indicate the reason(s) why the language failed to verify.
    These diagnostics should provide a trace (from the start symbol of the language) to help the language developer determine the location of the error. 
}

\requirement{Defer Typechecking}{Functional}{
    The metacompiler does not deal with semantic type checking, as type holes cannot be inferred at the language level. 
}

\requirement{Prevention of Arbitrary Recursion}{Functional}{
    The metacompiler should check that languages do not enable arbitrary recursion over data, as this would make termination impossible to show at the language level. 
}

\requirement{Extensibility}{Non-Functional}{
    The metacompiler should be designed in such a fashion that it can easily be extended in the future to accommodate further developments on the way to productisation. 
    These developments may occur at the module level (e.g. additional verification steps), or at the program level, with the addition of new modules. 
    This means that it must be modular. 
}

% subsection requirements_for_the_metacompiler (end)

\subsection{Out of Scope Requirements} % (fold)
\label{sub:out_of_scope_requirements}
Over the course of the project, certain portions of the system that were initially in-scope had to be moved out of scope due to concerns over completing the project on time.
The main casualty of these time restrictions was the creation of a complete metacompiler pipeline.\\

While it is a shame that the project will not result in a complete `product', this is less of a problem than it may initially seem.
The \textit{novel} work of the project is concentrated in the metacompiler front end, which ingests, parses and verifies the language. 
Any further code-generation from the defined language has already seen significant exploration, particularly by \citet{diehl1996semantics}.
To this end, the following requirements have been moved out of scope.\\

\requirement{DSL Compiler Generation}{Functional}{
    Generation of a compiler for the DSL specified in the input file from both the syntax and semantics contained therein.
    This DSL compiler must be capable of taking programs in the specified DSL and transpiling them to Haskell code.
    This resultant Haskell code should be ready for use via the C \gls{ffi}.
}

While an important part of `productising' this toolchain, the generation of the DSL compiler from the metaspec AST is out-of-scope.
This is because it is not novel work, and yet constitutes a significant amount of implementation effort. 
As it doesn't really contribute much to the state of the art, it is not considered as part of this project.\\

\requirement{Full Semantic Guard Checking}{Functional}{
    The metacompiler verification stage must ensure that all sets of guards for user-specified semantics are complete. 
    \textit{Complete} means that there is no set of values that the program can create which will not satisfy \textit{at least one} of the extant guards.
    This helps to ensure that the program always has defined semantics.
}

As discussed in Section~\ref{sub:guard_completeness_checking}, it is possible to develop a sophisticated mechanism for checking the completeness of the guards.
However, the significant development effort this would require is unlikely to be achievable within the project time-frame, and is hence ruled as out of scope. 
As mentioned in Section~\ref{sub:requirements_for_the_metacompiler} this doesn't, however, absolve the metacompiler of needing to check guard completeness. 

% subsection out_of_scope_requirements (end)

\subsubsection{Evaluating the Requirements Specification} % (fold)
\label{ssub:evaluating_the_requirements_specification}
The requirements contained in Sections~\ref{sub:requirements_for_the_metalanguage} and~\ref{sub:requirements_for_the_metacompiler} provide a high-level overview of the goals that the metacompiler toolchain is expected to meet. 
While it does not provide a high-level of detail, this is appropriate for the nature of the project.\\

Due to the research-based focus of the project, more specific requirements would be in a state of constant flux, while these higher-level specifications are broad enough that they can remain in place for the duration of the project.

% subsubsection evaluating_the_requirements_specification (end)

% section high_level_requirements_specification (end)

% chapter elucidation (end)

\chapter{Designing the Metalanguage} % (fold)
\label{cha:designing_the_metalanguage}
As a project, \gls{absol} has had a very heavy research bent. 
The experimental nature of the toolchain resulted in significant up-front design work and, combined with the highly theoretical nature of the language verification algorithms, this meant that language design and algorithmic development dominated the time spent on the project.
This section aims to illustrate the comprehensive design work that was put into the first of the two main project components: the metalanguage --- \gls{metaspec}. \\

\gls{metaspec} is the metalanguage for the \gls{absol} project, allowing the language designers to specify both the syntax and semantics of their DSL, as well as associated metadata, in a unified form. 
The final syntax for Metaspec is the result of significant design work, and consequentially the syntax discussed below has been through some evolution. \\

Metaspec is, in itself, a \gls{dsl}, and hence its design process was an interesting insight into how people might use the language to design their own \glspl{dsl}. 
The complete grammar for Metaspec can be found in Appendix~\ref{cha:the_metaspec_grammar}, and is written in standard \gls{ebnf} notation. 
The same notation will be used throughout this section of the document when referring to the metaspec grammar itself.

\section{The Top-Level Definitions} % (fold)
\label{sec:the_top_level_definitions}
The top-level structure of a Metaspec file consists of a series of ordered top-level definitions.
The presence of these definitions emerged from the seemingly disparate nature of a number of the requirements placed upon the language.
They are as follows:
\begin{enumerate}
    \item The language name (Requirement~\reqref{req:LanguageMetadata})
    \item The language version (Requirement~\reqref{req:LanguageMetadata})
    \item Language feature imports (Requirement~\reqref{req:ExtensionMechanisms})
    \item Ground truths for the proof engine (Requirement~\reqref{req:Ground-TruthSemantics})
    \item The language itself (Requirement~\reqref{req:IntegratedSyntaxandSemanticSpecification})
\end{enumerate}

Tying these somewhat disparate areas together is the requirement for language definition files to read in an ``intuitive'' fashion (Requirement~\reqref{req:IntuitiveFileStructure}). 
This provided an initial sense for the ordering of the language definitions, as each block assisted in providing the contextual foundation for the language definition itself. \\

To this end, the decision was made to enforce the ordering of these in the language grammar itself (as seen in Listing~\ref{lst:top_level_metaspec_definition_blocks}), with the ordering as above.

\begin{listing}[!htb]
\begin{minted}[firstnumber=156]{text}
metaspec-defblock = 
    name-defblock, rule-termination-symbol, 
    version-defblock, rule-termination-symbol, 
    using-defblock, rule-termination-symbol, 
    truths-defblock, rule-termination-symbol, 
    language-defblock, rule-termination-symbol;
\end{minted}
\caption{Top-Level Metaspec Definition Blocks}
\label{lst:top_level_metaspec_definition_blocks}
\end{listing}

Placing the metadata fields first was a natural way to provide some initial context as to the language and version, and also assists with at-a-glance checking the language version in a \gls{vcs}.
As the extensions import list, the \mintinline{text}{using-defblock}, can contain dependencies of the final two blocks, it made a significant amount of sense to put this next, as it aids in establishing context for the following blocks.\\

The ordering of the termination truths (\mintinline{text}{truths-defblock}) and the language definition itself (\mintinline{text}{language-defblock}) was similarly natural. 
While the truths are mainly depended upon by the termination proof mechanism (see \autoref{sub:metaverify_the_verification_engine}), they can also act as a guide to the language designer to indicate where they will be required to provide semantics for a language production. 
This quite clearly indicates that the termination truths should come before the language definition itself, as they provide additional context for the language designer.\\

While one might argue for the separation of these top-level elements, particularly the metadata, into multiple files, it seems far more natural to combine them all under one umbrella.
This means that each language definition is a self-contained unit with all the context required to understand it. 
While this single-file requirement does mean that definitions for large DSL could become unwieldy, taking a single-file approach significantly simplifies the metacompiler implementation.\\

Each of these top-level blocks have fairly distinct forms, and each was designed very carefully to aid the language designer's understanding of the file.
While it is not worth dwelling on their design, the most important design decisions should be noted.
\begin{itemize}
    \item \textbf{The Metadata Blocks:} While seemingly simplistic, the definitions of both the language name and version definition blocks was intended to not enforce any name or version scheme as \citet{raemaekers2014semantic} found the semantics of these to vary drastically between projects.
    As a result, both fields are utf-8 strings, parsed from the first non-whitespace character to the last occurring before the rule termination symbol (\mintinline{text}{;}), as seen in \autoref{lst:metadata_block_definitions}.
\begin{listing}[!htb]
\begin{minted}[xleftmargin=1cm, firstnumber=156]{text}
name-defblock = "name", where-symbol, { utf-8-char }-,;
version-defblock = "version", where-symbol, { utf-8-char }-,;
\end{minted}
\caption{Metadata Block Definitions}
\label{lst:metadata_block_definitions}
\end{listing}

    \item \textbf{The Imports Definition Block:} While the one-line-per-import style (oft used in programming languages) was considered due to its syntactic flexibility (allowing statements that modify the import --- e.g. \mintinline{haskell}{import qualified Data.Maybe as M}), a comma-separated list was chosen. 
    This is both more in-keeping with the style of metaspec as a language, and better accommodates the simple-keyword imports of the language. 
    It is hence defined as shown in \autoref{lst:the_using_definition_block}. 
\begin{listing}[!htb]
\begin{minted}[xleftmargin=1cm, firstnumber=175, escapeinside=||]{text}
using-defblock =
    "using",
    where-symbol,
    semantic-block-start,
    [ metaspec-feature, 
        { semantic-list-delimiter, metaspec-feature }]
    semantic-block-end; 
\end{minted}
\caption{The Using Definition Block}
\label{lst:the_using_definition_block}
\end{listing}

    \item \textbf{The Language Truths Block:} Understanding a language defined in Metaspec relies on the understanding of the trivial `base-cases' that are assumed to terminate by the proof engine. 
    The semantics of each truth can be understood as meaning ``this evaluation always terminates''.
    While it was initially thought that these could be generated automatically, doing so would have removed significant language flexibility, as much of said flexibility comes from the users specifying semantics exactly as they require. 
    To this end, users specify the termination cases for all symbols, even those they did not define.
    The format for doing so is taken almost directly from the language semantics (see \autoref{sec:specifying_the_language_semantics}) to allow for easy visual identification during language design, as can be seen in \autoref{lst:the_truths_definition_block}.
\begin{listing}[!htb]
\begin{minted}[xleftmargin=1cm, firstnumber=182]{text}
truths-defblock =
    "truths",
    where-symbol, 
    semantic-block-start,
    semantic-evaluation,
    { ", ", semantic-evaluation },
    semantic-block-end;
\end{minted}
\caption{The Truths Definition Block}
\label{lst:the_truths_definition_block}
\end{listing}

    \item \textbf{The Language Definition Block:} This block has a very simple design, acting as nothing but a container for the productions of the language, as seen in \autoref{lst:the_language_definition_block}.
    The form of these productions is discussed in ections~\ref{sec:specifying_the_language_syntax} and~\ref{sec:specifying_the_language_semantics}, and is the result of combining the two as discussed in Section~\ref{sec:combining_syntax_and_semantics}.
\begin{listing}[!htb]
\begin{minted}[xleftmargin=1cm, firstnumber=191]{text}
language-defblock =
    "language",
    where-symbol,
    semantic-block-start,
    language-definition,
    semantic-block-end;
\end{minted}
\caption{The Language Definition Block}
\label{lst:the_language_definition_block}
\end{listing}

    \item \textbf{Comments in Metaspec:} With comments being crucial in any programming or specification language, it was important to determine how they would look in Metaspec. 
    Requirement~\reqref{req:Comments} states that comments need not have any semantic meaning (contrasted with Python Docstrings, which are compiled with their functions \citep{python_docstrings}), and could hence be very simple and removed during preprocessing. 
    The choice of the comment symbols was informed by \gls{ebnf} for the block-comment style (using \mintinline{text}{"(*"} and \mintinline{text}{"*)"}), while the line-comment style was derived from the common C-style operator (\mintinline{text}{"//"}).
    Comments are thus defined as in \autoref{lst:comments_in_metaspec} below.
\begin{listing}[!htb]
\begin{minted}[xleftmargin=1cm, firstnumber=156]{text}
metaspec-comment =
    line-comment-start-symbol, { utf-8-char }, eol-symbol |
    block-comment-start-symbol, { utf-8-char }, 
        block-comment-end-symbol;
\end{minted}
\caption{Comments in Metaspec}
\label{lst:comments_in_metaspec}
\end{listing}
\end{itemize}

% section the_top_level_definitions (end)

\section{Specifying the Language Syntax} % (fold)
\label{sec:specifying_the_language_syntax}
Metaspec, as a hybrid metalanguage, needed to have the ability to specify the syntax of the language it describes. 
As discussed in \autoref{sub:choosing_a_syntactic_form}, the predominant notation for specifying language syntax that is used today is \gls{ebnf}.
Metaspec adapts the \gls{ebnf} syntactic specification language, taking the set of definitions for specifying productions almost directly.
This decision was made for a few main reasons:
\begin{itemize}
    \item \textbf{Flexibility of Syntactic Definition:} \gls{ebnf} is already capable of representing both context-free and context-sensitive language grammars.
    Furthermore, \gls{ebnf} places \textit{no restriction} on the format of the terminals of the language, which is important for allowing \glspl{dsl} to get as close to the domain notation as possible. 
    Any form of syntactic specification that would be devised as part of this project would likely be flawed in some way, shape or form, or not be as flexible as the extant notation in the form of \gls{ebnf}. 
    \item \textbf{User Familiarity:} Due to the prevalent use of \gls{ebnf} in the programming language community and its standardised nature, it was felt that providing a syntactic notation that the users would be familiar with would ease use of Metaspec. 
    \item \textbf{Correctness:} Adaptation of a pre-defined and well-studied metasyntax notation is far less likely to result in errors than implementing a new notation from scratch.
\end{itemize}

That is not to suggest, however, that the \gls{ebnf} productions were taken entirely as-is. 
As part of adapting the long-standing metasyntactic notation, there were a few changes made to its syntax.
These changes were in aid of providing better operation with other portions of the toolchain, as well as fixing some of the less-intuitive elements of the \gls{ebnf} grammar. 

\subsection{Adapting the Form of Non-Terminals} % (fold)
\label{sub:adapting_the_form_of_non_terminals}
Standard \gls{ebnf} represents non-terminals by strings of textual characters, including spaces.
This means that they can be difficult to visually identify for the reader of the grammar, and that concatenation of terminals and non-terminals is defined by the inelegant operator (\mintinline{text}{,}).
Metaspec adapts this through explicitly delimiting the non-terminals, as shown in \autoref{lst:non_terminals_in_metaspec}.

\begin{listing}[!htb]
\begin{minted}[numbers=none, fontsize=\blockfont]{text}
non-terminal-start = "<";
non-terminal-end = ">";
non-terminal-identifier = 
    textual-glyph, { textual-glyph | natural-number | "-" | "_"};
non-terminal = non-terminal-start, non-terminal-identifier, non-terminal-end;
\end{minted}
\caption{Non-Terminals in Metaspec}
\label{lst:non_terminals_in_metaspec}
\end{listing}

The use of the distinctive angle brackets both helps to visually distinguish the non-terminals of the language, and syntactically delimit the non-terminals.
As a result, there is no need for an explicit concatenation operator, reducing visual noise in the Metaspec definitions.\\

This altered syntax for non-terminals also brings benefits when it comes to parsing the semantics. 
The semantics of a language defined in metaspec may contain variable identifiers (as discussed in \autoref{sub:user_defined_semantics}), but also require access to elements of the syntax (see \autoref{sub:accessing_syntax_from_the_semantics}).
These identifiers look much like standard \gls{ebnf} non-terminal names, and so the alterations made for \gls{metaspec} provide some additional ability to disambiguate at the parser level. 

% subsection adapting_the_form_of_non_terminals (end)

\subsection{Specification of the Language Start Rule} % (fold)
\label{sub:specification_of_the_language_start_rule}
One of the things that \gls{ebnf} lacks is an explicit method of representing the `start rule' of a grammar. 
The start rule (or start symbol) in any grammar is the place from which a parse either \textit{starts}, or where a parse will \textit{end} \citep{slonneger1995formal}.
For automated tools working with syntax, this is clearly important to define. \\

Beyond the requirement to use the grammar portion of the specification for parser generation, the notion of a start symbol is also useful when it comes to verifying the language semantics.
The start symbol can act as the point from which the verification starts, and thus brings the ability to check for other criteria on the language (unused productions, syntax-only productions, etc). \\

While it could be simple as to provide a metadata field specifying the name of the start symbol for the \gls{dsl}, that seemed inelegant. 
Instead, a special type of non-terminal declaration was added; where non-terminals are enclosed in angle brackets --- \mintinline{text}{<nt>} --- the start symbol is enclosed in a set of double angle brackets: \mintinline{text}{<<start>>}. 
This means that it can be trivially identified during the parsing stage, and used as needed. 
The start symbol and start rule (the production associated with the start symbol) are hence defined as in \autoref{lst:the_start_symbol_and_start_rule_for_metaspec}.

\begin{listing}[!htb]
\begin{minted}[numbers=none, fontsize=\blockfont]{text}
start-symbol-start = "<<";
start-symbol-end = ">>";
start-symbol = start-symbol-start, non-terminal-identifier, start-symbol-end;
start-rule = start-symbol, defining-symbol, language-rule-body;
\end{minted}
\caption{The Start Symbol and Start Rule for Metaspec}
\label{lst:the_start_symbol_and_start_rule_for_metaspec}
\end{listing}

% subsection specification_of_the_language_start_rule (end)

\subsection{Removal of the Empty Syntax} % (fold)
\label{sub:removal_of_the_empty_syntax}
As part of its syntactic specification, \gls{ebnf} provides the notion of an `empty syntax'.
This is a piece of syntax that can exist when there is no actual syntax in a place where some would otherwise be expected. 
As one might imagine, this poses some significant difficulties to parsers, and hence has been removed in Metaspec for practical purposes. \\

There were some concerns that this would compromise the expressive power of the metasyntactic notation used in Metaspec, but careful examination of the \gls{ebnf} standard only showed one case where the empty syntax was used. 
\gls{ebnf} provides facilities for a syntactic exception, and this can be combined with the repetition notation and empty syntax to ensure that at least one repetition exists: \mintinline{text}{prod = {some-term}-;}.
While not allowing this notation does compromise the \textit{conciseness} of certain productions in Metaspec, it does not prevent representation of such terms --- the above can be represented as \mintinline{text}{prod = some-term, {some-term};}.

% subsection removal_of_the_empty_syntax (end)

\subsection{Altering the Assignment Operator} % (fold)
\label{sub:altering_the_assignment_operator}
One of the other changes that \gls{metaspec} makes to the standard \gls{ebnf} notation for language productions is to alter the assignment symbol.
Where \gls{ebnf} uses \mintinline{text}{=}, \gls{metaspec} uses \mintinline{text}{::=}.\\

The reasoning behind this is similar to some of the reasoning for redefining the form of non-terminals: it provides a more significant visual cue to the language designer. 
As the \mintinline{text}{=} symbol often appears in programming languages as some form of assignment, it is not unlikely that a \gls{dsl} designer would want to use it as such.
This means that it would have to occur in a production as \mintinline{text}{"="}. 
The \mintinline{text}{::=} as chosen by \gls{metaspec} is far less common, and this should help to improve the readability of the syntactic notation at a glance. 

% subsection altering_the_assignment_operator (end)

\subsection{The Final Language Syntax} % (fold)
\label{sub:the_final_language_syntax}
As a result, the final syntactic metalanguage for \gls{metaspec} is very similar to that of \gls{ebnf}.
Including the changes mentioned in the preceding sections, it is capable of the specification of flexible syntactic rules containing the following syntactic features:
\begin{itemize}
    \item Optionality of syntax.
    \item Grouping of syntax.
    \item Alternation of syntax terms.
    \item Repetition, in both the multiplicative (\mintinline{text}{3 * <nt>}) and repeated-group (\mintinline{text}) forms, for syntax terms.
    \item Syntactic exceptions (\mintinline{text}{<nt1> - <nt2>}).
    \item Arbitrary non-terminals and terminals.
\end{itemize}

Overall, this provides an excellent metasyntactic foundation for the metalanguage, and lets users describe a wide-set of possible languages.
Hence, it allows \gls{dsl} designers to get as close to their intended domain-specific notation as possible. 

% subsection the_final_language_syntax (end)

% section specifying_the_language_syntax (end)

\section{Specifying the Language Semantics} % (fold)
\label{sec:specifying_the_language_semantics}
While \gls{metaspec} now had a way of specifying language syntax, its hybrid nature meant that it also needed a way to specify the language semantics. 
As discussed in \autoref{sub:choosing_a_semantic_form}, the basic semantic descriptions in metaspec are based on big-step operational semantics (see Subsection~\ref{ssub:operational_semantics}). 
However, requirements dictated that the language support other forms of semantics, including environment accesses (Requirement~\reqref{req:EnvironmentAccesses}) and special-case semantic rules (Requirement~\reqref{req:SpecifyLanguageSemantics}).\\

As a result, the metalanguage has four different kinds of semantic rules, each of which is explored below:
\begin{itemize}
    \item \textbf{Standard Semantics:} These are an adaptation of the big-step operational semantics form.
    They provide restrictions and evaluation control, and are the main method for specifying language semantics.
    \item \textbf{Special-Case Semantics:} These special rules have a distinct syntactic form, and provide semantics that cannot be proved by the standard mechanism.
    \item \textbf{Environment Accesses:} Accesses to items stored in the global environment, akin to a key-value store.
    \item \textbf{Environment Stores:} The ability to store values in the global environment under a key. 
\end{itemize}

The following sections aim to explain the design process that the syntax for the semantic forms underwent, broadly grouping them into the categories of ``User-Defined Semantics'' (\autoref{sub:user_defined_semantics}), which took the most design effort, and ``Special-Case Semantics'' (\autoref{sub:special_case_semantics}).

\subsection{Accessing Syntax from the Semantics} % (fold)
\label{sub:accessing_syntax_from_the_semantics}
One of the first things to notice about the form of a big-step operational semantics rule (see \autoref{eq:example_of_big_step_operational_semantics_if} for an example) is that the sub-evaluations (those above the evaluation line) usually contain sub-terms of the main expression (below the line).
It should be noted that, in the general, non-metaspec case, these sub-evaluations do not \textit{have} to be purely sub-terms of the main expression. 

\begin{equation}
    [\text{if}] : \frac{\langle S_1, s \rangle \to s'}{\langle \text{if } b \text{ then } S_1 \text{ else } S_2, s\rangle \to s'} \text{ if } \mathbb{B}\llbracket b \rrbracket s = \text{true}
    \label{eq:example_of_big_step_operational_semantics_if}
\end{equation}

In a basic sense, every piece of syntax in Metaspec has some kind of semantics associated with it.
As a result, the syntax often defines the `main expression' of the semantic rule.
Given this fact, \gls{metaspec} needs a way for the semantic rules to `access' the syntax, that is: refer directly to portions of the syntax of a given production. \\

As has been established in \autoref{sec:specifying_the_language_syntax}, \gls{metaspec} supports a flexible notation for defining productions of the DSL. 
However, the important thing to recognise is that the only terms that one would need to refer to in the semantics are the non-terminals of the language. 
Any syntactic production decomposes to a set of terminals and non-terminals, and as terminals are fixed they can be explicitly written into the semantics where required. \\

\begin{listing}[!htb]
\begin{minted}[numbers=none]{text}
<example> ::= <foo> "(" <bar> "," <bar> ")"
\end{minted}
\caption{An Example Production for Syntax Access}
\label{lst:an_example_production_for_syntax_access}
\end{listing}

Consider an example production of the form seen in \autoref{lst:an_example_production_for_syntax_access}. 
The most intuitive way to access the syntax is to refer directly to the non-terminals directly.
This means that \mintinline{text}{<foo>} in the semantics would refer to the first instance of the non-terminal in the syntax, and similarly for \mintinline{text}{<bar>}.
However, this positional argument-based nature is a non-intuitive way to access the syntax, as it forces the \gls{dsl} designer to formulate their semantics in the order of the syntax. 
This would impose significant restrictions in the case of semantic special syntax (which have fixed argument positions), and was hence seen as undesirable.\\

As the above situation was deemed unsatisfactory, the final access form for \gls{metaspec} takes inspiration from array accesses in other programming languages, using the square-bracket notation (\mintinline{text}{[]}) for familiarity.
Each access into the syntax must hence specify the (zero-indexed) \textit{position} of the specific non-terminal to which it refers: \mintinline{text}{<bar>[1]} refers to the second instance of the non-terminal, for example.
These positions must hence be checked at language verification time (see \autoref{sub:user_defined_semantic_form_verification} for details).\\

This provides an intuitive way to access syntactic elements within the semantics, without restricting the form in which the semantics can be written. 

% subsection accessing_syntax_from_the_semantics (end)

\subsection{Semantic Typing} % (fold)
\label{sub:semantic_typing}
The semantics are the location where the notion of data types was designed into \gls{metaspec} with each semantic rule carrying some type information. 
This was imposed upon the language by Requirement~\reqref{req:SemanticTyping} for two main reasons:
\begin{itemize}
    \item \textbf{Conciseness:} A common feature of \glspl{dsl} is that they have a clear and concise syntax that matches the domain representation of knowledge as closely as possible (\autoref{sec:domain_specific_languages}). 
    As a result, having typing imposed at the syntactic level would add unnecessary visual clutter to the \gls{dsl} syntax. 
    \item \textbf{Clarity of Language Implementation:} It would likely introduce significant complexity to the form of the language semantics to introduce a method for providing syntactic typing to the language designer.
    Furthermore, the provision of such a feature would likely require alterations to the verification algorithm, and hence potentially impact correctness. 
\end{itemize}

To this end, it was decided to impose types at the semantic level.
In this case, each semantic expression (where the type could not be inferred) is required to have a result type, where the types may be one of the types allowed by the current language context. 
These types are expressions as part of the language grammar, and are contextually checked to be in scope at parse time (see \autoref{sub:metaparse_ast_generation}).\\

For information on the checking of types in the language specification, please see \autoref{sub:type_checking} on \autopageref{sub:type_checking}. 

% subsection semantic_typing (end)

\subsection{User-Defined Semantics} % (fold)
\label{sub:user_defined_semantics}
While \textit{all} semantics for a language designed in \gls{metaspec} are technically user-defined, this section focuses on the examination of the main form of the semantics --- those based upon big-step operational semantics. 
These semantic rules form core of languages written in \gls{metaspec}, as they provide a flexible way to specify the behaviour of the language. 

\subsubsection{Examining the form of Big-Step Operational Semantics} % (fold)
\label{ssub:examining_the_form_of_big_step_operational_semantics}
\begin{equation}
    [\text{if}] : \frac{\langle S_1, s \rangle \to s'}{\langle \text{if } b \text{ then } S_1 \text{ else } S_2, s\rangle \to s'} \text{ if } \mathbb{B}\llbracket b \rrbracket s = \textit{ true} 
    \label{eq:a_basic_big_step_operational_semantics_rule}
\end{equation}

A big-step operational rule such as the one seen in \autoref{eq:a_basic_big_step_operational_semantics_rule} can be broken down into a number of components:
\begin{itemize}
    \item \textbf{The Primary Expression:} \label{item:the_primary_expression} This is the portion of the syntax to be evaluated below the line, to the left of the evaluation arrow.
    Here, this is: $\langle \text{if } b \text{ then } S_1 \text{ else } S_2, s \rangle$.
    \item \textbf{The Evaluation Result:} The result of evaluating the primary expression, this is found to the right of the evaluation arrow below the line.
    Here it is: $s'$. 
    \item \textbf{The Restriction:} A restriction placed on the evaluation of this rule, such that the rule is only evaluated if the restriction holds. 
    Here, this is: $\text{if } \mathbb{B}\llbracket b \rrbracket s = \textit{ true}$.
    \item \textbf{The Sub-Evaluation:} A specification of how each component of the primary expression is evaluated, located above the line.
    Here, this is: $\langle S_1, s \rangle \to s'$. 
\end{itemize}

There are further components that are not indicated in such a rule. 
Consider the rule shown in \autoref{eq:an_alternate_big_step_operational_semantics_rule}.

\begin{equation}
    [+] : \frac{\langle A_1, s_0\rangle \to \langle n_1, s_1\rangle \;\;\;\; \langle A_2, s_1 \rangle \to \langle n_2, s_2\rangle}{\langle A_1 + A_2, s_0 \rangle \to \langle n, s_2 \rangle}\;\; n = n_1 + n_2
    \label{eq:an_alternate_big_step_operational_semantics_rule}
\end{equation}

This exhibits an additional set of components as follows:
\begin{itemize}
    \item \textbf{The Semantic Evaluation:} This is a portion of the rule that specifies how the result of the semantics is calculated, located to the right of the evaluation rule. 
    Here, it is: $n = n_1 + n_2$.
    \item \textbf{Multiple Sub-Evaluations:} This rule also illustrates that it is possible to have multiple sub-evaluations taking place as part of a given semantic rule. 
    The importance of this is the \textit{order} in which they are executed, as this may impact the results of computations.
    While this importance is not necessarily apparent at the level of a semantic specification, it becomes important when generating execution semantics from the definition. 
    For further discussion of this issue, see \nameref{ssub:verifying_the_evaluation_criterion} on \autopageref{ssub:verifying_the_evaluation_criterion}.
\end{itemize}

Each of these components must be accounted for in the semantic form that represents these semantic rules, and have been incorporated into the metasyntax for the semantics as discussed in \autoref{ssub:the_final_form_of_the_user_defined_semantics}.

% subsubsection examining_the_form_of_big_step_operational_semantics (end)

\subsubsection{Transforming Big-Step Operational Semantics} % (fold)
\label{ssub:transforming_big_step_operational_semantics}
Having identified the key portions of the big-step semantic rules, the next task was to represent these components effectively in a textual format. 
The working assumption for designing this syntactic representation is that these \gls{metaspec} language specifications would be written with nothing but a text editor, as specified by Requirement~\reqref{req:TextEditorReady}.
To this end, the notation should be simplistic and not require complex formatting (thus also simplifying the parsing process). \\

In one sense, semantics represent a flow of information or a \textit{pipeline} of evaluation. 
Semantically, each portion of the evaluation is predicated on the next portion of the evaluation, so they can be separated by what is effectively a \textit{where} clause. 
For the purposes of these semantic rules, the \mintinline{text}{:} operator was chosen to represent this.
To this end, the portions of the semantic rules discussed in \autoref{ssub:examining_the_form_of_big_step_operational_semantics} can be represented in order as follows:
\begin{enumerate}
    \item \textbf{The Output Variable:} A variable identifier and associated type (as discussed in \autoref{sub:semantic_typing}) as the leftmost portion of the rule: \mintinline{text}{<type> <identifier>}.
    This variable defines the final result of the semantic evaluations.
    \item \textbf{The Semantic Operations:} A list of operations that define how the results of the sub-evaluations are combined: \mintinline{text}{(<eval> {"," <eval>})}.
    These operations are defined in terms of an allowed set of semantic operations defined by metaspec. 
    \item \textbf{The Semantic Restrictions:} A list of restrictions that constrain the circumstances under which this semantic rule can operate.
    These are restrictions on the results of the sub-evaluations that act as boolean conditions, and hence look like a list of standard conditions: \mintinline{text}{([<key> <op> <key>] {"," <key> <op> <key>})}.
    Restrictions do not have to exist, however, and hence it is perfectly valid to have an empty restriction block.
    \item \textbf{The Semantic Evaluations:} These define evaluations on sub-terms of the syntax, with each sub-term addressing non-terminals as discussed in \autoref{sub:accessing_syntax_from_the_semantics}. 
    Each of these terms must define a result type and variable for use in the semantic operations section: \mintinline{text}{<type> <identifier> "<=" <syntax-access>}.
    As there can be multiple of these, they are given as a list.
\end{enumerate}

\begin{listing}[!htb]
\begin{minted}[firstnumber=314]{text}
semantic-evaluation-rule = 
    semantic-type,
    semantic-identifier,
    where-symbol,
    semantic-operation-list,
    semantic-restiction-list,
    where-symbol, 
    semantic-eveluation-list;
\end{minted}
\caption{The Semantic Evaluation Rule Grammar}
\label{lst:the_semantic_evaluation_rule_grammar}
\end{listing}

It is apparent at this point that the \textit{primary expression} identified on \autopageref{item:the_primary_expression} appears to have been ignored.
While it was initially included as a component in the semantic rule it was quickly noticed that the syntax of the language production itself suffices as the primary expression.
It was hence omitted from the semantics directly, with the syntax accesses (see \autoref{sub:accessing_syntax_from_the_semantics}) sufficing to combine the two. 
The final basic form of the syntax for these semantic rules can be seen in \autoref{lst:the_semantic_evaluation_rule_grammar}.\\

Combining these elements provides the form for a single evaluation rule.
A single rule is, however, not sufficient to represent a defined set of semantics for a given production. 
As a result, the grammar was expanded to accommodate a `semantic alternation', with multiple semantic rules separated by \mintinline{text}{|}, appearing as a logical disjunction.
This notion perfectly represents the concept, as only one semantic rule will evaluated based upon the evaluation of the guards. 
This is represented in the concept of the evaluation list, as defined in \autoref{lst:the_semantic_evaluation_rule_list}.

\begin{listing}[!htb]
\begin{minted}[firstnumber=310]
semantic-evaluation-rule-list = 
    semantic-evaluation-rule,
    { semantic-disjunction, semantic-evaluation-rule };
\end{minted}
\caption{The Semantic Evaluation Rule List}
\label{lst:the_semantic_evaluation_rule_list}
\end{listing}

Multiple portions of this syntax are required to match a certain form for verification to be able to occur. 
These restrictions are discussed in detail in \autoref{sub:user_defined_semantic_form_verification}, and are crucial to the ability of \gls{absol} to verify the language semantics.

% subsubsection transforming_big_step_operational_semantics (end)

\subsection{An Example of User-Defined Semantics} % (fold)
\label{sub:an_example_of_user_defined_semantics}
Having established the form of the semantics as above, it is now possible to formulate an example of the transformation from standard big-step operational semantics to the \gls{metaspec} syntax for the same.
Consider the rules for exponentiation as described in Equations~\ref{eq:big_step_exponentiation_special_case} and~\ref{eq:big_step_exponentiation_base_case} below.
These rules have a special case for illustrative purposes. 

\begin{align}
    [\string^] &: \frac{\la A_2, s_0 \ra \to \la n_2, s_1 \ra}
    {\la A_1 \string^ A_2, s_0 \ra \to \la n, s_1 \ra} n = 1 \text{ where } n_2 == 0 \label{eq:big_step_exponentiation_special_case} \\
    [\string^] &: \frac{\la A_2, s_0 \ra \to \la n_2, s_1 \ra \;\;\; \la A_1, s_1 \ra \to \la n_1, s_2 \ra}{\la A_1 \string^ A_2, s_0 \ra \to \la n, s_2 \ra} n = n_1^{n_2} \label{eq:big_step_exponentiation_base_case}
\end{align}

Working through the special-case rule in \autoref{eq:big_step_exponentiation_special_case} it is possible to compose the equivalent \gls{metaspec} expression as follows (assuming types and that $A_1, A_2$ are the same non-terminal): 
\begin{enumerate}
    \item \textbf{Determine the Output Variable:} In this case it is clear that it is $n$. 
    The expression is hence initially:
    \begin{minted}[xleftmargin=2cm, numbers=none]{text}
        number n : 
    \end{minted}
    \item \textbf{Determine the Evaluation Rule:} This is also clear as $n = 1$, in the presence of a restriction.
    The Metaspec rule is hence:
    \begin{minted}[xleftmargin=2cm, numbers=none]{text}
        number n : {n = 1}
    \end{minted}
    \item \textbf{Determine any Restrictions:} This rule has a restriction, and so this can also be found and added. 
    It is $n_2 == 0$, making the Metaspec rule into:
    \begin{minted}[xleftmargin=2cm, numbers=none]{text}
        number n : {n = 1}(n2 == 0) : 
    \end{minted}
    \item \textbf{Write the Sub-Evaluations:} This rule only has a single sub-evaluation as follows: $\la A_2, s_0 \ra \to \la n_2, s_1 \ra$.
    This is transformed into Metaspec, giving the resultant rule (as it is the \textit{second} instance of that `non-terminal' in the rule):
    \begin{minted}[xleftmargin=2cm, numbers=none]{text}
        number n : {n = 1}(n2 == 0) : {number n2 <= <a-nt>[1]}
    \end{minted}
\end{enumerate}

Performing the same process for the generic exponentiation rule in \autoref{eq:big_step_exponentiation_base_case} and combining using an alternation (as discussed on \autopageref{ssub:transforming_big_step_operational_semantics}), the final semantic expression can be found.
This is seen in \autoref{lst:metaspec_semantic_rules_for_exponentiation}, and can be assumed to be linked to a syntactic rule of the form \mintinline{text}{<exp> ::= <a-nt> "^" <a-nt>;} (or similar).

\begin{listing}[!htb]
\begin{minted}[numbers=none]{text}
number n : {n = 1}(n2 == 0) : {number n2 <= <a-nt>[1]} |
number n : {n = n1 ^ n2}() : 
    {number n2 <= <a-nt[1]}, {number n1 <= <a-nt>[0]}
\end{minted}
\caption{Metaspec Semantic Rules for Exponentiation}
\label{lst:metaspec_semantic_rules_for_exponentiation}
\end{listing}

% the user is able to decide which evaluations to make, and in which order

% subsection an_example_of_user_defined_semantics (end)

\subsubsection{The Final Form of the User-Defined Semantics} % (fold)
\label{ssub:the_final_form_of_the_user_defined_semantics}
At this point, the syntax for defining language semantics is concise and clear.
However, it has not yet been established exactly how one of these operations is evaluated.
Much like the big-step operational semantics from which they are derived, the evaluation semantics are crucial to understanding these expressions. \\

As a result, it is important to formalise the evaluation semantics of the metaspec expressions. 
A metaspec semantic rule is evaluated according to the algorithm given in \autoref{alg:metaspec_semantic_evaluation_algorithm}, taking an alternation of metaspec semantic rules as input.
Assume that \textsc{evaluate} is a function that evaluates the provided evaluation or computation, and that \textsc{checkRestriction} is a function that returns true if the restrictions are all true.
With formalised execution order, this ensures that the \gls{dsl} creators are provided with full control over the execution semantics of their language.

\begin{breakablealgorithm}
\caption{Metaspec Semantic Evaluation Algorithm}
\label{alg:metaspec_semantic_evaluation_algorithm}
\begin{algorithmic}
\State rules $\gets$ the alternation of semantic rules
\State
\ForAll{rule $\in$ rules}
    \State evals $\gets$ sub-evaluations in rule
    \State computations $\gets$ semantic evaluations in rule
    \State restrictions $\gets$ restrictions in rule
    \State 
    \ForAll {eval $\in$ evals}
        \State \textbf{evaluate}(eval)
    \EndFor
    \State
    \If{\Call{checkRestrictions}{restrictions}}
        \ForAll {computation $\in$ computations}
            \State \Call{evaluate}{computation}
            \Comment Order of evaluation enforced (see \autopageref{ssub:verifying_the_evaluation_criterion})
        \EndFor
        \State
        \State output the result
        \State \textbf{terminate}
    \Else
        \State \textbf{continue}
    \EndIf
\EndFor
\end{algorithmic}
\end{breakablealgorithm}

% subsubsection the_final_form_of_the_user_defined_semantics (end)

% subsection user_defined_semantics (end)

\subsection{Special-Case Semantics} % (fold)
\label{sub:special_case_semantics}
In addition to the user-defined semantics based upon the big-step operational semantics, Requirements~\reqref{req:EnvironmentAccesses} and~\reqref{req:ExtensionMechanisms} mean that other kinds of semantics must be defined.
These include the special-case semantics for providing language features that cannot be proved by the termination mechanism, and semantic rules for environment accesses and stores.
The below sections explore the design of these additional semantic features.

\subsubsection{Semantic Special Syntax} % (fold)
\label{ssub:semantic_special_syntax}
The semantic \glspl{ssr} are a method for \gls{metaspec} to provide additional, useful language features that cannot be proven by the usual termination proof mechanism. 
While they might technically be able to be expressed in big-step operational semantics, the restrictions placed on the standard semantic form (\autoref{sub:user_defined_semantics}) mean that they cannot be expressed in metaspec.
This includes features like data traversal and function calls. \\

The syntactic design for these pieces of special syntax was inspired by the notion of `library functions' in other languages.
Much like Haskell has \mintinline{haskell}{fmap fn s}, these pieces of special syntax provide advanced functionality to the user for little effort on their part.
As a result, the syntax was designed to look like a function call: \mintinline{text}{<ssr> "("[<arg>] {"," <arg>} ")"}.\\

\begin{listing}[!htb]
\begin{minted}[firstnumber=303, fontsize=\blockfont]{text}
special-syntax-rule = 
    semantic-special-syntax,
    special-syntax-start,
    [ syntax-access-block | environment-access-rule ],
    { semantic-list-delimiter, (syntax-access-block|environment-access-rule) },
    special-syntax-end;
\end{minted}
\caption{Semantic Special Syntax}
\label{lst:semantic_special_syntax}
\end{listing}

Syntactically, they consist of a language keyword and an argument list, and the exact semantics of each is defined in \autoref{sec:special_language_features}.
The argument list may consist of syntax accesses (see \autoref{sub:accessing_syntax_from_the_semantics} or environment accesses (see \autoref{ssub:environment_access_rules}), the latter of which allow them to retrieve values from the environment.
They are described by the \gls{ebnf} expression in \autoref{lst:semantic_special_syntax} and are subject to verification as described in \autoref{ssub:verification_of_special_syntax_rules}.\\

These \glspl{ssr} can occur inside the user-defined semantics (\autoref{sub:user_defined_semantics}) as they can provide useful functionality for defining the semantics of more complex expressions.
In allowing them to be nested in such expressions, it removes the need to indirect through another non-terminal where these might occur. 
This brings no additional complexity to the proof mechanism and brings the potential for additional simplicity to the language specification.
This is likely to somewhat reduce the potential for bugs. 

% subsubsection semantic_special_syntax (end)

\subsubsection{Environment Input Rules} % (fold)
\label{ssub:environment_input_rules}
Environment input rules are intended to provide mechanisms for \gls{dsl} designers to store and retrieve values in the environment. 
The environment was conceptualised as an unscoped key-value store to allow for variable definitions, function definitions and any other use that the user can come up with. \\

Syntactically, the environment was represented by a special reserved symbol in \gls{metaspec}, \mintinline{text}{e}. 
The notion behind this was to ensure that all environment stores and accesses could be found at a glance by the user, aiding their understanding of any stateful behaviour in the language semantics. 
Unfortunately this has turned out to not be entirely true, as some special syntax calls define (or access) elements in the environment without direct use of the symbol.\\

Beyond that, the store into the environment uses a newly introduced operator \mintinline{text}{<--}. 
This operator was designed to visually mimic the \mintinline{haskell}{<-} monadic extraction operator, and aims to imply getting something out of the syntax and storing it somewhere else. 
Stores in the environment operate in a key-value function, with the key and value separated by the \mintinline{text}{:} (defining) symbol.
In short: \mintinline{text}{e <-- <key>[n] : <values...>}, for some arbitrary non-terminals \mintinline{text}{<key>} and \mintinline{text}{<values>}.
Environment input rules are hence defined by the \gls{ebnf} expression given in \autoref{lst:environment_input_rules}.

\begin{listing}[!htb]
\begin{minted}[firstnumber=279]{text}
environment-input-rule =
    semantic-type, 
    semantic-environment-symbol,
    semantic-environment-input-symbol,
    syntax-access-block, (* key *)
    environment-defines-symbol,
    syntax-access-list;
\end{minted}
\caption{Environment Input Rules}
\label{lst:environment_input_rules}
\end{listing}

% subsubsection environment_input_rules (end)

\subsubsection{Environment Access Rules} % (fold)
\label{ssub:environment_access_rules}
Much like for environment inputs, environment accesses also took syntactic inspiration from features of common programming languages. 
Initial syntax for the environment accesses treated it much like a dictionary, where a key was given to retrieve a value: \mintinline{text}{e[<key>]}.
However, it is possible that certain structures in the environment have properties of their own, and it was immediately apparent that it could be syntactically awkward to access these.\\

To remove this awkwardness, the final syntax in \gls{metaspec} defines \mintinline{text}{.} as the environment access symbol, taking inspiration from object property access in most C++ derived languages.
The idea behind this is that it can be chained in a visually pleasing way, accessing properties in the environment and the environment's children: \mintinline{text}{e.<nt-1>.<nt-2> ...}
These accesses are based on the syntax of the triggering rule, and hence the \gls{ebnf} expression for such productions is as in \autoref{lst:environment_access_rules}.

\begin{listing}[!htb]
\begin{minted}[firstnumber=297]{text}
environment-access-rule = 
    semantic-environment-symbol,
    environment-access-symbol,
    syntax-access-block,
    { environment-access-symbol, syntax-access-block }; 
\end{minted}
\caption{Environment Access Rules}
\label{lst:environment_access_rules}
\end{listing}

% subsubsection environment_access_rules (end)

% subsection special_case_semantics (end)

% section specifying_the_language_semantics (end)

\section{Combining Syntax and Semantics} % (fold)
\label{sec:combining_syntax_and_semantics}
With a robust metasyntactic format for both syntax and semantics in \gls{metaspec}, the remaining challenge of the metalanguage was to combine the two. 
This combination is crucial to the design of the language, as stated in Requirement~\reqref{req:IntegratedSyntaxandSemanticSpecification}.
The combination of the two had to admit all of the four categories of semantic rule (see \autoref{sec:specifying_the_language_semantics}), and associate them with a syntactic production, and hence the choice of both \textit{syntax} and the \textit{combination point} was important. 

\subsection{Choosing the Combination Syntax} % (fold)
\label{sub:choosing_the_combination_syntax}
As the marriage of syntax and semantic specification is at the crux of \gls{metaspec} itself, the syntax for combining the two metasyntactic notations had some significance. \\

The initial inspiration for the syntax came from the notion of logical implication: that the syntax implied a given behaviour. 
There were concerns, however, about this being too literal an idea, and hence rather than \mintinline{text}{->}, the extended arrow operator (\mintinline{text}{-->}) was used. 
This was coupled with a block syntax, wrapping all the semantics for a given production up in braces (\mintinline{text}{{}}) to act as a visual delimiter.
The resultant \gls{ebnf} production can be seen in \autoref{lst:language_rule_semantics}. 

\begin{listing}[!htb]
\begin{minted}[firstnumber=267]{text}
language-rule-semantics = 
    semantic-behaves-as,
    semantic-block-start,
    semantic-rule,
    semantic-block-end;
\end{minted}
\caption{Language Rule Semantics}
\label{lst:language_rule_semantics}
\end{listing}

% subsection choosing_the_combination_syntax (end)

\subsection{Determining the Combination Point} % (fold)
\label{sub:determining_the_combination_point}
Just as important as the combination syntax was the choice of where to combine the syntactic and semantic descriptions.\\

The initial choice of the connection point went somewhat awry, with the grammar specifying that the semantics came directly after the full syntactic specification: \\
\mintinline{text}{production, "-->", "{", semantics, "}"}.
For some fairly obvious reasons, this did not work.
All of these boil down to one essential fact: syntactic productions contain alternations:
\begin{itemize}
    \item It is trivial to imagine a production where a non-terminal \mintinline{text}{<a>} occurs in the first alternation and is referred to in the semantics, but does not occur in the second alternation. 
    \item It is just as simple to imagine each alternations containing different terminal symbols (e.g. \mintinline{text}{<a> "+" <a> | <a> "-" <a>}) which may want to result in different behaviour. 
\end{itemize}

Clearly, the semantics should instead be specified at the level of the alternation instead: \mintinline{text}{<a> --> {} | <b> --> {} | ...} and so on.
This provides the maximum amount of flexibility to the language designers, allowing them to specify more complex productions that are logically grouped. An example of this is the `arithmetic operation' production shown in \autoref{lst:arithmetic_operations_in_metaspec}.\\

\begin{listing}[!htb]
\begin{minted}[numbers=none]{text} 
<arith-op> ::=
    <arith-expr> "+" <arith-expr> --> {
        num n : {n = n1 + n2}() :
            {num n1 <= <arith-expr>[0]}, {num n2 <= <arith-expr>[1]}
    } |
    <arith-expr> "-" <arith-expr> --> {
        num n : {n = n1 - n2}() :
            {num n1 <= <arith-expr>[0]}, {num n2 <= <arith-expr>[1]}
    } |
    <arith-expr> "*" <arith-expr> --> {
        num n : {n = n1 * n2}() :
            {num n1 <= <arith-expr>[0]}, {num n2 <= <arith-expr>[1]}
    } |
    <arith-expr> "/" <arith-expr> --> {
        num n : {n = n1 / n2}() :
            {num n1 <= <arith-expr>[0]}, {num n2 <= <arith-expr>[1]}
    } |
    <arith-expr> "^" <arith-expr> --> {
        num n : {n = n1 ^ n2}() :
            {num n1 <= <arith-expr>[0]}, {num n2 <= <arith-expr>[1]}
    };
\end{minted}
\caption{Arithmetic Operations in Metaspec}
\label{lst:arithmetic_operations_in_metaspec}
\end{listing}

It is hence clear that the decision to associate semantics with each block in a top-level alternation is the most flexible choice for \gls{metaspec} and affords the \gls{dsl} designers the most flexibility without creating an overly convoluted syntax. 

% subsection determining_the_combination_point (end)

% section combining_syntax_and_semantics (end)

% chapter designing_the_metalanguage (end)

\chapter{Architecture and Algorithms} % (fold)
\label{cha:architecture_and_algorithms}
Following on from the design work put into the \gls{metaspec} (see \autoref{cha:designing_the_metalanguage}), significant time and effort was then invested into the design of the metacompiler toolchain itself, as well as the core algorithms and theory that it uses. 
This chapter aims to provide a high-level overview of the \gls{absol} toolchain, showing the main system components and linking these to the high-level requirements.
It also provides a firm background to the design and development of the algorithms utilised by the metacompiler itself, and develops the rigour behind the special-case semantics. \\

It should be noted that unlike \gls{metaspec}, which was designed over one consistent period, the design and implementation periods for the \gls{absol} toolchain were interleaved heavily.
This provided the opportunity to iterate on the designs where necessary.
Instances of this occurring will be noted throughout this chapter, but the designs presented below will be the final ones. \\

This section explores three main components of the design work that has gone into \gls{absol}.
The first is the design of the software itself, including the architectural details and high-level functionality.
This is followed by a detailed exploration of the core algorithmic work that underpins the entire project.
These algorithms took some significant insight and development to ensure that they guaranteed the correctness of the language upon which they operate.
Finally, this section examines the designs for the special language features, from which the toolchain derives so much of its flexibility.

\section{Designing the Metacompiler --- ABSOL} % (fold)
\label{sec:designing_the_metacompiler_absol}
As for any large system, it is important to be able to visualise the way in which the individual system components interact and are integrated. 
\gls{absol} itself is composed at a high-level of two main modules, Metaparse and Metaverify.
These modules form a natural segmentation of the work that the compiler has to do, and is are naturally subdivided internally.
\gls{absol} is best visualised as a pipeline, and the arrows illustrate the flow of data through the metacompiler.
The pipeline-style architecture is very suitable for \gls{absol}, as each stage of the toolchain depends only on the output of the previous stage.
The high-level architecture of the metacompiler is illustrated in \autoref{fig:absol_high_level_architectural_diagram} below.\\

\begin{figure}[!htb]
    \centering
    \includegraphics[width=1\textwidth]{resources/images/metacompiler_pipeline_architecture.pdf}
    \caption{ABSOL High-Level Architectural Diagram}
    \label{fig:absol_high_level_architectural_diagram}
\end{figure}

The main components of the metacompiler can be described as follows:
\begin{itemize}
    \item \textbf{Metaparse:} Responsible for the generation of an \gls{ast} from the input metaspec file and verifying some precondition properties for the verification engine. 
    It has two main stages: lexing and parsing (which also performs precondition verification). 
    For further description of this stage please see \autoref{sub:metaparse_ast_generation}.
    \item \textbf{Metaverify:} Responsible for verifying the termination properties of the language.
    It consists of three main stages: \gls{ast} preprocessing, semantic inference and semantic verification.
    For further description of this stage please see \autoref{sub:metaverify_the_verification_engine}.
\end{itemize}

The following sections provide a detailed exploration of the design of these two main components, with a focus on the reasons behind the design choices that have been made.
It was known at design stage that the implementation language would be Haskell, so sporadic references are made to the language choice and how it may have impacted the system design.
Also below is an exploration of how the eventual intention for \gls{dsl} execution and type-checking have impacted the design of the metacompiler itself. 

\subsection{The Metacompiler Front-End} % (fold)
\label{sub:the_metacompiler_front_end}
The metacompiler front-end is the harness that wraps around both Metaparse and Metaverify to allow them to interact with the world. 
It is not truly part of the requirements for the toolchain, but its existence is necessary to enable the real-world applicability of the core features.
The front-end needed to handle the following features:
\begin{itemize}
    \item \textbf{Command-Line Option Parsing:} The metacompiler front-end needed to be capable of parsing options given to the tool on the command line.
    These options are used to provide inputs to the metacompiler (such as the file to run on) and configure its behaviour. 
    \item \textbf{File Loading:} The front-end also needed to be capable of loading the input language specification.
    \item \textbf{Pipeline Coordination:} The final task for the front-end was to pass data between the pipeline stages.
\end{itemize}

Initial design for the metacompiler front-end focused on the identification of separate logical concerns for the front end component.
Given that its responsibilities are entirely distinct, this posed little issue.
As any target language, Haskell included, would provide facilities for file loading and the movement of data, the only component that required any significant design work was the command-line options. 

\subsubsection{Designing the Command-Line Options} % (fold)
\label{ssub:designing_the_command_line_options}
With Haskell's robust support for parsing libraries, including those dedicated for parsing command-line options, little consideration had to be given to the design of the \gls{cli} argument parser itself.
As a counterpoint, the options required careful consideration as they would impact the abilities of the user to control the system behaviour. 
The design of the command-line arguments took place before the scope reduction (as discussed in \autoref{sub:removed_design_elements}) took place.
Hence, there are references in this list to options that are not used in the final design.\\

Determining appropriate configuration options resulted from an examination of the responsibilities of the metacompiler and a consideration of which portions of its behaviour might benefit from being configurable.
The analysis was further informed by the behaviour of common build tools such as the generic \mintinline{text}{gmake} (GNU-Make) and the Haskell-specific \mintinline{text}{stack}.
The following list of options resulted:
\begin{itemize}
    \item \textbf{Input Filename:} The metacompiler operates on a language specification, and so needs to be provided with the input file.
    \item \textbf{Analysis Verbosity:} An oft-encountered configuration parameter for processes with output to the command-line, being able to control the verbosity of the language analysis and error reporting process seemed useful.
    As a result, the options design included a flag to enable full reporting.
    \item \textbf{Log File:} It seemed reasonable for users to want to output the analysis results to a log-file rather than to \mintinline{text}{stdout}.
    This configuration parameter was intended to do exactly that, allowing the user to easily store the results of analysis of their language.
    \item \textbf{Output Directory:} Provided to serve both the log file output and any eventual build artefacts, many build tools provide the ability to specify the target directory for any output files.
    This seemed important enough for the option to be included in this list. 
    \item \textbf{Language Name / Version:} As discussed in \autoref{sec:the_top_level_definitions}, \gls{metaspec} specifications contain language metadata. 
    It is conceivable, however, that \gls{dsl} authors may want to temporarily override these settings at build time.
    To this end, flags for providing alternative language names and language versions should be provided. 
    \item \textbf{Reporting:} As the metacompiler was originally intended to perform code-generation, it was conceivable that the users of the tool might want to verify the language they are developing without generating code from the specification. 
    Providing a flag to enable this would shorten the write-compile-debug cycle so common in development, and hence improve \gls{dsl} development workflow.
    \item \textbf{Cleanup:} Also mostly intended to interact with the code-generation stage, the cleanup flag was meant to delete all build artefacts resulting from code-generation.
    This is a common feature provided by build systems (e.g. \mintinline{sh}{make clean} or \mintinline{sh}{stack cleanup}, and hence seemed important to include.
\end{itemize}

% subsubsection designing_the_command_line_options (end)

% subsection the_metacompiler_front_end (end)

\subsection{Metaparse --- AST Generation} % (fold)
\label{sub:metaparse_ast_generation}
Metaparse is the subsystem of \gls{absol} that performs generation of an \gls{ast} from the input language specification.
This meant that it was responsible for providing both a \gls{lexer} and a \gls{parser} for \gls{metaspec}, as stated in Requirement~\reqref{req:ParseMetaspec}.
During the design process, however, it became clear that the parser was also the most appropriate place to perform some initial precondition checking required by the verification algorithms at later stages of the pipeline. 
The design of Metaparse was therefore heavily influenced both by its \textit{location} in the metacompiler pipeline and the \textit{data} which it was dealing with.
The goals for the design of Metaparse were hence set out as follows:
\begin{itemize}
    \item \textbf{Produce an AST from a Metaspec File:} The lexing and parsing processes had to result in an AST from the input file, or in the case where the input file was incorrect, produce a helpful syntax error. 
    \item \textbf{Verify the Preconditions for Metaverify:} As a parser, Metaparse will have had to traverse all portions of the \gls{ast}, allowing it to have all the information necessary to validate the verification preconditions.
\end{itemize}

\subsubsection{The Parser and Lexer Design} % (fold)
\label{ssub:the_parser_and_lexer_design}
Initial considerations for the design of Metaparse called for a two-stage design, with separate lexing and parsing stages. 
The original intention was for the lexer to operate directly on the character stream of the \gls{metaspec} file, producing a token stream which would then be provided to the parser for the creation of an \gls{ast}. 
This design represented a good separation of concerns, and provided an appropriate level of decoupling between the two processes. \\

However, the design of Metaparse was later informed quite significantly by the choice of implementation technology. 
Megaparsec, as discussed in \autoref{sub:megaparsec_improved_parsing}, provides a significantly improved set of utilities for creating lexing primitives for a given language. 
This meant, from a design standpoint, that merging the lexing and parsing steps would bring significant benefits to the system design:
\begin{itemize}
    \item \textbf{Efficiency:} As the \gls{ast}-generation process would now only have to make one pass over the character stream, performing one process of characters $\to$ tokens (contrasted with the dual passes for characters $\to$ tokens and then tokens $\to$ \gls{ast}), this means that the process as a whole is more efficient.
    \item \textbf{Simplicity:} Merging the two processes serves to simplify the design due to the removal of a (mostly unnecessary) boundary between lexing and parsing.
    \item \textbf{Ease of Implementation:} This aforementioned simplicity will likely mean that the implementation of Metaparse is simplified.
    On a time-constrained project this is a big boon, and a big benefit of the integrated design.
\end{itemize}

While the merging of separate processes like this is often viewed as an improper separation of concerns, it is sometimes the case that certain technologies permit a novel approach that aids significantly in simplifying the design.
With Megaparsec, this is definitely the case as the lexing primitives it provides are \textit{intended} to be used in an integrated fashion to provide a fluid and easy to read parser. \\

Using Megaparsec for the parser also brings with it additional benefits for the design of the parser. 
Due to the provision of excellent parse-error reporting facilities, no design effort has to be dedicated to providing useful errors at parse time to the \gls{dsl} designer. 
Additionally, its nature as a parser combinator library means that the elements of the parser are clearly readable as expressions for the language grammar. 

% subsubsection the_parser_and_lexer_design (end)

\subsubsection{Designing the Metaparse AST} % (fold)
\label{ssub:designing_the_metaparse_ast}
Megaparsec operates on a very strongly-typed set of custom \gls{ast} data-types. 
This `tree' of types is one of the main components of the Metaparse design, and is created in accordance with the grammar for \gls{metaspec} that is outlined in \autoref{cha:designing_the_metalanguage}.
The process of creating this set of types is more a development task than a design one, as it is an almost direct translation of the \gls{metaspec} \gls{ebnf} grammar into Haskell's data types. \\

The key design element of this transformational process was to determine the level of abstraction at which it should take place:
\begin{itemize}
    \item \textbf{Representation of Productions:} Certain elements of the \gls{metaspec} grammar are not able to be directly represented in the form of Haskell data-types.
    As a result, the design process had to recognise where it was not possible to provide a direct representation and perform a transformation of the grammar while still remaining semantically correct.
    Due to the modular nature of the grammar, there were only a few instances where this was required to take place, an example of which can be seen below.
    \begin{minted}[xleftmargin=1cm, numbers=none]{text}
[ syntax-access-block | environment-access-rule ]
    \end{minted}
    This portion of the grammar exists within a more complex production, and representing the possibilities of this in Haskell would be difficult.
    Instead, it was factored out into a separate type representing this alternation, seen below
    \begin{minted}[xleftmargin=1cm, numbers=none]{text}
access-block-or-rule = 
    syntax-access-block | environment-access-rule ;
    \end{minted}
    \item \textbf{Indirections:} The structure of Haskell data-types (usually concerning type recursion) can preclude the representation of certain productions in \gls{ebnf}. 
    In such cases, it is often necessary to introduce an indirect type (equivalent to a production with a single node itself) to allow proper representation in the \gls{ast}. 
    An example of this indirection or `factoring out' can be seen below.
    The following production cannot be directly represented as Haskell data types:
    \begin{minted}[xleftmargin=1cm, numbers=none]{text}
semantic-rule = 
    environment-input-rule |
    environment-access-rule |
    special-syntax-rule |
    semantic-evaluation-rule-list;
    \end{minted}
    Some of the rule cases would be factored out as follows to ensure proper representation in Haskell while still retaining the correct grammatical structure.
    \begin{minted}[xleftmargin=1cm, numbers=none]{text}
semantic-rule = 
    environment-input-rule |
    environment-access-rule-proxy |
    special-syntax-rule-proxy |
    semantic-evaluation-rule-list;

environment-access-rule-proxy = environment-access-rule;
special-syntax-rule-proxy = special-syntax-rule;
    \end{minted}
\end{itemize}

These temporary alterations to the grammar in aid of designing the Metaparse \gls{ast} data-types were performed during implementation of the types themselves.
This means that the temporary alterations to the grammar were not retained in any consistent format, as they did not add any additional clarity to the \gls{ebnf} representation of \gls{metaspec}.
In most cases, they actually obscured the language grammar, which was undesirable.

% subsubsection designing_the_metaparse_ast (end)

\subsubsection{Precondition Verification in the Parser} % (fold)
\label{ssub:precondition_verification_in_the_parser}
The verification algorithms discussed in \autoref{sec:the_core_algorithms} are only correct in the presence of certain preconditions (Requirement~\reqref{req:VerifyLanguageConstruction}).
For more details on why these are required for verification correctness, please see \autoref{sub:verifier_precondition_validation} on \autopageref{sub:verifier_precondition_validation}.
These preconditions can briefly be described as follows:
\begin{itemize}
    \item All non-terminals that are \textit{used} must be \textit{defined} in scope (this may include non-terminals provided by language features).
    \item All non-terminals must be \textit{defined exactly once}.
    \item All used types must be \textit{in scope}.
    \item All used special-syntax must be \textit{in scope}. 
\end{itemize}

While these preconditions could easily be verified in a separate pass over the generated \gls{ast}, the parser itself has to touch every node in the AST as it generates the data structure.
This makes it an appropriate location to collect the necessary data to perform the verification.\\

The verification of these preconditions is an inherently stateful operation, and hence required some alterations to the initial design of the parser.
While, by default, a Megaparsec parser tracks some state (in order to provide appropriate parser error messages), this state is not accessible to the user. 
This means that the design must incorporate some additional state to track the data required to perform this verification.
It needs to record data as follows:
\begin{itemize}
    \item The non-terminals, types and special-syntax \textit{brought into scope} by the language imports.
    \item The non-terminals \textit{defined} by the language specification.
    \item The non-terminals \textit{used} by the language specification. 
\end{itemize}

For details of the precondition verification algorithm itself, see \autoref{ssub:the_precondition_checking_algorithm} on \autopageref{ssub:the_precondition_checking_algorithm}.

% subsubsection precondition_verification_in_the_parser (end)

\subsubsection{The Final Design for Metaparse} % (fold)
\label{ssub:the_final_design_for_metaparse}
Combining the results of the two-stage design process illustrated in the previous sections, the final design for Metaparse is as follows:
\begin{itemize}
    \item \textbf{Integrated Lexer and Parser:} The parsing process will use an integrated set of lexer primitives to allow a one-pass parse of the input \gls{metaspec} specification. 
    \item \textbf{Full AST Data-Type Tree:} The parser will operate over a comprehensive Haskell type-level representation of the \gls{metaspec} grammar. 
    This typed \gls{ast} will have undergone the necessary transformations so as to allow appropriate representation in Haskell while retaining grammatical correctness. 
    \item \textbf{Integration of Parser State:} The parser will have an integrated user state that is able to track all the data defined in \autoref{ssub:precondition_verification_in_the_parser}.
    This data will be tracked as the parse takes place. 
    \item \textbf{Utilisation of Parser State for Precondition Validation:}
    The parser will perform precondition validation on the tracked state using the algorithms described in \autoref{sub:verifier_precondition_validation}.
\end{itemize}

% subsubsection the_final_design_for_metaparse (end)

% subsection metaparse_ast_generation (end)

\subsection{Metaverify --- The Verification Engine} % (fold)
\label{sub:metaverify_the_verification_engine}
Metaverify is the second of the main subsystems in \gls{absol} and it is intended to perform verification of the input language. 
Verification, in the case of the metacompiler, means determining whether the language is guaranteed to always terminate and, in cases where it does not, providing diagnostics as to why.
This component aims to satisfy Requirements~\reqref{req:VerifySemanticForm}, \reqref{req:VerifySemanticGuards} and \reqref{req:GenerationofVerificationReports}, and hence has the following goals:
\begin{itemize}
    \item \textbf{Verification of the Semantic Form:} Checking the form of all of the semantics defined for the language to ensure that they satisfy the necessary criteria for guaranteed termination. 
    \item \textbf{Verification of the Semantic Guards:} Ensuring totality of all defined semantics.
    This means that there will always be a semantics for any portion of \gls{dsl} code. 
    \item \textbf{Generation of Reports:} The generation of reports on the termination status of the language. 
    In cases where the language cannot be shown to always terminate, these reports contain information specifying exactly why it does not. 
\end{itemize}

The verification engine consists of a multi-stage process, that first performs a preprocessing step (see \autoref{ssub:the_verification_preprocessor}), and then traverses the resultant language structure to determine if the language terminates. 
The algorithms used to check termination are not discussed in this section, and can instead be found in \autoref{sec:the_core_algorithms}.

\subsubsection{The Verification Preprocessor} % (fold)
\label{ssub:the_verification_preprocessor}
While it contains all necessary data to perform verification, the \gls{ast} that is generated by Metaparse is not the most efficient structure to verify the language. 
Simply put, this is because the verification algorithm requires some summary data, while other portions of the AST are just irrelevant.
To rectify this, the need for a \textit{preprocessing step} was identified at the design stage. \\

The role of the preprocessor is to take the full \gls{ast} that is output by Metaparse and extract the relevant data from it. 
At a high level this means it is an algorithm with inputs and outputs as follows, and no dependence on external state:
\begin{itemize}
    \item \textbf{Input(s):} The \gls{ast} for the input language $L$.
    \item \textbf{Output(s):} The truths defined for $L$, the start rule for $L$ and the additional language productions (non-terminals) defined for $L$. 
\end{itemize}

Due to the fact that the methods for traversing an \gls{ast} data-structure are inherently tied to the representation of the structure, it is somewhat difficult to represent this algorithm in a general format.
The algorithm is also responsible for collating these outputs in such a fashion that they are associated with the required tracking data for their termination state (for the contents of this data see \nameref{ssub:reporting_structures} on \autopageref{ssub:reporting_structures}).
This tracking data is known to the project as a \textit{tag}, and these are described in more detail later. 
The best algorithmic representation can thus be seen in \autoref{alg:the_verification_preprocessor_algorithm}, which assumes the existence of a number of functions as follows:
\begin{itemize}
    \item \textsc{findStartRule}() --- Finds the start rule in the AST and extracts it, before associating it with a tag.
    \item \textsc{defaultTag}() --- Generates a default tag for the rule (see \autoref{ssub:reporting_structures} for more information).
    \item \textsc{extractTruths}() --- Finds the defined truths for the languages, and extracts a list of the associated non-terminal symbols.
    These symbols are those known to terminate.
    \item \textsc{extractProductions}() --- Finds all productions (non-terminals) defined by the language and extracts them as pairs (Non-Terminal, Production Body). 
\end{itemize}

\begin{algorithm}[!htb]
\begin{algorithmic}
\Require{The language AST is available as input}
\Ensure{The collated output matches the relevant portions of the AST}
\State
\State ast $\gets$ input
\State
\State $P_s \gets$ \textsc{findStartRule}() 
\State startRule $\gets$ (\textsc{defaultTag}(), $P_s$)
\State $T \gets$ \textsc{extractTruths}()
\State $P_n \gets$ \textsc{extractProductions}()
\State $P \gets []$
\Comment The empty list
\ForAll{$(N, P_i) \in P_n$}
\Comment Transform productions into searchable form
    \State $P \gets (N, (\textsc{defaultTag}(), P_i)) : P$
    \Comment where $:$ is the list cons operator
\EndFor
\State
\State return (startRule, $T$, $P$)
\end{algorithmic}
\caption{The Verification Preprocessor Algorithm}
\label{alg:the_verification_preprocessor_algorithm}
\end{algorithm}

% subsubsection the_verification_preprocessor (end)

\subsubsection{Designing the Verification Engine} % (fold)
\label{ssub:designing_the_verification_engine}
While the verification engine itself is algorithmically dependent on the theory explored in \autoref{sec:the_core_algorithms}, this theory alone does not make a design. 
From an examination of these algorithms two main things are immediately apparent:
\begin{itemize}
    \item \textbf{Stateful Nature:} The algorithms operate in a stateful fashion.
    This is because they track the current state of the verification engine at any given point.
    \item \textbf{Integrated Traversal:} While there are two disparate semantic verification algorithms --- one dealing with semantic inference (\autoref{sub:semantic_inference}) and the other verifying defined semantics (\autoref{sub:user_defined_semantic_form_verification}) --- they both utilise the same traversal mechanism for the verification data. 
    This implies that both kinds of verification should be done in one pass, and that is indeed the form of the algorithm that performs the traversal in \autoref{sub:verifier_traversal}. 
\end{itemize}

Given that the bulk of this system component involves concrete implementations of the verification algorithms, the core design problem thus revolves around the state. 
The state for this algorithm provides three main functions:
\begin{itemize}
    \item \textbf{Tracking Verification Status:} The state must contain the current state of the verification for the language, including the tags for each production in the language.
    \item \textbf{Containing Algorithmic Resources:} The state must contain the information required by the verification algorithm.
    This predominantly means the outputs of the preprocessor algorithm as discussed in \nameref{ssub:the_verification_preprocessor}.
    \item \textbf{Tracking Temporary State:} As the algorithm progresses, certain portions of it need some awareness of where they have been.
    This helps to prevent mutual recursion between productions and other such non-terminating behaviour. 
\end{itemize}

% subsubsection designing_the_verification_engine (end)

\subsubsection{Reporting Structures} % (fold)
\label{ssub:reporting_structures}
While the generation of the reports themselves (Requirement~\reqref{req:GenerationofVerificationReports}) is nothing but an exercise in pretty-printing (and thus required no real design work), the recursive nature of the algorithm posed some difficulty for collating this data in the first place. 
Tracking the termination states, as defined in the verification algorithm (\autoref{ssub:termination_states}), is one of the more interesting design challenges for the Metaverify component.\\

From an examination of the core algorithm of the verifier (\autoref{sec:the_core_algorithms}), it is clear that the Metaverify component requires some practical method of tracking the termination state of each rule. 
However, this is not as simple as it might appear at first glance: in the cases where the rule is defined as non-terminating, the tag must also track the associated diagnostic data. \\

This is further complicated by the fact that the verification algorithms might involve arbitrary degrees of recursion. 
Consider, a production \mintinline{text}{<p> ::= <a> | <b> | <c>} where the algorithm recurses on each of the non-terminals in the alternation, for example. 
As each of these may have a different termination result, the result for \mintinline{text}{<p>} is the result of combining all the sub-results. 
The termination result combination is specified in Subsection~\ref{ssub:termination_states}, so the design problem here is how to retain the correct non-termination data. 
In the case of a non-termination, the following data should be retained:
\begin{itemize}
    \item The type of the non-termination.
    \item An error describing the exact nature of the non-termination.
    \item A trace of where the non-termination occurred in relation to the rule being tagged. 
\end{itemize}

At each combination point where the recursive calls return, therefore, two things need to happen:
\begin{enumerate}
    \item All of the non-termination reasons need to be stored.
    \item The non-termination trace needs to be updated for \textit{all} of the stored reasons. 
\end{enumerate}

% subsubsection reporting_structures (end)

\subsubsection{Finalising the Metaverify Design} % (fold)
\label{ssub:finalising_the_metaverify_design}
From the above design work, the final design for the Metaverify system component can be understood as follows:
\begin{itemize}
    \item \textbf{Verification Preprocessor:} The component must implement a preprocessor to transform the \gls{ast} into an appropriate form for verification.
    \item \textbf{Verification State:} Metaverify must provide a mechanism for tracking the state that the algorithm requires to operate (see \nameref{ssub:designing_the_verification_engine}).
    \item \textbf{Reporting Structures:} The verification engine must be able to accurately track the termination state (and associated metadata) of all of the portions of the language that it is verifying. 
    This is done via a reporting structure as discussed in \nameref{ssub:reporting_structures}.
\end{itemize}

% subsubsection finalising_the_metaverify_design (end)

% subsection metaverify_the_verification_engine (end)

\subsection{Type-Checking} % (fold)
\label{sub:type_checking}
As part of the process of designing \gls{absol}, significant consideration was given to performing type-checking at the language level. 
It emerged fairly quickly, however, that this would not be possible. \\

This unfortunate circumstance actually arises from one of the main points of flexibility designed into the typing discipline of the semantics. 
As \gls{metaspec} provides a number of generic functions (those for which the types involved are not known at language design time), it is impossible to typecheck the language at this stage.
This impossibility is concisely embodied in the \mintinline{text}{any} type (part of the \mintinline{text}{base} module), that acts as an effective type `hole'.
The presence of \mintinline{text}{any} as a type in the expression means that the type cannot be determined at the language level, as it can only be filled when types are instantiated at program compile time.
As a result, the process of type-checking the DSL programs is deferred until program compilation time, rather than performed at the language verification stage.\\

There was some initial worry during this process that not being able to verify the types would compromise the correctness of the language verification, but thankfully this is not the case:
\begin{itemize}
    \item The language checker ensures that any expression in the \gls{dsl} that can be parsed has associated semantics.
    \item It also guarantees that these semantics will always terminate.
    \item At compile-time, the \gls{dsl} compiler is able to reject the input program if the types for all different instantiations of the \mintinline{text}{any} type in the program do not resolve correctly. 
\end{itemize}

As a result, it is clear that not being able to typecheck at the level of the language semantics will not have an impact on the correctness of the verification mechanism.

% subsection type_checking (end)

\subsection{Removed Design Elements} % (fold)
\label{sub:removed_design_elements}
The design presented at the start of this section in \autoref{fig:absol_high_level_architectural_diagram} is not the original design for the system.
Towards the start of the design phase for \gls{absol}, the scope of the project was much more grand. 
In addition to the areas in the final design, the system was also originally intended to encompass the following:
\begin{itemize}
    \item \textbf{DSL Compiler Generation:} The original scope of the metacompiler called for the generation of a DSL compiler from the language specification provided by \gls{metaspec}. 
    While this was an important part of making the toolchain truly useful for designing and implementing \glspl{dsl}, it is far from novel.
    The code-generation stage would have been heavily based upon work explored in \autoref{sec:automating_the_generation_of_the_compiler}, and would require significant design and implementation effort, likely exceeding the amount of time available to the project.
    \item \textbf{Compilation of DSL Programs:} The generated code for the DSL compiler would be intended to be compiled itself. 
    The resultant build artefact would be capable of transpiling programs in the target DSL to Haskell.
    These haskell programs would be prepared for interaction via the C \gls{ffi} with any host language that would want to interact with the DSL.
\end{itemize}

Unfortunately, due to scoping concerns, these additional parts of the metacompiler toolchain were ruled as `out of scope' for the project. 
While disappointing, care was taken to ensure that the novel parts of the project (the metalanguage and language verification capabilities) remained in scope as part of the core toolchain. 

% subsection removed_design_elements (end)

% section designing_the_metacompiler_absol (end)

\section{The Core Algorithms} % (fold)
\label{sec:the_core_algorithms}
At the core of the design of \gls{absol} are the verification algorithms. 
Being so crucial to the functionality and success of the project, these underwent significant design work in isolation, and evolved over time during the implementation.
Much as for the design of the system components themselves, the algorithmic designs evolved over time, and so this section aims to present the final algorithms as well as discuss any changes that occurred over time. \\

The core algorithms can be separated into two main parts.
The first is the precondition validation algorithm that takes place at parse time (see \nameref{ssub:precondition_verification_in_the_parser} on \autopageref{ssub:precondition_verification_in_the_parser}).
This ensures that the \gls{ast} resultant from parsing the language does not contain certain states permitted by the grammar but not by the verification algorithms preconditions. 
The second is the validation algorithm itself.
This is separated into the inference and validation sections, the latter of which is further subdivided into validation for the different kinds of semantic rules encountered. \\

The verification algorithms aim to verify the language against two kinds of non-totality:
\begin{enumerate}
    \item \textbf{Divergent Programs:} Programs \gls{diverge} due to infinite loops or non-terminating recursion.
    \item \textbf{Undefined Programs:} Programs that, for some language expressions, do not have defined semantics or have indeterminate semantics.
\end{enumerate}

The key recognition about this verification process is that it is \textit{enabled} only by the restricted form of the semantics that can be represented in \gls{metaspec}.
In general, this kind of verification is an instance of the \gls{halting_problem}, and hence impossible.
It is through placing stringent restrictions on the form of the semantics for these \glspl{dsl} that it is possible to make this process work. \\

The algorithmic descriptions in this section use a few concepts from functional programming to enable the algorithms to be more concise.
These are as follows:
\begin{itemize}
    \item \textbf{Partial Function Application:} Functions can be given some of their arguments, with the function called when the remaining arguments are filled.
    This supports use of the next concept.
    \item \textbf{Mapping:} A form of data traversal that applies an operation to all elements of a structure.
    \item \textbf{Reduction:} Otherwise known as folding, this traverses a structure and collects all of its elements into one using a binary operation. 
\end{itemize}

\subsection{Verifier Precondition Validation} % (fold)
\label{sub:verifier_precondition_validation}
For the verification algorithm to produce correct results, there are a set of preconditions that must hold. 
These preconditions are assumed to hold by the algorithms used for verification, and so they must be ensured to hold in order to guarantee that the verification of the language is correct.
These preconditions are as follows, where scope is defined in all cases as in the first precondition:
\begin{itemize}
    \item \label{item:non_terminal_usage} \textbf{Non-Terminal Usage:} Any non-terminal \textit{used} on the right-hand-side of a rule definition must be \textit{defined} in the current scope. 
    The non-terminals in the current scope are those defined by the \gls{dsl} language specification, and those imported by the language features specified in the \mintinline{text}{using} definition block (see \autoref{sec:the_top_level_definitions}).
    If this precondition was not satisfied, it would be possible to have productions whose sub-terms do not have defined semantics and thus that the language would not be total. 
    \item \textbf{Single Definition:} Any non-terminal of a given name must be \textit{defined only once} in the language scope.
    Were this precondition not satisfied, it would be possible to have indeterminate semantics for a given program as the choice of production is undefined.
    This means that \glspl{dsl} cannot have non-deterministic semantics (unless explicitly specified via randomness). 
    \item \textbf{Types in Scope:} All types used by the language definition must be \textit{in scope}.
    As types have associated semantics at \gls{dsl} compile-time, having types not in scope could lead to the generation of incorrect programs. 
    This condition, however, does not lead to incorrectness at the language level, as all types have defined semantics.
    \item \textbf{Special-Syntax in Scope:} All special syntactic forms used by the language definition must be in scope.
    As special syntactic forms provide language semantics, not having the syntax in scope when it is used would lead to undefined semantics for programs in the \gls{dsl}. 
\end{itemize}

Verifying that these preconditions hold is, as mentioned previously, performed as part of Metaparse (\nameref{ssub:precondition_verification_in_the_parser} on \autopageref{ssub:precondition_verification_in_the_parser}).
The verification is a stateful process, and hence the parser is required to track the following information: the non-terminals defined and used, and the types and special syntax in scope. 

\subsubsection{The Precondition Checking Algorithm} % (fold)
\label{ssub:the_precondition_checking_algorithm}
The algorithm to check that the preconditions hold is fairly simple, and operates in a three stage process.
The first stage of the process involves gathering the required information at the start: finding the imported language features and determining which non-terminals, types and special syntax elements they define.
The second stage of the process operates during the parse, and checks every type and piece of special syntax for being in scope. 
At the same time, any non-terminal being defined is checked against a list of already-defined non-terminals and any non-terminal being used is stored.
The final stage of the process operates on the lists of defined and used non-terminals, checking that all members of the latter are members of the former.\\

The algorithm for precondition verification is illustrated in \autoref{alg:the_precondition_verification_algorithm}, and assumes the existence of the following functions:
\begin{itemize}
    \item \textsc{importedNTs}(): This function obtains the set of non-terminals defined by the imported language features.
    \item \textsc{importedTypes}(): This function obtains the list of imported types.
    \item \textsc{importedSpecialSyntax}(): This function obtains the list of special-syntax keywords imported by the language features. 
    \item \textsc{isNT/isType/isSpecial}(node): Determines whether the current AST node is a non-terminal, type or special-syntax keyword respectively.
    \item \textsc{defining}(node): Determines whether the current non-terminal being parsed is being defined or used. 
    \item \textsc{error}(msg): Generates an error message.
\end{itemize}

This algorithm is sufficient to check that the preconditions for verification hold, and \textit{must} be executed before the verification algorithm itself.

\begin{breakablealgorithm}
\caption{The Precondition Verification Algorithm}
\label{alg:the_precondition_verification_algorithm}
\begin{algorithmic}
\Require{The Language AST is Present}
\Ensure{The preconditions are validated}
\State
\State definedNTs $\gets$ \textsc{importedNTs}()
\Comment A set.
\State usedNTs $\gets$ []
\Comment [] is the empty list
\State types $\gets$ \textsc{importedTypes}()
\Comment A list.
\State syntax $\gets$ \textsc{importedSpecialSyntax}()
\Comment A list.
\State
\ForAll{$N_i$ $\in$ language}
\Comment $N_i$ is a node in the AST
    \State \Call{checkNode}{$N_i$, definedNTs, usedNTs, types, syntax}
\EndFor
\State
\ForAll{NT $\in$ usedNTs}
    \If{NT $\notin$ definedNTs}
        \State \textsc{error}(NT used but not defined)
    \EndIf
\EndFor
\State
\Function{checkNode}{$N_t$, definedNTs, usedNTs, types, syntax}
    \If{\textsc{isNT}($N_t$)}
        \If{\textsc{defining}($N_t$)}
            \If{$N_t \in$ definedNTs}
                \State \textsc{error}($N_t$ already defined)
            \Else
                \State definedNTs $\gets N_t +$ definedNTs
                \Comment $+$ is the set addition operator
            \EndIf
        \Else
            \State usedNTs $\gets N_t : $ usedNTs
            \Comment $:$ is the list cons operation
        \EndIf
    \ElsIf{\textsc{isType}($N_t$) $\land$ $N_t \notin$ types}
        \State \textsc{error}($N_t$ not a defined type)
    \ElsIf{\textsc{isSpecial}($N_t$) $\land$ $N_t \notin$ syntax}
        \State \textsc{error}($N_t$ not valid syntax)
    \Else
        \State continue
    \EndIf
\EndFunction
\end{algorithmic}
\end{breakablealgorithm}

% subsubsection the_precondition_checking_algorithm (end)

% subsection verifier_precondition_validation (end)

\subsection{Verifier Traversal} % (fold)
\label{sub:verifier_traversal}
The aim of the verification algorithm for \gls{absol} is to ensure that the semantics provided by the \gls{dsl} designer match the set of restrictions imposed to allow the termination proof. 
The exact nature of these restrictions varies based on the kind of semantics being considered (of the four available in \gls{metaspec} --- see \autoref{lst:language_rule_semantics}), but in all cases it is the job of this algorithm to ensure that the conditions hold for the language to be total. \\

The algorithm as a whole operates on a preprocessed version of the \gls{ast}, collated as discussed in \autoref{ssub:the_verification_preprocessor}. 
This preprocessor provides the following information to the algorithm:
\begin{itemize}
    \item \textbf{Truths:} The truths defined by the language designer.
    These are the base-cases for the main termination proof mechanism, and are assumed by this algorithm to always terminate.
    \item \textbf{Start Rule:} The body of the start rule for the language and a tag associated with it.
    This tag tracks the overall termination state of the language.
    \item \textbf{Non-Terminals:} A mapping from each non-terminal for which there was a user-supplied production to a pair of (tag, rule body).
    The tag is used to track the termination state of each non-terminal. 
\end{itemize}

As the algorithm aims to verify that all possible semantics \textit{used} in a language terminate, it is easily seen that it is naturally recursive. 
While it would also suffice to verify that all defined non-terminals terminate, this actually enforces a more strict condition. 
In the latter case, it is not possible to have non-terminals that define purely syntactic terms (e.g. \mintinline{text}{<excl> ::= "!"}), and also means that the \gls{dsl} designer cannot have partially defined but unused productions (e.g. during language development).\\

So as to best avoid placing overly limiting constraints on the \gls{dsl} designer, the algorithm instead aims to verify only the \textit{reachable} semantics for a given language. 
This means that it starts from the \textit{start rule}, verifying the semantics of all non-terminals in the rule body in a recursive fashion until it terminates. 
The results of this verification are then combined back up the recursive tree, marking each non-terminal with the appropriate tag, and determining the overall termination state for the language. \\

This traversal algorithm, though expressed recursively, is one part of a larger mutual recursion with the other verification functions. 
As a result, the algorithm in \autoref{alg:the_main_verification_traversal_algorithm} references algorithms defined in other portions of this chapter.
In the following, `productions' is a map of non-terminal to a pair of (tag, rule-body). 
The state is available across all portions of this algorithm.
It assumes the availability of the following functions:
\begin{itemize}
    \item \textsc{unpackInputs}(): This function takes the above inputs to the verification algorithm and unpacks them into a tuple.
    \item \textsc{getAlternatives}(rule): Gets the top-level syntax alternation from the rule. 
    Each of these alternatives may or may not have defined semantics.
    \item \textsc{hasSemantics}(alternative): \textit{True} if the alternative has user-defined semantics, \textit{false} otherwise.
\end{itemize}

\begin{algorithm}[!htb]
\begin{algorithmic}
\Require{The inputs to the algorithm are available and derived from the \gls{ast} result of Metaparse}
\Ensure{The verification result is `Terminates' if the language terminates, otherwise `DoesNotTerminate', `Untouched' or `Touched'}
\State
\State (truths, (startTag, startRule), productions) $\gets$ \textsc{unpackInputs}()
\Comment state
\State
\State startTag $\gets$ \Call{verifyRule}{startRule}
\State return startTag
\State
\Function{verifyRule}{rule}
    \State alternatives $\gets$ \Call{getAlternatives}{rule}
    \State results $\gets$ \Map{verifyAlternative}{alternatives}
    \State result $\gets$ \Reduce{\Call{tagPlus}{}}{Terminates}{results}
    \Comment tagPlus in Subsection~\ref{ssub:termination_states}
    \State \Return{result}
\EndFunction
\State
\Function{verifyAlternative}{alternative}
    \If{\Call{hasSemantics}{alternative}}
        \State \Return{\Call{verifyDefinedSemantics}{alternative}}
        \Comment see \autoref{alg:the_basic_semantic_verification_algorithm}
    \Else
        \State \Return{\Call{inferSemantics}{alternative}}
        \Comment see \autoref{alg:the_semantic_inference_algorithm}
    \EndIf
\EndFunction
\end{algorithmic}
\caption{The Main Verification Traversal Algorithm}
\label{alg:the_main_verification_traversal_algorithm}
\end{algorithm}

As part of this basic traversal another key algorithm is defined, to verify a single non-terminal. 
Non-terminals are given the result of the verification of their body, and this is where the main recursive call takes place. 
The verification result for a non-terminal is stored in the algorithm's state.
This is given by \autoref{alg:the_non_terminal_verification_algorithm}, and assumes the existence of the following functions:
\begin{itemize}
    \item \textsc{find}(map, key): \textit{True} if the key exists in the map, \textit{false} otherwise.
    \item \textsc{get}(map, key): Gets the value associated with the provided key in the given map.
    \item \textsc{setTag}(map, key, tag): Sets the tag in `map' for `key' to the value `tag'. 
    \item \textsc{elem}(val, list): \textit{True} if `val' is an element of `list', \textit{false} otherwise. 
    \item \textsc{tail}(list): \mintinline{haskell}{tail (x:xs) == xs}.
\end{itemize}

It also assumes the existence of a stack tracking the current trace of non-terminals to get to this point in the verification. 
This trace is used to avoid infinite recursion, and is known as `productionTrace'. 
It is stored as part of the state available to the verification algorithm and has two associated functions:
\begin{itemize}
    \item \textsc{pushFrame}(nt): Pushes the frame `nt' onto the top of the stack.
    \item \textsc{popFrame}(): Pops the top frame off the stack.
\end{itemize}

\begin{algorithm}[!htb]
\begin{algorithmic}
\Function{verifyNonTerminal}{nonTerminal}
    \State \Call{pushFrame}{nonTerminal}
    \If{\Call{elem}{nonTerminal, truths}}
        \State \Return{Terminates}
    \Else
        \If{\Call{find}{productions, nonTerminal}}
            \Comment Non-Terminal exists
            \State (tag, ruleBody) $\gets$ \Call{get}{productions, nonTerminal}
            \If{tag == Terminates $\lor$ tag == DoesNotTerminate}
                \State \Return{tag}
            \Else
                \State \Call{set}{productions, nonTerminal, Touched}
                \State traceTail $\gets$ \Call{tail}{productionTrace}
                \If{\Call{elem}{nonTerminal, traceTail}}
                    \Comment Avoid infinite loop
                    \State \Return{tag} 
                    \Comment works as tagged `Touched'
                \Else
                    \State result $\gets$ \Call{verifyRule}{ruleBody}
                    \Comment see \autoref{alg:the_main_verification_traversal_algorithm}
                    \State \Call{setTag}{productions, nonTerminal result}
                    \State \Return{result}
                \EndIf
            \EndIf
        \Else
            \Comment must exist in the truths block
            \State \Return{\Call{checkTruthsForTermination}{nonTerminal}}
        \EndIf
    \EndIf
\EndFunction
\Function{checkTruthsForTermination}{nonTerminal}
    \If{\Call{elem}{nonTerminal, truths}}
        \Comment `truths' in global state
        \State \Return{Terminates}
    \Else
        \State \Return{DoesNotTerminate}
    \EndIf
\EndFunction
\end{algorithmic}
\caption{The Non-Terminal Verification Algorithm}
\label{alg:the_non_terminal_verification_algorithm}
\end{algorithm}

\subsubsection{Termination States} % (fold)
\label{ssub:termination_states}
As part of this algorithm, each non-terminal in the language being analysed is marked with a given state.
In the project, these termination states are referred to as the \textit{tag} for the rule.
These tags may take one of four values:
\begin{itemize}
    \item \textbf{Untouched:} This is the initial state for all tags, and indicates that the entity associated with the rule (either a non-terminal or the start rule) has not yet been visited by the algorithm.
    \item \textbf{Touched:} The entity has been visited by the algorithm, but the algorithm is so far unable to assign it a termination state.
    \item \textbf{Does Not Terminate:} The algorithm has determined that the entity associated with this tag does not terminate. 
    \item \textbf{Terminates:} The algorithm has determined that the entity associated with this tag does not terminate.
\end{itemize}

As it is often the case that the tag result for a given production is the combination of the tag results for the non-terminals in the associated rule body, some thought was given to how best to combine these tag results.
The most intuitive way to define this combination was as a binary operation on values of the tag. 
After defining the operation, it became apparent that it was both associative and that an identity element existed, and hence the tag type is a \gls{monoid}.
The associative operation, `tagPlus' is defined as follows (using Haskell notation for convenience, the \mintinline{haskell}{_} refers to any value):
\begin{minted}[numbers=none]{haskell}
tagPlus :: Tag -> Tag -> Tag
tagPlus Untouched _                         = Untouched
tagPlus _ Untouched                         = Untouched
tagPlus DoesNotTerminate DoesNotTerminate   = DoesNotTerminate
tagPlus DoesNotTerminate _                  = DoesNotTerminate
tagPlus _ DoesNotTerminate                  = DoesNotTerminate
tagPlus Terminates Terminates               = Terminates
tagPlus Touched x                           = x
tagPlus x Touched                           = x
\end{minted}

The monoid has set $G = \{\text{touched, untouched, does not terminate, terminates}\}$, and satisfies the monoid axioms as follows:
\begin{itemize}
    \item Clearly, this operation satisfies the \textit{closure} property of the monoid, as the result type is always in the monoidal group $G$ (as given by the faux-function signature \mintinline{haskell}{tagPlus :: Tag -> Tag -> Tag}).
    \item Furthermore, it can be seen that the value of `touched' acts as the \textit{identity} element of the monoid: \mintinline{haskell}{Touched `tagPlus` x} = \mintinline{haskell}{x `tagPlus` Touched} = \mintinline{haskell}{x}.
    \item Finally, through observation, it is clear that this operation is also \textit{associative}.
\end{itemize}

This is useful at the implementation stage, as it can be used to simplify the code using common Haskell axioms (later discussed in \autoref{sub:tracking_error_information}). 

% subsubsection termination_states (end)

% subsection verifier_traversal (end)

\subsection{Semantic Inference} % (fold)
\label{sub:semantic_inference}
In some cases, the \gls{dsl} designer would potentially desire to write productions for which they define no semantics, or to utilise the semantics of existing productions directly.
While it is not possible to infer semantics for such rules in the general case (as the syntax of a production can be arbitrarily complex), \gls{metaspec} aims to support the obvious use case through a simple inference rule. \\

The core notion behind this inference rule is that designers may want to write productions which are just disjunctions of other non-terminals (e.g. \mintinline{text}{<n> ::= <i> | <d>}).
Under such circumstances it is trivial to see what the semantics of such a production should be: the semantics of whichever alternative is parsed. 
To this end, the inference algorithm implements this idea, allowing an alternative to have no user-defined semantics in one of two circumstances:
\begin{itemize}
    \item The alternative is a single non-terminal, in which case its semantics are the semantics of the non-terminal.
    \item The alternative is a single terminal, in which case it has no executable semantics.\footnote{It should be noted that this does not mean that terminal symbols cannot have semantics of their own, but only that they have no semantics where the user does not define them (the inference case).}  
\end{itemize}

This means that the inference will not take place if there is more than one terminal or non-terminal, or in the presence of any syntax indicating grouping or optionality. 
However, the inference will still operate in the presence of one syntactic construct.
It should be noted, however, that a rule for which semantics cannot be inferred can be trivially said to terminate through the inclusion of a ground-truth for the rule. 
If syntactic exception is present, this only serves to restrict the expressions that will match the production, and it has no impact on the semantics of the production.\\

The algorithm used to perform this inference is used as part of the main verification flow, and can be seen in \autoref{alg:the_semantic_inference_algorithm}.
It assumes the existence of the following functions:
\begin{itemize}
    \item \textsc{numTerms}: Obtains the number of syntactic terms in an alternative. 
    \item \textsc{getTerm}: Extracts the syntax term from an alternative with only one.
    \item \textsc{getFactor}: Extracts the syntactic factor from a syntax term.
    \item \textsc{getPrimary}: Extracts the syntax primary from the syntax factor.
    \item \textsc{isTerminal}: Returns \textit{true} if the input is a terminal, \textit{false} otherwise.
    \item \textsc{isNonTerminal}: Returns \textit{true} if the input is a non-terminal, \textit{false} otherwise.
\end{itemize}

This algorithm ensures that semantics are inferred in all cases where it is \textit{reasonable} to infer the semantics of a top-level alternative.
It would potentially be possible to define a set of `defaulting' rules to infer semantics in other situations.
However, the complexity of the syntactic grammar means that in many circumstances these additional rules would infer different semantics than intended by the \gls{dsl} designer, forcing them to manually define the semantics anyway. 

\begin{algorithm}[!htb]
\caption{The Semantic Inference Algorithm}
\label{alg:the_semantic_inference_algorithm}
\begin{algorithmic}
\Function{inferSemantics}{alternative}
    \State count $\gets$ \Call{numTerms}{alternative}
    \If{count $> 1$}
        \State \Return{DoesNotTerminate} 
        \Comment Cannot infer semantics
    \Else
        \Comment Only one term in this case.
        \State term $\gets$ \Call{getTerm}{alternative}
        \State factor $\gets$ \Call{getFactor}{term}
        \State \Return{\Call{verifyFactor}{factor}}
    \EndIf
\EndFunction
\State
\Function{verifyFactor}{factor}
    \If{\Call{hasRepeat}{factor}}
        \State \Return{DoesNotTerminate} 
        \Comment Cannot infer semantics
    \Else
        \State primary $\gets$ \Call{getPrimary}{factor}
        \State \Return{\Call{verifyPrimary}{primary}}
    \EndIf
\EndFunction
\State
\Function{verifyPrimary}{primary}
    \If{\Call{isTerminal}{primary}}
        \State \Return{Terminates}
        \Comment No defined semantics in this path.
    \ElsIf{\Call{isNonTerminal}{primary}}
        \State \Return{\Call{verifyNonTerminal}{primary}}
    \Else
        \State \Return{DoesNotTerminate} 
        \Comment Cannot infer semantics in all other cases
    \EndIf
\EndFunction
\end{algorithmic}
\end{algorithm}

% subsection semantic_inference (end)

\subsection{Defined Semantic Verification} % (fold)
\label{sub:defined_semantic_verification}
While \gls{absol} offers some limited capability for semantic inference, the vast majority of productions will have semantics defined by the \gls{dsl} designer. 
As discussed in \autoref{lst:language_rule_semantics}, there are four different kinds of semantics that can be defined for a production.
Each of these must be verified differently and so the explicit verification algorithms for each kind of semantics are defined in their respective sections.\\

The algorithm for verifying the case where the language semantics are explicitly defined by the \gls{dsl} designer is shown in \autoref{alg:the_basic_semantic_verification_algorithm}.
This algorithm assumes the existence of the following utility functions:
\begin{itemize}
    \item \textsc{decompose}(alternative): Returns a tuple of the syntax and semantics for the alternative.
    \item \textsc{setTag}(map, key, tag): Sets the tag in `map' for `key' to the value `tag'. 
    \item \textsc{isEnvironmentInputRule} / \textsc{isEnvironmentAccessRule} / \textsc{isSpecialSyntaxRule} / \textsc{isSemanticEvaluationRuleList}(semantics): \textit{True} if the semantics is of the kind in the name, \textit{false} otherwise.
    \item \textsc{find}(nonTerminal, list): Finds the number in `list' of pairs of (nonTerminal, number) corresponding to the input non-terminal.
\end{itemize}

\begin{breakablealgorithm}
\caption{The Basic Semantic Verification Algorithm}
\label{alg:the_basic_semantic_verification_algorithm}
\begin{algorithmic}
\Function{verifyDefinedSemantics}{alternative}
    \State (syntax, semantics) $\gets$ \Call{decompose}{alternative}
    \State ntCounts $\gets$ \Call{getNTList}{syntax}
    \ForAll{$(N_i, c) \in $ ntCounts}
        \State \Call{setTag}{productions $N_i$, Touched}
    \EndFor
    \State
    \If{\Call{isEnvironmentInputRule}{semantics}}
        \State \Return{\Call{verifyEnvironmentInputRule}{semantics, ntCounts}}
        \LineComment see \autoref{alg:environment_access_rule_verification}
    \ElsIf{\Call{isEnvironmentAccessRule}{semantics}}
        \State \Return{\Call{verifyEnvironmentAccessRule}{semantics}}
        \LineComment see \autoref{alg:environment_input_rule_verification}
    \ElsIf{\Call{isSpecialSyntaxRule}{semantics}}
        \State \Return{\Call{verifySpecialSyntaxRule}{semantics, ntCounts}}
        \LineComment see \autoref{alg:special_syntax_rule_verification}
    \ElsIf{\Call{isSemanticEvaluationRuleList}{semantics}}
        \State \Return{\Call{verifySemanticRuleList}{semantics, ntCounts}}
        \LineComment see \autoref{alg:user_defined_semantic_form_verification}
    \EndIf
\EndFunction
\State
\Function{getNTList}{syntax}
\Comment A list of pairs (nt, count)
    \State out $\gets$ []
    \Comment [] is the empty list
    \ForAll{$N_i \in$ syntax}
        \Comment $N_i$ is a non-terminal
        \If{\Call{elem}{$N_i$, out}}
            \State Increment the count for $N_i$.
        \Else
            \State Add $N_i$ to out with a count of 1
        \EndIf
    \EndFor
    \State \Return{out}
\EndFunction
\State
\Function{isValid}{(nonTerminal, index), ntList}
    \State count $\gets$ \Call{find}{nonTerminal, ntList}
    \If {index $<$ count}
        \State \Return{\textit{true}}
    \Else
        \State \Return{\textit{false}}
    \EndIf
\EndFunction
\end{algorithmic}
\end{breakablealgorithm}

As part of its operation it marks all non-terminals in the syntax for a given alternative as `touched'.
This ensures that if they are visited again that no infinite recursion will take place, but also means that they do not have to have their semantics verified.
The semantics of non-terminals in the syntax for a given rule (with user-defined semantics) will only be verified if they also appear in the semantics.

% subsection defined_semantic_verification (end)

\subsection{User-Defined Semantic Form Verification} % (fold)
\label{sub:user_defined_semantic_form_verification}
Verification of this semantic form lies at the very heart of the \gls{absol} project, with this ability providing much of the power inherent in \gls{metaspec}.
In order to design an appropriate verification algorithm for these semantics it is important to first develop the underlying theory.\\

As discussed in \autoref{sub:proving_termination}, \citet{nordstrom1988terminating} has shown that it is possible to define languages whose semantics can be reasoned about purely by well-founded induction.
In the case of \gls{metaspec}, this is not true of the entire language, but just the form of semantics that this portion of the verification engine operates on.
While this may seem to be a problem, it is possible (as discussed below) to verify that the remaining representable semantics also terminate, allowing the language to be shown to be total.\\

In order to show that the language semantic always terminate, the following relation for a program $M$ with unique configurations $s$ is defined:
\begin{equation*}
    s, M \to s', M'
    \label{eq:program_convergence}
\end{equation*}
where:
\begin{itemize}
    \item A configuration $s$ refers to any additional computational state (including the heap state, continuation, global environment, etc).
    \item The relation $\to$ is termed ``converges to'' and is inductively defined.
\end{itemize}

However, it is not sufficient for the convergence relation to be inductively defined to show that such a relation is total. 
In order to show totality, it has to be possible to show, by induction on the structure of $M$, that the rules by which $\to$ is defined are defined in such a way that the \textit{convergence hypothesis} for $M$ is given in terms of sub-programs of $M$.
If this is the case, then $M$ terminates as long as each sub-program terminates.
By induction, $M$ then terminates as long as the base-case semantics terminate.
This is formalised in \nameref{ssub:an_inductive_proof_of_the_theory} on page \autopageref{ssub:an_inductive_proof_of_the_theory}.\\

In the case of \gls{metaspec}, the program $M$ is a given instance of the language semantics, which provides the initial impetus for the proof mechanism. 
The proof mechanism for these semantics thus has to be capable of ensuring that the convergence hypothesis for a semantics $M$ is given purely in terms of sub-terms of $M$. 
It hence has to be capable of showing three things.
\begin{itemize}
    \item \textbf{The Sub-Program Criterion:} Each sub-semantics must be a strict sub-term of the semantic rule itself. 
    In the context of \gls{metaspec} this means ensuring both that the non-terminal in question exists in the syntax of the production, and that the syntax access address is not out of bounds. 
    \item \textbf{The Evaluation Criterion:} This emerges naturally from the sub-program criterion, and ensures that the evaluation portion of the semantics (the list of semantic operations) depends only on the results of evaluating the sub-terms, constants and their intermediaries. 
    If this did not hold, then it would be possible to define a non-terminating rule by including the output variable as part of the evaluation rules. 
    \item \textbf{The Inductive Criterion:} Each of the sub-program evaluations for a given semantic rule must also be shown to terminate by recursive verification.
\end{itemize}

With this theoretical foundation established, it is now possible to design an algorithm for verifying the convergence hypothesis.
The development of this algorithm is outlined throughout this section.

\subsubsection{An Inductive Proof of the Theory} % (fold)
\label{ssub:an_inductive_proof_of_the_theory}
The following inductive proof substantiates the claim above that $M$ terminates as long as each of its sub-programs terminate. 
Define a program $M$ whose semantics are defined inductively as a sequence $M_0, M_1, \dots, M_k$, where each $M_i \prec M_{i+1}$ (indicating that $M_i$ is a strict sub-term of $M_{i+1}$).
\begin{itemize}
    \item \textbf{Base Case:} The language \gls{metaspec} provides the ability to supply trivially terminating base cases.
    Therefore, the base case $M_0$ always terminates. 
    \item \textbf{Inductive hypothesis:} Assume that for some portion of the language semantics $M_{j-1}$ that this portion of the language semantics terminates.
    \item \textbf{Inductive Step:} By construction, the termination hypothesis for $M_j$ is given purely in terms of sub-programs of $M_j$. 
    By the inductive hypothesis, the subprograms of $M_j$ ($M_{j-1}$) terminate, and thus $M_j$ terminates.
\end{itemize}

It has thus been shown, by induction, that as long as the convergence hypothesis for the language semantics holds, then the language will always terminate.

% subsubsection an_inductive_proof_of_the_theory (end)

\subsubsection{Verifying the User-Defined Semantic Form} % (fold)
\label{ssub:verifying_the_user_defined_semantic_form}
From the theory established in \autoref{sub:user_defined_semantic_form_verification} it is possible to establish an algorithm for verifying the criteria required for the convergence hypothesis to hold.
In addition to verification of the convergence hypothesis, which proves the semantics always terminate, it is also important to check that the guards are complete and hence ensure that the semantics are always defined (see \autoref{sub:guard_checking}).
This is illustrated in \autoref{alg:user_defined_semantic_form_verification}.

\begin{breakablealgorithm}
\caption{User-Defined Semantic Form Verification}
\label{alg:user_defined_semantic_form_verification}
\begin{algorithmic}
\Function{verifySemanticRuleList}{semantics, ntCounts}
    \Comment `semantics' is a list of rules
    \State guardResult $\gets$ \Call{verifyGuards}{semantics}
    \Comment see \autoref{alg:user_defined_guard_verification}
    \State rulesResult $\gets$ \Call{verifySemanticRules}{semantics, ntCounts}
    \State \Return{\Call{tagPlus}{rulesResult, guardsResult}}
\EndFunction
\State
\Function{verifySemanticRules}{semantics, ntCounts}
    \Comment calls to checks
    \State evalCheck $\gets$ \Call{satisfiesEval}{semantics}
    \Comment see \autoref{alg:verification_of_the_semantic_evaluation_criterion}
    \State subtermCheck $\gets$ \Call{satisfiesSemanticForm}{semantics, ntCounts}
    \LineComment see \autoref{alg:verification_of_the_semantic_sub_term_criterion}
    \State \Return{\Call{tagPlus}{evalCheck, subtermCheck}}
\EndFunction
\end{algorithmic}
\end{breakablealgorithm}

% subsubsection verifying_the_user_defined_semantic_form (end)

\subsubsection{Verifying the Sub-Term Criterion} % (fold)
\label{ssub:verifying_the_sub_term_criterion}
As discussed above, verification that the sub-term criterion holds involves performing two checks:
\begin{enumerate}
    \item Ensuring that all sub-evaluations are strict sub-terms of the main rule body (the semantics).
    \item Ensuring that the semantics of the sub-evaluations also terminate until the base-cases are reached. 
\end{enumerate}

It is important to recognise that the second of the above criteria only applies to the non-terminals in the syntax that are \textit{used} in the semantics.
This means that non-terminals can be used in a purely syntactic form, and hence do not require any verification.
This two-stage process is illustrated in \autoref{alg:verification_of_the_semantic_sub_term_criterion} and assumes the existence of the following functions:
\begin{itemize}
    \item \textsc{getNTIndexPairs}(semantics): Returns a list if pairs of the form (nonTerminal, index).
\end{itemize}

\begin{algorithm}[!htb]
\begin{algorithmic}
\Function{satisfiesSemanticForm}{semantics, ntCounts}
    \State ntIndexPairs $\gets$ \Map{\Call{getNTIndexPairs}{}}{semantics}
    \State subTermsExist $\gets$ \Map{\Call{isValid}{ntCounts}}{ntIndexPairs}
    \State subTermsTerminate $\gets$ \Map{\Call{verifyNonTerminal}{}}{ntIndexPairs}
    \LineComment see \autoref{alg:the_non_terminal_verification_algorithm}
    \State allSubTermsExist $\gets$ \Reduce{($\land$)}{True}{subTermsExist}
    \State allSubTermsTerminate $\gets$ \Reduce{\Call{tagPlus}{}}{Terminates}{subTermsTerminate}
    \If{allSubTermsExist}
        \State \Return{\Call{tagPlus}{Terminates, allSubTermsTerminate}}
    \Else
        \State \Return{\Call{tagPlus}{DoesNotTerminate, allSubTermsTerminate}}
    \EndIf
\EndFunction
\end{algorithmic}
\caption{Verification of the Semantic Sub-Term Criterion}
\label{alg:verification_of_the_semantic_sub_term_criterion}
\end{algorithm}

This algorithm, however, is somewhat unsatisfactory due to the time constraints placed on the project. 
As it is, it has not been designed to cope with recursive productions that can never be parsed. 
While this technically allows it to admit a language with such productions (whose semantics are effectively undefined), such productions will never be parsed in an actual program (as they are infinite).
As the correctness of the termination proof depends on the finite nature of the syntax, the semantics of any possibly extant \gls{dsl} program are guaranteed to terminate.\\

For an illustration of such a circumstance assume the existence of two productions \mintinline{text}{<a> ::= <b>} and \mintinline{text}{<b> ::= "+" <a>}. 
The only expression that such a set of productions would be able to parse is an infinite string of \mintinline{text}{+} characters, and will hence never be utilised in a real program.
In such a case, whether the \gls{dsl} designer defines (technically terminating) semantics for these productions or the inference engine states that they terminate, the semantics for these productions can never actually appear in a real program, retaining language totality in any practical sense. 

% subsubsection verifying_the_sub_term_criterion (end)

\subsubsection{Verifying the Evaluation Criterion} % (fold)
\label{ssub:verifying_the_evaluation_criterion}
Verification of the evaluation criterion ensures that the \gls{dsl} designer is unable to create semantics that are self-recursive. 
Evaluations, otherwise known as the semantic operations, take the form (broadly) of a list of expressions of the kind \mintinline{text}{n = n1 <op> n2}, where \mintinline{text}{<op>} is some operator.
As part of the semantic rule, the output variable is defined as well. 
The output variable is assigned to by one rule, and any other target variables (temporaries), can be used freely.
The algorithm for ensuring this is required to check, for each rule, that the evaluations obey the following criteria:
\begin{itemize}
    \item The output variable must not occur in any of the temporary variables in the evaluation list.
    This prevents duplicate outputs, and hence undefined semantics.
    \item The output variable does not clash with any of the variables defined for the sub-evaluations of the semantics.
    This also avoids undefined semantics via avoiding duplicate outputs. 
    \item The output variable does not occur on the right-hand-side of any semantic assignment.
    This prevents divergent evaluation semantics.
    \item The evaluation variables (those defined by the semantic sub-evaluations) must not clash with any of the temporaries.
    This, too, helps to prevent the occurrence of undefined semantics.
    \item All variables used on the right-hand-side of an assignment must either be in the list of sub-evaluation variables or the list of temporaries 
    \item All temporaries must only be used to the left (in the list) of where they are declared.
    This prevents mutually- or self-recursive evaluation rules, and hence divergent semantics.
\end{itemize}

The algorithm for checking these criteria on the evaluation definitions in the semantics is shown in \autoref{alg:verification_of_the_semantic_evaluation_criterion} and assumes the existence of the following functions:
\begin{itemize}
    \item \textsc{getOutputEvalPair}(semanticRule): Extracts a pair of (output variable, evaluation rules) from a semantic rule ('evaluation rules' is a list). 
    \item \textsc{getEvalVariables}(outputVar, evalRules): Extracts a tuple of (output variable, temporaries, evaluated), where the `temporaries' is a list of the values assigned to (omitting the output variable), and `evaluated' is a list of variables on the RHS of any assignments.
    \item \textsc{getVarsInOrder}(pairs): Gets the list of pairs (var, [after]) where `var' is a target (assigned to) variable, and [after] is every variable that occurs after it in the list.
    \item \textsc{checkVarOrdering}(pair): Checks whether a variable is only used to the left of its definition.
    \item \textsc{notElem}(item, list): \textit{True} if `item' is not an element of `list', \textit{false} otherwise. 
\end{itemize}

\begin{breakablealgorithm}
\caption{Verification of the Semantic Evaluation Criterion}
\label{alg:verification_of_the_semantic_evaluation_criterion}
\begin{algorithmic}
\Function{satisfiesEval}{semantics}
    \State outputEvalPairs $\gets$ \Map{\Call{getOutputEvalPair}{}}{semantics}
    \State evalVariables $\gets$ \Map{\Call{getEvalVariables}{}}{outputEvalPairs}
    \State results $\gets$ \Map{\Call{checkEvalCriteria}{}}{\Call{zip}{outputEvalPairs, evalVariables}}
    \State orderedVars $\gets$ \Map{\Call{getVarsInOrder}{}}{outputEvalPairs}
    \State orderResult $\gets$ \Map{\Call{checkVarOrdering}{}}{orderedVars}
    \If{orderResult $\land$ (\Reduce{$(\land)$}{True}{results})}
        \State \Return{Terminates}
    \Else
        \State \Return{DoesNotTerminate}
    \EndIf
\EndFunction
\State
\Function{checkEvalCriteria}{(outVar, temps, evals), subEvalVars}
    \State subEvalVarsNotTemps $\gets$ \Reduce{$(\land)$}{True}{(\Map{\Call{notElem}{temps}}{subEvalVars})}
    \State usedVarsExist $\gets$ \Reduce{$(\land)$}{True}{(\Map{\Call{elem}{temps ++ subEvalVars}}{evals})}
    \State outNotInEvals $\gets$ \Call{notElem}{outVar, evals}
    \State outNotInTemps $\gets$ \Call{notElem}{outVar, temps}
    \State outNotInSubEvals $\gets$ \Call{notElem}{outVar, subEvalVars}
    \State \Return{subEvalVarsNotTemps $\land$ usedVarsExist $\land$ outNotInTemps $\land$ outNotInEvals $\land$ outNotInSubEvals}
\EndFunction
\end{algorithmic}
\end{breakablealgorithm}

% subsubsection verifying_the_evaluation_criterion (end)

% subsection user_defined_semantic_form_verification (end)

\subsection{Guard Checking} % (fold)
\label{sub:guard_checking}
In order to verify the guards for a given set of program semantics, the algorithm had to ensure that two criteria held:
\begin{enumerate}
    \item \textbf{The Completeness Criteria:} There must be a set of semantics defined for every possible combination of values of the guard variables, and hence there must be a guard admitting every possible combination of these values.
    \item \textbf{The Guard Variable Criteria:} The variables used in the guard expressions must be the result variables for the sub-evaluations of the semantic rules (see \autoref{sub:user_defined_semantic_form_verification}).
\end{enumerate}

Simply, this means that the result for verifying the guards is the combination of the results of the two separate verification criteria.
This is given in \autoref{alg:user_defined_guard_verification}, and depends only on the functions that check the two criteria for the guards. 
Both of these checks, however, depend on the availability of the following function:
\begin{itemize}
    \item \textsc{extractGuards}(semanticRules): Extracts the guard expressions from the provided semantic rules, returning these as a list. 
\end{itemize}

\begin{algorithm}[!htb]
\begin{algorithmic}
\Function{verifyGuards}{evaluationRules}
    \State guardsComplete $\gets$ \Call{verifyGuardsComplete}{evaluationRules}
    \Comment see \autoref{alg:guard_completeness_checking}
    \State guardVariablesComplete $\gets$ \Call{verifyGuardVariables}{evaluationRules}
    \Comment see \autoref{alg:guard_variable_checking}
    \State \Return{\Call{tagPlus}{guardsComplete, guardVariablesComplete}}
\EndFunction
\end{algorithmic}
\caption{User-Defined Guard Verification}
\label{alg:user_defined_guard_verification}
\end{algorithm}

\subsubsection{Checking Guard Completeness} % (fold)
\label{ssub:checking_guard_completeness}
Checking guards in the user-defined semantics was a point of significant design challenge, focused on ensuring that there would be defined semantics for all possible values of the guard variables. 
However, this was not as simple as it might initially sound.\\

Initial efforts around checking guard completeness focused on transforming the set of six guard (\mintinline{text}{==, !=, <, >, <=, >=}) operations into a more limited set of three (\mintinline{text}{==, <, >}) in the hope that this would ease the verification process.
Beyond that, this transformation also put all guards into a standard form, with a variable on the left and ordered lexically by that variable, and discarded nonsensical guards (that were nevertheless allowed by the grammar --- a design issue with the \gls{metaspec} language). 
Attempts were made to develop an algorithm for checking completeness of these transformed guards through enumeration of possible guard states, but it rapidly became apparent that this solution was at least $\mathcal{O}(2^n)$ in complexity and thus infeasible. 
While it was later discovered that such a problem could be solved via an application of linear programming, as discussed in \autoref{sub:guard_completeness_checking}, it was eventually ruled out of scope due to the time required to implement such a solution.\\

However, it was not possible to just \textit{assume} that the guards for a given set of semantics were complete, as this could lead to a language where the semantics were not total, and hence to erroneous verification of the language.
A simple (and unfortunately inelegant) solution was proposed to avoid this: ensure that every set of guards had at least one guard that admitted all values.
In other words, the goal of guard checking became ensuring that each set of guards had at least one `catch-all' guard. 

\begin{listing}[!htb]
\begin{minted}[numbers=none]{text}
<ifthen> ::= "if" <condition> "then" <statement> "else" <statement> --> {
    any n : {n = n2}(n1 == true) :
        {n1 <= <condition>[0]}, {n2 <= <statement>[0]} |
    any n : {n = n2}(n1 == false) :
        {n1 <= <condition>[0]}, {n2 <= <statement>[1]} |
    any n : {n = n2}() :
        {n1 <= <condition>[0]}, {n2 <= <statement>[0]}
};
\end{minted}
\caption{The Inelegance of Catch-All Guards}
\label{lst:the_inelegance_of_catch_all_guards}
\end{listing}

This is particularly inelegant as, even in cases where the \gls{dsl} creator has been able to create a complete set of guards, they still have to add some kind of semantic expression for the catch-all case. 
This inelegance can be seen in \autoref{lst:the_inelegance_of_catch_all_guards}, where a default case has to be provided despite completeness. 
While ugly, however, this does not impact on the potential for functionality within the language as guards are evaluated \textit{top-to-bottom}, with the first guard that is satisfied being evaluated. 
As a result, the finalised method for checking guard completeness can be found in \autoref{alg:guard_completeness_checking}.

\begin{algorithm}[!htb]
\begin{algorithmic}
\Function{verifyGuardsComplete}{evaluationRules}
    \State guards $\gets$ \Call{extractGuards}{evaluationRules}
    \ForAll{guard $\in$ guards}
        \If{guard == empty}
            \State \Return{Terminates}
        \EndIf
    \EndFor
    \State \Return{DoesNotTerminate}
\EndFunction
\end{algorithmic}
\caption{Guard Completeness Checking}
\label{alg:guard_completeness_checking}
\end{algorithm}

% subsubsection checking_guard_completeness (end)

\subsubsection{Checking Guard Variables} % (fold)
\label{ssub:checking_guard_variables}
The other of the criteria for the guard verification is to ensure that the guards depend only upon variables defined as part of the sub-evaluations in the language semantics.
The reasoning behind this criteria is that the guards are intended to restrict the circumstances under which computation can take place. 
Hence, if they depend on the final or intermediate variables of the semantic evaluations, then computation has to take place for the evaluation of the guards.\\

This means that the variables used in the guards must be constrained to those defined for the sub-evaluations of the semantic rule.
The procedure for verifying this is contained in \autoref{alg:guard_variable_checking}, and assumes the existence of the following functions:
\begin{itemize}
    \item \textsc{extractGuardVariables}(guards): Returns a list containing lists of the variables in each guard.
    \item \textsc{extractEvalVariables}(evaluations): Returns a list containing lists of the variables in the semantic evaluations for each rule.
    \item \textsc{zip}(a, b): Creates a list of pairs of the corresponding items in lists a and b.
\end{itemize}

\begin{breakablealgorithm}
\caption{Guard Variable Checking}
\label{alg:guard_variable_checking}
\begin{algorithmic}
\Function{verifyGuardVariables}{evaluationRules}
    \State guards $\gets$ \Call{extractGuards}{evaluationRules}
    \State guardVariables $\gets$ \Call{extractGuardVariables}{guards}
    \State evalVariables $\gets$ \Call{extractEvalVariables}{guards}
    \State guardEvalVarPairs $\gets$ \Call{zip}{guardVariables, evalVariables}
    \State results $\gets$ \Map{\Call{checkVars}{}}{guardEvalVarPairs}
    \State \Return{\Reduce{\Call{tagPlus}{}}{Terminates}{results}}
\EndFunction
\State
\Function{checkVars}{(guardVars, evalVars)}
    \State result $\gets$ True
    \ForAll{var $\in$ guardVars}
        \State result $\gets$ result $\land$ (var $\in$ evalVars)
    \EndFor
    \If{result == True}
        \State \Return{Terminates}
    \Else
        \State \Return{DoesNotTerminate}
    \EndIf
\EndFunction
\end{algorithmic}
\end{breakablealgorithm}

% subsubsection checking_guard_variables (end)

% subsection guard_checking (end)

\subsection{Verification of Other Semantic Forms} % (fold)
\label{sub:verification_of_other_semantic_forms}
While they are the main form of semantics expected to be used in \gls{metaspec} language definitions, the user semantic rules (\autoref{sub:user_defined_semantic_form_verification}) are not the only form of semantics.
Much like those rules, however, each form of semantics is required to undergo its own verification process. 

\subsubsection{Verification of Special-Syntax Rules} % (fold)
\label{ssub:verification_of_special_syntax_rules}
Special syntax rules exist in \gls{metaspec} to provide language features that can be shown to terminate (by independent proof), but not by the general termination proof mechanism. 
That does not mean, however, that the termination proof mechanism has nothing to do in this case.\\

As all special syntax rules take arguments, and these may be non-terminals, it is these arguments that must be shown to terminate. 
Hence, if the semantics of the special-syntax rule have been shown to terminate (by external proof), and the non-terminals involved have also been shown to terminate, then the special syntax rule itself should terminate.
The method for determining termination of such rules is illustrated in \autoref{alg:special_syntax_rule_verification}, and assumes the existence of the following functions:
\begin{itemize}
    \item \textsc{getAccesses}(nodeList): Extracts a list of syntax access blocks (see \autoref{sub:accessing_syntax_from_the_semantics}) from the nodes of the \gls{ast} used as arguments to the special-syntax rule.
    \item \textsc{extractNT}(syntaxAccessBlock): Extracts the non-terminal from the syntax access block.
\end{itemize}

\begin{breakablealgorithm}
\caption{Special-Syntax Rule Verification}
\label{alg:special_syntax_rule_verification}
\begin{algorithmic}
\Function{verifySpecialSyntaxRule}{semantics, ntCounts}
    \State accessList $\gets$ \Call{getAccesses}{semantics}
    \State validAccesses $\gets$ \Map{\Call{isValid}{}}{accessList}
    \State validAccess $\gets$ \Reduce{$(\land)$}{True}{validAccesses}
    \State nonTerminals $\gets$ \Map{\Call{extractNT}{}}{accessList}
    \State ntResults $\gets$ \Map{\Call{verifyNonTerminal}{}}{nonTerminals}
    \Comment see \autoref{alg:the_non_terminal_verification_algorithm}
    \State ntResult $\gets$ \Reduce{\Call{tagPlus}{}}{Terminates}{ntResults}
    \State
    \If{validAccess}
        \State \Return{\Call{tagPlus}{Terminates, ntResult}}
    \Else
        \State \Return{\Call{tagPlus}{DoesNotTerminate, ntResult}}
    \EndIf
\EndFunction
\end{algorithmic}
\end{breakablealgorithm}

% subsubsection verification_of_special_syntax_rules (end)

\subsubsection{Verification of Environment Input Rules} % (fold)
\label{ssub:verification_of_environment_input_rules}
Environment input rules provide a method for the \gls{dsl} designer to store values against keys in a globally accessible environment. 
While the semantics of such an operation might seem initially simple, it is important to recognise that the key and values might be derived from non-terminals.
Hence, the termination of an environment input rule is derived from two main components:
\begin{enumerate}
    \item \textbf{The Key and Value Terms:} The key and values might be non-terminals, and hence for the input rule to terminate, these non-terminals must also terminate.
    \item \textbf{The Input Itself:} Under more complex circumstances, it might be conceivable that writing terminating values to an environment might not terminate.
    However, the environment in \gls{metaspec} is particularly simple, and inputs into the environment are guaranteed to terminate, overwriting any previous value.
\end{enumerate}

As a result, the verification process for an environment input rule focuses on just verifying that the non-terminals that it uses also terminate.
This process is illustrated in \autoref{alg:environment_input_rule_verification} and assumes the availability of the following functions:
\begin{itemize}
    \item \textsc{getKey}(semantics): Gets the non-terminal and non-terminal index for the key under which the values are stored in the environment.
    \item \textsc{getValues}(semantics): Gets the non-terminals (and indices)for the values being stored in the environment.
\end{itemize}

\begin{breakablealgorithm}
\caption{Environment Input Rule Verification}
\label{alg:environment_input_rule_verification}
\begin{algorithmic}
\Function{verifyEnvironmentInputRule}{semantics, ntCounts}
    \State (key, ix) $\gets$ \Call{getKey}{semantics}
    \State vals $\gets$ \Call{getValues}{semantics}
    \State ntList $\gets$ (key, ix) $:$ vals
    \Comment $:$ is the cons operator
    \State terminationResults $\gets$ \Map{\Call{verifyNonTerminal}{}}{ntList}
    \Comment see \autoref{alg:the_non_terminal_verification_algorithm}
    \State validResults $\gets$ \Map{\Call{isValid}{ntCounts}}{ntList}
    \Comment see \autoref{alg:the_basic_semantic_verification_algorithm}
    \State terminationResult $\gets$ \Reduce{\Call{tagPlus}{}}{Terminates}{terminationResults}
    \State valid $\gets$ \Reduce{$(\land)$}{True}{validResults}
    \Comment finds if all nt accesses are valid
    \If{valid}
        \State \Return{\Call{tagPlus}{Terminates, terminationResult}}
    \Else
        \State \Return{\Call{tagPlus}{DoesNotTerminate, terminationResult}}
    \EndIf
    \Comment combines results where necessary
\EndFunction
\end{algorithmic}
\end{breakablealgorithm}

The fact that the environment just accepts the input with no error reporting is demonstrative of the comparative lack of design work that went into the environment feature. 
Were there more time allotted for the project, the concept of the environment would receive much additional design effort, allowing it to act as a scoped and error-proof store of environmental data (much more like a useful user state). 

% subsubsection verification_of_environment_input_rules (end)

\subsubsection{Verification of Environment Access Rules} % (fold)
\label{ssub:verification_of_environment_access_rules}
The verification of environment access rules is an interesting case as they are always guaranteed to terminate. 
Environment accesses act like retrievals from a key-value store.
In many such stores, however, not having a key for a value is a problem, returning some undefined or null value.
However, in the case of \gls{metaspec}, retrieval from an undefined key has well-defined semantics itself.\\

In such a case, the returned value is a defaulted value for the expected type.
With such a limited set of types available in the language (see \autoref{sec:special_language_features}), it is a trivial task to define defaulted values for each of them.
This ensures that any environment access will either return the previously stored value, or a default of the correct type. 
As a result, the algorithm is simple and shown in \autoref{alg:environment_access_rule_verification}.

\begin{algorithm}[!htb]
\begin{algorithmic}
\Function{verifyEnvironmentAccessRule}{semantics}
    \State \Return{Terminates}
\EndFunction
\end{algorithmic}
\caption{Environment Access Rule Verification}
\label{alg:environment_access_rule_verification}
\end{algorithm}

Using global defaults, however, is not always desirable behaviour. 
To this end, the \texttt{base} language feature provides special-syntax rules for accessing the environment with default values specified by the \gls{dsl} designer instead.
This is mainly due to the design for the environment not having been given a significant amount of thought at the time of designing the syntax.
Proofs of the termination properties for these rules (like all special syntax rules) are deferred to the respective language feature designs in \autoref{sec:special_language_features}.

% subsubsection verification_of_environment_access_rules (end)

% subsection verification_of_other_semantic_forms (end)

% section the_core_algorithms (end)

\section{Special Language Features} % (fold)
\label{sec:special_language_features}
% Talk about the design of each of the language features, and prove the required termination properties here.
% Talk about WHY things were thought of - the design process.

The special language features in \gls{absol} and \gls{metaspec} are best thought of as the `standard library' for \gls{dsl} implementation.
Each of these features aims to bring some important functionality to the toolchain, whether that be types for the semantics, non-terminals for parsing or special semantic functions to bring added flexibility. 
The main idea behind each of these features (which can also be referred to as \textit{packages}) is to aid the \gls{dsl} designer in creating their language. \\

The design process for these features was mainly derived from observation of the most useful features of other programming languages.
For \gls{absol} to be able to design useful \glspl{dsl}, it had to be able to support most of these features. 
To this end, the design process was initiated by making a list of all of the useful features that were commonly found in programming languages used today.
This list was refined through consideration of which features would be possible to represent in the `safe' environment of \gls{metaspec}, and which would be most useful (where useful is defined in a fashion mainly involving intuition).
What resulted was a minimal set of features that would allow languages designed in \gls{metaspec} to be competent and contain functionality to actually make them useful to the \gls{dsl} implementers and users. \\

Each feature definition will define the types that it provides, give definitions for each of the non-terminals it provides and also provide a list of language operations and/or special-syntax that it defines.
In the cases where the special syntax needs proof, the proof of its termination properties will be included. 
All proofs of special-syntax termination properties are given with the assumption that the arguments to the special syntax also terminate. 

\subsection{Feature --- \texttt{base}} % (fold)
\label{sub:feature_base}
The notion behind the \texttt{base} package is to provide an elementary set of non-terminals, types and features for defining a simple language grammar. 
In and of itself it is unlikely to be very useful, but it provides a foundation on which \glspl{dsl} can be created. 

\subsubsection{Types --- \texttt{base}} % (fold)
\label{ssub:types_base}
This language feature imports the following types into scope:
\begin{itemize}
    \item \texttt{any} --- The any type is for placement in any position where the type of an expression cannot be known until the time of \gls{dsl} compilation. 
    It is a type marker that is substituted for a concrete type at compile time.
    \item \texttt{none} --- A type annotation used to suppress the result type of a statement. 
    This allows for languages to either behave in an expression-oriented fashion (where the last expression in a block is a result), or a statement-oriented fashion, where there is no result of expressions.
    \item \texttt{bool} --- A basic boolean type that operates as expected. 
\end{itemize}

% subsubsection types_base (end)

\subsubsection{Non-Terminals --- \texttt{base}} % (fold)
\label{ssub:non_terminals_base}
This language feature defines the following non-terminals. 
All of these are guaranteed to terminate, and can hence be written in the truths definition block of the file. 
\begin{itemize}
    \item \mintinline{text}{<digit>} --- UTF-8 Digit Characters
    \item \mintinline{text}{<nondigit>} --- All UTF-8 non-digit characters except newlines and whitespace.
    \item \mintinline{text}{<whitespace>} --- All UTF-8 whitespace characters excluding newlines.
    \item \mintinline{text}{<newline>} --- All UTF-8 newline characters.
    \item \mintinline{text}{<utf-8-char>} --- All UTF-8 characters.
    \item \mintinline{text}{<bool> ::= "true" | "false" ;} --- Boolean literals.
\end{itemize}

% subsubsection non_terminals_base (end)

\subsubsection{Operations and Syntax --- \texttt{base}} % (fold)
\label{ssub:operations_and_syntax_base}
This feature defines the following operations over the set of types it defines:
\begin{itemize}
    \item \mintinline{text}{&&, ||, &, |, ==, !=, <, >, <=, >=} --- Defined as expected for a C-like programming language.
    \item \mintinline{text}{envStore(key, value)} --- A more versatile way of storing values in the environment that can be used from anywhere special syntax can.
    This trivially terminates as long as its arguments do, as the under-the-hood semantics are those of a standard environment store.
    \item \mintinline{text}{envGet(key)} --- A more versatile way of retrieving values from the environment that can be used from anywhere special syntax can. 
    This also trivially terminates as long as its arguments do, as the under-the-hood semantics are those of a standard environment store.
    \item \mintinline{text}{envGetDefault(key, defaultValue)} --- Similar to the above, but allows the \gls{dsl} designer to specify a default value in the case the key does not exist.
    \item \mintinline{text}{nodeLength(nodeList)} --- Gets the number of nodes contained by a non-terminal. 
    This is useful in the context of function definitions and function calls where the executed semantics may differ based on the number of arguments given. 
    As all syntax is finite, determining the length of a finite syntax list is also finite and hence guaranteed to terminate. 
    \item \mintinline{text}{semanticsOf(list)} --- Given a list of non-terminals (enclosed in \mintinline{text}{{}}), it gives the production the semantics of the list items evaluated sequentially.
    As the list is finite, it is clear that this has defined semantics as long as each list item also has defined semantics.
    If given a non-list portion of the syntax it is evaluated directly, and if given comma-separated values they are evaluated in order according to the above rules. 
\end{itemize}

% subsubsection operations_and_syntax_base (end)

% subsection feature_base (end)

\subsection{Feature --- \texttt{number}} % (fold)
\label{sub:feature_number}
The \texttt{number} package provides types and non-terminals for a number of standard numerical types found in many programming languages.
It also has a set of operations defined on these types, and two special-syntax calls for useful numeric operations.

\subsubsection{Types and Non-Terminals --- \texttt{number}} % (fold)
\label{ssub:types_number}
This package provides the following types and corresponding non-terminals:
\begin{itemize}
    \item \texttt{natural} / \mintinline{text}{<natural>} --- An unbounded, unsigned integer type.
    \item \texttt{integer} / \mintinline{text}{<integer>} --- An unbounded, signed integer type.
    \item \texttt{int32} / \mintinline{text}{<int32>} --- A 32-bit signed integer type.
    \item \texttt{uint32} / \mintinline{text}{<int64>} --- An unsigned 32-bit integer type.
    \item \texttt{int64} / \mintinline{text}{<uint32>} --- A signed 64-bit integer type.
    \item \texttt{uint64} / \mintinline{text}{<uint64>} --- An unsigned 64-bit integer type.
    \item \texttt{float} / \mintinline{text}{<float>} --- An IEEE 754 32-bit floating-point number.
    \item \texttt{double} / \mintinline{text}{<double>} --- An IEEE 754 64-bit floating-point number.
    \item \texttt{integral} / \mintinline{text}{<integral>} --- For any integral type.
    \item \texttt{floating} / \mintinline{text}{<floating>} --- For any floating-point type.
    \item \texttt{number} / \mintinline{text}{<number>} --- For any of the numeric types imported by this feature.
\end{itemize}

% subsubsection types_and_non_terminals_number (end)

\subsubsection{Operations and Syntax --- \texttt{number}} % (fold)
\label{ssub:operations_and_syntax_number}
This package provides the following operations on the above types:
\begin{itemize}
    \item \mintinline{text}{+, -, *, /, ^, \%} --- Standard arithmetic operations.
    \item \mintinline{text}{ciel(x)} --- Special syntax to calculate the ceiling of \mintinline{text}{x}.
    \item \mintinline{text}{floor(x)} --- Special syntax to calculate the floor of \mintinline{text}{x}.
\end{itemize}

% subsubsection operations_and_syntax_number (end)

% subsection feature_number (end)

\subsection{Feature --- \texttt{string}} % (fold)
\label{sub:feature_string}
This language feature provides a basic, efficient, packed-UTF-8 string type. 
It provides one type and non-terminal: \mintinline{text}{string} / \mintinline{text}{<string>}.
The non-terminal parses strings delimited by double-quotes. 
The operations over string are as follows:
\begin{itemize}
    \item \mintinline{text}{+} --- Concatenates two strings. 
    This terminates trivially.
    \item \mintinline{text}{s[x]} --- Addressing into the string at position x, obtaining the corresponding utf-8 glyph.
    This is guaranteed to return a valid string value (and always terminate): if the index is out of range, it will return the empty string.
    \item \mintinline{text}{rev(s)} --- Special syntax for reversing the string.
    Guaranteed to terminate as strings are finite.
    \item \mintinline{text}{split(s, d)} --- Special syntax for splitting a string \mintinline{text}{s} on the delimiter \mintinline{text}{d}. 
    As strings are finite, this will always terminate.
    Must have \texttt{list} imported to work as it will return the string unchanged if list is not imported. 
    \item \mintinline{text}{join(s, d)} --- Special syntax for joining a list of strings \mintinline{text}{s} with the delimiter \mintinline{text}{d}.
    As lists and strings are finite this is guaranteed to terminate.
    Much like the above, it must have \texttt{list} imported to work. 
\end{itemize}

% subsection feature_string (end)

\subsection{Feature --- \texttt{list}} % (fold)
\label{sub:feature_list}
This feature provides a doubly-linked list to contain homogeneous data with the contained type determined at compile time. 
It defines the type \mintinline{text}{list}, and the corresponding non-terminal \mintinline{text}{<list>}, for the parsing of list literals of the form \mintinline{text}{[a, b, c, d, ...]}.
It defines the following operations:
\begin{itemize}
    \item \mintinline{text}{[]} --- An empty list literal, usable anywhere in the \gls{metaspec} where a value is expected. 
    This may also be given with comma-separated items of the same type, in which case a list literal with values is constructed.
    \item \mintinline{text}{l[x]} --- Accesses the list \mintinline{text}{l} at position \mintinline{text}{x}. 
    As lists are finite this operation is always guaranteed to terminate.
    If the index \mintinline{text}{x} does not exist, the linearly closest element of the list is returned. 
    \item \mintinline{text}{+} --- List concatenation.
    Trivially guaranteed to terminate.
    \item \mintinline{text}{:} --- The list cons operator, also trivially guaranteed to terminate.
\end{itemize}

% subsection feature_list (end)

\subsection{Feature --- \texttt{matrix}} % (fold)
\label{sub:feature_matrix}
This language feature provides a $m \times n$ matrix type to contain homogeneous data. 
It defines the type \mintinline{text}{matrix}, and the corresponding non-terminal \mintinline{text}{<matrix>} for matrix literals.
Matrix literals have the form \mintinline{text}{| a, b, ... ; c, d ... ; ...}, with each row having the same number of columns.
It defines the following operations:
\begin{itemize}
    \item \mintinline{text}{+, *, /, ^, -} --- Standard matrix arithmetic, defined as for matrices.
    As matrices are finite, this is always guaranteed to terminate.
    \item \mintinline{text}{||} --- Constructs an empty matrix, or in the presence of values as for matrix literals, constructs the corresponding matrix literal.
    \item \mintinline{text}{m|x, y, ...|} --- Access the values in matrix \mintinline{text}{m} at the position given. 
    For any dimension the value \mintinline{text}{:} may be given to access the entire dimension, or \mintinline{text}{a:b} to access the portion of the dimension defined by the semi-closed interval $[a,b)$.
    This is guaranteed to terminate as matrices are finite, and much as for the list, it will return the linearly closest item(s) if the indices are out of range.
\end{itemize}

% subsection feature_matrix (end)

\subsection{Feature --- \texttt{traverse}} % (fold)
\label{sub:feature_traverse}
Traverse is a special-syntax only package that defines methods for traversing collection data-structures.
It provides the following operations as special syntax, and requires \texttt{funcall} to be imported into scope:
\begin{itemize}
    \item \mintinline{text}{map(fn, collection)} --- This operation traverses \mintinline{text}{structure}, applying the function \mintinline{text}{fn} to each element of the structure. 
    The termination property for this is not initially obvious, but by \citet{nordstrom1988terminating}, any recursion over a finite structure will terminate as long as the the size of the structure decreases with every recursive call. 
    As a result, the function uses an implementation under the hood akin to the following for some illusory typeclass \mintinline{haskell}{Traversible}:
\begin{minted}[xleftmargin=1.5cm, numbers=none]{haskell}
map :: (Traversible t) => (n -> m) -> t n -> t m
map fn [] = []
map fn (x:xs) = (fn x) : map fn xs 
\end{minted}
    As the argument to the function always gets smaller until it reaches a base case, this is guaranteed to terminate.
    \item \mintinline{text}{fold(fn, val, collection)} --- This operation traverses \mintinline{text}{structure}, performing a left fold to reduce the values in collection to a single value. 
    The function \mintinline{text}{fn} must be a binary operation taking two arguments of the type contained in structure, and it uses \mintinline{text}val} as the initial value.
    Much like for the above, the termination proof for this relies on the fact that finite recursion terminates as long as the size of the structure being recursed over decreases with each recursive call. 
    It would use an implementation akin to the following Haskell, using the same typeclass:
\begin{minted}[xleftmargin=1.5cm, numbers=none]{haskell}
fold :: (Traversible t) => (a -> a -> a) -> a -> t a -> a
fold fn value [] = value
fold fn value (x:xs) = fold fn (fn value x) xs
\end{minted}
    In the case of a matrix, the signature is slightly different as it results in a matrix (that may be a single value) to allow for reduction across rows. 
    \item \mintinline{text}{filter(fn, collection)} --- This operation traverses \mintinline{text}{structure}, only retaining the elements for which \mintinline{text}{fn}, a predicate, returns true.
    Much like for the above, the termination proof in this case comes from the fact that the size of the structure decreases with each recursive call, as in this example implementation:
\begin{minted}[xleftmargin=1.5cm, numbers=none]{haskell}
filter :: (Traversible t) => (a -> Bool) -> t a -> t a
filter pred [] = []
filter pred (x:xs) = if pred x 
    then 
        (x : filter pred xs) 
    else 
        (filter pred xs)
\end{minted}
    In the case of a matrix, it operates across a given dimension, and hence produces a new matrix of a potentially different size. 
\end{itemize}

This language feature also provides variants for matrices taking an additional argument (at the end) that specifies the dimension (by number) to traverse. 
This ensures that the matrix invariants are not broken and that everything operates as expected. 

% subsection feature_traverse (end)

\subsection{Feature --- \texttt{funcall}} % (fold)
\label{sub:feature_funcall}
The final language feature designed during this project is \texttt{funcall}. 
This language feature provides primitives for defining functions and procedures.
The distinction between these two may not be obvious, and can be described as follows:
\begin{itemize}
    \item \textbf{Function:} A function is a block of code that takes arguments and returns a value. 
    It may be called from within a procedure, but not from within another function.
    This avoids recursion.
    Functions are used as arguments to the special-syntax calls in \texttt{traverse}.
    \item \textbf{Procedure:} A procedure is a top-level block of code that takes arguments from \textit{the host language} and returns a value to a host language.
    Procedures may call functions, and there is no mechanism to call other procedures except via the \gls{dsl} \gls{ffi}
    This avoids the ability to recurse in procedures or mutually recurse between functions and procedures. 
\end{itemize}

This is a risky area, and so special care has been taken to ensure that the following pieces of special syntax retain the termination properties of the language. 
They are defined as follows, and are intended to be used to create frameworks within a DSL for defining functions and procedures at the program level:
\begin{itemize}
    \item \mintinline{text}{defproc(name, args)} --- This defines a procedure called \mintinline{text}{name}, with arguments of the types of \mintinline{text}{args}.
    These procedures are automatically connected to the \gls{dsl} \gls{ffi} to allow for the DSL to actually be called from the host language. 
    \item \mintinline{text}{deffun(name, args)} --- Much like for \mintinline{text}{defproc}, this defines a function called \mintinline{text}{name}, with arguments of the types of \mintinline{text}{args}. 
    This function is defined into the semantic environment, and can be called from within a procedure through use of the next special syntax call. 
    \item \mintinline{text}{callfun(name, args)} --- Calls a function named \mintinline{text}{name}, with arguments given by \mintinline{text}{args}.
    This call is a \gls{noop} when executed from any place other than inside a defined procedure. 
    This ensures that no recursive function calls (direct or indirect) can be made, and thus ensures that the language using these constructs will always terminate.
    % Additionally, this syntax can be used to defer execution of a given piece of semantics until another
\end{itemize}

% subsection feature_funcall (end)

\subsection{Discounted Language Features} % (fold)
\label{sub:discounted_language_features}
The features that actually went through proper design work for use in \gls{metaspec} were not the only ones considered. 
In addition to those outlined in the preceding sections, the following features were considered safe to implement for \gls{metaspec}, but were not considered crucial enough to its functionality.
Omitting these from the project scope was a useful method of constraining the number of features that were required to be implemented.
They are as follows:
\begin{itemize}
    \item \textbf{Associative Arrays:} The ability to represent associative-array container types and the operations on them in metaspec. 
    This would be useful for representing key-value mappings and any other kind of named dictionary structure.
    \item \textbf{Random Number Generation:} It could be useful to be able to generate random numbers.
    Such a feature would likely include probabilistic distributions to assist in mathematical modelling.
    \item \textbf{Maybe:} A `Maybe' type, similar to that in Haskell.
    This would aid in the representation of error conditions in types and functions.
    \item \textbf{Either:} Also similar to Haskell, this would allow the representation of alternates in result types, and hence also assist in the representation of error conditions.
    \item \textbf{State:} The provision of a user controlled state (and associated operations) that can be passed around functions and interacted with entirely separately from the semantic environment.
    \item \textbf{Special Numbers:} A feature providing additional number types such as complex numbers, financial numbers and arbitrary precision floating-point types.
    \item \textbf{Mathematical Operations:} Sets of additional mathematical operations defined for both the standard \mintinline{text}{number} and the above special numbers features. 
\end{itemize}

All of these features would provide the appropriate operations, types and non-terminal definitions to allow working with these language features as required. 

% subsection discounted_language_features (end)

% section special_language_features (end)

% chapter architecture_and_algorithms (end)

% Present an overview of the software system, and a high-level discussion of the implementation proces.
% Reflection on the choice of languages, tooling and techniques (project management).
% Make sure to focus on the MAIN issues with implementation.
% Follow the same architectural ordering as the design section.

\chapter{Implementation} % (fold)
\label{cha:implementation}
This section will:
\begin{itemize}
    \item Follow a similar structure to algorithms, using code listings to illustrate how the abstract algorithms were turned into a concrete system. 
    \item Examine the compromises or changes to the algorithm that had to take place.
\end{itemize}

\section{Building the Application Framework} % (fold)
\label{sec:building_the_application_framework}
% Designing state into the parser (two stage impl)
% How did the NT tracker come about - why was it insufficient to track all nt parses the same way? -> they are all parsed the same, but they have different semantic meaning for the precondition verification. 

% section building_the_application_framework (end)

\section{Building the Lexer and Parser} % (fold)
\label{sec:building_the_lexer_and_parser}

% section building_the_lexer_and_parser (end)

\subsection{Building the Verification Framework} % (fold)
\label{sub:building_the_verification_framework}

% subsection building_the_verification_framework (end)

\section{Tooling and Language Choices} % (fold)
\label{sec:tooling_and_language_choices}

\subsection{Reflecting on the Language Choice} % (fold)
\label{sub:reflecting_on_the_language_choice}

% subsection reflecting_on_the_language_choice (end)

% section tooling_and_language_choices (end)

% chapter implementation (end)

\chapter{Testing} % (fold)
\label{cha:testing}
Like any well-designed piece of software, \gls{absol} underwent a significant amount of testing during its development.
As \gls{absol} was mainly developed as an embodiment of the theoretical algorithms developed as part of the project, rather than as a software product itself, the ad-hoc testing approach seemed to prove sufficient for the context of the project.\\

This chapter of the document aims to both outline the testing approach taken, with an examination of how it integrated with the development process.
It also provides examples of the tests carried out with an examination of their result. 
Finally, it provides an example of a language developed in \gls{metaspec} and verified by \gls{absol}, with a description of that language's development process, in order to showcase the utility of the metacompiler toolchain.

\section{The General Testing Approach} % (fold)
\label{sec:the_general_testing_approach}
\gls{absol} is a large software system, and like any large software system it is likely to have bugs.
This meant that having some testing approach was paramount in order to ensure that the software operated correctly.
With the clear need for testing, the choice to be made was as to what kind of testing.\\

Initial examination of the testing ecosystem in Haskell highlighted two main complementary testing approaches that could be unified by tooling.
Both of these testing approaches could be unified under a common testing toolchain integrated with Stack in the form of Tasty \citep{tasty_haskell}. 
Tasty is a testing framework for Haskell that provides the ability to combine diverse testing approaches into a single test suite that can be run from within the Stack build tool. 
The approaches were as follows:
\begin{itemize}
    \item \textbf{HUnit:} Akin to the JUint framework for Java, HUnit provides a unit testing framework for Haskell that allows software engineers to write tests in terms of assertions on results of functions \citep{hunit}.
    \item \textbf{QuickCheck:} An automated property-based testing framework for Haskell, QuickCheck provides specifications of function behaviour that are then automatically tested over a large random search space \citep{quick_check}.
    With integrated QuickCheck support in Megaparsec, this would have integrated nicely with an automated testing approach.
\end{itemize}

However, despite the obviously robust support for software testing provided by the Haskell ecosystem, the project did not take an automated approach.
While it would have brought significant benefits to the project, initial evaluation of such a testing strategy in the context of the non-product nature of the \gls{absol} toolchain indicated that the effort to maintain the tests would outweigh the benefits they would bring.
This seemed to be especially true of the QuickCheck specifications, which are written in a \gls{dsl} themselves. \\

As an alternative, the project decided to rely on a dual-pronged approach that combined manual software testing with the significant strengths of Haskell as a language: its strong type-safety:
\begin{itemize}
    \item \textbf{Haskell's Type-Safety:} It is often heard in folklore around Haskell that ``if it compiles, it probably works''. 
    This quote stems from the fact that in specifying the type of a function, and keeping functions small and composable, it is possible to encode a significant amount of information about the function itself.
    This meant that the first part of this testing approach involved the strict specification of types for all functions, not relying on Haskell's robust type-inference mechanism.
    If the resultant code compiled then there was some guarantee of safety.
    It should be noted that this is not on par with the kinds of tests that can be encoded in the type system using a \gls{dependently_typed} system.
    \item \textbf{Manual Testing:} The type-safety alone, however, was far from sufficient. 
    While it can guarantee that the functions making up a program operate in the correct domain, it cannot necessarily show that the behaviour of these functions is \textit{correct}.
    In order to ensure correctness, the decision was made to additionally perform manual testing using a test file.
    This test file would be modified to test the piece of functionality that was currently a concern.
    The test file was also intended to act as some kind of regression test, as no working piece of syntax or semantics would have been removed from it. 
\end{itemize}

The main benefit of such a testing approach, and the main reason why it was chosen over more formal automated testing was that it required little maintenance effort.
In a project so constrained for time, this was seen as a large boon. 
To this end, this was the chosen approach for the project, with the hope that it would be sufficient. 
The chosen testing strategy was interleaved with the development process, testing each feature as they were developed. 

\subsection{Examining the Testing Approach in Hindsight} % (fold)
\label{sub:examining_the_testing_approach_in_hindsight}
While the testing approach chosen did \textit{help} the development of \gls{absol} it turned out to be woefully insufficient in hindsight.
The main failure of the chosen testing approach was that it provided very little in the way of \textit{regression testing}: tests to ensure that further development did not break or alter the function of existing features. \\

The \gls{metaspec} test file, which can be seen in \autoref{cha:the_absol_testing_file} on \autopageref{cha:the_absol_testing_file}, was able to act as some kind of regression testing.
This file was used to test the new features as they were added, testing both accepting and rejecting scenarios.
However, as a feature became `finished' or development moved on from that point, the corresponding part of the file had to be left in an `accepting' state, meaning that it only tested the success criteria.
This meant that while the test file acted somewhat like a regression test, it would only detect breakages where something correct or valid was no longer accepted by \gls{absol}.
As a result, it entirely ignored functionality in the cases where things should be broken.\\

In order to attempt to compensate for this difficulty, manual breakages were introduced to the file at regular intervals in order to examine the behaviour in these cases. 
However, it was all too easy to forget to re-test a certain case, and this led to multiple occasions where alterations to the verification engine or parser caused breakages that were not detected until much later.
This late discovery of bugs often made fixing them more difficult due to the mental context switch involved. \\

In hindsight, it would have been far better to have applied the extra work required to maintain both HUnit and QuickCheck tests for \gls{absol}.
A fully automated test suite would have acted as an efficient regression testing mechanism, and also provided other testing guarantees that would potentially have allowed more rapid feature development. 
The HUnit-based tests would have been able to check the behaviour of the support modules and Metaverify, while sets of QuickCheck properties could have helped automate the testing of both the parser and the verification engine.
Overall, the chosen testing strategy turned out to be a large misstep during the development of \gls{absol}.

% subsection examining_the_testing_approach_in_hindsight (end)

% section the_general_testing_approach (end)

\section{Testing During Development} % (fold)
\label{sec:testing_during_development}
Testing during the development process took place using manual test cases created in \gls{metaspec} (the language) itself.
These test cases were kept throughout development, and can be seen in \autoref{cha:the_absol_testing_file}.
While most of the tests that took place operated successfully, often due to the correct Haskell type-signatures constraining function behaviour, there were some occurrences where this manual testing approach was able to expose significant problems.
These usually occurred either with the \textit{algorithms} underlying the metacompiler or the \textit{implementation} of these algorithms.

\subsection{Infinite Recursion in Mutually Recursive Productions} % (fold)
\label{sub:infinite_recursion_in_mutually_recursive_productions}
As mentioned in \nameref{ssub:the_implementation_influencing_design} on \autopageref{ssub:the_implementation_influencing_design}, it was not uncommon for the development process to influence the design of the underlying algorithms.
In the particular case highlighted in the above section, the verification engine was failing to terminate for a set of mutually recursive productions specifically designed to test behaviour in this case. 
They are as seen in \autoref{lst:the_test_case_for_mutually_recursive_semantic_verification}, and are still visible in the metaspec test file:
\begin{listing}[!htb]
\begin{minted}[numbers=none, fontsize=\blockfont]{text}
<arith-expr> ::= <my-number> | <arith-op> ;
...
<arith-op> ::= 
    <arith-expr> "+" <arith-expr> --> {
        number n : {n = n1 + n2}() :
            {number n1 <= <arith-expr>[0]}, {number n2 <= <arith-expr>[1]}
    } | ... ;
\end{minted}
\caption{The Test Case for Mutually Recursive Semantic Verification}
\label{lst:the_test_case_for_mutually_recursive_semantic_verification}
\end{listing}

In this case it is obvious that the verification for \mintinline{text}{<arith-expr>} depends on the verification for \mintinline{text}{<arith-op>}, and the converse is also true. 
This test case, was initially written into the file at the time of testing the parser itself, but once the development of user-semantic verification occurred for Metaverify, it was able to expose the issue in verifying such productions.
The metacompiler was observed to hang, and using the test case contained within the testing file it was possible to diagnose the issue and implement the solution as described in the final version of the non-terminal verification algorithm (see \autoref{sub:verifier_traversal}).

% subsection infinite_recursion_in_mutually_recursive_productions (end)

\subsection{False Successes for Semantic Evaluations} % (fold)
\label{sub:false_successes_for_semantic_evaluations}
Another case where the testing approach was able to identify an issue was in one of the rules for verification of the user-defined semantics. 
While \autoref{alg:verification_of_the_semantic_evaluation_criterion} had provided the correct set of criteria to validate these evaluations since it was designed (see \autoref{ssub:verifying_the_evaluation_criterion} on \autopageref{ssub:verifying_the_evaluation_criterion}), the testing approach highlighted a bug in the implementation of one of the criteria checks.\\

As stated in the above section, all temporary variables declared in the evaluation list must only be used \textit{to the left} of where they were declared. 
The test case created to check this verification was working as intended was as seen in \autoref{lst:testing_part_of_the_semantic_evaluation_criteria}.

\begin{listing}[!htb]
\begin{minted}[numbers=none, fontsize=\blockfont]{text}
<arith-op> ::= 
    ... |
    <arith-expr> "^" <arith-expr> --> {
        ... |
        number n : {n = n1 ^ n3, n3 = n3 * 0 + n2}() :
            {number n1 <= <arith-expr>[0]}, {number n2 <= <arith-expr>[1]}
    };
\end{minted}
\caption{Testing Part of the Semantic Evaluation Criteria}
\label{lst:testing_part_of_the_semantic_evaluation_criteria}
\end{listing}

This test case was designed to fail immediately, but it turned out that it validated without issue. 
Careful tracing of this result through the verification algorithm allowed the discovery that the collation algorithm for obtaining the variables used after each definition had not been implemented properly. 
Fixing this implementation allowed the metacompiler to act as expected, failing the language with this semantic rule:

\begin{minted}[numbers=none]{text}
Incorrect Semantic Form.
    REASON: Malformed semantic operation(s).
    IN: <statement> -> <arith-expr> -> <arith-op>

Incorrect Semantic Form.
    REASON: Malformed semantic operation(s).
    IN: <statement> -> <assignment> -> <arith-expr> -> <arith-op>
\end{minted}

% subsection false_successes_for_semantic_evaluations (end)

% section testing_during_development (end)

\section{Testing Error States} % (fold)
\label{sec:testing_error_states}
In general, the testing approach focused on establishing that each feature of the metacompiler worked as expected as it was completed. 
The following section aims to provide some evidence of the correct functionality of \gls{absol} through an application of the standard testing approach. \\

As Metaparse was the first system component requiring testing to be developed, initial testing focused on the correct operation of the parser.
As discussed in \autoref{sub:metaparse_ast_generation}, the parser in \gls{absol} has two main functions:
\begin{itemize}
    \item Parsing the \gls{metaspec}
    \item Ensuring that the preconditions are met for the verifier.
\end{itemize}

Each of these pieces of functionality was manually tested as it was developed. 

\subsection{Syntax Errors} % (fold)
\label{sub:syntax_errors}
All of the syntax error detection and reporting functionality built into \gls{absol} comes as a result of the use of Megaparsec, which provides these abilities by default.
As a result, testing this was as simple as introducing syntax errors into the input file and ensuring that the resultant parse error made sense. 
Consider the introduction of the syntax error in \autoref{lst:introducing_a_syntax_error}, which omits the closing \mintinline{text}{>} of a non-terminal. 
The resultant error, shown in \autoref{lst:the_syntax_error_diagnostic}, correctly diagnoses what's wrong as part of the built in functionality of Megaparsec. 

\begin{listing}[!htb]
\begin{minted}[numbers=none]{text}
<my-number> ::= <integer> | <floating ;
\end{minted}
\caption{Introducing a Syntax Error}
\label{lst:introducing_a_syntax_error}
\end{listing}

\begin{listing}[!htb]
\begin{minted}[numbers=none]{text}
metaspec/simple_test.meta:26:38:
unexpected space
expecting '>' or alphanumeric character
\end{minted}
\caption{The Syntax Error Diagnostic}
\label{lst:the_syntax_error_diagnostic}
\end{listing}

While the syntax error reporting behaviour does degrade in the presence of backtracking parsing, testing shows that it is still able to retain some fairly sensible diagnostics in the case of errors in the backtracking portion of the parser. 
If you omit the type before a special syntax rule, for example, it has enough context to suggest a list of types: \mintinline{text}{expecting "any", "bool", ...}, an appropriate suggestion based on the parser structure and language grammar.

% subsection syntax_errors (end)

\subsection{Precondition Verification Errors} % (fold)
\label{sub:precondition_verification_errors}
More interesting is the testing of the precondition verification algorithm.
The errors here are fully part of the \gls{absol} implementation and thus not provided by a library. 
This means that it was very important to test both that the errors manifested as expected, \textit{and} that they provided the appropriate diagnostic information.
The precondition verification algorithm aims to check four different criteria (see \autoref{sub:verifier_precondition_validation}), each of which was tested independently

\subsubsection{Testing Used Non-Terminals Defined In Scope} % (fold)
\label{ssub:testing_used_non_terminals_defined_in_scope}
To test this it is as simple as introducing the usage of a non-terminal into the language definition that does not exist in the language scope. 
Conversely, it is possible to remove the import for a given non-terminal from the \mintinline{text}{using} definition block for the language.
To test this a non-terminal \mintinline{text}{<tmp>} is added to the start-rule production of the test file. 
This means that the non-terminal is used but never defined, and should hence cause an error.

\begin{listing}[!htb]
\begin{minted}[numbers=none]{text}
metaspec/simple_test.meta:82:1:
The following Non-Terminals are used but not defined: <bar>... 
\end{minted}
\caption{Error for a Non-Terminal Used While Not In-Scope}
\label{lst:error_for_a_non_terminal_used_while_not_in_scope}
\end{listing}

Doing this results in the error seen in \autoref{lst:error_for_a_non_terminal_used_while_not_in_scope}, which correctly diagnoses the issue.
In the context where the missing non-terminal is defined by a language feature, it would also be capable of suggesting the corresponding import to the \gls{dsl} designer.

% subsubsection testing_used_non_terminals_defined_in_scope (end)

\subsubsection{Testing the Single Definition Principle} % (fold)
\label{ssub:testing_the_single_definition_principle}
Much like the above, it is very simple to test.
To do so it is sufficient to introduce a secondary definition for a non-terminal that has already been defined. 
In this case, a duplicate definition of \mintinline{text}{<integer>} is added.
This non-terminal is defined by the \mintinline{text}{number} language feature.

\begin{listing}[!htb]
\begin{minted}[numbers=none, fontsize=\blockfont]{text}
metaspec/simple_test.meta:28:11:
Non-Terminal with name "integer" already defined. Defined by language feature(s): number.
\end{minted}
\caption{Error for Duplicate Non-Terminal Definitions}
\label{lst:error_for_duplicate_non_terminal_definitions}
\end{listing}

As seen in \autoref{lst:error_for_duplicate_non_terminal_definitions}, this produces a helpful error. 
In this case, the error is able to recognise that, rather than being defined in the document body itself, the non-terminal was originally defined by a language feature, and that information is provided. 

% subsubsection testing_the_single_definition_principle (end)

\subsection{Testing Types and Special-Syntax in Scope} % (fold)
\label{sub:testing_types_and_special_syntax_in_scope}
Much like the other portions of the precondition verifier, the testing of both of these was simple. 
The basic test file makes use of the \mintinline{text}{map} special syntax, as well as heavy use of the \mintinline{text}{<integer>} type, and so removing the relevant imports produced errors as seen in \autoref{lst:error_for_types_and_special_syntax_missing_from_scope}.
In both cases, the missing elements are defined by language features, and so the metacompiler is able to suggest the relevant imports to help the \gls{dsl} designer. 

\begin{listing}[!htb]
\begin{minted}[numbers=none, fontsize=\blockfont]{text}
metaspec/simple_test.meta:74:12:
Special Syntax "map" not in scope. Please import one of the following: traverse.

metaspec/simple_test.meta:14:14:
Type "integer" not in scope. Defined in language feature(s): number.
\end{minted}
\caption{Error for Types and Special-Syntax Missing from Scope}
\label{lst:error_for_types_and_special_syntax_missing_from_scope}
\end{listing}

% subsection testing_types_and_special_syntax_in_scope (end)

% subsection precondition_verification_errors (end)

% section testing_error_states (end)

\section{Testing Metaverify Errors} % (fold)
\label{sec:testing_metaverify_errors}
Having established that the parser worked properly, it was time to develop the verification engine. 
Much like Metaparse, the verification engine also underwent manual testing to ensure that it detected all the possible conditions that it was meant to detect. 
This, too, used the same testing methodology to help establish whether Metaverify was able to operate correctly, and thus that its conclusions were also correct.\\

Metaverify was the second system component that was developed for \gls{absol} that required testing.
However, it was developed piece-by-piece so the testing approach focused on each portion of the verification algorithm in turn. 

\subsection{Testing the Semantic Inference} % (fold)
\label{sub:testing_the_semantic_inference}
One of the major features of the verification algorithm is its ability to infer the semantics for simple productions that just consist of alternations. 
Ensuring this operates correctly requires both checking that it does not infer semantics when it reasonably cannot, but also that it will infer semantics correctly where possible.
The first test was to ensure that productions of the form where it \textit{should} operate correctly had their semantics inferred. \\

The test file used for most of the testing of the toolchain contains multiple examples of productions who satisfy the required form for semantic inference. 
One of these can be seen in \autoref{lst:a_production_with_inferred_semantics} below. 
When executing the metacompiler on this file, the semantics of the language are verified correctly, stating that it terminates.
This implies that the semantic inference is working in this case. 

\begin{listing}[!htb]
\begin{minted}[numbers=none]{text}
<arith-expr> ::= <my-number> | <arith-op> ;
\end{minted}
\caption{A Production with Inferred Semantics}
\label{lst:a_production_with_inferred_semantics}
\end{listing}

It remains to be shown, however, that the inference fails in appropriate cases.
In any situation other than a single non-terminal or terminal in an alternation, the inference engine should fail with an appropriate message. 
Altering the above production to read \mintinline{text}{... | <arith-op> "+" ;} introduces a situation in which the inference algorithm should fail, and indeed it does (as seen in \autoref{lst:inference_failure}).

\begin{listing}[!htb]
\begin{minted}[numbers=none]{text}
Unable to infer semantics for rule.
    REASON: Cannot infer semantics for rule.
    IN: <statement> -> <arith-expr>
\end{minted}
\caption{Inference Failure}
\label{lst:inference_failure}
\end{listing}

% subsection testing_the_semantic_inference (end)

\subsection{Testing the Alternative Semantic Forms} % (fold)
\label{sub:testing_the_alternative_semantic_forms}
The second kind of semantic verification that was developed were the special types of semantic rule.
This was broadly because their verification is generally simple, revolving around special-case semantic proofs, and the termination of the non-terminals involved.
The termination of all special semantic forms (special syntax rules, environment input rules and environment access rules) is guaranteed as long as the involved non-terminals terminate and exist in the associated syntactic production. \\

As the verification procedure is the same for each of these, the following example pertains only to the verification of special-syntax rules. 
The test file can already be shown to terminate with the valid special-syntax rule on line 73, but this does not show that it fails appropriately.
Consider a production (defined in the test file) as shown in, and introduce some semantic failure into \mintinline{text}{<arith-op>}.
As shown in \autoref{lst:a_failing_special_syntax_rule}, this produces the appropriate errors in the metacompiler output. 

\subsubsection{Testing Special-Syntax Rules} % (fold)
\label{ssub:testing_special_syntax_rules}
The termination of special syntax rules depends both on the external termination proof (as seen in \autoref{sec:special_language_features}), and the termination of all the non-terminals used in the semantic rule. 
Testing this is hence simple, introducing an error into one of the non-terminals used in the rule.

\begin{listing}[!htb]
\begin{minted}[numbers=none, fontsize=\blockfont]{text}
PRODUCTION: <ssr-fail>
STATUS: Does not terminate.

Refers to non-existent subterms.
    REASON: Non-terminal <arith-op> with index 2 is not defined in this production.
    IN:

Incorrect Semantic Form.
    REASON: Malformed semantic operation(s).
    IN: <arith-op>

Incorrect Semantic Form.
    REASON: Malformed semantic operation(s).
    IN: <arith-op>
\end{minted}
\caption{A Failing Special-Syntax Rule}
\label{lst:a_failing_special_syntax_rule}
\end{listing}

The other kinds of alternative semantic form were similarly tested, and the verification algorithms for each kind of semantics all appeared to operate correctly.
This meant that they detected the appropriate error conditions and produced the correct corresponding error messages.

% subsubsection testing_special_syntax_rules (end)

\subsection{Testing the User-Defined Semantics Checks} % (fold)
\label{sub:testing_the_user_defined_semantics_checks}
Testing the verification algorithm for the user-defined semantics is a touch more involved due to the number of separate components that exist within it. 
Each of these individual component tests was developed individually, and so was tested individually. 

\subsubsection{Testing Sub-Term Criterion Verification} % (fold)
\label{ssub:testing_sub_term_criterion_verification}
The sub-term criterion states that all sub-evaluations in a rule must be strict subterms of the main body, and that the semantics of these sub-terms also terminate. 
The implementation is a direct adaptation of the algorithm in \nameref{ssub:verifying_the_sub_term_criterion} on \autopageref{ssub:verifying_the_sub_term_criterion}, and this provides the grounds to test the implementation.\\

Introducing a sub-term that does not exist in the syntax should result in an error as it is not a sub-term of the evaluated semantics.
Introducing a bad index into the \mintinline{text}{<arith-op> production} does indeed result in an appropriate error, as seen in \autoref{lst:failing_due_to_a_non_existent_sub_term}.

\begin{listing}[!htb]
\begin{minted}[numbers=none, fontsize=\blockfont]{text}
PRODUCTION: <arith-op>
STATUS: Does not terminate.

Refers to non-existent subterms.
    REASON: Non-terminal <arith-expr> with index 2 is not defined in this production.
    IN:
\end{minted}
\caption{Failing Due to a Non-Existent Sub-Term}
\label{lst:failing_due_to_a_non_existent_sub_term}
\end{listing}

By the same token, adapting the semantic rule to contain non-terminating productions will also cause the verification to fail.

% subsubsection testing_sub_term_criterion_verification (end)

\subsubsection{Testing the Evaluation Criterion} % (fold)
\label{ssub:testing_the_evaluation_criterion}
The second main component of verifying the user semantic form is ensuring that the evaluation rules match the required form (described in \nameref{ssub:verifying_the_evaluation_criterion} on \autopageref{ssub:verifying_the_evaluation_criterion}).
It is possible to break this criterion in multiple ways, and so it is not worth showcasing all of the tests that were run on it here.\\

Instead, this demonstration will focus on a case where the conditions are broken by using a non-existent evaluation variable as part of the computation.
Working from the basic, valid, test file, the evaluation on line 42 can be transformed to read as in \autoref{lst:testing_the_evaluation_criterion} to introduce a dependency on a non-existent variable \mintinline{text}{n3}.

\begin{listing}[!htb]
\begin{minted}[numbers=none]{text}
<arith-expr> "+" <arith-expr> --> {
    number n : {n = n1 + n3}() :
        {number n1 <= <arith-expr>[0]}, {number n2 <= <arith-expr>[1]}
} | ...
\end{minted}
\caption{Testing the Evaluation Criterion}
\label{lst:testing_the_evaluation_criterion}
\end{listing}

Running the metacompiler on this file with the alteration produces an error in the output, as expected. 
The error can be seen in \autoref{lst:failure_to_verify_the_semantic_operations}, and it clearly diagnoses the issue. 

\begin{listing}[!htb]
\begin{minted}[numbers=none]{text}
PRODUCTION: <arith-op>
STATUS: Does not terminate.

Incorrect Semantic Form.
    REASON: Malformed semantic operation(s).
    IN:
\end{minted}
\caption{Failure to Verify the Semantic Operations}
\label{lst:failure_to_verify_the_semantic_operations}
\end{listing}

% subsubsection testing_the_evaluation_criterion (end)

\subsubsection{Testing the Guard Checking} % (fold)
\label{ssub:testing_the_guard_checking}
\begin{listing}[!htb]
\begin{minted}[numbers=none]{text}
 <arith-expr> "^" <arith-expr> --> {
    number n : {n = 1}(n1 == 1) :
        {number n1 <= <arith-expr>[0]}, {number n2 <= <arith-expr>[1]} |
    number n : {n = n1 * n1}(n2 == 2) :
        {number n1 <= <arith-expr>[0]}, {number n2 <= <arith-expr>[1]}
\end{minted}
\caption{A Set of Invalid Guards}
\label{lst:a_set_of_invalid_guards}
\end{listing}

The final element of checking the user-defined semantics is to ensure that there is always a semantic rule to execute, no matter the values of the guards. 
This is fairly simple to perform in this mode, checking just that there is a catch-all guard, as discussed in \autoref{sub:guard_checking}. 
This makes checking that it works as intended simple - remove the catch-all guard from a set of semantic rules and check that it detects it appropriately.
It is simple enough to alter the semantic rules beginning on line 58 of the test file to not include the catch-all guard, as seen in \autoref{lst:a_set_of_invalid_guards}. \\

Executing the metacompiler on the file containing the error produces a diagnostic message as expected.
This message can be seen in \autoref{lst:an_invalid_guards_error}.

\begin{listing}[!htb]
\begin{minted}[numbers=none]{text}
PRODUCTION: <arith-op>
STATUS: Does not terminate.

Guards incomplete.
    REASON: Guards must contain a catch-all clause.
    IN:
\end{minted}
\caption{An Invalid Guards Error}
\label{lst:an_invalid_guards_error}
\end{listing}

% subsubsection testing_the_guard_checking (end)

% subsection testing_the_user_defined_semantics_checks (end)

% section testing_metaverify_errors (end)

\section{An Example Language} % (fold)
\label{sec:an_example_language}
The other main component to testing \gls{absol} was the creation of a (potentially) useful \gls{dsl} as a proof-of-concept for the metacompiler system. 
This section aims to outline the process through which a simple \gls{dsl} could be created, and demonstrate the final language here. 

\subsection{The Idea for the Language} % (fold)
\label{sub:the_idea_for_the_language}
The notion for this language, known as \textit{spreadsheet}, is that it provides a \gls{dsl} for the manipulation of homogeneous, spreadsheet-based data. 
This means that it needs to be able to:
\begin{itemize}
    \item Ingest and output `spreadsheets' to the host language via the FFI (and so package \mintinline{text}{funcall} is needed).
    \item Perform basic operations over spreadsheets, allowing users to define functions to apply to rows and columns (\mintinline{text}{traverse}) will be required. 
    \item The spreadsheets should be able to contain numeric data for processing (hence \mintinline{text}{number} is required). 
\end{itemize}

Considering all of these language design goals, another import required will be \mintinline{text}{matrix}, providing a table-like container for homogeneous data. 
Finally, \mintinline{text}{base} should be included as it provides useful special features for the implementation of any language. 

% subsection the_idea_for_the_language (end)

\subsection{Designing the Example Language} % (fold)
\label{sub:designing_the_example_language}
This section aims to guide the reader through the design process for this language, emphasising the use of the various features of \gls{metaspec} to produce a language that is successfully verified by \gls{absol}. 
The design process involves multiple stages. 

\subsubsection{Specifying the Language Metadata} % (fold)
\label{ssub:specifying_the_language_metadata}
The metadata fields in a \gls{metaspec} provide a foundation from which to develop the language in question. 
To that end, the language is contextualised with a name and initial version, and the established list of imports is added to the \mintinline{text}{using} defblock. 
In addition to this, it is worth inserting both the truths defblock (though empty), and the language defblock with a basic start-rule. 
Doing this ensures that you can check if the language parses and terminates in its current state.
From this, the current state of the language is seen in \autoref{lst:the_initial_version_of_spreadsheet}. 

\begin{listing}[!htb]
\begin{minted}[numbers=none]{text}
name : spreadsheet;

version : 0.0.1;

using : {
    base,
    number,
    matrix,
    traverse,
    funcall
};

truths : {

};

language : {

<<spreadsheet>> ::= "";

};
\end{minted}
\caption{The Initial Version of Spreadsheet}
\label{lst:the_initial_version_of_spreadsheet}
\end{listing}

% subsubsection specifying_the_language_metadata (end)

\subsubsection{Designing the Language Itself} % (fold)
\label{ssub:designing_the_language_itself}
Designing the language itself is a difficult process to describe as it relies on both experience and intuition. 
It was built using a top-down process, envisioning what the user would be required to do to write useful programs in `spreadsheet', and developing the productions from there.
Nevertheless, the development of the \gls{dsl} relied heavily on the metacompiler to provide feedback about the termination state of the language. 
At any given time it was possible to run the metacompiler on it, and let the analysis explain what is invalid and needs to be improved, changed or completed. 
The final definition of `spreadsheet' can be seen below:

\begin{minted}[fontsize=\blockfont]{text}

name : spreadsheet;

version : 0.0.1;

using : {
    base,
    number,
    matrix,
    traverse,
    funcall,
    string
};

truths : {
    {number n <= <number>},
    {matrix n <= <matrix>},
    {any n <= <nondigit>},
    {any n <= <digit>},

    // Termination for literals
    {any n <= <proc-name>},
    {any n <= <variable-name>}
};

language : {

<<spreadsheet>> ::= <procedure-def> | <function-def> ;

// The procedure definitions are available in the host language. 
<procedure-def> ::= 
    "proc " <proc-name> "(" <argument-list> ")" "{" <proc-body> "}" ";" --> {
        // Suppress any value returned.
        none defproc(<proc-name>[0], <argument-list>[0], <proc-body>[0])
    };

// Restricted form for procedure naming, is a literal so is in truths
<proc-name> ::= <nondigit> { <nondigit> | <digit> | "-" | "_" };

// The argument list is a list of literals or variables
<argument-list> ::= <arg-nt> { "," <arg-nt> } --> {
    any semanticsOf(<arg-nt>[0], <arg-nt>[1])
};

// Argument non-terminals can either be literals or variables
<arg-nt> ::= <literal> | <variable-name> | <statement> ;

// These have default terminating semantics by the truths
<literal> ::= <matrix> | <number> ;

// Variable names take a standard form
<variable-name> ::= { <utf-8-char> } ;

// The procedure body is a list of statements.
<proc-body> ::= { <statement> } --> {
    any semanticsOf(<statement>[0])
};

// Statements make up the program in the language
<statement> ::= <process-sheet> | <reduce-sheet> | <binary-op> ;

// Process-sheet allows the programmer to specify a function to operate on
<process-sheet> ::= 
    "process " "(" <function-name> "," <arg-nt> "," <dim> ")" --> {
    any map(<function-name>[0], <arg-nt>[0], <dim>[0])
};

// Reduce-sheet allows reduction of the rows or columns of the sheet with a 
// function and value
<reduce-sheet> ::= 
    "reduce " "(" <function-name> "," <arg-nt> "," <arg-nt> "," <dim> ")" --> {
        any fold(<function-name>[0], <arg-nt>[0], <arg-nt>[1], <dim>[0])
    };

// Dimension markers for traversal
<dim> ::= "0" | "1" ;

// Mathematical operations to provide to process-sheet and reduce-sheet
<binary-op> ::= <number> "+" <number> --> {
        number n : {n = n1 + n2}() :
            {number n1 <= <number>[0]}, {number n2 <= <number>[1]}
    } |
    <number> "-" <number> --> {
        number n : {n = n1 - n2}() :
            {number n1 <= <number>[0]}, {number n2 <= <number>[1]}
    } |
    <number> "*" <number> --> {
        number n : {n = n1 * n2}() :
            {number n1 <= <number>[0]}, {number n2 <= <number>[1]}
    } |
    <number> "/" <number> --> {
        number n : {n = n1 / n2}() :
            {number n1 <= <number>[0]}, {number n2 <= <number>[1]}
    };

// Functions may only have single expressions in their bodies. 
<function-def> ::= 
    "fun " <function-name> "(" <argument-list> ")" "{" <fun-body> "}" ";" --> {
        none deffun(<function-name>[0], <argument-list>[0], <fun-body>[0])
    };

// These both get inferred semantics.
<function-name> ::= <proc-name> ;
<fun-body> ::= <statement> ;

};

\end{minted}

% subsubsection designing_the_language_itself (end)

% subsection designing_the_example_language (end)

\subsection{An Example Program} % (fold)
\label{sub:an_example_program}
Having  defined the language `spreadsheet', it remains to define an example program in this \gls{dsl} to exemplify what it is capable of.
For the purpose of this demonst.ration, this program will sum all the columns in an input matrix, and then get the sum of those results. 
It is defined in \autoref{lst:an_example_spreadsheet_program}.

\begin{listing}[!htb]
\begin{minted}[]{text}
proc sum_matrix (matrix) {
    // Sum down columns then across the results (rows)
    reduce(sum, 0, reduce(sum, matrix, 1), 0)
};

fun sum (val1, val2) { val1 + val2 };
\end{minted}
\caption{An Example Spreadsheet Program}
\label{lst:an_example_spreadsheet_program}
\end{listing}

This program defines a top-level procedure, with the return value being the value of the last statement. 
It computes the sum of each column, which results in a row vector (a $1 \times n$ matrix), which is then reduced over the row dimension to compute the overall sum.
This procedure relies on the function \mintinline{text}{sum}, which computes the sum of its two arguments. 

% subsection an_example_program (end)

\subsection{Reflecting on the Language Design Process} % (fold)
\label{sub:reflecting_on_the_language_design_process}
Overall, the process of designing this language was fairly simple.
Using the design documentation specified in \autoref{sec:special_language_features} on \autopageref{sec:special_language_features} it was clearly possible to define the language as required:
\begin{itemize}
    \item \textbf{Truths:} It was quite clear that when a production was only intended to have a literal meaning that it should be put in the \mintinline{text}{truths} definition block.
    Choosing the type for this was sometimes non-obvious, however, but if in doubt the \mintinline{text}{any} type would suffice as it would be bound properly at \gls{dsl} compile-time.
    \item \textbf{Meta-Thinking:} The main issue in defining the language was to think at the \textit{language} level, rather than the \textit{program} level.
    There were multiple occurrences where behaviour became confused between the language and a program in that language, requiring alterations to the language definition. 
    \item \textbf{Metaspec Deficiencies:} It was an oversight to not allow for the definition of constants as arguments to the special-syntax forms. 
    This would allow the creation of more elegant interfaces to some of these forms, and hence a better experience for the language designer.
    An example of this is illustrated in \autoref{lst:an_alternative_definition_for_reduce_sheet}, which would allow a more fluent interface for users by using better domain terminology, rather than syntax enforced by the special syntax form. 
    \item \textbf{Metacompiler Support:} Much like the edit-compile-test interactive development cycle when writing programs, the \gls{absol} metacompiler enables a similar kind of feedback for the people defining the \gls{dsl}.
    During the course of developing `spreadsheet' it was quickly made apparent that this support, and the notion that the user would immediately be alerted to errors with their language definition, allowed the \gls{dsl} designer a greater ability to experiment with both syntax and semantics.
    This liberty to experiment led to the creation of more intuitive syntax for the \gls{dsl}, but also allowed the language to remain correct.
    It should be noted that, in part, this interactivity is enabled by the performance of \gls{absol}, which requires no noticeable wait on the behalf of the language designer. 
\end{itemize}

\begin{listing}[!htb]
\begin{minted}[numbers=none, fontsize=\blockfont]{text}
<reduce-sheet> ::= 
    "reduce " "(" <function-name> "," <arg-nt> "," <arg-nt> "," <dim> ")" --> {
        any n : {n = n1}(n2 == "rows") : 
            {any n1 <= any fold(<function-name>[0], <arg-nt>[0], <arg-nt>[1], 0)},
            {any n2 <= <dim>[0]} |
        any n : {n = n1}(n2 == "cols") : 
            {any n1 <= any fold(<function-name>[0], <arg-nt>[0], <arg-nt>[1], 1)},
            {any n2 <= <dim>[0]} |
        any n : {n = n1}() : 
            {any n1 <= any fold(<function-name>[0], <arg-nt>[0], <arg-nt>[1], 0)},
            {any n2 <= <dim>[0]}
    };

<dim> ::= "rows" | "cols" ;
\end{minted}
\caption{An Alternative Definition for \texttt{<reduce-sheet>}}
\label{lst:an_alternative_definition_for_reduce_sheet}
\end{listing}

% subsection reflecting_on_the_language_design_process (end)

% section an_example_language (end)

% chapter testing (end)

\chapter{Evaluation} % (fold)
\label{cha:evaluation}
The design and development of both \gls{metaspec} and the \gls{absol} metacompiler toolchain has involved an immense amount of careful thought, time and effort.
The final results of the project are impressive, but flawed.
This section aims to examine and discuss both the major successes of the \gls{absol} project, including how it succeeded at its high-level goals and advanced the state of the art in language verification.
It will also provide a discussion of the major failings of the project, particularly around scoping, design and the development process itself.
Finally, it will suggest future work in the area of \gls{dsl} verification, and improvements that could be made to the \gls{absol} toolchain.

\section{Successes of the Project} % (fold)
\label{sec:successes_of_the_project}
The key successes of the \gls{absol} project are twofold. 
The first major success is how well both \gls{metaspec} and the metacompiler toolchain itself were able to meet its high-level requirements: it has been able to achieve all of the requirements detailed for it.
The second major success is the contribution that this project has been able to make to the state of the art in language verification: through its constrained form of language semantics it has been able to provide a flexible and useful toolchain for the creation of provably correct languages. 

\subsection{Meeting the High-Level Goals} % (fold)
\label{sub:meeting_the_high_level_goals}
One of the main concerns during the design and development of both \gls{metaspec} and \gls{absol} has been meeting the high-level system requirements set out in \autoref{sec:high_level_requirements_specification} on \autopageref{sec:high_level_requirements_specification}. 
Both the metalanguage and the metacompiler program have been able to meet all of these requirements due to the comprehensive design process that both underwent. \\

\gls{metaspec} has been a particular success of the project as the detailed and careful design process for the language, as explored in \autoref{cha:designing_the_metalanguage}, has ensured that the metalanguage really appears as one \textit{integrated} entity, rather than being composed out of parts.
One of the most successful portions of the metaspec language is the method by which it provides an integrated syntax and semantic specification (Requirement~\reqref{req:IntegratedSyntaxandSemanticSpecification}).
While the syntax definition format required little adaptation from the \gls{ebnf} it was based upon, the linking of syntax and semantics through the `syntax-access' blocks (\autoref{sub:accessing_syntax_from_the_semantics}) is far more intuitive than would initially have been expected. 
The success of the combined specification is complemented by the amount of design work that went into defining the syntax (and hence features) of the user-defined semantics (\autoref{sub:user_defined_semantics}).
The comprehensive nature of the translation from big-step operational semantics to this language has retained the full set of semantic features, and is a big boon in the expression of \gls{dsl} semantics.\\

The second major success of the metalanguage arises from the ease with which it allows users to define languages. 
While part of this ease comes from the support of the metacompiler tools, the language itself provides obvious ways to accomplish the goals of the \gls{dsl} designer. 
While in part this comes from the neat integration of syntactic and semantic definitions, the designer also benefits significantly from the approach it takes to defining languages. 
Through the provision of the \mintinline{text}{truths} defblock, it becomes simple for language designers to declare when they know things terminate (such as the literals shown in \autoref{sec:an_example_language}).
Beyond this, the provision of semantic special-forms, especially the special-syntax rules, means that language designers are \textit{not} overly constrained when it comes to defining the capabilities of their language.\\

As much as the metalanguage has been one of the largest successes of the project, the \gls{absol} toolchain itself has also been a major success. 
Much like \gls{metaspec}, the metacompiler has also been able to meet all of its high-level goals placed upon it.
Within these goals, however, it has achieved two highly-significant successes.\\

The first of these successes comes through the verification capabilities of the metacompiler. 
Though the design process for the verification algorithms (\autoref{sec:the_core_algorithms}) was complex and involved, the final implementation of the verification theory has turned out to be elegant and fairly simple.
This means that the success is twofold: the verification portion of the metacompiler remains extensible, but also in the capabilities it affords the user.
In being able to verify the language as it is being developed, it provides the user with significant assurance through this development process.
Through the practice of developing example languages, such as `spreadsheet', discussed in \autoref{sec:an_example_language}, it became apparent that the support from the metacompiler allowed for more experimentation and less mental load.
As the verification of the language was offloaded to \gls{absol}, the language designer could better concern themselves with the \textit{form} of the language and the \textit{local} semantics, rather than constantly ensuring global semantic correctness.
While this success comes from a very limited sample-size, the benefits felt to a highly-experienced user could imply that the metacompiler would be an even more significant boon to the less-experienced user. \\

The other major success of the metacompiler itself is the language diagnostic facility that it provides.
Though derived from a difficult design task, the use of the \mintinline{haskell}{RuleTag} type and associated monoidal properties for tracking error information have allowed the metacompiler to provide the user with comprehensive diagnostics.
The resultant diagnostics are very obvious, providing both a message describing the problem and a diagnostic trace of where the problem arises from.
Hence, they allow the users of the metacompiler to quickly identify and fix errors in their language definition, enhancing the utility of the metacompiler.

% subsection meeting_the_high_level_goals (end)

\subsection{Contributions to the State of the Art} % (fold)
\label{sub:contributions_to_the_state_of_the_art}
% - Funcall and the other special forms are important for the quality of the project.
% - The ability to verify at the language level while still retaining useful functionality for the creation of capable languages. 
% - Integrated syntax and semantic specifications that operate together intuitively. 

As discussed above, the project has seen success in terms of meeting its high-level goals.
However, alone that does not make the project worthy of note.
Much of the major success of this project arises from the way in which it contributes to the state-of-the-art in the area of formal language verification.\\

While it would be possible to define a language purely based on big-step operational semantics, the restrictions required to prove termination about this language would make it \textit{useless}.
The main contribution of \gls{absol} to the state of the art is, resultantly, the ability to define useful (though still limited) languages while retaining the ability to prove them correct.
This contribution to the state of the art arises from two main factors:
\begin{enumerate}
    \item \textbf{The User-Defined Semantics:} The users of \gls{metaspec} are given the ability to define the semantics of their own \gls{dsl} within the restrictions imposed on these semantics.
    While sufficient for defining many common operations, they are not enough to make a useful language on their own.
    \item \textbf{The Special Semantic Forms:} The special semantic forms provide known-to-terminate functionality to the \gls{dsl} designer while allowing the implementation of more complex, useful language features.
\end{enumerate}

It is the \textit{combination} of both of these features in a context where the termination properties can be proved that is the real contribution of the project.
Previous frameworks have existed for specifying complex language semantics, such as Mosses' Action Semantics \citep{mosses1992action}. 
However, such semantic frameworks become too complex to automatically verify languages based upon them.
As a counterpoint simpler semantic frameworks exist, such as restricted forms of operational semantics, that \textit{can} be automatically proven to terminate (under certain restrictions) \citep{Zhang:2004:SSD:981009.981013}.
Much like the more complex frameworks these also have an issue, except here it is that they are incapable of representing more complex (and hence useful) language semantics while still being provable.\\

To this end, the combination of \gls{metaspec} and \gls{absol} lies in the middle of these two roads.
While it provides restricted methods for users to define semantics, allowing semantics forms such that they can automatically be shown to terminate,
it \textit{also} provides additional semantic forms.
These additional forms, while they have to be proved total externally, can then interact with the proof mechanism simply. 
It is this combination that represents the major contribution of the project, allowing the definition of capable languages while still retaining the ability to prove the language total.
This means that all programs written in these languages will also be total.\\

The second major contribution of this project to the state of the art in language verification is the creation of \gls{metaspec}. 
Prior to the creation of this language there appeared to be no metalanguages that allowed for the combined specification of both \textit{syntax} and \textit{semantics} in an integrated form. 
While \autoref{sub:specifying_language_semantics} explores multiple methods (of varying complexities) for the specification of language semantics, research found that none of them properly integrated with the syntactic descriptions of languages.
This separation meant that it was often difficult to reason about the semantics of a given program fragment, given that syntax and semantics are so intrinsically linked in a verification context.\\

This highlights that the creation of the \gls{metaspec} metalanguage offers a significant contribution to the state of the art in that it provides an \textit{integrated} method for specifying both syntax and semantics in one file.
This contrasts dramatically with existing approaches that keep syntax and semantics separate, or annotate existing metasyntactic notations rather than using a custom-designed notation.
As a result, it is clear that this integrated approach brings significant benefits to the usability of defining languages.

% subsection contributions_to_the_state_of_the_art (end)

% section successes_of_the_project (end)

\section{Failings of the Project} % (fold)
\label{sec:failings_of_the_project}

As much as the design and development of the project has been able to meet its high-level goals, this does not mean to imply that it has been without its problems. 
During the course of the project there have been multiple instances where portions of the project could have been better-designed, or processes being used could have been refined. 
This section aims to examine these major project failings and discuss how they could have been mitigated. 

\subsection{Scoping the Project} % (fold)
\label{sub:scoping_the_project}
One of the major failings of the project was an initial inability to define the scope of the system appropriately.
Initial requirements and the corresponding system designs intended to incorporate both the full semantic guard verification algorithm and to perform full code-generation for the \gls{dsl} compiler, as discussed in \autoref{sub:out_of_scope_requirements}. \\

While this would not have been a major issue had the project scope been decreased quickly, the project progressed under the impression that both of these features would be feasible to develop within the provided time frame.
This meant that some not insignificant effort was wasted performing design work for them. 
In particular, significant time was spent attempting to define an algorithm for the checking of guard completeness before this was determined as difficult enough to rule out of scope (see \nameref{ssub:checking_guard_completeness} on \autopageref{ssub:checking_guard_completeness}).
While this could have been ruled out of scope sooner through more targeted research (see \nameref{ssub:guard_completeness_and_project_scope} on \autopageref{ssub:guard_completeness_and_project_scope}), it was thought of as an easy problem and not worthy of detailed research. 
This was very much incorrect.\\

The other major scoping issue on the project was the insistence on the creation of a full `product' pipeline for \gls{absol}.
This meant that the generation of code for the \gls{dsl} compiler was considered as in scope for a significant portion of the project.
While it was always recognised as a large body of work (and a non-novel portion of the project), it did not have as much design work dedicated to it as the guard completeness problem had.
Nevertheless, the project under-appreciated how much work would be contained within the development of the parser and verification engine.
When this work became apparent, the code-generation module was swiftly ruled out of scope. \\

While the now useless design and development work dedicated to these portions of the project has not impacted on the project's completion in any undue fashion, it is clear that having this time available for other tasks would have achieved a much more efficient time utilisation.
As a result, the failure to scope the project appropriately is one of the major failings of the project.

% subsection scoping_the_project (end)

\subsection{Metalanguage Design Issues} % (fold) 
\label{sub:metalanguage_design_issues}
While \gls{metaspec} represents a big step in the design of metalanguages for syntactic and semantic specification, it is not without some significant issues.
The majority of these issues arise from portions of the language that were not given enough consideration during the design process, though some also arise from issues with the language grammar. 

\subsubsection{Issues with the Metaspec Grammar} % (fold)
\label{ssub:issues_with_the_metaspec_grammar}
One of the main issues with the current form of the \gls{metaspec} language grammar is that it is under-constrained in certain circumstances. 
This means that, from a grammatical standpoint, it will admit syntax that is not actually valid at the metalanguage semantics level.
There are a few main examples of this:
\begin{itemize}
    \item \textbf{Special-Syntax Arguments:} The grammar in its current form admits any number of arguments to each special syntax form. 
    This means that the verification of these arguments is not part of parsing, but later at the level of \gls{dsl} compilation time (when the types of the arguments are also checked).
    While this does not impinge on the correctness of the verification result, it would be a significant improvement to enforce the correct number of arguments for each piece of special syntax at the level of the language grammar. 
    \item \textbf{Semantic Associations:} The semantic association was joined to the existing syntactic grammar at the level of the alternation: each term in \mintinline{text}{<a> | <b> | ...} could each provide its own semantics.
    However, this addition did not recognise the fact that alternations can contain sub-alternations (e.g. \mintinline{text}{(<a> | <b>) | <c>}).
    At the level of the grammar, it is possible to write \gls{metaspec} statements which assign semantics to both \mintinline{text}{<a>} and \mintinline{text}{<b>}.
    While these don't have any semantic meaning and are ignored by all stages of the metacompiler pipeline, it would ideally not be possible to write such expressions in the grammar. 
\end{itemize}

Fixing both of these flaws in the current metalanguage design would result in minor, but still useful improvements to the metalanguage as a whole. 

% subsubsection issues_with_the_metaspec_grammar (end)

\subsubsection{The Design of the Environment} % (fold)
\label{ssub:the_design_of_the_environment}
%   - No direct use of environment accesses in basic semantics, requiring indirection.
%   - Fairly awkward use of the environment (no ability to delete properties, for example). 
% Environment not particularly well thought-out.

The notion behind the semantic environment was that it would be able to act as a global key-value store. 
However, this idea was not thought through to its full extent, leaving the current version of the environment with the following issues:
\begin{itemize}
    \item \textbf{Defaulting:} When the semantic rules attempt to retrieve a value for which a key does not exist, the environment will return a default for the appropriate type. 
    While this ensures that the semantics are always defined, it is relatively inelegant. 
    Ideally, even in the basic environment rules, the user would be able to specify the default value that they want to return, and even check for the presence of a given key.
    Neither of these are insurmountable obstacles, but would require careful semantic analysis and integration in the current design of metaspec.
    \item \textbf{Usability in Semantic Rules:} The current metaspec grammar precludes environment accesses from occurring in a number of locations where they would be useful (especially in semantic evaluation rules).
    This means that special-syntax provides functionality for performing this in these locations, but it would be much more elegant to just allow the use of standard environment syntax (which would bring back the benefit of all environment-based accesses being instantly visible --- see \autopageref{ssub:environment_input_rules}).
    \item \textbf{No Property Deletion:} Currently, any value stored under a given key in the environment is unable to be deleted.
    While it can be overwritten with some value that is semantically equivalent (in the \gls{dsl} domain) to \mintinline{text}{null}, this is not the same as being able to delete a property from the environment entirely.
    Future work on the environmental design would be able to introduce this feature and ensure that it works in an always-terminating manner. 
    \item \textbf{No Scoping:} The semantic environment is a purely un-scoped store of information at the moment.
    While this does work from a basic perspective, it restricts the ability of \gls{dsl} designers to define scope-based constructs in their language.
    While this is not a \textit{particularly onerous} restriction in the context of \glspl{dsl}, it would not impact the semantic correctness of the environment to allow it to work in a scoped fashion.
    Ideally, future revisions of \gls{metaspec} would allow this.
    \item \textbf{No Error States:} Currently, if a key already exists in the semantics, it will just have its value overwritten silently for any new input with that key.
    Ideally, this would instead provide information to the \gls{dsl} designer via some kind of feedback mechanism, allowing them to reliably ensure that the correct value is associated with the correct key in the environment.
\end{itemize}

Future work on \gls{metaspec} would definitely benefit from revising the nature of the environment to make it more useful to \gls{dsl} designers.
As it is, it's not a particularly useful feature due to the lack of thought that went into its design.

% subsubsection the_design_of_the_environment (end)

\subsubsection{Special-Syntax Flexibility} % (fold)
\label{ssub:special_syntax_flexibility}
While able to be rectified at the grammar level, this is a more in-depth issue than those discussed above. 
In their current form, arguments to a special-syntax rule can only take the form of two things: environment accesses, and syntax access blocks. 
However, as discussed in \autoref{sub:reflecting_on_the_language_design_process}, it could be quite useful to be able to pass semantic constants as arguments to the special-syntax rules as well.\\

The main reason for this is that the current inability to use constants breaks one of the core tenets of a dsl (see \autoref{sec:domain_specific_languages}) --- they should match the notation of their domain as closely as possible.
Much like the example in \autoref{sub:reflecting_on_the_language_design_process}, the ability to use values of the domain-syntax and translate these into the syntax of the underlying semantic functions provides a much more succinct interface at the level of the \gls{dsl}.
As key as the domain-notation-based nature of \glspl{dsl} is to their usage and utility, this is a \textit{major} oversight in the design of the metalanguage.\\

As allowing constants does not alter the termination properties of any of the special syntax rules, this is purely an issue at the metalanguage design level.
Future work on this project could significantly benefit from removing this restriction.
This would allow \gls{metaspec}-defined \glspl{dsl} to far better match their domain notation.

% subsubsection special_syntax_flexibility (end)

% subsection metalanguage_design_issues (end)

\subsection{Metacompiler Design Issues} % (fold)
\label{sub:metacompiler_design_issues}
Much like the metalanguage, the \gls{absol} toolchain itself has some not-insignificant flaws in its design. 
While it, too, provides some novelty and advances the state of the art through the form of its verification algorithms, the rectification of these design issues would improve it as a tool.

\subsubsection{Error Reporting Issues} % (fold)
\label{ssub:error_reporting_issues}
With the diagnostic capabilities of \gls{absol} proving so integral to the \gls{dsl} development process, it is clear that the clarity and explanatory power of these diagnostics is key.
However, while a diagnostic message is available for every error the metacompiler can diagnose, sometimes these messages are not specific enough.\\

This is particularly true of the criteria defined for the semantic operation evaluations.
As the algorithm in \nameref{ssub:verifying_the_evaluation_criterion} on \autopageref{ssub:verifying_the_evaluation_criterion} states, there are six criteria that must be satisfied.
However, the current error reporting groups all of these issues under a single error message, making it difficult for the user of the toolchain to find and correct the issue.\\

This error-reporting deficiency is also true in general.
While many of the errors that exist today \textit{are} helpful, most of these could still provide additional contextual information (e.g. exactly \textit{which} evaluation is wrong).
Future development of the toolchain should definitely dedicate some time to the improvement of these error messages.

% subsubsection error_reporting_issues (end)

\subsubsection{Precondition Verification} % (fold)
\label{ssub:precondition_verification}
It is the responsibility of the metacompiler to verify the preconditions for the verification algorithms are correct, and it does this.
However, from a user-utility perspective, the method by which this is performed is suboptimal.\\

As the algorithm in use performs many of its checks \textit{during} the parse of the input file, any error encountered and output will stop the parse entirely.
This means that even if there are other issues present in the file, the user has to go through this iterative cycle of `check' / `fix'.
It would be much more user-friendly to collate all of these errors during the parsing process and output them in a comprehensive fashion at the end.
This would provide a better experience for the user, and hence should be improved in future work on the metacompiler. 

% subsubsection precondition_verification (end)

% subsection metacompiler_design_issues (end)

\subsection{Development Issues} % (fold)
\label{sub:development_issues}
As discussed in \autoref{sub:the_development_process} and \autoref{sec:the_general_testing_approach}, the approaches to both development and testing on this project were somewhat ad-hoc. 
While the project has been completed successfully, it was not without its headaches, and many of these could have been mitigated or avoided entirely through use of more formal processes.
The lack of formalisation seen throughout this project is that it was mainly treated as a theoretical project, rather than a software-development project.
While it \textit{has been} highly theoretical, in hindsight the application of more rigorous processes and quality standards would have been beneficial.\\

The lack of formalisation of the design process, in particular, is what has led to many of the issues identified in \autoref{sub:metalanguage_design_issues} above.
Ideally, a more formal design process would have been employed by the project, bringing the benefits of additional design rigour.
Doing this would have likely improved the software architecture, algorithmic design and better identified the features required for the metalanguage. 
Applying such a process would likely have led to the environment being removed entirely as a feature until it could have been better developed, and given the list of issues with illustrated in \nameref{ssub:the_design_of_the_environment} above, this would not have been entirely negative. \\

Similarly, \autoref{sub:examining_the_testing_approach_in_hindsight} identifies some major problems with the lack of a rigorous (and automated) testing approach.
The regressions in the parser and verification engine, in particular, that were not caught until much later on cost the project some not-insignificant time to fix.
Employing a more rigorous testing procedure, much like a more rigorous design process, would most likely have identified these regressions immediately, or even prevented them from happening in the first place.\\

The initial intent behind the lack of formal processes was to allow a flexible and unrestricted approach to the design and development of \gls{absol}.
With the benefit of hindsight, however, this approach created more problems than it solved.
Doing the project again would involve far more rigorous design and development processes, as well as the employment of a fully automated test suite.

% subsection development_issues (end)

% section failings_of_the_project (end)

\section{Future Work} % (fold)
\label{sec:future_work}

While \gls{absol} is a substantial achievement, it is far from the end of the evolution of language verification. 
With the contributions to the state-of-the art that this project has made, the theoretical and practical advancements that have resulted from it should be able to be integrated into future work in this space. 
Similarly, the toolchain itself is still not complete, and future evolution of this project should focus on developing the final code-generation stage to take it from theory to useful tool.

\subsection{Evolving the Art of DSL Verification} % (fold)
\label{sub:evolving_the_art_of_dsl_verification}
As discussed above, the project has made two major contributions to the art of \gls{dsl} verification.
Taking these contributions forward could involve either of the following:
\begin{itemize}
    \item \textbf{Developing the Metalanguage:} Providing a novel approach to combining syntax and semantics for language development through its syntax access methodology, \gls{metaspec} provides a robust foundation for the further development of metalanguages.
    \item \textbf{Dual-Level Semantics:} The other innovation of this project was the combination of multiple levels of semantics.
    In doing so it was possible to both provide useful features for \gls{dsl} designers, but also enable provable correctness for these languages.
\end{itemize}

Future work in this area could involve expansion in both of these areas, as outlined below.
However, by the same token, it could involve totally unintended directions for this work.

\subsubsection{The Future of Metaspec} % (fold)
\label{ssub:the_future_of_metaspec}
While \gls{metaspec} itself is limited to the scope of the \gls{absol} project, this does not mean that the principles behind the metalanguage cannot be employed in other avenues. 
The key innovation in metaspec that is more broadly applicable is the syntax-access blocks, by which the syntax and semantics are tied together directly. 
Future work could certainly evolve the metalanguage, using this integration principle to combine the syntactic specification with more complex semantic specifications (e.g. Action Semantics).
These more complex semantic specifications would bring significantly increased expressive power.\\

While these expanded specifications may not always be able to be automatically proved, though \citet{Mosses:2009:CS:1596486.1596489} has pursued significant work in that area, this integrated approach brings significant benefits over existing approaches.
Combining syntax and semantics in future would make it much easier to reason about the semantic properties of languages, much as it has done for \gls{absol}.

% subsubsection the_future_of_metaspec (end)

\subsubsection{Multi-Level Semantic Proofs} % (fold)
\label{ssub:multi_level_semantic_proofs}
The multi-level semantic proofs could also be expanded.
Theoretically, this could be taken in two directions:
\begin{enumerate}
    \item \textbf{Additional Proof Levels:} Further levels of semantics could be implemented, allowing the proof of more complex language properties.
    \item \textbf{Creation of Additional Capabilities:} Additional capabilities could be added within the existing proof framework.
\end{enumerate}

Both of these are likely to be fruitful endeavours, though it will be important to ensure that the scope within which these operate is still constrained.
Without appropriate constraints on the language being constructed, the proof of semantics becomes impossible.\\

An alternate direction for the semantic proofs is to move the semantic checking to \textit{program} compile time.
While that is far out of scope for this project, it enables the proof of many more properties while increasing the expressive potential of the language.
This would rely more on the work of \citet{hinze2010reasoning} and \citet{nordstrom1988terminating} on the termination of general recursion than any insight brought by this project, but the development and use of \gls{absol} has demonstrated just how useful such proof-based approaches can be.

% subsubsection multi_level_semantic_proofs (end)

% subsection evolving_the_art_of_dsl_verification (end)

\subsection{Improving the ABSOL Toolchain} % (fold)
\label{sub:improving_the_absol_toolchain}
Aside from the developments of the state of the art based upon the work of this project, \gls{absol} itself is still unfinished.
Future work could certainly involve the development of the incomplete portions of the toolchain, and evolutions of existing features.
Some proposals for further development are as follows:
\begin{itemize}
    \item \textbf{Code Generation:} In order for \gls{absol} to become a truly useful tool, it needs to be capable of generating a compiler for the defined DSL.
    As the \gls{metaspec} files contain both syntax and semantics all of the necessary information is available to the metacompiler.
    Generation of syntax parsers based on the grammar is easy, especially with the use of combinator-based parsing.
    While the attribution of semantics to these new productions is harder, work by \citet{diehl1996semantics} has shown that it is doable.
    \item \textbf{Further Special Features:} As discussed in \autoref{sub:discounted_language_features}, there are still multiple special language features that could be proven to terminate and then implemented.
    These would bring additional expressive power and expand the capabilities of \gls{absol} \glspl{dsl}.
    \item \textbf{Proper Guard Verification:} As has been discussed throughout the document, particularly in \autoref{sub:guard_checking}, the checking of guards for completeness is a difficult problem.
    However, \autoref{sub:guard_completeness_checking} discusses a possible method for accomplishing it, and hence it is likely to be a fruitful avenue to pursue in the future. 
\end{itemize}

In addition to these undeveloped features of the toolchain, there is a wealth of improvements to be made, as discussed in the above sections.
Ideal future development of \gls{absol} would fix these issues, and then continue on to finalise toolchain development as discussed above.

% subsection improving_the_absol_toolchain (end)

% section future_work (end)

% chapter evaluation (end)

\chapter{Conclusion} % (fold)
\label{cha:conclusion}
As a project, \gls{absol} has been a great success.
Despite several failures of the project, the end result is a novel contribution to the state of the art in language verification. 
Despite the flaws in both the metalanguage and metacompiler, the achievements of this project cannot be ignored. 
The main successes of this project are twofold:
\begin{itemize}
    \item \textbf{Advancing the State of the Art in Language Verification:} With the combination of the integrated syntactic and semantic specification in the form of \gls{metaspec}, and the novel verification algorithms implemented in \gls{absol} itself, this project provides two major novel contributions to the language verification ecosystem.
    \item \textbf{An Extensible Metacompiler:} The final version of the metacompiler is a functional and extensible solution to the language verification problem for \glspl{dsl} with limited semantics.
    It allows the language designers to create in relative safety, providing clear and concise diagnostics to help diagnose errors.
\end{itemize}

As a tool, \gls{absol} came very close to meeting its initial project goals, with only the omission of the code-generation step really marring that record. 
Nevertheless, this toolchain goes a long way to making provably correct \glspl{dsl} a reality in the context of the kind of systems that inspired this project.
It allows for the definition of capable \glspl{dsl}, and also for their verification, ensuring that the language semantics are total and thus eliminating entire classes of bug in the host systems.
With the implementation of the remaining system components as outlined in \autoref{sub:improving_the_absol_toolchain} it would be a fully-fledged system for \gls{dsl} creation.\\

During the development of this system, the metalanguage and algorithmic design processes went incredibly smoothly, resultant from the weeks of careful thought that had been put into both. 
This meant that the corresponding development proceeded well, working directly from these comprehensive designs.
Conversely, the lack of rigid processes imposed on the development and testing of the system, resulted in a not-insignificant number of bugs being found far later than was ideal.
Despite this, the system works as intended (even though it's not \textit{quite} complete), and its operational nature is a big success. \\

The final result, then, is this: \gls{absol}, while a flawed system, is a complex and powerful tool for the creation of provably correct \glspl{dsl}. 
It has produced novel contributions to the state of the art in language verification through the creation of the \gls{metaspec} metalanguage and the dual-part semantic verification engine that allows automated semantic verification while retaining useful language semantics.
It hence fills a niche in the language verification ecosystem, bridging the gap between complex semantic systems and systems so inflexible as to be unusable.
While it still needs some work, the future is bright, with \gls{absol} both having a clear path for further development, and advancing the state of the art. 

% chapter conclusion (end)


% Bibliography
\bibliographystyle{abbrvnat}
\bibliography{resources/bibliography}

\begin{appendix}

\chapter{Software Readme} % (fold)
\label{cha:software_readme}
This section will contain a guide to the project as a whole, as well as the necessary details to write languages and run the metacompiler on them.

\section{The Project Structure} % (fold)
\label{sec:the_project_structure}
The implementations for each of the metacompiler components are contained in the following files. 
Each of these files is given relative to the project root in \mintinline{text}{absol/}:
\begin{itemize}
    \item \textbf{The Metacompiler Front-End:} The files for the front-end are contained in:
    \begin{itemize}
        \item \mintinline{text}{./app/Main.hs}
        \item \mintinline{text}{./app/Cmdargs.hs}
    \end{itemize}
    \item \textbf{Metaparse:} The files for metaparse are contained in:
    \begin{itemize}
        \item \mintinline{text}{./src/Absol/Metaparse.hs}
        \item \mintinline{text}{./src/Absol/Metaparse/Grammar.hs}
        \item \mintinline{text}{./src/Absol/Metaparse/Parser.hs}
        \item \mintinline{text}{./src/Absol/Metaparse/Utilities.hs}
        \item \mintinline{text}{./src/Absol/Metalex.hs}
        \item \mintinline{text}{./src/Absol/Metaspec.hs}
        \item \mintinline{text}{./src/Absol/Metaspec/Special.hs}
    \end{itemize}
    \item \textbf{Metaverify:} 
    \begin{itemize}
        \item \mintinline{text}{./src/Absol/Metaverify.hs}
        \item \mintinline{text}{./src/Absol/Metaverify/Collate.hs}
        \item \mintinline{text}{./src/Absol/Metaverify/Diagnostics.hs}
        \item \mintinline{text}{./src/Absol/Metaverify/RuleTag.hs}
        \item \mintinline{text}{./src/Absol/Metaverify/State.hs}
    \end{itemize}
    \item \textbf{Misc:} There are also stub implementation files for other components of the metacompiler toolchain that have since been declared as out of project scope. 
\end{itemize}

% section the_project_structure (end)

\section{Building and Executing ABSOL} % (fold)
\label{sec:building_and_executing_absol}
To build and execute the metacompiler please follow the instructions in the following sections.

\subsection{System Requirements} % (fold)
\label{sub:system_requirements}
In order to both build and execute \gls{absol} you will need to have the following software available:
\begin{itemize}
    \item \textbf{Stack:} The Haskell build management tool is able to handle the downloading and installing of both the required compiler version and required libraries.
    \item \textbf{Internet Connection:} Stack requires access to the internet to download and build the packages that are depended upon by \gls{absol}. 
\end{itemize}

% subsection system_requirements (end)

\subsection{Building the Metacompiler} % (fold)
\label{sub:building_the_metacompiler}
Building \gls{absol} is a simple prospect as long as the system requirements are met.
It can be done as follows:
\begin{enumerate}
    \item Open a terminal.
    \item Navigate to the root directory of the project as described in \autoref{sec:the_project_structure}.
    \item Execute the command \mintinline{text}{stack setup}. This will download and install the appropriate versions of GHC and its support libraries. 
    \item Execute the command \mintinline{text}{stack build}. 
\end{enumerate}

This will result in a binary being built for your particular system configuration.
The binary can then be executed as described in \autoref{sub:executing_the_metacompiler}.

% subsection building_the_metacompiler (end)

\subsection{Executing the Metacompiler} % (fold)
\label{sub:executing_the_metacompiler}
Once the metacompiler has been built, it can be executed on a provided test file or one of your own creation.
Test files are provided in the \mintinline{text}{./metaspec/} directory. 
One of these is the example language developed in \autoref{sec:an_example_language}, while the other is the test file discussed in \autoref{sec:testing_during_development}.
The metacompiler can be executed as follows (assuming that you have a terminal open in the same directory as described above):
\begin{enumerate}
    \item Choose a metaspec file and note its relative path to your current directory. 
    This is denoted \mintinline{text}{FILE} in the following command.
    \item First run \mintinline{text}{stack exec absol -- --help}. 
    This will display the help text for all of the command-line options.
    \item Next run \mintinline{text}{stack exec absol -- FILE -l}.
    This will execute the metacompiler on \mintinline{text}{FILE} in verbose reporting mode. 
\end{enumerate}

If you have used one of the example files you will see that the metacompiler is able to verify the file such that it terminates.
Please feel free to introduce errors into either of these files to examine how the metacompiler behaves in the presence of errors. 
If you continue to include the \mintinline{text}{-l} flag in these circumstances you will see comprehensive diagnostic information about whatever error you introduce. 

% subsection executing_the_metacompiler (end)

% section building_and_executing_absol (end)

% chapter software_readme (end)

\chapter{The Metaspec Grammar} % (fold)
\label{cha:the_metaspec_grammar}
Metaspec, as a language, has a fairly complex syntactic structure. 
The syntax for the language is represented using \gls{ebnf}, and is included in its entirety below. \\

For an overview of standard \gls{ebnf} syntax, please see the official standard for the metasyntactic notation published in \citet{standard1996ebnf}.

\begin{minted}[fontsize=\blockfont]{text}

(*
    This file defines the grammar for the Syntax of the metalanguage 'metaspec'.

    It uses the EBNF syntax defined in Sections 4 and 5 of the ISO-14977 EBNF
    standard.

    The grammar does not care about whitespace except in the case of single-line
    comments which are ended by an EOL character (\n, \r, \r\n), in a
    platform-specific manner.

    Comments, defined by the metaspec-comment grammar element, are stripped 
    before parsing.
*)

(*
    This section defines literals useful in the definition of the language.
    The UTF-8 literal is defined as all graphemes that can be represented by the
    UTF-8 transformation format as defined in RFC 3629.
    For reference, the special symbols have the following meaning:

        *       repetition
        -       except
        ,       concatenate
        |       disjunction / definition separator
        =       defining
        ;       rule terminator
        []      optional
        {}      repetition
        ()      group
        ? ?     special sequence
*)

utf-8-char = ? all-utf-8-glyphs ?;
text = { utf-8-char }-;
digit = "0" | "1" | "2" | "3" | "4" | "5" | "6" | "7" | "8" | "9";
natural-number = { digit }-,;
integer = [ "+" | "-" ], natural-number;
floating-point-number = natural-number, [ ".", natural-number ];
number = natural-number | integer | floating-point-number;
textual-glyph = utf-8-char - digit;
eol-symbol = ? EOL ?;
literal-quote = ? ASCII-double-quote-symbol ?;
newline = ? \n and \r ?

(*
    This section defines the terminal symbols of the language itself, including:
    - Symbols for grammar definition
    - Symbols for comments in the language
*)

(* Terminals used for defining the grammar *)

repeat-count-symbol = "*";
except-symbol = "-";
disjunction-symbol = "|";
defining-symbol = "::=";
rule-termination-symbol = ";";

optional-start-symbol = "[";
optional-end-symbol = "]";
group-start-symbol = "(";
group-end-symbol = ")";
repeat-start-symbol = "{";
repeat-end-symbol = "}";

special-sequence-start-symbol = "<?";
special-sequence-end-symbol = "?>";

start-symbol-start = "<<";
start-symbol-end = ">>";
non-terminal-start = "<";
non-terminal-end = ">";

(* Comment Symbols *)

line-comment-start-symbol = "//";
block-comment-start-symbol = "(*";
block-comment-end-symbol = "*)";

(* Semantic Definition Symbols *)

semantic-behaves-as = "-->";
evaluates-to = "<=";
where-symbol = ":";
semantic-and = ",";
semantic-assign = "=";

semantic-environment-symbol = "e";
semantic-environment-input-symbol = "<--"
environment-access-symbol = ".";
environment-defines-symbol = ":";
semantic-list-delimiter = ",";
semantic-disjunction = "|";

semantic-block-start = "{";
semantic-block-end = "}";
restriction-block-start = "(";
restriction-block-end = ")";

syntax-access-start-symbol = "[";
syntax-access-end-symbol = "]";

special-syntax-start = "(";
special-syntax-end = ")";

(*
    This section defines the allowed types of identifiers in metaspec.
*)

non-terminal-identifier = 
    textual-glyph, { textual-glyph | natural-number | "-" | "_"};
terminal-string =  { utf-8-char - newline };
semantic-identifier = 
    (textual-glyph, { textual-glyph | natural-number | "-" | "_"})
    - (semantic-type | metaspec-feature | semantic-special-syntax);
string-literal = literal-quote, text, literal-quote;
semantic-type
    = "any"
    | "none"
    | "bool" 
    | "natural"
    | "integer"
    | "int32"
    | "int64"
    | "uint32"
    | "uint64"
    | "float"
    | "double"
    | "integral"
    | "floating"
    | "number"
    | "string"
    | "list"
    | "matrix" ;

(*
    This section defines the grammar of metaspec itself.
    The start symbol is 'metaspec'.
*)

metaspec = metaspec-defblock; (* file cannot be empty *)

(* 
    Comments in metaspec are defined as follows, and are removed in a 
    preprocessing step. They are hence not represented in the rest of the 
    language grammar in an explicit fashion. 
*)
metaspec-comment =
    line-comment-start-symbol, { utf-8-char }, eol-symbol |
    block-comment-start-symbol, { utf-8-char }, block-comment-end-symbol;

(* 
    All of these blocks must be defined once in order.
*)
metaspec-defblock = 
    name-defblock, rule-termination-symbol, 
    version-defblock, rule-termination-symbol, 
    using-defblock, rule-termination-symbol, 
    truths-defblock, rule-termination-symbol, 
    language-defblock, rule-termination-symbol;

(* 
    Names can be arbitrary, parses to from first to last non-whitespace character
    before the `;`.
*)
name-defblock = "name", where-symbol, { utf-8-char }-,;

(* 
    Version strings can be arbitrary, parses to from first to last 
    non-whitespace character before the `;`.
*)
version-defblock = "version", where-symbol, { utf-8-char }-,;

using-defblock =
    "using",
    where-symbol,
    semantic-block-start,
    [ metaspec-feature, { semantic-list-delimiter, metaspec-feature }]
    semantic-block-end;

truths-defblock =
    "truths",
    where-symbol, 
    semantic-block-start,
    semantic-evaluation,
    { ", ", semantic-evaluation },
    semantic-block-end;

(* For defining the language itself *)
language-defblock =
    "language",
    where-symbol,
    semantic-block-start,
    language-definition,
    semantic-block-end;

(* 
    These features import language features into scope.

    The syntax and usage of these features is tbc, and they appear in semantic
    portions of the defined language.
*)
metaspec-feature
    = "base"
    | "number"
    | "string"
    | "list"
    | "matrix"
    | "traverse"
    | "funcall" ;

(*
    The language described by the metalanguage is defined in terms of rules 
    that combine syntax definitions and semantics.

    The start symbol must be defined first, followed by the productions of the
    language.
*)
language-definition = start-rule, { language-rule };

non-terminal = non-terminal-start, non-terminal-identifier, non-terminal-end;
terminal = literal-quote, terminal-string, literal-quote;
start-symbol = start-symbol-start, non-terminal-identifier, start-symbol-end;

start-rule = start-symbol, defining-symbol, language-rule-body;

language-rule = non-terminal, defining-symbol, language-rule-body;

language-rule-body = 
    syntax-expression,
    rule-termination-symbol;

(* These NTs use definitions adapted directly from ISO 14977 - EBNF *)
syntax-expression = 
    syntax-alternative, { disjunction-symbol, syntax-alternative };

syntax-alternative = syntax-term, { syntax-term }, [ language-rule-semantics ];

syntax-term = syntax-factor, [ except-symbol, syntax-exception ];

syntax-exception = 
    ? a syntax-factor that can be replaced by one containing no NT symbols ?;

syntax-factor = [ integer, repeat-count-symbol ], syntax-primary;

syntax-primary = 
    syntax-optional | 
    syntax-repeated |
    syntax-grouped |
    syntax-special |
    non-terminal |
    terminal;

syntax-optional = optional-start-symbol, syntax-expression, optional-end-symbol;

syntax-repeated = repeat-start-symbol, syntax-expression, repeat-end-symbol;

syntax-grouped = group-start-symbol, syntax-expression, group-end-symbol;

syntax-special =
    special-sequence-start-symbol,
    text,
    special-sequence-end-symbol;

(* These productions define the syntax of the semantic definition blocks *)
language-rule-semantics = 
    semantic-behaves-as,
    semantic-block-start,
    semantic-rule,
    semantic-block-end;

semantic-rule = 
    environment-input-rule |
    environment-access-rule |
    special-syntax-rule |
    semantic-evaluation-rule-list;

environment-input-rule =
    semantic-type, 
    semantic-environment-symbol,
    semantic-environment-input-symbol,
    syntax-access-block, (* key *)
    environment-defines-symbol,
    syntax-access-list;

syntax-access-block = non-terminal, syntax-accessor;

syntax-accessor = 
    syntax-access-start-symbol,
    natural-number,
    syntax-access-end-symbol;

syntax-access-list = 
    syntax-access-block, { semantic-list-delimiter, syntax-access-block };

environment-access-rule = 
    semantic-environment-symbol,
    environment-access-symbol,
    syntax-access-block,
    { environment-access-symbol, syntax-access-block };

special-syntax-rule = 
    semantic-special-syntax,
    special-syntax-start,
    [ syntax-access-block | environment-access-rule ],
    { semantic-list-delimiter, (syntax-access-block|environment-access-rule) },
    special-syntax-end;

semantic-evaluation-rule-list = 
    semantic-evaluation-rule,
    { semantic-disjunction, semantic-evaluation-rule };

semantic-evaluation-rule = 
    semantic-type,
    semantic-identifier,
    where-symbol,
    semantic-operation-list,
    semantic-restiction-list,
    where-symbol, 
    semantic-eveluation-list;

semantic-eveluation-list = 
    semantic-evaluation,
    { semantic-list-delimiter, semantic-evaluation };

semantic-evaluation = 
    semantic-block-start
    semantic-type, 
    semantic-identifier,
    evaluates-to,
    [ syntax-access-block | special-syntax-rule ],
    semantic-block-end;

semantic-operation-list = 
    semantic-block-start,
    semantic-operation-assignment
    { semantic-list-delimiter, semantic-operation-assignment },
    semantic-block-end;

semantic-operation-assignment =
    semantic-identifier,
    semantic-assign,
    semantic-operation;

// Revise this section in response to live code changes
semantic-operation 
    = semantic-identifier
    | semantic-value
    | semantic-identifier-access
    | "(", semantic-operation, ")"
    | prefix-semantic-unary-operator, semantic-operation
    | semantic-operation, postfix-semantic-unary-operator
    | semantic-operation, semantic-binary-operator, semantic-operation;

semantic-identifier-access = 
    semantic-identifier, "|", natural-number, "|" | "[", natural-number, "]";

semantic-restriction-list = 
    restriction-block-start
    semantic-restriction,
    { semantic-list-delimiter, semantic-restriction },
    restriction-block-end;

semantic-restriction = semantic-identifier, semantic-restriction-check-operator,
    semantic-value | identifier;

(* The symbols here are dependent on the language using imports *)
semantic-restriction-check-operator = "==" | "!=" | "<" | ">" | "<=" | ">=";

semantic-value = string-literal | number | semantic-boolean;

semantic-boolean = "true" | "false";

semantic-binary-operator = 
    "+" |
    "-" |
    "*" |
    "/" |
    "%" |
    ":" |
    "^" |
    "|" |
    "||" |
    "&&" |
    "&" |
    "==" |
    "!=" |
    "<" |
    ">" |
    "<=" |
    ">=";

prefix-semantic-unary-operator = "!" | "-" | "++" | "--";

postfix-semantic-unary-operator = "--" | "++";

semantic-special-syntax 
    = "map"
    | "fold"
    | "filter"
    | "defproc"
    | "deffun"
    | "callproc"
    | "callfun"

\end{minted}

% chapter the_metaspec_grammar (end)

\chapter{The ABSOL Testing File} % (fold)
\label{cha:the_absol_testing_file}
This appendix contains the test file used to check the functionality of the metacompiler both during and after its development. 
The file is contained in the following listing and defines a toy language that doesn't do anything useful.
Its main intention is to test the parsing functionality of \gls{absol}, as well as the semantic verification facilities. 

\begin{minted}[fontsize=\blockfont]{text}
// Language metadata
name : simple_test;

version : 0.0.1a;

using : {
    number,
    base,
    traverse
};

// Specification of the base-case semantics for this test language
truths : {
    {integer n <= <integer>},
    {floating n <= <floating>},
    {number n <= <number>}
};

language : {

// The start symbol
<<simple_test>> ::= <statement> | <a> ;

// Productions whose semantics should be inferred.
<statement> ::= <my-number> | <number> | <arith-expr> | <test> | <assignment> ;
<my-number> ::= <integer> | <floating> ;

// Should have inferred semantics, mutually recurses with <arith-op>.
<arith-expr> ::= <my-number> | <arith-op> ;

// Mutually recursive productions with infinite syntax.
// These should not impact on successful verification.
<a> ::= <b> ;
<b> ::= <a> ;

// A production that should not have semantics as it's never used semantically.
<empty> ::= ""; 

// Testing user-defined semantics with both kinds of alternations. 
// Testing guard verification.
<arith-op> ::= 
    <arith-expr> "+" <arith-expr> --> {
        number n : {n = n1 + n2}() :
            {number n1 <= <arith-expr>[0]}, {number n2 <= <arith-expr>[1]}
    } |
    <arith-expr> "-" <arith-expr> --> {
        number n : {n = n1 - n2}() :
            {number n1 <= <arith-expr>[0]}, {number n2 <= <arith-expr>[1]}
    } |
    <arith-expr> "*" <arith-expr> --> {
        number n : {n = n1 * n2}() :
            {number n1 <= <arith-expr>[0]}, {number n2 <= <arith-expr>[1]}
    } |
    <arith-expr> "/" <arith-expr> --> {
        number n : {n = n1 / n2}() :
            {number n1 <= <arith-expr>[0]}, {number n2 <= <arith-expr>[1]}
    } |
    <arith-expr> "^" <arith-expr> --> {
        number n : {n = 1}(n1 == 1) :
            {number n1 <= <arith-expr>[0]}, {number n2 <= <arith-expr>[1]} |
        number n : {n = n1 * n1}(n2 == 2) :
            {number n1 <= <arith-expr>[0]}, {number n2 <= <arith-expr>[1]} |
        number n : {n = n1 ^ n2}() :
            {number n1 <= <arith-expr>[0]}, {number n2 <= <arith-expr>[1]}
    };

// Test for parsing and checking of environment input rules. 
<assignment> ::= <statement> "=" <arith-expr> <empty> --> {
    none e <-- <statement>[0] : <arith-expr>[0]
};

// Test for parsing and checking of special syntax rules. 
<test> ::= <statement> "traverse" <statement> --> {
    any map(<statement>[0], <statement>[1])
};

<env-access> ::= <statement> --> {
    any e.<statement>[0]
};

};

\end{minted}

% chapter the_absol_testing_file (end)


\end{appendix}

\end{document}
